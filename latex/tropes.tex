\section{Bureaucratic Tropes\label{sec:tropes}}

Any sufficiently large organization will have a diverse group of bureaucrats. From that diversity certain tropes emerge. The value of recognizing these recurring patterns is to leverage their strengths so that you and that person can be more effective. 

\begin{itemize}
    \item Excited intern or new employee. \\
    \textit{What you can do}: Their enthusiasm can be channeled to take on challenges other bureaucrats are scared of taking on for political reasons. Their naivete should be remedied by pairing with a mentor who can explain the history of the organization and specific challenges.
    
    \item Retired in place. \\
    \textit{What you can do}: Recognizing this attitude allows you to temper your own expectations about their productivity. 
    
    \item Font of institutional memory. Likes to explain how the current situation relates to his experience 10 years ago. \\
    \textit{What you can do}: You can benefit from talking to this person about historical processes and what has been tried before. 
    
    \item ``Let them eat cake'' managers who are removed from challenges and deny the problem is a problem. \\
    \textit{What you can do}: It is worth figuring out whether these managers actually don't care, or if they are busy fighting battles you are not exposed to.
    
    \item ``Don't rock the boat'' people who tried changing the system, didn't succeed, and are emotionally beaten down such that they no longer try. \\
    \textit{What you can do}: These people are often creative and can be consulted before you tackle a challenge they have experience with. 
    
    \item The Patriot who is willing to sacrifice for the organization. \\
    \textit{What you can do}: Educate them that there are alternative ways to be effective in a bureaucracy.
    
    \item Mismatched -- person's skills and interests don't match their role, so they are despondent and frustrated. \\
    \textit{What you can do}: You can provide value to this person by networking with them and finding better opportunities.
    
    \item Friendly grouch: friendly and helpful but unhappy in their job due to bureaucracy and other hurdles. \\
    \textit{What you can do}: Provide a copy of this book to increase their effectiveness and improve their spirits. They will be grateful to you and the organization will benefit.
    
    \item Detail-oriented: ask them what time it is and they'll tell you how to build a clock. \\
    \textit{What you can do}: You can serve as their ambassador to other members of the team or organization.  
\end{itemize}


% see related but not relevant
% https://blog.codinghorror.com/which-online-discussion-archetype-are-you/
The above personality tropes apply to individuals. If your team works on multiple projects, you may identify project tropes:
\begin{itemize}
    \item We've always done it this way.\\
    \textit{Assessment}: The team is scared of change. This change may be known (and be undesirable) or unknown. Change disrupts power.\\
    \textit{What you can do}: You can brainstorm with the team ways to incrementally improve. 

    \item Individual solutions to systemic challenges.\\
    \textit{Assessment}: Demonstrating small wins helps build credibility. \\
    \textit{What you can do}: Plan to ramp up in complexity (towards the systemic challenges) and spend the earned reputation.
\end{itemize}
