\subsection*{Who is a Bureaucrat?}

A cashier in a gas station is a bureaucrat. 
\index{exemplar!store clerk}
The shared resource is the gas and other items for sale. The ``policy'' might be ``take money from customer in exchange for items sold in the store and gas from the pumps,'' but the subjective application of that policy leaves a lot of room for the cashier to shape the customer's experience. Does the cashier greet the customer when the customer enters the store? Does the cashier look at the customer to acknowledge the customer? Smile? How quickly does the cashier engage the customer? Minor nuances left to the cashier in the execution of the store policy mean there is room for subjective application of the policy. 

% examples of bureaucrats
A bank teller, a loan officer, and a bank's \href{https://en.wikipedia.org/wiki/Technical_support}{technical support} 
\index{Wikipedia!\href{https://en.wikipedia.org/wiki/Technical_support}{technical support}}
\marginpar{[Wikipedia] Technical\\support}
are all bureaucrats. 
\index{exemplar!bank employees}
The shared resource includes money the bank manages and the expertise associated with managing money. Each employee subjectively enforces policies on behalf of the organization. Each role has a different amount of impact on the bank's finances. Of these three roles, the loan officer's interactions with bank customers provide the most straightforward route for feedback based on profit. The loan officer doesn't act alone though -- the customer's interactions with tellers and the bank's technical systems also matter to the customer's decision about how and where to manage money. 


This same discretionary application of policy applies to commercial bureaucrats like sandwich makers, car salespeople, oil well drillers, grocery clerks, retail clerks, 
\index{exemplar!store clerk}
and plumbers. Public school teachers, state and federal police, 
\index{exemplar!law enforcement officers (LEO)}
military members, 
\index{exemplar!military}
tax collectors, and other state workers are government bureaucrats. 


% bureaucracy is not limited to white collard office workers
Factory line workers subjectively apply policies, with the assembly line and factory being the shared resource. Enforcement of quality standards is a subjective policy. The pacing of work is a negotiation with management that directly impacts productivity and profits. The feedback loop for factory workers from profit generated by the product is weak and diffuse. %The shared resources governed by the worker's policies include the work environment, product quality, productivity, and profit.

Structured environments like sports featuring well-defined rules do not eliminate bureaucracy. Teammates use subjective policies on who to work with and how to best leverage their strengths and exploit the opponent's weaknesses. The policies are set in part by the coach. Referees make subjective determinations about rules.

Sometimes bureaucrats do not work directly with customers or citizens or products. Then the bureaucratic process is inflicted on fellow bureaucrats. In that scenario, a bureaucrat is subjectively applying a policy to other bureaucrats. 

\ \\


\begin{quizbox}{
      \textit{Self-administered exam}: 
      \textbf{Am I a Bureaucrat?}
}
Answer Yes or No for each of the following descriptions:
\begin{itemize}
    \item I rely on the knowledge of others. 
    \item The outcome of other people's decision-making informs my actions. 
    \item I administer subjectively determined policies for a shared resource. 
\end{itemize}
If all three of these are true for you in a specific context, then you are a bureaucrat. In situations where these conditions are not applicable, you are not a bureaucrat.
\label{box:self-administered-exam}
\end{quizbox} 
See Figure~\ref{fig:am-I-a-bureaucrat} \iftoggle{haspagenumbers}{ on page~\pageref{fig:am-I-a-bureaucrat}}{}
for a more detailed evaluation.

\ \\

Identifying yourself as a bureaucrat matters because the label informs the responsibilities of your role. The risk of not self-identifying as a bureaucrat is that you won't grasp how much control you have in enacting and enforcing policy. If you think of yourself as having to blindly follow rules, you will harm the people you are applying the rules to and the organization you are applying the rules for. The value of having capacity for judgment is so you can adapt policies to circumstances. Power is merely decisions that alter the opportunities of other people, so your self-perception of what decisions are available determine how much power you have.

In a similar sense from the consumer's or citizen's perspective, if you don't think you are interacting with a bureaucracy, you won't perceive the opportunity to negotiate.  If you view rules as unchanging and inflexible, you will harm your ability to make progress. If a human made a rule, then that rule is flexible. Who made the rule? Who enforces the rule? If you can talk to them, could they be convinced to make a modification or an exception? Sometimes the answer is yes.

If you don't think about the bureaucratic framing, you might think the store clerk is enforcing a policy because they don't like you. Assigning personality conflict as the cause might lead to a different conversation with their manager (the person who created the policy). 

If you don't think of yourself as a bureaucrat, you may behave passively in your job. The paradigm of ``just tell me what to do'' is the default (learned in school) and you won't know how to engage with coworkers and bosses since their role is distinct from that of a friend. You will be less likely to understand how to leverage members of your organization. Thinking from a bureaucrat's perspective explains why communicating within your organization is critical to your success and the organization's success. 
%You'll see your coworkers as competitors for promotion.

% https://graphthinking.blogspot.com/2020/10/impact-of-self-identifying-as-not.html
If you think of yourself as merely a cog in a machine, you are less likely to notice that you exert influence in the process and you are less likely to recognize the autonomy available to you. 
If you think ``I have a real job (e.g., nurse, cashier, teacher); therefore I'm not a bureaucrat", you are less likely to recognize the subjective power you have in interactions with the public.
If you think, ``I'm at the bottom of my organization's hierarchy; therefore I do not have power", you are less likely to notice autonomy when it is available.

If you don't consider yourself to be part of a bureaucratic process, you'll behave differently in interactions with bureaucrats.  You won't perceive opportunities to negotiate because the processes seem fixed instead of subjective. 
You won't recognize motives and incentives of bureaucrats, so their activities will seem incomprehensible. Once you identify as an important part of the bureaucratic process, you will be able to effectively engage, negotiate, and make meaningful changes happen.


