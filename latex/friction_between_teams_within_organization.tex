\subsection*{Friction between teams within an organization}

Ideally there is a clear division of responsibilities among different teams. (Because a team's scope tends to align with skills of the team members, the division of  responsibilities isn't always clear.)
With separate teams there is necessarily some interaction among teams -- one team may depend on the output from another team. Coordination among the teams regarding transfer of data or products or projects or knowledge is critical to the smooth operation of the organization. 

An organization has finite staffing, money, and time. Therefore, teams within the organization face a \href{https://en.wikipedia.org/wiki/Zero-sum_game}{zero-sum}
\index{Wikipedia!\href{https://en.wikipedia.org/wiki/Zero-sum_game}{zero-sum game}}
distribution of resources.

% TODO
Differences in decision-making perspective based on local conditions, stupidity, or different incentives, different definitions of success

When attempting to resolve friction between teams, there is an authority common to the teams, but that person lacks the nuanced insight, doesn't have time to get involved in every challenge, and doesn't want to micromanage multiple teams.