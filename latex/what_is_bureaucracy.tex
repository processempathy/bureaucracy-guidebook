\section{What is bureaucracy?\label{sec:define_bureaucracy}}

Bureaucracy is the specialization of roles necessitated by scaling and complexity for the distribution of common resources or widespread policy

In the course of carrying out someone else's subjectively defined policy, you have to make your own subjective decisions in the execution and enforcement in the course of carrying out someone else's subjectively defined policy, you have to make your own subjective decisions in the execution and enforcement


While you may know it when you see it or experience it, for this book definitions are useful. There are three distinct roles in bureaucracy: policy creator, policy enforcer, and the person upon whom policy is inflicted.

The existence of bureaucracy is independent of an organization's purpose.


The policy creator is either a politician or a bureaucrat. 

A \gls{bureaucrat} is a person subjectively interpreting policies on behalf of an organization and has discretionary enforcement to facilitate coordination of stakeholders. 

Let's break that down piece-by-piece. First, ``subjective interpretation'' means there is a person making a decision about how to do something. The subjectivity arises from different reasons one might choose an option over a competing option.  ``Policies" is a set of actions in a given circumstance. ``An \gls{organization}" is the collection of people for who the policy is made. ``Discretionary enforcement'' means the person is choosing how to apply the policy in the specific circumstances. ``Facilitating coordination'' means bureaucracy is about getting multiple people (or sometimes a person at different instances in time) to work together. The ``stakeholders'' is a group of people who care about the application of the action in each circumstance.  That's still pretty dense, so the rest of the book is spent expanding on the nuances and implications of this definition.

Bureaucracy is neither good nor bad. Bureaucracy is not tied to politics, or any specific institution (corporations, governments, academics). Bureaucracy is not defined to be efficient nor, does it have to be inefficient. Bureaucracy is not restricted to paperwork, or record keeping, or quantification, or gathering metrics. 

Bureaucracy is about delegation of control, communication, decision making, coordination, and processes. Involves negotiation, primarily informal. 

An organization comprised of bureaucrats is a \gls{bureaucracy}. The definition of bureaucracy used in this book is independent of government. Nothing in this definition involves paperwork or an office building. Definitions that limit the concept of bureaucracy to specific contexts result in a decreased ability to describe complex large-scale human organizations. 

The protagonist within a \gls{bureaucracy} is the \gls{bureaucrat} -- the person who is a member of an organization and is responsible for subjective implementation of policy for the organization. The person that a bureaucrat's decisions are inflicted on a \gls{subject}.  Depending on context, a subject may be a student (when the bureaucrat is a teacher) or a subject may be a citizen if the bureaucrat is a police officer or government official. Sometimes a bureaucrat's decisions are inflicted on other bureaucrats-as-subjects, such as when a Chief of Police creates guidelines for police in their district, or when a senior diplomat sets policy for embassy employees. 

A critical aspect of bureaucracy is that everything is made up, specifically by other humans. The consequence is that everything is negotiable. You (in the role of either a subject or a bureaucrat) need to know who to negotiate with and how to negotiate the desired changes. The only actual rules are mathematical physics that describe nature. Everything else is either naturally occurring macroscopic emergent phenomena (e.g., chemistry, biology) or humans making up labels and norms. 

Bureaucracy arises when there is no common objectively quantifiable feedback mechanism for individual participants in the organization. This aspect is why governments, schools, and prisons are characterized as bureaucratic. The military doesn't rank soldiers by ``number of enemies killed'' and is bureaucratic. Even profit-driven commercial organizations are bureaucratic when the impacts of individual employees are not coupled to sales metrics. 

Profit-based feedback makes some roles in a business context slightly more predictable and understandable, though there are still trade-offs like long-term profit versus short-term profit and externalization of harm. 

The concept of bureaucracy is most visible for complex, long lasting, and recurring situations involving many people. The apparent friction can be lower when there are only a few people involved (``I'm just talking to my collaborator" or ``I'm just buying groceries from a clerk at the store'' or ``I'm using a website for a government service''), but there is a continuous gradient. 

There is the external resource (mail delivery for USPS, public safety for FBI, environment for EPA) and there are resources internal to the bureaucracy. The focus of this book is on internal resources. In that context, bureaucracy is for the disseminated responsibility for use of resources: attention, skill, expertise. Time, money, staffing are proxy measures.



A useful way to think about bureaucracy is as a system for distributed knowledge and distributed decision making. That is in contrast to easier-to-understand concepts like centralized knowledge and centralized decision making. A government run by dictatorship is easy to conceptualize compared to democracies because there is a central character around which a narrative can be formed. Similarly, telling stories about the \href{https://en.wikipedia.org/wiki/Chief_executive_officer}{CEO} of a company is much easier than capturing the thousands of interactions conducted by the many employees of that company. Linear story-telling with a small number of protagonists does not map well to the complexities of bureaucracy. 
% are there alternatives to Bureaucracy that accomplish the same non-centralized non-consensus approach to complexity?


% https://graphthinking.blogspot.com/2017/09/market-friction-and-bureaucratic.html
Distributed knowledge and distributed decision making are hindered by
\begin{itemize}
    \item limited bandwidth between people, specifically the bureaucrats involved
    \item non-zero latency of information between people, specifically the bureaucrats involved
    \item the cost of measurement (getting data)
    \item the cost of analysis of the data
    \item making decisions that are suboptimal
\end{itemize}


\subsection{Bureaucracy as an economics model}
Firms exist in a market because negotiating contracts and prices for every interaction is burdensome. 
% https://www.kellogg.northwestern.edu/faculty/hubbard/htm/research/ec174/lectures/3coase.htm

Doesn't address small vs large companies, and doesn't distinguish between profit-oriented and non-profit and government. 

\subsection{Bureaucracy as emergent phenomenon}
Bureaucracy as a set of many bilateral interactions may not need to invoke emergence. However, there's a universality that hints at emergence. 

Above the threshold for emergence, there is scale-free behavior. The same patterns are observable at large organizations and extremely large organizations.

All those choices faced by the individual are not independent choices with respect to other bureaucrats in their environment. There is a flocking behavior of my choices are informed by the choices of those around me. Not necessarily in space.

Everyone is playing by different rules and has different objectives and everything is dynamic (both individuals and the conditions). 

Bureaucracy as a macroscopic phenomenon is emergent at sufficient scale. The scale is important because there is no longer dependence on individual relationships (beyond \href{https://en.wikipedia.org/wiki/Dunbar\%27s_number}{Dunbar's number}. There are people in the organization that you don't know and for which there is no common accountability. An organization subdivided into team recursively until there is local person-to-person accountability.  

The local rules bureaucrats employ to enable distributed decisions using distributed knowledge is meetings, processes, and communications. 

The relevance of making a claim that something is emergent is that there is behavior occurring at the macroscopic scale, and Knowing that individual motives and actions of every player at the microscopic level is not relevant.

% https://www.preposterousuniverse.com/podcast/2021/10/11/168-anil-seth-on-emergence-information-and-consciousness/
What does ``emergent'' mean? Nominal emergence example: a circle is emergent from a collection of points. Weak emergence is measurable using \href{https://en.wikipedia.org/wiki/Granger_causality}{Granger causality} or, equivalently\footnote{https://arxiv.org/abs/0910.4514}, \href{https://en.wikipedia.org/wiki/Transfer_entropy}{transfer entropy} (information theory). 


\subsection{Bureaucracy in Game Theory}
Bureaucracy does not fit cleanly into game theory categories of cooperative or competitive.

Maybe all the interactions within a bureaucracy are a bunch of small games?

Bureaucracy is self-modifying. 

Bureaucracy is in constant flux due to external conditions, externally imposed constraints, staff turn-over, internal dilemmas, disagreements of individuals. 


\subsection{Bureaucracy resists characterization}
Actually, bureaucracy is worse than emergent - the system rules can be altered or ignored by the stakeholders. \href{https://en.wikipedia.org/wiki/Wicked_problem}{Wicked problem}. This is why coming up with a holistic theory of bureaucracy is difficult. 

As soon as a claim is made, then a group can respond to that claim by behaving in an opposing manner. 

\subsection{Money as a fitness function}
Commercial businesses have a different accountability -- money. Common across all participants within the organization, and common with external stakeholders. The goal of a company is to generate profit. Commercial businesses have people who make subjective decisions and enforce policies, but there is a common metric for feedback. The feedback mechanism is not perfect. Being a good commercial bureaucrat does not necessarily result in monetary success.

Prisons, schools, medical, government, military all consume and spend money, but money isn't the goal. When faced with a decision, choice is not guided by which will generate more profit. 


\subsection{Bureaucracy as evolutionary outcome}

Biological, Genetic -- individual level
Biological, Genetic -- Group selection
Memetic

\subsection{Bureaucracy as Psychological Phenomenon}

Are you doing what's best for you, the group you're in, or everyone?
Altruistic or reciprocal? Retaliation
The answer changes time the time and situation of situation and person to person

Just a mixture of pathologies?