% Claim: you already have habits; these are similar to processes use by bureaucrats in organizations

Because bureaucratic policies and processes seem convoluted, let's start from a more relatable point: your personal habits.
You might eat at the same time every day, or you might go to bed at the same time. Those are both examples of personal policies. Sticking to a routine decreases the need to think about options. In the same way, bureaucrats in organizations seek routines to decrease uncertainty. 

You have experience with creating and using personal policies (your habits).  Bureaucratic processes for organizations are similar.
Both are routines which ease the burden of decision-making. A habit can be intentional (e.g., my policy is to brush my teeth before going to bed), just as a process can be designed. Habits can be unconscious (e.g., arriving at work without recalling driving there), just as processes can arise without an intentional design. 

Consider your intentional habits -- they are likely motivated by a personal policy. Similarly, bureaucrats acting on behalf of an organization apply processes that support policies of the organization.

\ \\

The sections of this chapter address 
the \hyperref[sec:definition-of-process]{what}\ifhaspagenumbers
(page~\pageref{sec:definition-of-process}), 
\else
, 
\fi
\hyperref[sec:why-processes-exist]{why}\ifhaspagenumbers
(page~\pageref{sec:why-processes-exist}), 
\else
, 
\fi
and 
\hyperref[sec:process-chaos]{how} 
\ifhaspagenumbers
(page~\pageref{sec:process-chaos}) 
\fi
of processes. 
The last sections of this chapter 
(\hyperref[sec:design-of-processes]{design} 
\ifhaspagenumbers
on page~\pageref{sec:design-of-processes} 
\fi
and 
\hyperref[sec:change-a-process]{change}\ifhaspagenumbers
on page~\pageref{sec:change-a-process})
\else
)
\fi
 
assume a level of control and autonomy that you may not think of you have. Regardless of where your role is in the hierarchy, your purpose is to provide value to the organization. That value can be negotiation with fellow bureaucrats on improvements to the design of task workflows. 

The foundation of your \gls{process empathy} is based on the sections in this chapter, whereas previous chapters have been focused on your skills as a bureaucrat. 