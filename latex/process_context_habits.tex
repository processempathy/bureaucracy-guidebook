% Claim: you already have habits; these are similar to processes use by bureaucrats in organizations

Because bureaucratic policies and processes seem convoluted, let's start from a more relatable point: your own personal habits.
You might eat at the same time every day, or you might go to bed at the same time. Those are both examples of personal policies. Sticking to a routine decreases the need to think about options. In the same way, bureaucrats in organizations seek routines to decrease uncertainty. 

You have experience with creating and using personal policies (your habits).  Bureaucratic processes for organizations are similar.
Both are routines which ease the burden of decision making. A habit can be intentional (e.g., my policy is to brush my teeth before going to bed), just as a process can be designed. Habits can be unconscious (e.g., arriving at work without recalling driving there), just as processes can arise without an intentional design. 

Consider your own intentional habits -- they are likely motivated by a personal policy. Similarly, bureaucrats acting on behalf of an organization apply processes that support policies of the organization.

