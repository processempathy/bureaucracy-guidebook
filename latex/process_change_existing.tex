\section{Change Existing Processes\label{sec:change-a-process}}

New processes can arise organically or be issued as top-down edicts. Processes evolve to fit the constraints of available resources and skills of bureaucrats and subjects. These causes of consistent flux mean updates to policies and processes are ongoing. Whether you are a bureaucrat administering a process, a subject burdened by a process, or a bureaucrat responsible for revising processes, you have input on changes.



Processes change over time because the conditions change. When a bureaucratic process is no longer providing value to the organization or the participants, change is needed. 

Changing complex processes is hard because individuals involved in the process may depend on steps that are not visible to other people.

Making change more complicated is the entanglement of interdependent processes. Processes do not exist in isolation. A version of Hyrum's Law~\cite{2017_Hyrum} 
% https://news.ycombinator.com/item?id=29848295
modified for a bureaucratic context is:
\begin{quote}
``With a sufficient number of stakeholders of a process,
it does not matter what you describe as the steps;
all aspects of your process will be depended on by somebody."
\end{quote}


When  a change to processes is attempted, change is slow to happen because of lag in feedback loops. 

\subsection*{Bureaucratic Inertia}

Creating an organization, recruiting staff, getting funding, getting office space, setting up communication technologies, and creating processes is a significant investment of time, money, and people. 
That same work applies when making changes to the size and scope of an organization (making it larger or smaller). 
Because this work and attention take time, there is a lag associated with changes to an organization that better reflect the scope of tasks and the number of tasks. 
The lag of changes as they ripple through an organization characterizes \href{https://en.wikipedia.org/wiki/Bureaucratic_inertia}{bureaucratic inertia}. 
\index{Wikipedia!\href{https://en.wikipedia.org/wiki/Bureaucratic_inertia}{bureaucratic inertia}}

On a more personal scale, the causes of bureaucratic inertia are attributable to changes to processes or creating new processes. A process displaces the need for bureaucrats to think, so new processes force bureaucrats to think about their roles or develop new skills. 


%Seemingly simple tasks like buying pens incurs significant overhead of work and time. 

Another source of bureaucratic inertia is the over-subscription of tasks to the amount of attention available and resources available. In the situation where there aren't enough resources or attention, then the response to some tasks is necessarily delayed or left unaddressed. Delayed tasks block the workflows that depend on those outcomes. 

%\subsection*{Why Processes Change}


%Any \href{https://en.wikipedia.org/wiki/Nash_equilibrium}{Nash equilibrium}
%\index{Wikipedia!\href{https://en.wikipedia.org/wiki/Nash_equilibrium}{Nash equilibrium}}
%is constantly being upset by the change in conditions and change in people (who have varying motives).


\subsection*{Why Change a Process}
% https://graphthinking.blogspot.com/2016/11/reflecting-on-mistake-leads-to-insight.html
Why changes to a process occur in a bureaucratic organization:
\begin{itemize}
    \item To make an improvement to an existing process that is working as desired (i.e., a more clever or efficient solution).
    \item Improving an existing process by undoing mistakes previously made.
    \item Inventing a new process where there previously was not one.
    \item If utility and improvement are difficult to quantify bureaucratic activities, then bureaucrats may be promoted based on change instead of whether value was delivered.
\end{itemize}

If you recognize that processes are evolutionary, then the right response is to allow and look for iterative change (fail fast) rather than trying to create static processes.

\subsection*{How to Change a Process}
% https://graphthinking.blogspot.com/2016/06/top-down-and-bottom-up-approaches-to.html

Change to a process can come from the top of the organization's hierarchy or from the bottom. 

When change is driven from the top-down,\\
\textit{Benefit}: Unified vision enables global optimization.\\
\textit{Inefficiency}: Can't see all the details from the top, so solutions may not fit well.\\
\textit{Resolution}: Better reporting up the chain.

When change is driven from the bottom up,\\
\textit{Benefit}: Each component in the hierarchy has local control, sees local aspects, and creates solutions for the local problem.\\
\textit{Inefficiency}: Local optimization\footnote{
``Blessed are those who optimize locally, for there is no glory in making the whole system work better.''~\cite{1996_unknown}} across multiple components in a workflow can yield suboptimal outcomes.\\
\textit{Resolution}: Each local component acts with the same objective.

Regardless of where you are in the hierarchy, there are generally applicable tips for changing a process. The first is to segment the stakeholders, the second is to find critical \href{https://en.wikipedia.org/wiki/OODA_loop}{OODA loops},
\index{Wikipedia!\href{https://en.wikipedia.org/wiki/OODA_loop}{OODA loop}} and the third is to alter how you notify stakeholders.

Even if you use the tactics described here, you are likely to encounter resistance. 
Transitioning from legacy processes to new processes impacts the bureaucrats involved. Changes have to overcome legacy investments. People can feel threatened by the change of role or skills.  

\subsection*{Tactic: Segment Stakeholders}

The people a process is inflicted upon are not all the same. Some example categories of bureaucratic subjects are
normal users, power users, and malicious users.
In a similar manner, 
\href{https://en.wikipedia.org/wiki/Market_segmentation}{segmenting}
\index{Wikipedia!\href{https://en.wikipedia.org/wiki/Market_segmentation}{Market segmentation}}\iftoggle{WPinmargin}{\marginpar{$>$Wikipedia: market segmentation}}{}
the stakeholders for a potential change can help you distinguish different levels of support or resistance. 

The concept of a \gls{stakeholder} is broader than just the people in various roles carrying out the tasks associated with a bureaucratic process. 
To be more effective in enacting change, identify people in the following groups. 
\begin{itemize}
 \item People interested in active collaboration. They may not share your zeal, but finding shared activities is helpful for participation.
    \item People who passively support the activity but do not provide resources. May provide feedback or enlarge the coalition.
    \item People who don't care and are not engaged.
    \item People who disagree with you. Seek these contrarians out to refine the idea or scope. Negotiate! Be open to evolving the concept to address or at least acknowledge their concerns.
    \item People who are actively working against you. Try to understand their motives. Not through speculation, but by direct discussion. Written communication is inadequate; face-to-face is best since you are more likely to be treated like a human rather than an idea or concept. 
\end{itemize}

Which segment a person is part of changes as your scope and timeline shifts. Their activities and priorities may cause their position to evolve. 

\subsection*{Tactic: Find Critical OODA loops}

Influence is not distributed equitably among members of a team or organization. For that reason, surveys may be misleading. If you want to change a process, focusing your attention on crucial influencers is more effective than trying to spend the same amount of time learning from every bureaucrat. 

The colloquial use of ``influencer'' doesn't reflect the bureaucratic sense of shaping the direction of an organization or team. Because the term ``leader'' is over-used, a more precise label is ``maven.'' Some mavens are highly visible, while other mavens work behind the scenes.
Mavens may or may not be highly placed in the hierarchy of the organization, and they may not have fancy titles. 

The approach to changing a process being described here takes two steps: figure out who matters in an organization or team, then change how those influencers think and act.

The first step is to use a social implementation of \href{https://en.wikipedia.org/wiki/PageRank}{PageRank}
\index{Wikipedia!\href{https://en.wikipedia.org/wiki/PageRank}{PageRank}}
to find the relevant mavens for a given topic or process. You start your search with random bureaucrats (sampled from across the organization). In a one-on-one interaction, ask the person who they would recommend talking to.
``Who else would you recommend talking to about this topic?" is the last question in the first conversation.
To start this search process, start with your first-order social connections. Cover both low-level bureaucrats and higher in the chain of command.

% https://graphthinking.blogspot.com/2016/01/methodology-for-people-acting-as.html
Once you've started your search for mavens, leverage the trust already in the social network by starting conversations with ``When I spoke with Bob he recommended I talk to you about $<$name of topic$>$."


Once you've identified mavens for the topic or process you want to change, the next step is to change how the mavens think. 

% https://graphthinking.blogspot.com/2016/03/how-to-evolve-organization-community-or.html
A model for human decision-making is the 
\href{https://en.wikipedia.org/wiki/OODA_loop}{OODA loop}.
\index{Wikipedia!\href{https://en.wikipedia.org/wiki/OODA_loop}{OODA loop}}
If you want to change a process, change the people; if you want to change people, change their OODA loop. That means changing the data the person is aware of (what they observe), changing the external incentives (how they orient), and teaching them new decision frameworks (e.g., cost-benefit models).



\subsection*{Tactic: Approval, Forgiveness, Opposition\label{sec:approval-forgiveness-opposition}}
% https://graphthinking.blogspot.com/2017/10/flipping-approval-mentatlity.html

Bureaucracy involves distributed decision-making. 
A common bureaucratic task is seeking consensus regarding action or spending resources. There are distinct options about how to get that consensus:
\begin{itemize}
    \item Seek approval before taking action. This approach incurs both providing justification and waiting.
    \item Ask forgiveness after taking action. Often viewed as being in contrast to seeking approval. Less delay, and usually works if things go well or if no one notices. 
    \item Notification of Intent with deadline for response. The window for response should be sufficient to actually allow feedback. If no response is provided default is for action to be taken.
    \item Solicit opposition before taking action\footnote{\href{https://www.dailykos.com/stories/2009/2/11/696188/-}{``Unless Otherwise Directed" in Iraq}}. This is a different framing from approval or forgiveness. It decreases the risk the approver has to take on.
\end{itemize}
The best way to proceed depends on the personalities of the people involved in building consensus and their relationships. 

Most organizations default to an approval-based  processes. Each new idea needs to be signed off as approved by a sequential list of \glspl{bureaucrat}. The sequential (not concurrent) process may be known in advance, or it may be ad hoc if the request is novel.

Relying on approval is harmful to innovation because sign-off by each bureaucrat is interpreted as ``I am 100\% in agreement with this.'' Each stakeholder has to bless innovation and tie their reputation to the outcome.

Soliciting opposition and a response of ``I won't stop this'' is a more useful paradigm. With the consensus process language changed to ``I won't stop this," then the bureaucrat reviewing the idea can avoid taking responsibility for the idea and therefore is not tying their reputation to the result.

% https://news.ycombinator.com/item?id=15407757

\noindent\hrulefill

\ \\

% TRANSITION to next section: process_deployment
The section above described tactics for changes to processes driven by changes of resources or staffing but assuming the underlying policy had remained the same. The next section describes the case where changes to a policy require that processes be revised. 