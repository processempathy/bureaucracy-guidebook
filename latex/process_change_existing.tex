\section{Change Existing Processes\label{sec:change-a-process}}

\subsection*{Bureaucratic Inertia}

Creating an organization, staffing, funding, getting office space, setting up communication technologies, and creating processes is a significant investment of time and money and people. Changes to the size of an organization (making it larger or smaller) to better reflect the scope of task and the number of tasks has a lag. Changes to processes or creating new processes displace how participants think of their roles, or rely on new skills. 

Seemingly simple tasks like buying pens incurs significant overhead of work and time. 

Another source is the over-subscription of tasks to amount of attention available and resources available. The response to some tasks is necessarily delayed or left unaddressed, impeding workflows that depend on those outcomes. 



\subsection*{Why Processes Change}


When impact of activities in an organization is unclear, bureaucrats are promoted based on change, not improvement.

Any \href{https://en.wikipedia.org/wiki/Nash_equilibrium}{Nash equilibrium}
\index{Wikipedia!\href{https://en.wikipedia.org/wiki/Nash_equilibrium}{Nash equilibrium}}
is constantly being upset by the change in conditions and change in people (who have varying motives).

Processes change over time because the conditions change. Processes are implemented differently than intended because the people implementing them are not the same people who came up with and designed them


\subsection*{Why Change a Process}
% https://graphthinking.blogspot.com/2016/11/reflecting-on-mistake-leads-to-insight.html
Why change:
\begin{itemize}
    \item To make an improvement to an existing processes that are working as desired (i.e., a more clever solution).
    \item Improving an existing process by undoing mistakes previously made.
    \item Inventing a new process where there previously was not one.
\end{itemize}

If you recognize that processes are evolutionary, then the right response is to allow and look for iterative change (fail fast) rather than trying to create static processes.

\subsection*{How to Change a Process}
% https://graphthinking.blogspot.com/2016/06/top-down-and-bottom-up-approaches-to.html
% How to change:
Top-down:\\
\textit{benefit}: unified vision enables global optimization.\\
\textit{inefficiency}: can't see all the details from the top, so solutions may not fit well.\\
\textit{resolution}: better reporting up the chain.

Bottom up:\\
\textit{benefit}: each component in the hierarchy has local control, sees local aspects, and creates solutions for the local problem.\\
\textit{inefficiency}: local optimization across multiple components in a workflow can yield suboptimal outcomes.\\
\textit{resolution}: each local component acts with same objective

\ \\

\href{https://en.wikipedia.org/wiki/Market_segmentation}{Segment}
\index{Wikipedia!\href{https://en.wikipedia.org/wiki/Market_segmentation}{Market segmentation}}
your stakeholders into 

\begin{itemize}
 \item people interested in active collaboration. They may not share your zeal, but finding shared activities is helpful for participation.
    \item people who passively support the activity but do not provide resources. May provide feedback or enlarge the coalition
    \item people who don't care and are not engaged.
    \item people who disagree with you. Seek these contrarians out to refine the idea or scope. Negotiate
    \item people who are actively working against you. Try to understand their motives. Not through speculation, but by direct discussion. Written communication is inadequate. 
\end{itemize}

Which segment a person is part of changes as your scope and timeline shifts. Their activities and priorities may cause their position to evolve. 

\ \\

% https://graphthinking.blogspot.com/2016/03/how-to-evolve-organization-community-or.html
% How to change
\begin{enumerate}
    \item Humans use \href{https://en.wikipedia.org/wiki/OODA_loop}{OODA loops}.
    \index{Wikipedia!\href{https://en.wikipedia.org/wiki/OODA_loop}{OODA loop}}
    \item To change an organization, change the OODA loops of participants.
    \item Not all humans are equally important in an organization. For that reason, surveys may be misleading. Organizations typically have a few dominant participants -- the mavens. These people may or may not be leaders in the organization.
    \item Use a social implementation of Page Rank to find the relevant participants. In a one-on-one interaction, ask the person who they would recommend talking to.
``Who else would you recommend talking to about this topic?" is the last question in the first conversation.
To start this search process, simply start with your first-order social connections. Cover both low-level participants and the chain of command.
\end{enumerate}

% https://graphthinking.blogspot.com/2016/01/methodology-for-people-acting-as.html
Leverage the trust already in the social network by starting conversations with ``When I spoke with Bob he recommended I talk to you about $<$name of topic$>$."





\ \\

Changing complex processes is hard because individuals involved in the process may depend on steps that are not visible to other people.


Processes do not exist in isolation
\footnote{\href{https://www.hyrumslaw.com/}{https://www.hyrumslaw.com/}}. % https://news.ycombinator.com/item?id=29848295



\subsection*{Tactic: Approval, Forgiveness, Opposition\label{sec:approval-forgiveness-opposition}}
% https://graphthinking.blogspot.com/2017/10/flipping-approval-mentatlity.html

Bureaucracy involves distributed decision-making. 
A common bureaucratic task is seeking consensus regarding action or spending resources. There are distinct options about how to get that consensus:
\begin{itemize}
    \item Seek approval before taking action. This approach incurs both providing justification and waiting.
    \item Ask forgiveness after taking action. Often viewed as being in contrast to seeking approval. Less delay, and usually works if things go well or if no one notices. 
    \item Notification of Intent with deadline for response. The window for response should be sufficient to actually allow feedback. If no response is provided default is for action to be taken.
    \item Solicit opposition before taking action\footnote{\href{https://www.dailykos.com/stories/2009/2/11/696188/-}{``Unless Otherwise Directed" in Iraq}}. This is a different framing from approval or forgiveness. It decreases the risk the approver has to take on.
\end{itemize}
The best way to proceed depends on the personalities of the people involved in building consensus and their relationships. 

Most organizations default to an approval-based  processes. Each new idea needs to be signed off as approved by a sequential list of \glspl{bureaucrat}. The sequential (not concurrent) process may be known in advance, or it may be ad hoc if the request is novel.

Relying on approval is harmful to innovation because sign-off by each bureaucrat is interpreted as ``I am 100\% in agreement with this.'' Each stakeholder has to bless innovation and tie their reputation to the outcome.

Soliciting opposition and a response of ``I won't stop this'' is a more useful paradigm. With the consensus process language changed to ``I won't stop this," then the bureaucrat reviewing the idea can avoid taking responsibility for the idea and therefore is not tying their reputation to the result.

% https://news.ycombinator.com/item?id=15407757

\noindent\hrulefill

\ \\

% TRANSITION to next section: process_deployment
The section above described tactics for changes to processes driven by changes of resources or staffing but assuming the underlying policy had remained the same. The next section describes the case where changes to a policy require that processes be revised. 