\subsection{Change Existing Processes\label{sec:change_a_process}}

\subsubsection{Bureaucratic Inertia}

Creating an organization, staffing, funding, getting office space, setting up communication technologies, and creating processes is a significant investment of time and money and people. Changes to the size of an organization (making it larger or smaller) to better reflect the scope of task and the number of tasks has a lag. Changes to processes or creating new processes displace how participants think of their roles, or rely on new skills. 

Seemingly simple tasks like buying pens incurs significant overhead of work and time. 

Another source is the over-subscription of tasks to amount of attention available and resources available. The response to some tasks is necessarily delayed or left unaddressed, impeding workflows that depend on those outcomes. 



\subsubsection{Why Processes Change}


When impact of activities in an organization is unclear, bureaucrats are promoted based on change, not improvement.

Any Nash equilibrium is constantly being upset by the change in conditions and change in people (who have varying motives).

Processes change over time because the conditions change. Processes are implemented differently than intended because the people implementing them are not the same people who came up with and designed them


\subsubsection{Why Change a Process}
% https://graphthinking.blogspot.com/2016/11/reflecting-on-mistake-leads-to-insight.html
Why change:
\begin{itemize}
    \item To make an improvement to an existing processes that are working as desired (ie a more clever solution)
    \item improving an existing process by undoing mistakes previously made
    \item inventing a new process where there previously was not one
\end{itemize}

If you recognize that processes are evolutionary, then the appropriate response is to allow and look for for iterative change (fail fast) rather than attempting to create static processes.

\subsubsection{How to Change a Process}
% https://graphthinking.blogspot.com/2016/06/top-down-and-bottom-up-approaches-to.html
% How to change:
Top-down
* benefit: unified vision --> global optimization
* inefficiency: can't see all the details from the top, so solutions may not fit well
* resolution: better reporting up the chain
Bottom up:
* benefit: each component in the hierarchy has local control, sees local aspects, and creates solutions for the local problem
* inefficiency: local optimization across multiple components in a workflow can yield suboptimal outcomes
* resolution: each local component acts with same objective


% https://graphthinking.blogspot.com/2016/03/how-to-evolve-organization-community-or.html
% How to change
1) humans use OODA loops
2) to change an organization, change the OODA loops of participants
3) not all humans are equally important in an organization. For that reason, surveys may be misleading. Organizations typically have a few dominant participants -- the mavens. These people may or may not be leaders in the organization.
4) use a social implementation of Page Rank to find the relevant participants. In a one-on-one interaction, ask the person who they would recommend talking to.
``Who else would you recommend talking to about this topic?" is the last question in the first conversation.
To initiate this search process, simply start with your first-order social connections. Cover both low-level participants and the chain of command.
% https://graphthinking.blogspot.com/2016/01/methodology-for-people-acting-as.html
Leverage the trust already in the social network by starting conversations with ``When I spoke with Bob he recommended I talk to you about $<$name of topic$>$."





*********************

Changing complex processes is hard because individuals involved in the process may depend on steps that are not visible to other people.


Processes do not exist in isolation
\footnote{https://www.hyrumslaw.com/} % https://news.ycombinator.com/item?id=29848295
