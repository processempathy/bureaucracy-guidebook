\subsection*{Make Effective Presentations\label{sec:effective-presentations}}

% https://graphthinking.blogspot.com/2011/10/presentation-notes.html

Speaking is vital at decision points in your career progress -- at interviews, competitions, gaining new collaborators at conferences. 
Because you want to avoid mistakes in those situations, you will need to prepare and practice.
Aim to impress your peers, supervisors, and subordinates. 

Breaking down the variables of your presentation can help identify questions to consider. 
\begin{itemize}
    \item \textbf{Purpose}: What's your goal? What does your audience want? Are those aligned? 
    \item \textbf{Scope}: What is the minimum information you need to convey your point, while balancing that against the need to provide enough evidence of your claims?
    \item \textbf{Audience}: What's their background? What are they seeking? What are they expecting?
    \item \textbf{Speaker}: What should your appearance be? Do you need to alter your normal enunciation, volume, or rate?
    %, verbal accent, enunciation, smell.
    \item \textbf{Slides}: What is the layout of your content? What's the content? 
    \item \textbf{Props}: Are there artifacts that demonstrate the point of your presentation?
    \item \textbf{Venue}: How big is the room? What technology is available? How is the lighting? Will you be competing with a noise source? Other visual distractions?
    \item \textbf{Time}: How much time do you have to prepare? Do you have access to the venue before the event? How much time do you have for presentation? For questions?
\end{itemize}

% from https://graphthinking.blogspot.com/2023/01/more-questions-to-ask-when-you.html
\label{sec:extending-Heilmeier}
Expanding on the question of purpose for your talk, a standard framing is 
\href{https://en.wikipedia.org/wiki/George_H._Heilmeier\%23Heilmeier's_Catechism}{Heilmeier's Catechism}.
\index{Wikipedia!Heilmeier's Catechism@\href{https://en.wikipedia.org/wiki/George_H._Heilmeier\%23Heilmeier's_Catechism}{Heilmeier's Catechism}}
Heilmeier's list of questions is concise enough to be memorable, but skips some relevant aspects. The list I recommend reflecting on builds on the original set:
\begin{itemize}
    \item (Heilmeier asks) What are you trying to do? Articulate your objectives using  no jargon.
    \begin{itemize}
        \item Should the solution be technical, social, or a process?
        \item Which aspects are quantifiable and which are qualitative?
    \end{itemize}
    \item (Heilmeier asks) How is it done today, and what are the limits of current practice?
    \begin{itemize}
        \item How did we get to the current situation? 
    \end{itemize}
    \item (Heilmeier asks) What's new in your approach and why do you think it will be successful?
    \begin{itemize}
        \item What has been tried before? Why did those efforts not succeed?
    \end{itemize}
    \item (Heilmeier asks) Who cares? If you're successful, what difference will it make?
    \begin{itemize}
        \item Who are the stakeholders? What involvement do you expect from each stakeholder?
    \end{itemize}
    \item (Heilmeier asks) What are the risks and the payoffs?
    \item What are the constraints?
    \begin{itemize}
        \item (Heilmeier asks) How much will [each milestone] cost?
        \item (Heilmeier asks) How long will [each milestone] take?
        \item What skills are needed for each milestone?
        \item How do you know that set of constraints is correct? Complete?
    \end{itemize}
    \item (Heilmeier asks) What are the midterm and final ``exams" to check for success?
    \begin{itemize}
        \item Who is evaluating the milestone artifacts?
        \item What will be measured to determine the success of each milestone?
        \item You should pre-register what counts as failure to enable accountability.
    \end{itemize}
\end{itemize}








In addition to the questions above, another way to think ahead is the sequential tasks associated with the presentation. 
The following can serve as a checklist.

\subsubsection*{Planning your presentation}

Evaluate your audience's experience and education before creating the presentation. The question you are addressing in your presentation may be independent of the audience, but the level of delivery depends on the audience's background.

When speaking to an audience outside your field, aim to use jargon your audience is familiar with. Another tactic is to look for areas of commonality and then build on that.

When introducing your topic, there are a few ways to open the talk:
\begin{itemize}
    \item \textbf{Compliment} the audience.
    \item \textbf{Humor}: tell a relevant joke.
    \item \textbf{Vulnerability}: Tell how this work makes you feel.
    \item \textbf{Numbers}: Explain context and relevance in terms of money, number of people involved, and size of the system.
\end{itemize}


%\begin{itemize}
%    \item . For example, a physicist uses ``quantization of energy" whereas an audience of mathematicians may be more comfortable with ``discrete energy levels."
%    \item Look for commonality. For example, both physicists and mathematicians use assumptions and build models.
%\end{itemize}

Prepare the content:
\begin{itemize}
    \item Tell a coherent story with a unified theme. Each slide should be logically connected to the following slide. Don't just put a bunch of slides with data or pictures together. You risk disorienting your audience.
    \item Run spell check before presenting.
    \item Translate your slides from your native language to the language of your audience.
\end{itemize}


For the visual content in presentations:
\begin{itemize}
    \item Switching between dark and light slides in a dark room stresses the eyes. The audience needs time to adjust to varying light levels.
    \item Images should use both color contrast and distinct symbols; this is called redundant coding. 
    \item Pick colors that accessible for colorblind audiences. For example green and magenta is better than green and red. 
    % https://jfly.uni-koeln.de/color/
    % prevoiusly http://jfly.iam.u-tokyo.ac.jp/color/
    \item Making slides appear ``professional" means adding non-informational content. This added content should be consistent, not distracting.
\end{itemize}

%Software:
Common choices include \LaTeX (specifically \href{https://en.wikipedia.org/wiki/Beamer_(LaTeX)}{Beamer}), 
\index{Wikipedia!Beamer@\href{https://en.wikipedia.org/wiki/Beamer_(LaTeX)}{Beamer software}}\iftoggle{WPinmargin}{\marginpar{$>$Wikipedia: Beamer}}{}
Microsoft \href{https://en.wikipedia.org/wiki/Microsoft_PowerPoint}{PowerPoint}, 
\index{Wikipedia!PowerPoint@\href{https://en.wikipedia.org/wiki/Microsoft_PowerPoint}{PowerPoint software}}
and Apple's \href{https://en.wikipedia.org/wiki/Keynote_(presentation_software)}{Keynote}. 
\index{Wikipedia!Keynote software@\href{https://en.wikipedia.org/wiki/Keynote_(presentation_software)}{Keynote software}}
%Less well-known are Prezi (and its open source equivalent, InkScape+Sozi add-on).
You are sending a few messages to your audience if you use Microsoft Word, a document PDF, Notepad, or any other non-presentation software to make a presentation.
The two messages are first, the audience isn't worth the time you needed to develop a proper presentation, and second, you are not technically savvy.

\ \\

%Presenting multiple topics within one presentation:
%Disparate topics require a segue to show why you are transitioning.
%Inter-relate the multiple topics 

Live demos can invigorate a boring presentation. Besides the interactivity aspect, the audience is excited by the risk of technical failure. 
If you plan to give a live demo, have screenshots of the process in the presentation. That way, if the demo fails you can show what was supposed to happen in the slides. If the demo works, skip the slides.


Practice the presentation (out loud in real-time) at least once.
Use the setup as close to reality as possible for practice sessions. Project onto the screen using the projector. This experiment will show if the color contrast is sufficient.

\ \\
%Prior to giving a talk at a remote (or unfamiliar) environment

Questions to ask your host:
\begin{itemize}
    \item Will I have a projector and screen for the presentation?
    \item For the technology I'm using (PowerPoint, PDF, Keynote), which version is provided? 
    Or does the speaker bring their computer? 
    Or is there internet access?
    \item How to best get the presentation content to the host? Bring a USB drive or laptop, or email the presentation file to the host?
\end{itemize}


\subsubsection*{Day of talk}

Appearance (suit and tie or jeans and t-shirt?):
\begin{itemize}
    \item Underdressed = I don't respect the audience.
    \item Overdressed = I'm better than you.
    \item Similar level of dress = I'm a peer.
\end{itemize}

\ \\

Some of these suggestions should seem glaringly obvious. They are here because I have seen them in ``official" presentations given by a ``professional." For example, do not curse. No profanity while speaking, in the slide presentation, or even the name of the file.

\subsubsection*{Before the talk begins}
\begin{itemize}
    \item Ensure your facility has power, a screen, a projector, a pointer, and any other necessary equipment. (Don't rely on your host to think of these things for you.)
    \item Ensure all equipment works and functions together.
    \item Before the presentation begins, use a slide to make announcements and reminders:
\begin{itemize}
    \item List the agenda if there are multiple speakers.
    \item Time talk begins, how long it will last.
    \item Reminder: Turn OFF Cell Phones. Airplane mode is insufficient since previously set alarms can ring.
\end{itemize}
    \item Announce whether to ask questions during the talk (i.e., the audience should interrupt) or to hold questions until done.
\end{itemize}

\subsubsection*{During the talk}
\begin{itemize}
    \item Opening: Thank hosts, inviters, and organizers. Establish a connection between you (the speaker) and the audience.
    \item Do not pace.
    \item Do not stand frozen.
    \item If you are the only person laughing at your joke, it isn't funny.
\end{itemize}

 \subsubsection*{End of Your Presentation}

\begin{itemize}
    \item Let the person asking the question finish. Even if you think you know what they are going to ask, wait.
    \item Restate the question to make sure the audience heard it and that you understood it.
    \item Aim for shorter responses. Save your longer answer for a follow-up discussion after the audience has been released.
\end{itemize}


%\subsubsection*{After the Presentation}
