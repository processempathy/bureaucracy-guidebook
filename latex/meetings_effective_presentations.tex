\subsection*{Make Effective Presentations\label{sec:effective-presentations}}

% https://graphthinking.blogspot.com/2011/10/presentation-notes.html

Speaking is vital at decision points in your career progress - at interviews, competitions, gaining new collaborators at conferences, impressing peers or supervisors or subordinates. It's best not to make these mistakes in those situations.

Breaking down the variables of presentation can help identify questions to consider. 
\begin{itemize}
    \item Audience. What's their background? What are they seeking? What are they expecting?
    \item Speaker. What should your appearance be? Do you need to alter your normal enunciation or volume or rate?
    %, verbal accent, enunciation, smell.
    \item Slides. What's the goal? Is the narrative logical? What is the layout of content? What's the content? 
    \item Props. Are there artifacts that demonstrate the point of your presentation?
    \item Venue. How big is the room? What technology is available? How is the lighting? Will you be competing with a noise source? Other visual distractions?
    \item Lead time. How much time do you have to prepare? Do you have access to the venue? 
\end{itemize}

In addition to those variables, another way to think ahead is the sequential tasks associated with the presentation. 
The following can serve as a checklist.

\subsubsection*{Planning your presentation}

Evaluate the experience and education background of your audience prior to creating the presentation. The question being answered in the presentation may be independent of audience, but the level of delivery depends on audience background.

When speaking to an audience outside your field aim to use jargon your audience is familiar with. Another tactic is to look for areas of commonality and then build on that.

In introducing your topic, there are a few ways to open the talk:
\begin{itemize}
    \item Compliment the audience.
    \item Humor: tell a relevant joke.
    \item Vulnerability: How does this work make me feel?
    \item Numbers: Explain context and relevance in terms of money, number of people involved, size of system.
\end{itemize}


%\begin{itemize}
%    \item . For example, a physicist uses ``quantization of energy" whereas an audience of mathematicians may be more comfortable with ``discrete energy levels."
%    \item Look for commonality. For example, both physicists and mathematicians use assumptions and build models.
%\end{itemize}

Prepare the content:
\begin{itemize}
    \item Tell a coherent story with a unified theme. Each slide should be logically connected to the following slide. Don't just put a bunch of slides with data or pictures together. You risk disorienting your audience.
    \item Run spell check before presenting.
    \item Translate your slides from your native language to the language of your audience.
\end{itemize}


For the visual content in presentations:
\begin{itemize}
    \item Switching between dark and light slides in a dark room stresses the eyes (need time to adjust to varying light levels).
    \item In figures, use both color contrast and distinct symbols to deal with colorblind audiences. See also \href{http://jfly.iam.u-tokyo.ac.jp/color/}{http://jfly.iam.u-tokyo.ac.jp/color/}.
    \item Making slides appear ``professional" means adding non-informational content. This added content should be consistent, not distracting.
\end{itemize}

%Software:
Common choices include \LaTeX (specifically \href{https://en.wikipedia.org/wiki/Beamer_(LaTeX)}{beamer}), 
\index{Wikipedia!\href{https://en.wikipedia.org/wiki/Beamer_(LaTeX)}{beamer}}
\marginpar{[Wikipedia] Beamer}
Microsoft \href{https://en.wikipedia.org/wiki/Microsoft_PowerPoint}{Powerpoint}, 
\index{Wikipedia!\href{https://en.wikipedia.org/wiki/Microsoft_PowerPoint}{Powerpoint}}
and Apple's \href{https://en.wikipedia.org/wiki/Keynote_(presentation_software)}{Keynote}. 
\index{Wikipedia!\href{https://en.wikipedia.org/wiki/Keynote_(presentation_software)}{Keynote}}
%Less well-known are Prezi (and its open source equivalent, InkScape+Sozi add-on).
If you use Microsoft Word, a document PDF, notepad, or any other non-presentation software to make a presentation, then you are sending a few messages to your audience:
first, the audience isn't worth your time needed to develop or learn a proper presentation, and second, you are not technically savvy.

\ \\

%Presenting multiple topics within one presentation:
%Disparate topics require a segue to show why you are transitioning.
%Inter-relate the multiple topics 

Live demos can invigorate a boring presentation. Beside the interactivity aspect, the audience is excited by the risk of technical failure. 
If you plan to give a live demo, have screenshots of the process in the presentation. That way, if the demo fails you can show what was supposed to happen in the slides. If the demo works, skip the slides.


Practice the presentation (out loud in real time) at least once.
If possible, use the setup as close to reality as possible for practice sessions. Project onto the screen using the projector. This will show if the color contrast is sufficient.

\ \\
%Prior to giving a talk at a remote (or unfamiliar) environment

Questions to ask your host:
\begin{itemize}
    \item Will I have a projector and screen for the presentation?
    \item For the technology I'm using (Powerpoint, PDF, Keynote), which version is in provided? 
    Or does the speaker bring their own computer? 
    Or is there internet access?
    \item How to best get the presentation content to the host? Bring a USB drive or laptop, or email the presentation file to the host?
\end{itemize}


\subsubsection*{Day of talk}

Appearance (suit and tie or jeans and t-shirt?):
\begin{itemize}
    \item Under-dressed = I don't respect the audience.
    \item Overdressed = I'm better than you.
    \item Similar level of dress = I'm a peer.
\end{itemize}

\ \\

Some of these suggestions may seem glaringly obvious. The reason they are here is because I have personally seen them in ``official" presentations given by a ``professional."
\begin{itemize}
    \item Do not curse (profanity while speaking, or in the slide presentation, or even the name of the file).
    
    \item 
\end{itemize}
\subsubsection*{Before the talk begins}
\begin{itemize}
    \item Make sure your facility has power, a screen, a projector, pointer, any other necessary equipment. [Your host may not think of these things for you.]
    \item Make sure all equipment works and functions together.
    \item Before the presentation begins, use a slide to make announcements and reminders:
\begin{itemize}
    \item List the agenda if there are multiple speakers.
    \item Time talk begins, how long it will last.
    \item Reminder: Turn OFF Cell Phones. Airplane mode is insufficient since previously set alarms can ring.
\end{itemize}
    \item Announce whether to ask questions during talk (interrupt) or to hold questions until done.
\end{itemize}

\subsubsection*{During the talk}
\begin{itemize}
    \item Opening: Thank hosts, inviters, and organizers. Establish a connection between you (the speaker) and the audience.
    \item Do not pace.
    \item Do not stand frozen.
    \item If you are the only person laughing at your joke, it isn't funny.
\end{itemize}

 \subsubsection*{End of Your Presentation}

\begin{itemize}
    \item Let the person asking the question finish.
    \item Restate the question to make sure the audience heard it and that you understood it.
    \item Give the minimal answer. Save the long answer for followup.
\end{itemize}


%\subsubsection*{After the Presentation}
