%\subsection*{Financial models of communication}

%This section analyzes communication from a financial perspective.
%Being late to a meeting is theft. 


\subsection*{Meeting time compared to Theft\label{sec:financial-models-of-communication}}
% https://graphthinking.blogspot.com/2021/02/organizations-value-things-more-than.html

In large organizations there can be significant attention given to small purchases. A multi-step review process may be incurred for a \$200 acquisition. (Typically the cost of the review process in terms of person-hours spent isn't part of the calculus.)

Another measurement of value is that if an employee were to steal even \$200 worth of materials, the organization would likely punish that employee.

In the book High Output Management~\cite{1995_Grove}, Grove points out that those metrics apply to tangible goods, but not to people's time. Consider a meeting of 10 people and each person's cost is \$200 per hour. 
\marginpar{$>$ Math}%
A wasted meeting is not unusual and would not incur bureaucratic review processes. The cost to the organization is fiscally the same -- \$2000. Similarly, consider an employee who is late and causes a loss of productivity. Merely depriving the organization of \$200 worth of time is not punished in the same way theft is.

In practice, organizations default to meetings (even recurring meetings) rather than not meet. And being late (by a few minutes) to a meeting is commonly accepted. 
We can debate the differences between theft of materials and theft of time, but these are financially indistinguishable. 


 