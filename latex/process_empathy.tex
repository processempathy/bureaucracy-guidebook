\section{Process Empathy\label{sec:process-empathy}}

Empathy typically refers to understanding the emotions of another person. To detect the emotional state of another person you rely on facial expressions (smiles for happiness, frowns for frustration), body language (arms crossed, turning away), speech (yelling in anger, singing for joy), and actions (hitting something). From these signals you evaluate a person's emotional state by recalling instances where you felt similar emotions. Emotional empathy helps you understand another person. 

A distinct form of empathy is the ability to understand the reasoning another person uses.  
\hyperref[sec:intellectual-empathy]{Intellectual empathy} is helpful for enabling a common understanding of an issue. That applies both one-on-one and with a team of people. Intellectual empathy is enacted by thinking about how another person would assess a situation and they might respond. Another technique is to imagine what reasoning they would use when presented with a set of facts. 


A third form of empathy is the focus of this book: the ability to perceive why individuals working together behave in recurring patterns. What incentives do people in each role have, what information does each person have, and how does that manifest as actions by the team or organization? Applying the paradigm of \gls{process empathy} helps you answer questions like ``Why is this taking so long?'' and ``Why is this action so apparently inefficient?'' The answers come from recognizing \hyperref[sec:unavoidable-hazards]{unavoidable hazards} of bureaucracy. As another example, the question ``Why are people under-trained for their role?'' is answered by constant \hyperref[sec:turnover]{turnover}. Process empathy relies on understanding bureaucracy so that you can distinguish which patterns are specific to the people involved versus specific to this organization versus generic to every organization.

Reasoning about a process is slightly different from process empathy. Reasoning assumes a holistic external view (taken by Gall in~\cite{2002_Gall}), whereas process empathy is about the perspective of individual people operating within their local context. Reasoning about teams and the aggregate organization invariably leads to observations of illogical and absurd outcomes, whereas a view based on process empathy reveals the causes of apparent paradoxes.

Process empathy is comprised of thinking beyond professional relationships with individuals and deeper than the abstraction of team-as-entity and organization-as-entity. 
Because process empathy is generic across bureaucracies, the benefit of this perspective is that you can gain insight about people you don't know doing work you don't understand.
Each application will have nuances specific to the people involved and the shared resource being managed, but process empathy provides a useful framework for navigating bureaucracy.


Process empathy involves asking questions about your situation:
\begin{itemize}
    \item What incentives do process participants face?
Also consider the (lack of) \hyperref[sec:feedback-loop-and-ripples]{feedback loops} (page~\pageref{sec:feedback-loop-and-ripples}).
    \item What unaligned goals do individual bureaucrats have that cause friction?
    \item What are the different paths individual bureaucrats take to work towards their goals?
See \hyperref[sec:dilemma-trilemma]{dilemmas} on page~\pageref{sec:dilemma-trilemma}.
    \item How to leverage the interplay between processes and interpersonal professional relationships?
See \hyperref[sec:reputation]{reputation} management on page~\pageref{sec:reputation}.
\end{itemize}

Rather than asking about ``what benefits the team?'' or ``what supports the organization?'' the paradigm of process empathy focuses on people and incentives. For you this translates to specific actions. For example, to be effective you need to 
identify rules (written and unwritten), determine who enforces what, consider the detection mechanics, and weigh the consequences of subversion against the benefits of breaking rules.


The question of ``when is process empathy applicable?" is answered by ``when bureaucracy is present?" which requires a definition of bureaucracy. 
