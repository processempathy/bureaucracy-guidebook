\subsection*{Decisions and Processes Create Bureaucratic Debt\label{sec:bureaucratic-debt}}

% https://graphthinking.blogspot.com/2017/09/bureaucratic-debt-and-what-to-do-about.html

Suppose a \gls{process} is enacted and later found to be ineffective. Some work is needed to revise the process and hopefully improve effectiveness (or an \hyperref[sec:exceptions-to-process]{exception} is needed).
\ifsectionref
; see section~\ref{sec:exceptions-to-process}).
\fi
\gls{bureaucratic debt} is 
\marginpar{[Tag] Definition} the work needed to change a process\footnote{Similar to \href{https://en.wikipedia.org/wiki/Technical_debt}{technical debt}
\index{Wikipedia!\href{https://en.wikipedia.org/wiki/Technical_debt}{technical debt}}
}. 
The bureaucratic debt is caused by choosing an easy solution now (with limited information or insufficient resources) instead of using an approach that would take longer to design and enact but be more robust.


Decisions made by \glspl{bureaucrat} occur in a resource constrained environment.
Getting information (measurement) and analysis are costly in terms of money, time, skill, and labor.
Each decision made results in options that are not explored. These missed opportunities are associated with short-term versus long-term trade-offs of costs.

The \href{https://en.wikipedia.org/wiki/Opportunity_cost}{opportunity costs}
\index{Wikipedia!\href{https://en.wikipedia.org/wiki/Opportunity_cost}{opportunity cost}}
(options the organization doesn't take) alters which future decisions become available.

\ \\

The purpose of defining bureaucratic debt as a concept is to capture the work resulting from decisions that would otherwise be unaccounted for.
Once the concept of bureaucratic debt is understood it can be tracked.

To document bureaucratic debt, you need to record aspects of decisions as they are made:
\begin{itemize}
    \item What is the decision to be made?
    \item When the decision was identified?
    \item When the decision was made?
    \item Who made the decision?
    \item What options were identified?
    \item Which option was chosen?
    \item Why was that option was chosen over the other options?
\end{itemize}
The purpose of documenting decisions is to enable both aversion to bad decisions and attraction to good decisions. That may sound strange, but the default of decision-makers is to apply the same behavior in future decisions. 
Without documenting decisions, there is no transparency, accountability, historical measure of progress, or ability to track dependencies. 

Creating a record of decisions is necessary but not sufficient. The documentation of decisions needs to be disseminated to stakeholders to enable accountability. This should occur as promptly as possible. 

Every bureaucrat exercises policies that apply to subjects, even if the subjects are other bureaucrats. What I'm describing above is beyond merely documenting what the processes and policies are for subjects. Documenting bureaucratic debt is for use internal to the team or organization.  

The scale of decision impact determines the level of documentation. ``Do I choose pencil or pen?" incurs negligible bureaucratic debt; therefore the documentation needed is also negligible. Projecting impact of decisions is a subjective prediction. 

%Similarities of tech debt and bureaucratic debt.
%In developing software, there are three artifacts: the software, documentation on how to use the software, and documentation on why to use the software. The two distinct types of documentation are typically combined in one document. Each of these three artifacts are independent. The ramification of this is that each artifact can be created independently, and it takes work to maintain synchronization of the artifacts. 

