\section{Processes Involving Two Organizations\label{sec:processes-two-organizations}}

When two independently-managed processes interact across organizational boundaries there can be friction. Process friction, especially between two  organizations that lack shared coordination, results in either \hyperref[sec:exceptions-to-process]{exceptions to the process}\iftoggle{haspagenumbers}{ (see page~\pageref{sec:exceptions-to-process}),}{,}
lying, or a \hyperref[sec:change-a-process]{change in process}\iftoggle{haspagenumbers}{ (see page~\pageref{sec:change-a-process}).}{.}


The following stories illustrate process friction between teams within an organization and process friction between organizations. The teams within an organization have shared objectives set by the organization, whereas separate organizations lack incentives to coordinate. 


Two teams operating within an organization, with a shared intent and working relationship, may be stuck with the process they have even when the friction is clear. When organizations lack a common objective,  friction can be even more significant. 

\subsection*{Example of coordinating processes between two Organizations}
This story is about receipts from one process being submitted for reimbursement to another process.
Submitting the receipt shifts the accountability and justification burden to the accepting bureaucrat.

% https://graphthinking.blogspot.com/2017/02/financial-motivation-in-bureaucracy.html
%\marginpar{[Tag] Story Time}
\index{story time!dental insurance}
%\begin{storytime}{Dental Insurance}
\begin{mdframed}[frametitle={Dental Insurance},frametitlerule=true,frametitlealignment=\centering]
I have dental insurance. I visited the dentist on December 2. One of the procedures during a routine cleaning was ``bitewing x-rays." This procedure is covered by my insurance, so I was surprised when I received a bill for it from my dentist a few weeks after my visit.

I called my dentist and they explained that the dental insurance had declined to pay for the procedure. I called the insurance company on January 3 and they confirmed that the procedure was covered by my policy. 
Every time I called the insurance provider I had to provide my social security number, date of birth, and zip code twice -- once to the automated system, a second time to the person I talk with. The insurance company had made a mistake and said they would cover the cost of the procedure. I followed up with my dentist and explained the situation.

I called the dentist to see if they had received payment yet. They had not, so I called the dental insurance provider again on February 6 and 8. On February 8 the insurance company said they would process the payment within 7-10 business days. I called again on February 20 and the claim hadn't been started within the insurance company. I spoke to the supervisor and she said she would personally visit the claims office within the insurance company.

From the perspective of the dentist, they are seeking money for the service they provided me.

From the perspective of the insurance company, delaying payment on a claim makes good financial sense -- the policyholder is likely to just pay the balance to avoid going to court with the dentist.

From my perspective, the question is whether chasing this issue makes financial sense. I think of my hourly rate as \$40, so after an hour the charge of \$38 would have been better to pay out of pocket. Effectively I'm devaluing my time. The emotional stress and thought-cycles spent are also relevant, though harder to quantify.

Streamlining bureaucratic processes does not occur automatically. There needs to be both incentive to change and authority to make the change. 
%\end{storytime}
\end{mdframed}

In the above story, the dental office wanted to be paid for services provided. The insurance company wanted to minimize the number of payments made. And I wanted to minimize my costs. While none of those are conflicting, each organization has separate objectives.

\subsection*{Deploying Bureaucrats to Different Teams or Organizations\label{sec:prisoner-exchange}}

One method of addressing friction between teams or organizations is to deploy a person across boundaries. This may not result in a specific change, but it can help members of both the originating team and the receiving team better understand their counterparts. For the rest of this section I'll use ``Team~A" as the origin and ``Team~B'' as the receiver, though the concept applies to an exchange of organization members. The name of the bureaucrat in this example is Mark. This example covers an informal deployment in which Mark remains employed with Team~A. While Mark is with Team~B he provides the management of Team~A with a weekly activity report. 

The easiest question regarding Mark's deployment to Team~B is ``How long?'' That depends on the purpose of the exercise and the complexity of the work Mark will be doing with Team~B, as well as how long Team~A can operate without Mark's contributions. What is a successful outcome for Mark? For Team~A? For Team~B? For the organization? How long will the integration of Mark onto Team~B take? Does Mark need time on Team~A to delegate his current work?

A deployment harms the productivity of originating Team~A by loss of staff. The deployment also harms Mark's productivity for his work with Team~A. The deployment harms the receiving Team~B since they have to train or integrate Mark. 

Incentives for this investment include cross-training for Mark, better process empathy for both teams, and temporary (exceptional) support. What is Team~B expecting from Mark? How will Team~A benefit? What is Mark expecting to gain from the deployment? Or is this a sacrifice on Mark's part? 

Cross-training Mark can improve Mark's productivity  and the teams. 
Team~B gets to hear an outsider's perspective from Mark, and Mark returns to Team~A with a broader perspective of the organization.
A deployment can build relationships among the teams. Both teams are better able to address process friction or exceptional interactions. 


Potential engagement modes for Mark: consultant (providing knowledge to Team~B), integree (increasing the capacity of Team~B), or shadowing (learning from Team~B). Shadowing can be of an individual on Team~B, or Mark can shadow Team~B by attending meetings. 


Mark could be deployed full-time for the entire duration, split his time between teams, or build up and then decrease over the course of the deployment.

Are there criteria for early termination of the deployment? Are there criteria for extending Mark's deployment?

If this sounds useful to your team, there are a few considerations. Are you on the originating team A or the receiving team B? Who gets deployed? How is Mark picked? Was the opportunity advertised? If you're receiving Mark, what are your acceptance criteria? How many concurrent deployments can your team support?


\subsection*{Service-level Agreements\label{sec:sla}}

When two teams or two organizations need to interact repeatedly, a formalized approach is to create a \href{https://en.wikipedia.org/wiki/Service-level_agreement}{service-level agreement}. 
\index{Wikipedia!service-level agreement@\href{https://en.wikipedia.org/wiki/Service-level_agreement}{service-level agreement}} 
While a service-level agreement (SLA) is constructive for outlining what each party expects from the other, within a bureaucracy an SLA is typically not a legally binding contractual agreement. Instead of a judge resolving disputes, an SLA within a bureaucracy may be adjudicated by a supervisor common to the two teams.

A service-level agreement within a bureaucracy is dependent on the goodwill and honor of the signatories. Since an SLA is a formalization of a relationship, it is subject to revision when there is turnover of signatories in the teams. 

A service-level agreement should include providing historical and live data to stakeholders so violations can be measured. Measurements are needed because enforcement of the SLA is by the signatories. Observability is key to accountability. 

A supervisor common to the parties may be invoked when an SLA is not met or if there is a dispute over the interpretation of an SLA. Usually the threat of invoking oversight is enough to coerce a change in negotiations. 
When the common supervisor gets involved, there are not many options available for sanctions. The supervisor could withhold bonuses or promotions,  assign a new team lead, or re-negotiate the SLA. 

Because creation of SLAs is burdensome to all parties involved, monitoring to enable enforcement incurs work, and punishment is limited, most cooperation between teams and between organizations is informal and ad hoc. Relying on personal relationships is easier than formal agreements. Reputations, both for individual bureaucrats and for teams and organizations, are then formed based on performance and reliability.

%\ \\

% TRANSITION to next section: process_mistakes
%The next section addresses mistakes, a common source of process friction. 
