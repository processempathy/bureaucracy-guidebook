% see https://en.wikibooks.org/wiki/LaTeX/Glossary

\newglossaryentry{organization}{
  name={organization},
  description={an assembly of teams. The name of the concept might be a corporation, an agency, a department, a bureau, or any other aggregation of smaller organizations. }
}

\newglossaryentry{culture}{
name={culture},
description={norms, expectations around interaction among people}
}

\newglossaryentry{stakeholder}{
name={stakeholder},
description={a person who cares about the process or the outcome; distinct from a participant}
}

\newglossaryentry{participant}{
name={participant},
description={a person who is expected to take action or make contribution}
}


\newglossaryentry{bureaucracy}{
    name=bureaucracy,
    description={subjective implementation of policies to facilitate coordination of stakeholders.}
}

\newglossaryentry{visible bureaucracy}{
name={visible bureaucracy},
    %name={bureaucracy, visible},
    description={procedures and processes are written down and can be discovered by stakeholders}
}
\newglossaryentry{invisible bureaucracy}{
name={invisible bureaucracy},
    %name={bureaucracy, invisible},
    description={procedures and processes are known to some stakeholders and are conveyed verbally to some of the other stakeholders.}
}

\newglossaryentry{process}{
name={process},
description={a task broken into a specified set of subtask dependencies.}
}

% https://graphthinking.blogspot.com/2021/07/bureaucracy-book-outline.html
\newglossaryentry{bureaucrat}{
    name=bureaucrat,
    description={a person responsible for subjective implementation of someone else's intent, with unquantifiable results. Examples of a bureaucrat role: teacher, police, government employee. Not bureaucrats: factory line worker, student}
}