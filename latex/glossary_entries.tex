% see https://en.wikibooks.org/wiki/LaTeX/Glossary
% and https://www.overleaf.com/learn/latex/Glossaries

\newglossaryentry{organization}{
  name={organization},
  plural={organizations},
  description={An assembly of teams. The name of the concept might be a corporation, an agency, a department, a bureau, or any other aggregation of smaller organizations}
}

\newglossaryentry{culture}{
name={culture},
plural={cultures},
description={Norms; expectations around interaction among people}
}

\newglossaryentry{reputation}{
name={reputation},
description={Your reputation as a bureaucrat is what other people expect from you}
}

\newglossaryentry{stakeholder}{
name={stakeholder},
plural={stakeholders},
description={A person who cares about the process or the outcome; distinct from a participant. Stakeholders can be subdivided by who is affected and who has control}
%descriptionplural={people who cares about the process or the outcome; distinct from participants}
}

\newglossaryentry{participant}{
name={participant},
description={A person who is expected to take action or make contribution}
}

\newglossaryentry{essential bureaucracy}{
name={essential bureaucracy},
text={Essential bureaucracy},
description={The minimum processes and staffing and skills necessary to address the complexity of the problem space}
}

\newglossaryentry{org chart}{
name={org chart},
plural={org charts},
description={\href{https://en.wikipedia.org/wiki/Organizational_chart}{organizational chart}  identifies formal roles and the formal relations among roles.}
}



\newglossaryentry{bureaucratic debt}{
name={bureaucratic debt},
plural={bureaucratic debts},
text={Bureaucratic debt},
% https://graphthinking.blogspot.com/2017/02/schedules-slip.html
description={The cost of work need to change a process caused by choosing an easy solution now instead of using a better approach that would take longer. Bureaucratic debt is a trade-off, conscious or unconscious, of what work to do and risks to take. More effort (work, time) could be spent building a better product, but customers want solutions now}
}

\newglossaryentry{subject}{
    name={subject},
    plural={subjects},
    description={The person experiencing bureaucracy. See page XIV of Lipsky~\cite{1983_Lipsky}}
}

\newglossaryentry{Prisoner's dilemma}{
    name={Prisoner's dilemma},
    description={Two or more people with incomplete information of a situation will make suboptimal choices compared to someone with perfect knowledge of the situation}
}



\newglossaryentry{reference experience}{
name={reference experience},
plural={reference experiences},
description={Something you have done or observed in the past}
}

\newglossaryentry{thought terminating}{
name={thought terminating},
description={\href{https://en.wikipedia.org/wiki/Thought-terminating_clich\%C3\%A9}{Thought terminating statements} at first sound reasonable but, upon reflection and analysis, are incorrect}
}


\newglossaryentry{policy}{
  name={policy},
  plural={policies},
  description={Formalized opinion that can be applied uniformly across scenarios with particular conditions}
}

% https://graphthinking.blogspot.com/2021/07/bureaucracy-book-outline.html
\newglossaryentry{bureaucrat}{
    name={bureaucrat},
    plural={bureaucrats},
    description={A person who is a member of an organization and is responsible for subjective implementation of policy for the organization. Conventional examples of a bureaucrat role: teacher, police, government employee}
}

\newglossaryentry{process empathy}{
name={process empathy},
description={what incentives do people in each role have, what information does each person have, and how does that manifest as actions by the team or organization}
}

\newglossaryentry{shared resource}{
name={shared resource},
plural={shared resources},
description={Tangible items (e.g., land, air, water) and expertise. Examples:
\begin{itemize}
\item A play put on by actors for an audience.
\item A movie in the theater.
\item A road used by cars and trucks and buses.
\item A parking lot -- public or private.
\item A kitchen with multiple users.
\item A viewing platform overlooking a waterfall.
\item A server hosting a webpage accessed publicly or only available internally to members of an organization.
\item Education - teachers
\item Health care - doctors, dental
\item Insurance
\item A hiking trail.
\item Law enforcement
\item Judicial system
\item A library.
\item Religion is a bureaucracy premised on the constraint of finite expertise concerning what God wants you to do. The shared resource is the expert's knowledge.
\item A stairwell.
\item A hallway used by renters in a multi-unit building, or used by hotel guests.
\item A house for a family. For example, a policy of taking your shoes off when you enter. 
\item Parks -- public (city, county, state, national) or private (e.g., paid access).
\item Safety of passengers on a plane, bus, train; quiet, no strong smells, no flashing lights.
\item Safety of cars and trucks on a road. 
\end{itemize}
Private or public is irrelevant; commercial or non-profit is irrelevant
}
}

\newglossaryentry{simple decision}{
  name={simple decision},
  description={Has one correct or beneficial choice and one or more wrong or harmful choices}
}

% https://tex.stackexchange.com/questions/69567/uppercase-word-in-glossary-lowercase-in-text
\newglossaryentry{bureaucracy}{
    name={bureaucracy},
    plural={bureaucracies},
    text={bureaucracy},
    description={
    An organization of bureaucrats comprises a bureaucracy. A bureaucracy facilitates coordination of stakeholders. 
    Bureaucracy is how large organizations make distributed decisions using distributed knowledge.    \\
    Everything in a bureaucracy is made up by other participants. \\
    Bureaucracy is a macroscopic phenomenon emergent at sufficient scale. The scale is important because there is no longer dependence on individual relationships. \\
    Bureaucracy arises when there is no common objectively quantifiable feedback mechanism for individual participants in the organization.\\
    Bureaucracy is a \href{https://en.wikipedia.org/wiki/Wicked_problem}{wicked problem}\cite{1973_Rittel}}
}

\newglossaryentry{feedback loop}{
name={feedback loop},
plural={feedback loops},
description={Consequences for the decision maker}
}

\newglossaryentry{ripple}{
name={ripple},
plural={ripples},
description={A decision maker selects an option and that propagates to schedule, what is possible, and dependent tasks}
}

\newglossaryentry{visible bureaucracy}{
name={visible bureaucracy},
    %name={bureaucracy, visible},
    description={Processes are written down and can be discovered by stakeholders}
}
\newglossaryentry{invisible bureaucracy}{
name={invisible bureaucracy},
    %name={bureaucracy, invisible},
    description={Processes are known to some stakeholders and are conveyed verbally to some of the other stakeholders}
}

\newglossaryentry{process}{
name={process},
plural={processes},
% aka procedure
% https://graphthinking.blogspot.com/2017/02/breaking-down-bureaucracy-process-roles.html
description={A task broken into a specified set of subtask dependencies, typically with subtasks in a sequence. 
Each task is associated with the application of policies enforced by bureaucrats. 
% Also known as a procedure. 
Two distinguishing features in the context of bureaucracy are authorization and justification.  
A process has inputs and outputs. 
A process can be decomposed into other processes. 
Processes operate on both information and tangible objects. 
Processes require \href{https://en.wikipedia.org/wiki/Work_(physics)}{work} and time. 
Processes are carried out by people or machines}
}

\newglossaryentry{process friction}{
name={process friction},
description={
process friction is caused by simplification of assumptions and the neglect of specific circumstances. Process friction results in the waste of resources (tangible or expertise), temporal inefficiency, emotional frustration, and social distrust of institutions
}
}