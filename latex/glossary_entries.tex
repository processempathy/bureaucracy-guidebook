% see https://en.wikibooks.org/wiki/LaTeX/Glossary
% and https://www.overleaf.com/learn/latex/Glossaries

\newglossaryentry{organization}{
  name={organization},
  plural={organizations},
  description={an assembly of teams. The name of the concept might be a corporation, an agency, a department, a bureau, or any other aggregation of smaller organizations}
}

\newglossaryentry{culture}{
name={culture},
plural={cultures},
description={norms, expectations around interaction among people}
}

\newglossaryentry{stakeholder}{
name={stakeholder},
plural={stakeholders},
description={a person who cares about the process or the outcome; distinct from a participant}
%descriptionplural={people who cares about the process or the outcome; distinct from participants}
}

\newglossaryentry{participant}{
name={participant},
description={a person who is expected to take action or make contribution}
}

\newglossaryentry{essential bureaucracy}{
name={essential bureaucracy},
description={the minimum processes and staffing and skills necessary to address the complexity of the problem space}
}

\newglossaryentry{bureaucratic debt}{
name={bureaucratic debt},
plural={bureaucratic debts},
description={the cost of work need to change a process caused by choosing an easy solution now instead of using a better approach that would take longer}
}

\newglossaryentry{subject}{
    name={subject},
    plural={subjects},
    description={the person experiencing bureaucracy}
}
\newglossaryentry{Prisoner's dilemma}{
    name={Prisoner's dilemma},
    description={Two or more people with incomplete information of a situation will make suboptimal choices compared to someone with perfect knowledge of the situation}
}

\newglossaryentry{thought terminating}{
name={thought terminating},
description={\href{https://en.wikipedia.org/wiki/Thought-terminating_clich\%C3\%A9}{thought terminating statements} initially sound reasonable but, upon reflection and analysis, are incorrect}
}

\newglossaryentry{presence creates priority}{
name={presence creates priority},
description={being physically at a person's desk motivates that person to respond better than calling them or emailing them}
}

% https://tex.stackexchange.com/questions/69567/uppercase-word-in-glossary-lowercase-in-text
\newglossaryentry{Bureaucracy}{
    name={bureaucracy},
    plural={bureaucracies},
    text={Bureaucracy},
    description={subjective implementation of policies to facilitate coordination of stakeholders}
}

\newglossaryentry{visible bureaucracy}{
name={visible bureaucracy},
    %name={bureaucracy, visible},
    description={procedures and processes are written down and can be discovered by stakeholders}
}
\newglossaryentry{invisible bureaucracy}{
name={invisible bureaucracy},
    %name={bureaucracy, invisible},
    description={procedures and processes are known to some stakeholders and are conveyed verbally to some of the other stakeholders}
}

\newglossaryentry{process}{
name={process},
plural={processes},
description={a task broken into a specified set of subtask dependencies}
}

% https://graphthinking.blogspot.com/2021/07/bureaucracy-book-outline.html
\newglossaryentry{bureaucrat}{
    name={bureaucrat},
    plural={bureaucrats},
    description={the person who is a member of an organization and is responsible for subjective implementation of policy for the organization. Conventional examples of a bureaucrat role: teacher, police, government employee}
}