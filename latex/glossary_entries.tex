% see https://en.wikibooks.org/wiki/LaTeX/Glossary
% and https://www.overleaf.com/learn/latex/Glossaries

\newglossaryentry{organization}{
  name={organization},
  plural={organizations},
  description={An assembly of teams, with teams being comprised of people. Alternative names for the concept might be a corporation, an agency, a department, a bureau, or any other aggregation of smaller organizations}
}

\newglossaryentry{culture}{
name={culture},
plural={cultures},
description={Norms; expectations each person has regarding interactions with other people}
}

\newglossaryentry{reputation}{
name={reputation},
description={Your reputation as a \gls{bureaucrat} is what other people expect from you}
}

\newglossaryentry{stonewalling}{
name={stonewalling},
description={The recipient of a request or question doesn't reply}
}


\newglossaryentry{slow-rolling}{
name={slow-rolling},
description={You get a response to your request or question, but the response isn't helpful. There is a delay in the outcome because you have to iterate to get an answer}
}

\newglossaryentry{bikeshedding}{
name={bikeshedding},
description={The recipient of a question or request focuses on unimportant details relative to the primary topic}
}

\newglossaryentry{red herring}{
name={red herring},
description={You send a message to another person and their response to your message is misleading, whether intentional or not. The respondent provides what looks like a reasonable answer but results in unproductive work}
}

\newglossaryentry{stakeholder}{
name={stakeholder},
plural={stakeholders},
description={A person who cares about the \gls{process} or the outcome; distinct from a \gls{participant}. Stakeholders can be subdivided by who is affected and who has control}
%descriptionplural={people who cares about the process or the outcome; distinct from participants}
}

\newglossaryentry{participant}{
name={participant},
description={A person who is expected to take action or make contribution}
}

\newglossaryentry{essential bureaucracy}{
name={essential bureaucracy},
text={Essential bureaucracy},
description={The minimum 
\iftoggle{glossarysubstitutionworks}{\glspl{process}}{processes}, 
staffing, and skills necessary to address the complexity, latency, and scale of the challenge}
}

\newglossaryentry{org chart}{
name={org chart},
plural={org charts},
description={\href{https://en.wikipedia.org/wiki/Organizational_chart}{Organizational chart} 
\index{Wikipedia!organizational chart@\string\href{https://en.wikipedia.org/wiki/Organizational_chart}{organizational chart}}
identifies formal roles and the formal relations among roles. The consequence of these relations is delegation of action and decision authority}
}

\newglossaryentry{bureaucratic debt}{
name={bureaucratic debt},
plural={bureaucratic debts},
text={Bureaucratic debt},
% https://graphthinking.blogspot.com/2017/02/schedules-slip.html
description={The cost of work needed to change a \gls{process} caused by choosing an easy solution now instead of using a better approach that would take longer or cost more money. Bureaucratic debt is a trade-off, conscious or unconscious, of what work to do and risks to take. More effort (work, time) could be spent building a better product, but subjects want solutions for access to the shared resource now}
}

\newglossaryentry{policymaker}{
   name={policymaker},
   plural={policymakers},
   description={The person who determines access to a shared resource by setting a \gls{policy}}
}

\newglossaryentry{subject}{
    name={subject},
    plural={subjects},
    description={The person experiencing \gls{bureaucracy}. The subject participates because they are seeking a shared resource and are coerced (through threat of violence or sanction) to participate. See page XIV of Lipsky~\cite{1983_Lipsky}}
}

\newglossaryentry{Prisoner's dilemma}{
    name={Prisoner's dilemma},
    description={Two or more people with incomplete information of a situation will make suboptimal choices compared to someone with perfect knowledge of the situation}
}

\newglossaryentry{reference experience}{
name={reference experience},
plural={reference experiences},
description={Something you have done or observed in the past that informs your current decision-making}
}

\newglossaryentry{thought-terminating}{
name={thought-terminating},
description={\href{https://en.wikipedia.org/wiki/Thought-terminating_clich\%C3\%A9}{Thought-terminating statements}
\index{Wikipedia!Thought-terminating cliche@\string\href{https://en.wikipedia.org/wiki/Thought-terminating_clich\%C3\%A9}{Thought-terminating clich\'e}}
at first sound reasonable but, upon reflection and analysis, are incorrect}
}


\newglossaryentry{policy}{
  name={policy},
  plural={policies},
  description={Formalized opinion that can be applied uniformly and impersonally across scenarios with particular conditions. Used in bureaucracy to 
  adjudicate access to \iftoggle{glossarysubstitutionworks}{\glspl{shared resource}.}{shared resources.} Every policy 
  used in a \gls{bureaucracy} is subjective because policies are made up by humans. The subjectivity of human-made policies is distinct the subjective enactment by bureaucrats}
}



% https://graphthinking.blogspot.com/2021/07/bureaucracy-book-outline.html
\newglossaryentry{bureaucrat}{
    name={bureaucrat},
    plural={bureaucrats},
    description={The person responsible for subjectively enacting someone else's policy addressing access to 
\iftoggle{glossarysubstitutionworks}{\glspl{shared resource}.}{shared resources.}
    Conventional examples of roles filled by bureaucrats: teacher, police, government employee}
}

\newglossaryentry{emotional empathy}{
name={emotional empathy},
description={Feeling how another person feels. This form of \href{https://en.wikipedia.org/wiki/Empathy}{empathy}
\index{Wikipedia!empathy@\string\href{https://en.wikipedia.org/wiki/Empathy}{empathy}}
is concerned with emotions like sadness, happiness, anger}
}

\newglossaryentry{intellectual empathy}{
name={intellectual empathy},
description={Thinking about how another person thinks. If you possess a capacity for \href{https://en.wikipedia.org/wiki/Theory_of_mind}{Theory of Mind}, 
\index{Wikipedia!Theory of mind@\string\href{https://en.wikipedia.org/wiki/Theory_of_mind}{Theory of Mind}}
you can predict another person's intent}
}

\newglossaryentry{decision archaeology}{
name={decision archaeology},
description={You can identify the evolutionary sequence of decisions that started with a specific issue and led to the current situation. 
Each successive contributor may have had good intentions, but their improvement may add complexity by introducing new dependencies. 
There are things that you might know that the original decision maker did not, and vice versa.
People with good intentions who don't know what they are doing or have limited context create suffering and waste.
In 
\href{https://en.wikipedia.org/wiki/G._K._Chesterton\%23Chesterton\%27s_fence}{Chesterton's fence}
\index{Wikipedia!Chesterton's fence@\href{https://en.wikipedia.org/wiki/G._K._Chesterton\%23Chesterton\%27s_fence}{Chesterton's fence}}
the admonition to, ``Go away and think" should be to go do decision archaeology}
}

\newglossaryentry{motive}{
name={motives},
plural={motives},
description={Internal motives are chosen by you.
For example, fear, social recognition, predictability}
}

\newglossaryentry{incentive}{
name={incentive},
plural={incentives},
description={What other people can offer you to change your behavior.
For example pay, promotion, titles, stability, and awards}
}

\newglossaryentry{process empathy}{
name={process empathy},
description={Cultivating process empathy requires considering
\begin{itemize}
\item Internal 
\iftoggle{glossarysubstitutionworks}{\glspl{motive}}{motives} and external 
\iftoggle{glossarysubstitutionworks}{\glspl{incentive}}{incentives}
that each person in each role has.
\item What information each person has.
\item The dilemmas that each person faces.
\item The interplay of motives, incentives, and information manifests as actions by the team or organization.
\item The \gls{decision archaeology} of determining why things are the way they are, what is the overarching intent.
\item How to modify existing processes and create new processes for  diverse stakeholders and complicated constraints.
\end{itemize}
Your Process Empathy rests on the definition of \gls{bureaucracy}}
}
% the goal of ABM is to search for explanatory insight into the collective behavior of agents obeying simple rules
% https://en.wikipedia.org/wiki/Agent-based_model

% solving specific practical or engineering problems.
% https://en.wikipedia.org/wiki/Multi-agent_system

% NO: 
% https://en.wikipedia.org/wiki/Actor%E2%80%93network_theory

\newglossaryentry{shared resource}{
name={shared resource},
plural={shared resources},
description={A community resource -- either tangible items (e.g., land, air, water) or intangible concepts like expertise and information. Access to the shared resource is constrained by subjective policies and enforcement. \\
Notably, shared resources do not have to be scarce. For example marriage licenses are not constrained by scarcity. Because marriage licenses confer special privileges legally and financially, access is limited. \\
The label of ``shared resoure'' is used here even when the status of whether the resource is shared is contentious. For example, a private property owner of land may view their land as not-shared, but another person may walk across that land. This may invoke involvement of the police to arrest the tresspasser. There's contention for access to the resource. \\
More examples of shared resources:
\begin{itemize}
\item A play put on by actors for an audience. The play is the shared resource; the norms of watching a play are the informal policies -- this is not a bureaucracy. There are also policies set by the theater. This constitutes a bureaucracy because ushers are the bureaucrats and the audience are subjects. 
%\item A movie in the theater. The informal norms of watching are enforced by the audience. The rules of the theater are another source of policy.
\item A road used by cars, trucks, and buses is a shared resource. Police are bureaucrats and drivers are the subjects.
\item A parking lot -- public or private. Formal policies set by the owner are inflicted by tow truck drivers; vehicle owners are the subjects.
\item You are a shared resource. That applies physically -- you can do manual labor, or mentally -- your attention. You set policies regarding access  like ``I don't work for free.'' This does not constitute bureaucracy because the roles are not held by different people. 
\item A kitchen with multiple users. In a residential apartment with a few roommates there is no separation between policymaker, bureaucrat, subject. At small scale (no formal roles) this is not a bureaucracy; this is just a negotiation amongst stakeholders.
\item A prison is a bureaucracy; guards are bureaucrats enforcing policy. The shared resource is the space and food and noise of the prison. Inmates are the subjects.
\item A viewing platform overlooking a waterfall is a shared resource. Policies are stated by signs (an artifact of bureaucracy) and enforced by park staff (the bureuacrats) on visitors (the subjects).
\item A server hosting a webpage accessed publicly or only available internally to members of an organization. The subjective policy for access is set by the server's administrator. The subjects are visitors to the webpage. There is no bureaucrat, so this is not a bureaucracy. 
\item Education is a shared resource. Teachers are bureaucrats when applying the school's policy. Students are subjects. The teacher enforcing a policy they made is not a bureaucracy.  
\item In the education system a teacher's expertise is a shared resource, and there are policies made regarding access to the teacher's expertise. 
\item A bulletin board for community announcements is a shared resource. Policies about what can be posted are enforced by the owner.
\item Health care is a shared resource; specifically Doctors and Dentists are experts accessed by the community. Bureaucrats in the health care provider system mediate access.
\item Insurance is a shared resource. Claims adjusters are bureaucrats.
\item A hiking trail is a shared resource. The norms associated with passing and overtaking on a hiking trail
relies on social convention and is informal. This does feature bureaucracy. 
\item Safety of cars and trucks on a road is a shared resource. Police are bureaucrats, drivers are subjects.
\item Law enforcement for a safe and just society is a shared resource. Law enforcement officers (e.g., police) are bureaucrats. 
Even though laws are written down (anyone could read them), there's still distributed knowledge and distributed decision-making. No one person knows all the laws. Decisions are made by a sprawling workforce.
\item The judicial system is a shared resource. Judges are bureaucrats and sometimes policymakers. Defendants and plaintiffs are the subjects. 
\item A library space is a shared resource, as are books in the library. Librarians are bureaucrats, patrons (readers) are subjects. Books in a personal library are not a shared resource since no access policy is needed.
\item A swimming pool with staff is a shared resource. If there is a single person managing the pool there is no bureaucracy since the \gls{policymaker} role and bureaucrat role are one person -- then there is no room for conflict or confusion in the interpretation of the policy regarding access to the pool. If the pool has multiple staff and one person makes rules and another person administers the rules for swimmers, then bureaucracy exists.
\item In the military, the warfighter is a shared resource. The equipment (tanks, guns, planes, boats) is the shared resource. The subject is the country's citizens. 
\item A church is a shared resource among the congregation.
\item Religion is a bureaucracy premised on constrained access to expertise concerning what God wants you to do. The shared resource is the expert's insights.
\item A stairwell is a shared resource. Example policy: don't sleep in the stairwell; don't sell items in the stairway. Here the bureaucrats are the building manager and the fire marshal. 
\item A hallway used by renters in a multi-unit building or used by hotel guests.
\item A house for a family. For example, a policy of taking your shoes off when you enter. The bureaucrat is the homeowner.
\item Parks -- public (city, county, state, national) or private (e.g., paid access). 
\item Safety of passengers on a plane, bus, or train; quiet, no strong smells, no flashing lights.
\end{itemize}
Private or public is irrelevant; commercial or non-profit is irrelevant. \\
Not a shared resource: personal property, both durable and consumable, that no one is contesting access to. If no one is seeking access, then no policy is needed. \\
Bureaucracies typically manage more than one shared resource.
As an example, the \href{https://en.wikipedia.org/wiki/Department_of_transportation}{Department of Transportation}
\index{exemplar!Department of Transportation@\string\href{https://en.wikipedia.org/wiki/Department_of_transportation}{Department of Transportation}}
manages  transit systems like roads and railways, the expertise to manage those resources, and money to fund the development and maintenance of transit systems}
}

\newglossaryentry{simple decision}{
  name={simple decision},
  description={Has one correct or beneficial choice and one or more wrong or harmful choices}
}

% https://tex.stackexchange.com/questions/69567/uppercase-word-in-glossary-lowercase-in-text
\newglossaryentry{bureaucracy}{
    name={bureaucracy},
    plural={bureaucracies},
    text={bureaucracy},
    description={
The process of enacting policies through subjective decisions made by individual participants, typically in the context of overseeing a 
\gls{shared resource}.
The resource can be either tangible or in the form of expertise or information. \\
Managing shared resources does not require bureaucracy (e.g., if there is consensus, or through the use of violence) and bureaucracy can exist without a shared resource being managed (e.g., malicious  processes inflicted by (threat of) violence.) \\
\index{Wikipedia!wicked problem@\string\href{https://en.wikipedia.org/wiki/Wicked_problem}{wicked problem}}
    Bureaucracy is a \href{https://en.wikipedia.org/wiki/Wicked_problem}{wicked problem}~\cite{1973_Rittel}} because bureaucracy necessarily involves humans. \\
    There are two subtopics that depend on the number of people involved: core 
    \iftoggle{glossarysubstitutionworks}{\gls{core bureaucracy}}{bureaucracy}
    and distributed
    \iftoggle{glossarysubstitutionworks}{\gls{distributed bureaucracy}}{bureaucracy}
}

\newglossaryentry{distributed bureaucracy}{
    name={distributed bureaucracy},
    text={bureaucracy},
    description={Within an \gls{organization} comprised of multiple 
    \iftoggle{glossarysubstitutionworks}{\glspl{bureaucrat}}{bureaucrats} 
    the concept of \gls{core bureaucracy} applies. A bureaucrat may set policy for other bureaucrats, and that policy is inflicted on other bureaucrats. \\
    %     The process of bureaucracy is how a group of individuals make distributed decisions using distributed knowledge
    No one bureaucrat can carryout the policy (due to complexity or scale), so  distributed knowledge and distributed decision-making are essential}
}

\newglossaryentry{core bureaucracy}{
    name={core bureaucracy},
    text={bureaucracy},
    description={The separate roles of \gls{policymaker}, \gls{bureaucrat}, and \gls{subject}. A minimum of three separate people occupy the three roles. \\
    Examples not meeting the definition of core bureaucracy:
    \begin{itemize}
        \item The manager of swimming pool that has no other staff.
        \item A ``no parking" sign.
    \end{itemize}
These two scenarios have a shared resource, policy maker, and subject but no separate bureaucrat}
}

\newglossaryentry{feedback loop}{
name={feedback loop},
plural={feedback loops},
description={Consequences for the decision-maker. 
\index{Wikipedia!skin in the game@\string\href{https://en.wikipedia.org/wiki/Skin_in_the_game_(phrase)}{skin in the game}}
Also known as ``\href{https://en.wikipedia.org/wiki/Skin_in_the_game_(phrase)}{skin in the game}"}
}

\newglossaryentry{ripple}{
name={ripple},
plural={ripples},
description={A decision-maker selects an option and that propagates to schedule, what is possible, and dependent tasks. There is no or very little consequence felt by the decision-maker}
}

\newglossaryentry{visible bureaucracy}{
name={visible bureaucracy},
    %name={bureaucracy, visible},
    description={
\iftoggle{glossarysubstitutionworks}{\Glspl{process}}{Processes}
are written down and can be discovered by 
\iftoggle{glossarysubstitutionworks}{\glspl{stakeholder}}{stakeholders}}
}

\newglossaryentry{invisible bureaucracy}{
name={invisible bureaucracy},
    %name={bureaucracy, invisible},
    description={
\iftoggle{glossarysubstitutionworks}{\Glspl{process}}{Processes}
are known to some 
\iftoggle{glossarysubstitutionworks}{\glspl{stakeholder}}{stakeholders} 
and are conveyed verbally to some of the other stakeholders}
}

\newglossaryentry{process}{
name={process},
plural={processes},
% aka procedure
% https://graphthinking.blogspot.com/2017/02/breaking-down-bureaucracy-process-roles.html
description={A sequence of tasks for the subject of bureaucracy. 
Each task is associated with the application of 
\iftoggle{glossarysubstitutionworks}{\glspl{policy}}{policies}
enforced by 
\iftoggle{glossarysubstitutionworks}{\glspl{bureaucrat}.}{bureaucrats.}
% Also known as a procedure. 
Two common tasks in the context of bureaucracy are getting authorization and providing justification.  
A process has inputs and outputs. 
A process can be decomposed into other processes. 
Processes can operate on both information and tangible objects. 
Processes require \href{https://en.wikipedia.org/wiki/Work_(physics)}{work} 
\index{Wikipedia!work, physics@\string\href{https://en.wikipedia.org/wiki/Work_(physics)}{work}}
and time. 
Processes are carried out by people or machines}
}

\newglossaryentry{process friction}{
name={process friction},
description={
Caused by the simplification of assumptions and the neglect of specific circumstances. Process friction results in the waste of resources (tangible or expertise), temporal inefficiency, emotional frustration, and social distrust of institutions}
}