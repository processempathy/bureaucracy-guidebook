\section{Scope of Bureaucracy}
Bureaucracy is considered a trait of an organization: we call organizations bureaucratic. This is consistent with the definition of bureaucracy based on managing \glspl{shared resource}. Examples of organizations considered bureaucratic include
  \begin{itemize}
      \item The \href{https://www.epa.gov/}{Environmental Protection Agency} (EPA) manages access to bureaucrats who have expertise with environmental regulations governing water, air, workplace safety, etc.
      \item The \href{https://www.intelligence.gov/}{United States Intelligence Community} manages access to bureaucrats who have expertise in collecting foreign intelligence.
      \item The information technology department in a large organization manages staff with expertise in repairing computers.
      %\item Human resources department
      \item Bureaucrats in the \href{https://www.fda.gov/}{Food and Drug Administration} (FDA) use their expertise regulating the safety of food to serve their community.
      \item Bureaucrats of the \href{https://www.faa.gov/}{Federal Aviation Administration} (FAA) have expertise on flight safety to protect their community.
      \item The military manages the shared resource of bureaucrats with expertise in war-making.
  \end{itemize}

That description of bureaucracy is conceptually consistent but doesn't provide value to you, the practicing bureaucrat. Bureaucracy is not a distant faceless concept. Bureaucracy permeates your environment and is the basis for many interactions you have with other people. 

When you recognize bureaucracy specific to your environment, you will be able to respond appropriately. If you recognize the systemic incentives driving a person's behavior, you are less likely to see the person as malicious or incompetent. 