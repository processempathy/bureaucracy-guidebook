\section{The Scope of Bureaucracy}
% THESIS FOR THIS SECTION
Bureaucracy is widespread and therefore worth learning about so you can be effective.  When you recognize bureaucracy specific to your environment, you will be better able to respond appropriately. If you recognize the systemic incentives driving a person's behavior, you are less likely to see the person as malicious or incompetent. You are also more likely to be able to negotiate with them and enact the changes that you seek.

%\ \\

% THESIS FOR THIS PARAGRAPH
The definition of bureaucracy in this book (managing access to 
\iftoggle{glossarysubstitutionworks}{\glspl{shared resource}}{shared resources}) 
is consistent with conventional perceptions of bureaucratic organizations. 
%Bureaucracy is a trait of an organization: we call organizations bureaucratic. This is consistent with the definition of bureaucracy based on .  \iftoggle{glossaryinmargin}{\marginpar{[Glossary]}}{}
%Bureaucracy arises when management of a shared resource is necessary.
That resource can be external to the organization or internal to the organization. Examples of external shared resources include:

  \begin{itemize}
      \item The \href{https://en.wikipedia.org/wiki/United_States_Environmental_Protection_Agency}{Environmental Protection Agency} (EPA) manages access to bureaucrats who have expertise with environmental regulations governing water, air, workplace safety, etc.
      \index{exemplar!Environmental Protection Agency (EPA)}
      \index{Wikipedia!Environmental Protection Agency@\href{https://en.wikipedia.org/wiki/United_States_Environmental_Protection_Agency}{Environmental Protection Agency}}%

      \item Mail delivery by the \href{https://en.wikipedia.org/wiki/United_States_Postal_Service}{United States Postal Service}.
\index{exemplar!United States Postal Service (USPS)}%
\iftoggle{WPinmargin}{\marginpar{$>$Wikipedia: USPS}}{}
\index{Wikipedia!United States Postal Service@\href{https://en.wikipedia.org/wiki/United_States_Postal_Service}{United States Postal Service}}

      \item Public safety for the \href{https://en.wikipedia.org/wiki/Federal_Bureau_of_Investigation}{Federal Bureau of Investigation} (FBI).
\index{exemplar!Federal Bureau of Investigation (FBI)}%
\index{Wikipedia!Federal Bureau of Investigation@\href{https://en.wikipedia.org/wiki/Federal_Bureau_of_Investigation}{Federal Bureau of Investigation}}%

      \item The \href{https://www.intelligence.gov/}{United States Intelligence Community} manages access to bureaucrats who have expertise in collecting foreign intelligence.

      \item Bureaucrats in the \href{https://www.fda.gov/}{Food and Drug Administration} (FDA) use their expertise to regulate food safety to serve their community.
      \index{exemplar!Food and Drug Administration (FDA)}


      \item Bureaucrats of the \href{https://www.faa.gov/}{Federal Aviation Administration} (FAA) have expertise on flight safety to protect their community.
      \index{exemplar!Federal Aviation Administration (FAA)}


      \item The information technology department in a large organization manages staff with expertise in repairing computers.%
      \index{exemplar!information technology department}
      %\item Human resources department
      \item The military manages the shared resource of bureaucrats with expertise in war-making.
      \index{exemplar!military}
  \end{itemize}

There are also resources internal to an organization like attention, skill, and expertise. Those shared internal resources get quantified as time, money, and staffing. While talking about trade-offs of time, money, and staffing are easy, keep in mind they are proxy measures for the central intangible resources like attention and expertise.
The bureaucracy of managing access to shared resources is conceptually consistent but doesn't provide actionable insight to you, whether you are a practicing bureaucrat or subject. 
Bureaucracy is not a distant, faceless concept. Bureaucracy permeates your environment and is the basis for many interactions you have with other people. 
Actionable concepts are highlighted throughout this book in the margins.
