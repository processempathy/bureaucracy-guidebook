\subsection*{A Well-run Meeting\label{sec:well-run-meeting}}

Being an effective bureaucrat means leveraging the distributed knowledge present on your team and in your organization.
Explaining how to run an effective meeting (the scope of this section) is easy compared to the work of implementing those techniques. 
% DESCRIPTION
The extra work required to facilitate an effective meeting means the default of unproductive discussions is a \href{https://en.wikipedia.org/wiki/Nash_equilibrium}{Nash equilibrium}. 
\index{Wikipedia!Nash equilibrium@\href{https://en.wikipedia.org/wiki/Nash_equilibrium}{Nash equilibrium}}\iftoggle{WPinmargin}{\marginpar{$>$Wikipedia: Nash equilibrium}}{}
In general, a Nash equilibrium is when each participant in a scenario can improve their situation if no one else changes their approach. 
%No single participant benefits by reducing the number of meetings or effectiveness even though the group would benefit. 
We all sit in ineffective meetings because that is the easiest thing to do. 

% PRESCRIPTION
Good news: you can choose to take action that disrupts the equilibrium. The positive view of that action is that you can serve as a role model and demonstrate behaviors of other people can learn from. The negative view is that you are relieving the need for other people to do work.

\ \\

Most bureaucratic meetings lack the formality and structure  covered by \href{https://en.wikipedia.org/wiki/Robert\%27s_Rules_of_Order}{Robert's Rules of Order} (a set of parliamentary procedures).\footnote{I have not used Robert's Rules of Order for a meeting internal to a bureaucracy. Robert's Rules of Order does not appear to be commonly used inside bureaucracies. Some public-facing meetings may be, but not internal to the organization.} 
\index{Wikipedia!Robert's Rules of@\href{https://en.wikipedia.org/wiki/Robert\%27s_Rules_of_Order}{Robert's Rules of Order}}\iftoggle{WPinmargin}{\marginpar{$>$Wikipedia: Robert's Rules of Order}}{}
The level of formality for meetings in bureaucratic organizations varies widely. Most commonly bureaucrats rely on personal relationships and social norms in meetings of less than 15 people. In that context there are tips for effective interactions. Although many of these may seem obvious, each of them requires extra work. 

\index{list of tips!meetings}

\subsubsection*{Before the Meeting}

\ \\
\begin{samepage}
\textit{Meeting suggestion}: \textbf{Form relationships and understand constituents before the meeting}.\\
To minimize surprise during meetings, socialize the concepts before the meeting. This concept is called 
\href{https://en.wikipedia.org/wiki/Nemawashi}{nemawashi}.
\index{Wikipedia!nemawashi@\href{https://en.wikipedia.org/wiki/Nemawashi}{nemawashi}}\iftoggle{WPinmargin}{\marginpar{$>$Wikipedia: Nemawashi}}{}
\end{samepage}

If you don't have relationships or if the topics are a surprise to meeting participants, time will be spent educating and setting definitions. This is an inefficient use of time when different participants have different backgrounds and motives.

\ \\
\begin{samepage}
\textit{Meeting suggestion}: \textbf{Don't invite everyone}.\\
This may seem counter-intuitive since bureaucracy is about coordination, but having too many attendees wastes the time of people who have no input on the issue. 
To identify essential attendees you need to think ahead about which decisions are going to be made and who has the authority to make a decision. If someone does not need to be present, tell them in advance that you will share the meeting notes afterward. See the \hyperref[table:dilemma-personal-scope-of-speaking]{Dilemma of Speaking Scope}.
\marginpar{See page~\pageref{table:dilemma-personal-scope-of-speaking}.}
\end{samepage}

\ \\
\begin{samepage}
\textit{Meeting suggestion}: \textbf{Create and use an Agenda}.\\
% description
\textit{Bad}: Having no meeting agenda. As a consequence, much time is spent rehashing previous topics. No action items are recorded so accountability is more challenging. \\
% prescription
\textit{Good}: Have an agenda for the meeting. \\
\textit{Better}: \underline{Share the agenda with other participants} before the meeting. Having an agenda keeps attendees focused.  Enables tracking of progress during the meeting so participants are more likely to get to all topics.\\
\textit{Best}: For formal meetings, \underline{share the agenda in writing before meeting}. Sharing the agenda in advance allows attendees to research topics, gather data, do analysis, and form opinions. Some attendees might be good at quick responses, others need time to think about a topic of discussion. Opinions formed before the meeting decrease the risk of groupthink. 
\end{samepage}

% Why "bad" is the default
The primary reason agendas don't happen is that takes time to create an agenda. When an agenda is created and shared, attendees might not take the time to read the agenda before the meeting. 

Attendees may not stick to the agenda during the meeting. Enforcing agenda discipline requires someone willing to combat the natural entropy of human interactions. 



\ \\
\begin{samepage}
\textit{Meeting suggestion}: \textbf{Ensure facilities are adequate}.\\
For formal in-person meetings verify the meeting venue has enough space, seating, and working IT equipment. For formal virtual meetings ensure participants are familiar with virtual meeting controls. 
\end{samepage}

%  why logistics and infrastructure matter in a bureaucracy
As a bureaucrat you may not see taking care of details like this as your responsibility. Ensuring well-run logistics and effective use of infrastructure may not be your area of expertise, you lack training in this domain, it's not in your job description, and you won't get promoted for taking care of it. 

Attention to detail outside the scope of your official duties helps your reputation. Proactive concern for the smooth operation of the bureaucracy enables efficiency in time and resources.

% TODO: forces conspiring against logistics and infrastructure


\subsubsection*{During the Meeting}

\ \\
\begin{samepage}
\textit{Meeting suggestion}: \textbf{Body language matters}.\\
If you are not speaking, are you reclined or leaning forward on the edge of your seat? Are you looking at the speaker?
\end{samepage}

If your eyes are closed, other people don't know if you're picturing something or falling asleep. 

If you approve of something but don't want to verbally interject, a thumbs up is useful signaling. 

\ \\
\begin{samepage}
\textit{Meeting suggestion}: \textbf{Take and share Meeting Notes}.\\
Meeting notes can be more detailed than the agenda but less detailed than a transcript of who said what. Meeting notes synthesize the discussion. Meeting notes specify who is taking which follow-on actions with what deadlines. 
\end{samepage}

There is a spectrum of options for meeting notes. 
\begin{itemize}
    \item No one takes notes. The arguments used to defend this approach include, ``I'm too busy participating in the discussion," ``I'm not able to write summaries during the interaction," and ``Taking notes isn't my responsibility." In practice the memory of each attendee is not as good as the attendee might think.
    \item One person takes notes for everyone. The person in the role of scribe could be the same at every meeting or on a rotating basis among participants.
    \item Each person takes notes; no sharing and no aggregation.
    \item Each person takes notes; with aggregation and sharing.
\end{itemize}

Changing the cultural norms for a team or organization can come from top-down directives, you can try to convince coworkers of the value of taking notes, or you can ignore cultural norms and take initiative and just focus on what you can do.


See also the discussion on \hyperref[sec:written-comm-does-not-happen]{why meeting notes do not get taken}.
\marginpar{See page~\pageref{sec:written-comm-does-not-happen}}

\ \\
\begin{samepage}
\textit{Meeting suggestion}: \textbf{Facilitator ensures Presenters are Capable}.\\
Does the presenter know how to project slides or video? If audio is needed, does it work for everyone in the audience?
See advice on \hyperref[sec:bad-presentations]{presentations}.
\marginpar{See page~\pageref{sec:bad-presentations}.}
\end{samepage}

\ \\
\begin{samepage}
\textit{Meeting suggestion}: \textbf{Facilitator's ground rules}.\\
To run a smooth and productive meeting, I explicitly state two ground rules to the attendees:
\end{samepage}
\begin{itemize}
    \item If you want to talk, raise your hand and I will call on you. If multiple people want to talk, I'll track the order of speakers.
    \item If you talk too long, I'll cut you off. 
\end{itemize}
This approach is critical when there are many people present, when people with diverse backgrounds are present, or when there is a mixture of dominant and submissive personalities present. 
If a visual signal like hand raising is not used, reliance on verbal interruption defaults to dominant personalities. Waiting for a person to finish speaking doesn't work for everyone because some participants will use more than their fair share of time. Speaking for a long time needs to be addressed regardless of whether an intentional effort to exclude others or a consequence of verbosity.

As a facilitator, my focus is on structure (distinct phases of the discussion) and ensuring participation. I remove myself from taking part in the discussion.

The facilitator does not need to share all their guardrails with participants. You should have a structure in mind before starting the meeting. For example,
\begin{enumerate}
    \item Establish understanding of what the topic is. Without a shared focus and a common goal a meeting is unlikely to be productive. 
    \item Set aside topics that are not the focus. Either discuss outside the current meeting or defer to a later meeting. Managing scope is critical to accomplishing the goal. 
    \item Establish a shared language specific to the topic and the participants.
    \item Establish each participant's viewpoint on the topic.
    \item Brainstorm options.
    \item Build consensus or nominate a decider.
\end{enumerate}


\ \\
\begin{samepage}
\textit{Meeting suggestion}: \textbf{Facilitate Asking Dumb Questions without Feeling Intimidated}.\\
\nopagebreak % this didn't help :( https://texfaq.org/FAQ-nopagebrk
Asking a question of an expert from a position of ignorance can feel intimidating. You may worry you're wasting the expert's time. In a recurring meeting a facilitator can address this by having participants write questions on paper and submit them anonymously. The questions or discussion topics can then be raised at following meetings. 
\end{samepage}

To facilitate anonymity every participant must be given paper and pen, and every participant must write something on the paper. The facilitator then has to collect the paper from each participant. For contributors who don't have a question, they can write down feedback about the meeting. 

This technique allows the expert to get the information needed for a response, or to figure out who the best person to respond is. 

\subsubsection*{After the Meeting}

\ \\
\begin{samepage}
\textit{Meeting suggestion}: \textbf{Share meeting notes}.\\
Attendees can check that their input was captured and clarify if additions or changes are needed. Notes should be provided quickly after a discussion while the observations are fresh in each person's mind.
\end{samepage}

\ \\
\begin{samepage}
\textit{Meeting suggestion}: \textbf{Collect feedback from Attendees on How to Improve}.\\
To learn what meeting attendees wanted from the interaction you have to ask. Or you can hope that complaints are verbalized and shared with you. By engaging with attendees you can help them feel valued.
\end{samepage}
