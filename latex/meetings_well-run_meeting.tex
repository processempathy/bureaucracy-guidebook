\subsection*{A Well-run meeting\label{sec:well-run-meeting}}

Being an effective bureaucrat means leveraging the distributed knowledge present on your team and in your organization.
Explaining how to run an effective meeting (the scope of this section) is easy compared to the work of implementing those techniques. The extra work required to facilitate an effective meeting means the default of unproductive discussions is a \href{https://en.wikipedia.org/wiki/Nash_equilibrium}{Nash equilibrium}. 
\index{Wikipedia!\href{https://en.wikipedia.org/wiki/Nash_equilibrium}{Nash equilibrium}}
\marginpar{[Wikipedia] Nash \\ equilibrium}
No one participant benefits by reducing the number of meetings or effectiveness even though the group would benefit. So we all sit in ineffective meetings. You can choose to take action that disrupts the equilibrium. The positive view of that action is that you can serve as a role model behaviors of other people can learn from. The negative view is that you are relieving the need for other people to do work.



The scope of this section is distinct from the processes covered by \href{https://en.wikipedia.org/wiki/Robert\%27s_Rules_of_Order}{Robert's Rules of Order}\footnote{I have not used Robert's Rules of Order for a meeting internal to a bureaucracy. Inside bureaucracies I'm aware of Robert's Rules of Order does not appear to be commonly used.}. 
\index{Wikipedia!\href{https://en.wikipedia.org/wiki/Robert\%27s_Rules_of_Order}{Robert's Rules of Order}}
\marginpar{[Wikipedia] Robert's \\Rules of Order}
The level of formality for meetings in bureaucratic organizations varies widely. Most commonly bureaucrats rely on personal relationships in meetings of less than 15 people. In that setting there are many tips for effective interactions.

\index{list of tips!meetings}

\ \\
\begin{samepage}
\textit{Meeting tip}: \textbf{Form relationships and understand constituents before the Meeting}.\\
To minimize surprise during meetings, socialize the concepts prior to the meeting. This concept is referred to as 
\href{https://en.wikipedia.org/wiki/Nemawashi}{Nemawashi}.
\index{Wikipedia!\href{https://en.wikipedia.org/wiki/Nemawashi}{Nemawashi}}
\marginpar{[Wikipedia] Nemawashi}
\end{samepage}

If you don't have relationships or if the topics are a surprise to meeting participants, time will be spent educating and setting definitions. This is an inefficient use of time when different participants have different backgrounds and motives.

\ \\
\begin{samepage}
\textit{Meeting tip}: \textbf{Don't invite everyone}.\\
This may seem counter-intuitive since bureaucracy is about coordination, but having too many attendees wastes the time of people who have no input on the issue. 
To identify essential attendees you need to think ahead about which decisions are going to be made and who has the authority to make a decision. If someone does not need to be present, tell them in advance that you will share the meeting notes afterwards. See the \hyperref[table:dilemma-scope-of-speaking]{Dilemma of Speaking Scope}.
\marginpar{See page~\pageref{table:dilemma-scope-of-speaking}.}
\end{samepage}

\ \\
\begin{samepage}
\textit{Meeting tip}: \textbf{Create and use an Agenda}.\\
\textit{Bad}: Having no meeting agenda.\\
\textit{Good}: Have an agenda for the meeting. \\
\textit{Better}: \underline{Share the agenda with other participants}. Having an agenda keeps attendees focused.  Enables tracking of progress during the meeting so participants are more likely to get to all topics.\\
\textit{Best}: For formal meetings, \underline{share the agenda in writing before meeting}. Sharing the agenda in advance allows attendees to prepare.
\end{samepage}

The primary reason agendas don't happen is that takes time to create an agenda. When an agenda is created and shared, attendees might not take the time to read beforehand. Attendees may not stick to the agenda during the meeting.

\ \\
\begin{samepage}
\textit{Meeting tip}: \textbf{Ensure facilities are adequate}.\\
For formal in-person meetings verify meeting venue has enough space, seating, working IT equipment. For formal virtual meetings ensure participants are familiar with virtual meeting controls. 
\end{samepage}

%  why logistics and infrastructure matter in a bureaucracy
As a bureaucrat you may not see taking care of details like this as your responsibility. Ensuring well-run logistics and effective use of infrastructure may not be your area of expertise, you lack training in this domain, it's not in your job description, and you won't get promoted for taking care of it. 

Attention to detail outside the scope of your official duties benefits your reputation. Proactive concern for the smooth operation of the bureaucracy enables efficiency in time and resources.

% TODO: forces conspiring against logistics and infrastructure

% TODO
Fire alarm or other emergencies. 

\ \\
\textit{Meeting tip}: \textbf{Body language matters}

If you are not speaking, are you reclined or leaning forward on the edge of your seat? Are you looking at the speaker?

If your eyes are closed, other people don't know if you're picturing something or falling asleep. 

If you approve of something but don't want to verbally interject, a thumbs up is useful signaling. 

\ \\
\begin{samepage}
\textit{Meeting tip}: \textbf{Take and share Meeting Notes}.\\
Meeting notes are more detailed than the agenda but less detailed than a transcript of who said what. Meeting notes synthesize the discussion. Meeting notes specify follow-on who is taking which actions with what deadlines. 
\end{samepage}

See also the discussion on \hyperref[sec:written-comm-does-not-happen]{why meeting notes do not get taken}.
\marginpar{See page~\pageref{sec:written-comm-does-not-happen}}

\ \\
\begin{samepage}
\textit{Meeting tip}: \textbf{Facilitator ensures Presenters are Capable}.\\
Does the presenter know how to project materials? How to present slides?
\end{samepage}

\ \\
\begin{samepage}
\textit{Meeting tip}: \textbf{Facilitator's ground rules}.\\
To run a smooth and productive meeting, I explicitly state two ground rules to the attendees:
\end{samepage}
\begin{itemize}
    \item If you want to talk, raise your hand and I will call on you. If there are multiple people wanting to talk, I'll track the order of speakers.
    \item If you talk too long, I'll cut you off. 
\end{itemize}
This approach is critical when there are many people present, when people with diverse backgrounds are present, or when there is a mixture of dominant and submissive personalities present. 
If a visual signal like hand raising is not used, reliance on verbal interruption defaults to dominant personalities. Waiting for a person to finish speaking doesn't work for everyone because some participants will use more than their fair share of time. speaking for a long time needs to be addressed regardless of whether an intentional effort to exclude others or a consequence of verbosity.

As a facilitator, my focus is on structure (distinct phases of the discussion) and ensuring participation. I remove myself from taking part in the discussion.

\ \\
\begin{samepage}
\textit{Meeting tip}: \textbf{Facilitate Asking Dumb Questions without Feeling Intimidated}.\\
\nopagebreak % this didn't help :( https://texfaq.org/FAQ-nopagebrk
Asking a question of an expert from a position of ignorance can feel intimidating. You may worry you're wasting the expert's time. In a recurring meeting a facilitator can address this by having participants write questions on paper and submitting anonymously. The questions or discussion topics can then be raised at following meetings. 
\end{samepage}

To facilitate the anonymity every participant must be given paper and pen, and every participant must write something on the paper. The facilitator then has to collect the paper from each participant. For contributors who don't have a question, they can write down feedback about the meeting. 

This technique allows the expert to get the information needed for a response, or to figure out who the best person to respond is. 

\ \\
\begin{samepage}
\textit{Meeting tip}: \textbf{Collect feedback from Attendees on How to Improve}.\\
To understand what meeting attendees wanted you have to ask, or hope that complaints are verbalized and shared with you. By engaging with attendees you can help them feel valued.
\end{samepage}