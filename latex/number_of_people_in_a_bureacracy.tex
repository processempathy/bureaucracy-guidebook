\subsection{Number of people in a bureaucracy}
Although bureaucracy can be present for one person, and bureaucracy is often apparent on teams (e.g., 3 to 20 people), this book primarily focuses on the situation of multiple teams comprising an organization. This might be a few hundred people (above \href{https://en.wikipedia.org/wiki/Dunbar's_number}{Dunbar's number}) up to millions of people. 

There are tens of companies that employ more than a million people\footnote{see \href{https://en.wikipedia.org/wiki/List_of_largest_employers}{Wikipedia's list of largest employers}}, including Walmart, Amazon, and McDonald's. Bureaucracy emerges at a far smaller scale, is scale invariant, and is generic across sectors. 

Small companies with a few people incur bureaucracy. Non-profit organizations encounter bureaucracy. The complexity of the tasks may be different, but the same scale-independent patterns can emerge because of a common factor: human behavior.

Size of bureaucracy scales with the complexity of the problem. 

Scaling a bureaucracy up is easier than scaling down.