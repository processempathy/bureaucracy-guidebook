\subsection*{Factors Influencing Bureaucracy}

\textit{The size of an organization and the complexity of managing shared resources drive bureaucracy.}

Although bureaucracy can be present for one person, and bureaucracy is often apparent in teams (e.g., 3 to 20 people), this book focuses on the situation of multiple teams comprising an organization. The size of an organization might be a few hundred people (above \href{https://en.wikipedia.org/wiki/Dunbar\%27s_number}{Dunbar's number})\iftoggle{WPinmargin}{\marginpar{$>$Wikipedia: Dunbar's number}}{}
\index{Wikipedia!Dunbar's number@\href{https://en.wikipedia.org/wiki/Dunbar\%27s_number}{Dunbar's number}}
 up to millions of people. 
Examples of companies that employ more than a million people\footnote{See Wikipedia's \href{https://en.wikipedia.org/wiki/List_of_largest_employers}{list of largest employers}.
\index{Wikipedia!list of largest employers@\string\href{https://en.wikipedia.org/wiki/List_of_largest_employers}{list of largest employers}}
} include Walmart, Amazon, and McDonald's. Size isn't a requirement for bureaucracy. Small companies with a few people incur bureaucracy because of the need to coordinate subjective policies governing shared resources. 


% https://graphthinking.blogspot.com/2017/05/population-sizes-needed-to-support.html
Bureaucracy scales with the complexity associated with a shared resource. For example, if participants in a society only used hand tools they could make themselves, then there is little need for bureaucracy. Mining and producing small amounts of metal is feasible for an individual, though the relevance of specialization becomes clearer. A society large enough to support the technology of writing (beyond the use of clay tablets) seems to coincide with the bureaucratic need for writing. Getting to technology like the telegraph and radio requires a society that supports complex processes and specialization -- key features of bureaucratic systems. While decreasing accidental bureaucracy is feasible, there is some essential bureaucracy\footnote{The distinction of accidental and essential complexity is from Brook's 
``\href{https://en.wikipedia.org/wiki/No_Silver_Bullet\%23Summary}{No Silver Bullet}''~\cite{1986_brooks}.
\index{Wikipedia!No Silver Bullet@\string\href{https://en.wikipedia.org/wiki/No_Silver_Bullet}{No Silver Bullet}}}
necessary for maintaining the current level of technological sophistication. 

In addition to essential task complexity, the size of bureaucracy depends on accidental factors like how old the bureaucracy is, how big the community being supported is, and how diverse the community is.

Money is uncorrelated with bureaucracy. Bureaucracy occurs in commercial companies, government, and non-profit organizations. Money can help align incentives and provide feedback loops to inform behavior, but it can also be the source of administrivia and policies. 

Bureaucracy emerges in small organizations, has scale-invariant patterns, and is generic across sectors. The complexity of the tasks may differ, but the same scale-independent patterns emerge because of a common factor: human behavior. 

% TRANSITION
The scale-independent patterns and underlying human behavior are the focus of the next section. The Fundamentals of Bureaucracy maps the formal definition of bureaucracy used in this book to commonly observed features of bureaucracy: decision-making, hierarchy, communication, and feedback loops. Understanding these fundamental aspects helps develop your process empathy by knowing why these features exist. 

