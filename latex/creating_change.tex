\subsection{Creating change in the organization\label{sec:creating_change}}

If the organization you are a participant in has no problems or challenges, this subsection can be skipped. For bureaucrats in organizations that do have structural problems, this part of the guidebook provides points to ponder independent of the specific problem.

As a bureaucrat, you have unique insight on the problems the organization faces, and you have unique leverage to alter the situation.  While you could proceed haphazardly, an effective bureaucrat has vision, goals that break down the vision, plans on how to achieve each goal, and milestones which indicate whether the plan is proceeding. 

Perspectives to consider when assessing change include what the situation is, what the situation could be, and what the situation looks like from different perspectives

Confounding your ability to improve the organization, there are people around you who have conflicting visions or no vision. There are different views on whether something is actually a problem, different prioritizations, and different approaches to addressing problems.

A trade-off to consider is that having niche impact is easier than broad change. There's also a trade-off of the quick fix versus more robust solutions.

For a given structural problem in an organization, options include technical solutions, changing policy, or changing cultural norms of participants.

determine social/political/technical impediments. 

\textit{Tip}: Before starting a new effort, check to see whether this has been tackled before. Learn the history of the problem. Why hasn't this been solved?

\textit{Tip}: Query your first and second order social network

\textit{Tip}: get feedback early; don't polish

\textit{Tip}: advertise the result.

\textit{Tip}: hear criticism and respond
