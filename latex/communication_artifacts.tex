%*************************************
When a challenge that spans multiple bureaucrats (and may span teams) is recognized in a bureaucracy, a typical response is to hold more meetings (which might help with decision-making) and send more emails (which might help with sharing information). Those tactics are easy to enact but have a deficit: both meetings and emails are ephemeral. To enable long-term success and avoid repeating mistakes, bureaucrats need to be able to find and reference past communications. The ability to learn from past lessons requires share information using channels that are discoverable (e.g., searchable) by people who are not present at the time of the discussion. 

\ \\
%*************************************

In this book I don't address paper memos versus email versus phone calls versus video chat versus Skype versus Slack. 
The attributes relevant to distinguishing channels are agnostic to specific technologies and implementations: synchronous versus asynchronous (is there a delay between participants); discoverable or not discoverable (do participants have access to content); searchable or not searchable (is content indexed). 
%In that context, there might be some interesting bureaucratic specific things to think about.
