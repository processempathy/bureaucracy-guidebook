\subsection*{Slowing Communication\label{sec:slowing-communication}}

Tactics which interfere communication within a bureaucratic organization are stonewalling, slow rolling, bikeshedding, and red herrings. By learning these concepts you will better be able to identify and then respond to their use.

\index{responsiveness!stonewalling}
\Gls{stonewalling} is when the recipient of a request or question simply doesn't reply. There are legitimate reasons for a lack of response. The person may be busy and didn't see your message, or they did see your message but didn't have a chance to reply yet because a response to you is lower priority than other tasks they have. I'm unable to differentiate those from when the recipient doesn't want to enable me to proceed. They may disagree with my objective and see silence as less confrontational than explicit rejection. 

One way of circumventing stonewalling is to ask if the respondent is opposed to your idea. 
\marginpar{[Tag] Actionable Advice} 
\index{actionable advice}
Then a lack of response indicates no opposition. This tactic applies if you are confident the recipient will read or hear the message.

An unintentional source of stonewalling is when you ask on the wrong channel. Sending an email may result in what appears to be stonewalling if the person relies on chat messages or the phone. The solution for this 
\marginpar{[Tag] Actionable Advice} 
\index{actionable advice}
is for the person who only uses certain channels to explicitly indicate that. An automatic out-of-office email that says, ``Contact me by phone'' tells the sender the preferred channel.

If you need time to think or gather information before responding, 
\marginpar{[Tag] Actionable Advice} 
\index{actionable advice}
tell the person asking that you acknowledge their message and will follow up in more detail later (with a specific timeline). 

\index{responsiveness!slow rolling}
\Gls{slow rolling} is when you get a response to your request or question, but the response isn't helpful. There is a delay in the outcome because you have to iterate to get an answer. There are valid reasons for a slow roll and there are uncool reasons for a slow roll response. Perhaps the person wants to acknowledge your request but doesn't currently have time to provide a full explanation. The person may need to gather more information for the complete response. Or the person may understand your question and does not want to enable your progress. 

The reason for a slow roll should be made explicit by the respondent, 
\marginpar{[Tag] Actionable Advice}
\index{actionable advice}
and a timeline for a complete response is helpful. 


\index{responsiveness!bikeshedding}
\Gls{bikeshedding} is when the recipient of a question or request \href{https://en.wikipedia.org/wiki/Law_of_triviality}{focuses on unimportant details relative to the primary topic}. 
\index{Wikipedia!\href{https://en.wikipedia.org/wiki/Law_of_triviality}{Law of Triviality}}
Whether this behavior is intentional or not, the best response is to refocus the conversation on the core issue. The amount of time allocated for various topics should be proportional to their consequence. 

\index{responsiveness!red herring}
A \gls{red herring} response is misleading, whether intentional or not. The respondent provides what looks like a reasonable answer but results in unproductive work. Occasionally there is a coincidental benefit of discovering something unexpected, but that wasn't the intent of the respondent. 
