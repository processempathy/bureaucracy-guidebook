\subsection*{Slowing Communication\label{sec:slowing-communication}}

In an ideal scenario there would be no delay associated with communication -- you would get the information you need when you need it. In practice, there are various causes for why your progress is blocked when you depend on other people. 

Tactics that delay communication within a bureaucratic organization are stonewalling, slow-rolling, bikeshedding, and red herrings. By learning these concepts you will be better able to identify and then respond to their use.

\index{responsiveness!stonewalling}
\iftoggle{glossaryinmargin}{\marginpar{[Glossary]}}{}
\iftoggle{glossarysubstitutionworks}{\Gls{stonewalling}}{Stonewalling} 
is when the recipient of a request or question  doesn't reply. There are \hyperref[sec:email-responsiveness]{legitimate reasons for the lack of response}\iftoggle{haspagenumbers}{; see page~\pageref{sec:email-responsiveness}.}{.}

The person may be busy and didn't see your message, or they did see your message but didn't have a chance to reply yet because a response to you is lower priority than other tasks they have. You cannot differentiate those reasonable causes from when the recipient doesn't want to enable you to proceed. They may disagree with your objective and see silence as \href{https://en.wikipedia.org/wiki/Passive-aggressive_behavior}{less confrontational}
\index{Wikipedia!\href{https://en.wikipedia.org/wiki/Passive-aggressive_behavior}{passive-aggressive behavior}}\iftoggle{WPinmargin}{\marginpar{[Wikipedia] Passive-\\aggressive behavior}}{}
than explicit rejection. 

One way of circumventing stonewalling is to ask if the respondent is opposed to your idea. 
\marginpar{$>>$ Actionable Advice} 
\index{actionable advice}
Then a lack of response indicates no opposition. This tactic applies if you are confident the recipient will read or hear the message.

An unintentional source of stonewalling is when you ask on the wrong channel. Sending an email may result in what appears to be stonewalling if the person relies on chat messages or the phone. The solution for this 
\marginpar{$>>$ Actionable Advice} 
\index{actionable advice}
is feasible, but action is required by the person who doesn't respond. The person who only uses certain channels should explicitly indicate that. An automatic out-of-office email that says, ``Contact me by phone'' tells the sender the \hyperref[sec:communication-preferences]{preferred channel}. 
\marginpar{See page~\pageref{sec:communication-preferences}.}

From the view of the person doing the stonewalling, if you need time to think or gather information before responding, 
\marginpar{$>>$ Actionable Advice} 
\index{actionable advice}
tell the person who sent a request that you acknowledge their message and will follow up in more detail later (with a specific timeline). While better than no response, this leads to the next challenge.

\index{responsiveness!slow-rolling}
\iftoggle{glossaryinmargin}{\marginpar{[Glossary]}}{}
\iftoggle{glossarysubstitutionworks}{\Gls{slow-rolling}}{Slow-rolling} 
is when you get a response to your request or question, but the response isn't helpful. Progress is delayed because you have to iterate to get an answer. There are valid reasons for a slow-roll and there are uncool reasons for a slow-roll response. Perhaps the person wants to acknowledge your request but doesn't currently have time to provide a full explanation. The person may need to gather more information for the complete response. Or the person is \href{https://en.wikipedia.org/wiki/Passive-aggressive_behavior}{passive-aggressive}
\index{Wikipedia!\href{https://en.wikipedia.org/wiki/Passive-aggressive_behavior}{passive-aggressive behavior}}\iftoggle{WPinmargin}{\marginpar{[Wikipedia] Passive-\\aggressive behavior}}{}
and may understand your question but does not want to enable your progress. 

The reason for a slow-roll should be made explicit by the respondent, 
\marginpar{$>>$ Actionable Advice}
\index{actionable advice}
and a timeline for a complete response is helpful. 


\index{responsiveness!bikeshedding}
\iftoggle{glossaryinmargin}{\marginpar{[Glossary]}}{}
\iftoggle{glossarysubstitutionworks}{\Gls{bikeshedding}}{Bikeshedding} 
is when the recipient of a question or request 
\href{https://en.wikipedia.org/wiki/Law_of_triviality}{focuses on unimportant details relative to the primary topic}. 
\index{Wikipedia!\href{https://en.wikipedia.org/wiki/Law_of_triviality}{Law of Triviality}}\iftoggle{WPinmargin}{\marginpar{[Wikipedia: Law\\of Triviality]}}{}
Whether this behavior is intentional or not, the best response is to refocus the conversation on the core issue. The amount of time allocated for various topics should be proportional to their consequence. 

\index{responsiveness!red herring}
A \gls{red herring}\iftoggle{glossaryinmargin}{\marginpar{[Glossary]}}{}
response is misleading, whether intentional or not. The respondent provides what looks like a reasonable answer but results in unproductive work. Occasionally there is a coincidental benefit of discovering something unexpected, but that wasn't the respondent's intent. 


Even though bikeshedding and a red herring can be a dark pattern, that may not be the intent of the speaker or author. Perception matters more than intent.
\marginpar{$>>$ Mantra}
\index{mantra!perception matters more than intent}