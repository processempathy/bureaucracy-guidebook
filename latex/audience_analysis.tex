\section{Audience Analysis}

The primary audience for this book is self-identified novice bureaucrats and people planning to work in a bureaucracy. The value of this book is advice beyond ``be a good person'' and specific to the role of a bureaucrat, while avoiding adjacent domains like project manager, team lead, software developer. 

The secondary audience is white collar workers who have experience working in a bureaucracy and are beginning to realize they are bureaucrats. This audience will find some of the information to be already known.  The value for this audience is a re-framing of the environment and problems. 

The tertiary audience is non-bureaucrats who want to better understand why bureaucratic systems are challenging. Americans consider themselves ``individuals" and neglect the necessary integration of operating within a society. This book won't be a popular best seller, but it is intended for a lay audience and does not assume prior academic exposure to the topic.

Lastly, researchers of bureaucracy. This book is not a theoretical systemic analysis. 

\ \\

I assume the reader has a college education. 
% according to
% https://federaljobs.net/college-degrees/
% "sixty percent of all federal workers do not have a college degree"


% Citations for literacy level:
% https://nces.ed.gov/naal/
% 1999: https://nces.ed.gov/pubs99/1999470.pdf
% https://nces.ed.gov/surveys/piaac/