\section{Automating Processes\label{sec:automating-processes}}

% TODO: is this mentioned already?
% why automation doesn't happen by default:
% Automation and discovery and search require investment and skills ancillary to the purpose of the team 

% https://graphthinking.blogspot.com/2022/07/automating-management.html

For bureaucratic organizations that deal with intangibles (referred to as ``\href{https://en.wikipedia.org/wiki/Knowledge_worker}{knowledge work}'') evolving towards automation is well-trod.  (When physical objects are involved, then robotics is necessary for automation.)

The evolution of automating a process involves
\begin{enumerate}
    \item Identify who is involved in the process. How did the process get created and how has it evolved? Who evaluates success of the process? 
    \item For existing processes transition verbal folklore to written documentation.
    \item Transition from plain text documentation to structured data. Examples of structured computer-parsable text include  decision trees and workflow diagrams. Implementation could be in Graphviz or Powerpoint.
    \item Collect metrics (for example, number of times application was opened, time spent on webpage) for existing manual processes to enable cost evaluation for enacting automation. This is the modern version of a \href{https://en.wikipedia.org/wiki/Time_and_motion_study}{time and motion study}.
    \index{Wikipedia!\href{https://en.wikipedia.org/wiki/Time_and_motion_study}{time and motion study}}
    \footnote{see \href{https://xkcd.com/1205/}{https://xkcd.com/1205/} and \href{https://xkcd.com/1319/}{https://xkcd.com/1319/}}
    \index{xkcd!\href{https://xkcd.com/1319/}{1319}}
    \index{xkcd!\href{https://xkcd.com/1205/}{1205}}
    \item Implement stand-alone software (scripts) for each task using personal automation tools like \href{https://pyautogui.readthedocs.io/en/latest/}{PyAutoGUI} or \href{https://www.autoitscript.com/site/}{AutoIt}.
    \item Tie the scripts in the previous step to the task workflows from step 2.
    \item Create an automation assistant that actively monitors your activities (keystrokes, mouse clicks, on-screen events) to detect repetitive actions that are candidates for automation.
\end{enumerate}

The path described above is not accessible to many bureaucrats for multiple reasons. 
The barriers to automating bureaucracy depend on having relevant skills, IT resources, and predictable workflows that justify the
\index{Wikipedia!\href{https://en.wikipedia.org/wiki/Return_on_investment}{return on investment}}
\href{https://en.wikipedia.org/wiki/Return_on_investment}{return on investment}. Compounding these hurdles, the skill of identifying opportunities for automation depends on your practical skills (what are you capable of) and your experience.

Where you are in the hierarchy of your bureaucratic organization effects your options. 
Automating bureaucracy from the bottom-up can be augmented and supported by management. 
At the organizational level, there are system-level changes like
\begin{itemize}
    \item Hire people who desire automation and the skills to enact automation.
    \item Use tracking bits in
    \index{Wikipedia!\href{https://en.wikipedia.org/wiki/PDF}{PDF}}
    \href{https://en.wikipedia.org/wiki/PDF}{PDFs} and webpages to collect data on how systems are used. This decreases the burden on process participants to report metrics and can be less biased since it does not rely on self-reporting.
    \item Provide a metrics aggregation service. The ability to store metrics from multiple teams enables a holistic perspective on the organization and can help identify chokepoints. Without the ability to view the entire organization, premature optimization is likely.
    \item Align incentives (e.g., pay, promotion) with implementation of automation.
    \item Transition from paper to 
    \href{https://en.wikipedia.org/wiki/PDF}{PDF} 
    \index{Wikipedia!\href{https://en.wikipedia.org/wiki/PDF}{PDF}}
    to website to \href{https://en.wikipedia.org/wiki/API}{API}. 
    \index{Wikipedia!\href{https://en.wikipedia.org/wiki/API}{API}}
    To read about how this worked out at Amazon see \footnote{\href{https://gist.github.com/chitchcock/1281611}{https://gist.github.com/chitchcock/1281611}}
    \footnote{\href{https://news.ycombinator.com/item?id=3101876}{https://news.ycombinator.com/item?id=3101876}}
    \footnote{\href{https://news.ycombinator.com/item?id=27566676}{https://news.ycombinator.com/item?id=27566676}}.
    \item Members of bureaucratic teams should be trained to develop software, share code, use \href{https://en.wikipedia.org/wiki/API}{API}s, 
    \index{Wikipedia!\href{https://en.wikipedia.org/wiki/API}{API}}
    create APIs, and maintain APIs.
\end{itemize}

\ \\

% TRANSITION to next section: process_exceptions
When automating bureaucratic processes, exceptions disrupt the expected workflow of tasks that comprise the process. Handling exceptions is a tricky subjective topic because exceptions are sometimes critical to allow for and can be abused. Abuse of exceptions to a process can originate from the subjects of a process or from the bureaucrats administering the process.