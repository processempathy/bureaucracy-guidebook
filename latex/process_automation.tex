\section{Automating Processes\label{sec:automating-processes}}



% https://graphthinking.blogspot.com/2022/07/automating-management.html

For bureaucratic organizations that deal with intangibles (referred to as ``\href{https://en.wikipedia.org/wiki/Knowledge_worker}{knowledge work}''), the desire to automate processes is wide-pread.  (When physical objects are involved, then robotics is necessary for automation.) Automation has the benefits of consistency, predictability, decreased labor costs, and easier troubleshooting. 

% why automation doesn't happen by default:
While the benefits of automation are known to participants, there are barriers to automation you will need to address.  
As with making information discoverable and searchable, automation requires an investment of time, money, and skills that are ancillary to the purpose of the organization. A transition from manual processes to automated workflows requires doing the original work while enabling automation. Creating the conditions that are needed for automation has a delayed pay-off. 

The evolution of automating a process involves
\begin{enumerate}
    \item Identify who is involved in the process. How did the process get created and how has it evolved up to now? Who evaluates the success of the process? 
    \item For existing processes, transition verbal folklore to written documentation. The purpose of this step is to write down the workflow, business logic, and relevant decisions. 
    \item Transition the documentation in the previous step from plain text documentation to structured data. Examples of structured text that can be parsed by a computer include  decision trees and workflow diagrams. Implementation could be in \index{Wikipedia!Graphviz@\href{https://en.wikipedia.org/wiki/Graphviz}{Graphviz}}
    \href{https://en.wikipedia.org/wiki/Graphviz}{Graphviz} or \index{Wikipedia!PowerPoint@\href{https://en.wikipedia.org/wiki/Microsoft_PowerPoint}{PowerPoint software}} \href{https://en.wikipedia.org/wiki/Microsoft_PowerPoint}{PowerPoint}.
    \item Collect metrics (for example, the number of times software application was opened, how frequently the folder was reviewed, how much  time was spent on the webpage) for existing manual processes to enable cost evaluation for implementing automation. This is the modern version of a
    \index{Wikipedia!time and motion study@\href{https://en.wikipedia.org/wiki/Time_and_motion_study}{time and motion study}}
    \href{https://en.wikipedia.org/wiki/Time_and_motion_study}{time and motion study}.\footnote{\href{https://xkcd.com/1205/}{https://xkcd.com/1205/} -- the trade-off of task frequency versus temporal savings per task instance.}
    \index{xkcd!xkcd.com/1205@\href{https://xkcd.com/1205/}{1205}}
    \item Implement stand-alone software (scripts) for each task using personal automation tools like \href{https://pyautogui.readthedocs.io/en/latest/}{PyAutoGUI} (a free Python-based automation package) or \href{https://www.autoitscript.com/site/}{AutoIt} (a free domain-specific language for Windows GUI applications).
    \item Tie the scripts in the previous step to the task workflows from step 2.
    \item Create an automation assistant that actively monitors your activities (keystrokes, mouse clicks, on-screen events) to detect repetitive actions that are candidates for automation.
\end{enumerate}

The path described above is not accessible to many bureaucrats for multiple reasons. 
The barriers to automating bureaucracy depend on having relevant skills, IT resources, and predictable workflows that justify the
\index{Wikipedia!return on investment@\href{https://en.wikipedia.org/wiki/Return_on_investment}{return on investment}}
\href{https://en.wikipedia.org/wiki/Return_on_investment}{return on investment}. Compounding these hurdles, the skill of identifying opportunities for automation depends on your practical skills (what are you capable of) and your experience.

Where you are in the hierarchy of your bureaucratic organization affects your options. 
Automating bureaucracy from the bottom-up can be augmented and supported by management. 
At the organizational level, there are system-level changes like
\begin{itemize}
    \item Hire people who desire automation and the skills to enact automation.
    \item Use tracking bits in
    \index{Wikipedia!PDF@\href{https://en.wikipedia.org/wiki/PDF}{PDF}}
    \href{https://en.wikipedia.org/wiki/PDF}{PDFs} and webpages to collect data on how systems are used. This decreases the burden on process participants to report metrics and can be less biased since it does not rely on self-reporting.
    \item Provide a metrics aggregation service. The ability to store metrics from multiple teams enables a holistic perspective on the organization and can help identify chokepoints. Without the ability to view the entire organization, premature optimization is likely.
    \item Align incentives (e.g., pay, promotion) with the implementation of automation.
    \item Transition from paper to 
    \href{https://en.wikipedia.org/wiki/PDF}{PDF} 
    \index{Wikipedia!PDF@\href{https://en.wikipedia.org/wiki/PDF}{PDF}}
    to a website to an%
    \index{Wikipedia!API@\href{https://en.wikipedia.org/wiki/API}{API}}
    \href{https://en.wikipedia.org/wiki/API}{API} (application programming interface).\footnote{To read about how this worked out at Amazon, see \href{https://gist.github.com/bhpayne/49c8379a3ea880b7cc079fc8d32c87a7}{Yegge's 2011 description}~\cite{2011_Yegge}.}
%and \href{https://news.ycombinator.com/item?id=3101876}{https://news.ycombinator.com/item?id=3101876}}
%and \href{https://news.ycombinator.com/item?id=27566676}{https://news.ycombinator.com/item?id=27566676}.}
    \item Members of bureaucratic teams should be trained to develop software, share code, use \href{https://en.wikipedia.org/wiki/API}{API}s, 
    \index{Wikipedia!API@\href{https://en.wikipedia.org/wiki/API}{API}}
    create APIs, and maintain APIs.
\end{itemize}

%\ \\

%\noindent\hrulefill

Automation can reduce labor costs, but designing, implementing, maintaining, and updating automation requires significant investment, a change of culture, and new skills for the workforce.  


% TRANSITION to next section: process_exceptions
When automating bureaucratic processes, exceptions disrupt the expected workflow of tasks that comprise the process. Handling exceptions is a tricky subjective topic because exceptions are sometimes critical to allow for and can be abused. Abuse of exceptions to a process can originate from the subjects of a process or from the bureaucrats administering the process.