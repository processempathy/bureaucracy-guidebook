\chapter{Working with other Bureaucrats\label{sec:working-with-other-bureaucrats}}

% TODO
Why would you need to interact with other bureaucrats?
\begin{itemize}
    \item Delegation of tasks
    \item Asking for help
    \item Seeking input
    \item Offering to help
    \item Offering input
\end{itemize}


% https://graphthinking.blogspot.com/2016/07/if-everyone-is-doing-right-then-why-are.html
If everyone participating in a process thinks they are doing the right thing, then why are there suboptimal outcomes? Potential reasons include
\begin{itemize}
    \item Each participant in the process may have limited view of other parts of the process. Each person talks to only to adjacent participants.
    \item Each participant typically has limited scope of responsibility. They don't need to know everything. 
    \item Each participant typically has limited scope of authority. Even if the person knows more than just their role, they lack control over other aspects of the process.
    \item Participants typically have insufficient time, resources, and expertise to implement improvements.
    \item Participants in the process may have a common goal, but different methods for addressing the challenges.
\end{itemize}
Those are all potential constraints; they don't exclude being effective or improving the process. 
You can learn what someone else's priorities are and why those are priorities.


When you can't do everything yourself, you rely on your team and members of the wider organization. You can treat this loss of independence within the organization (\href{https://en.wikipedia.org/wiki/Deindividuation}{deindividuation}) as either sublimation or an extension of your capacity and abilities.


% https://graphthinking.blogspot.com/2017/05/presence-with-attention-time-alertness.html
When you engage fellow bureaucrats as part of a process, you are providing your time, attention, and alertness. To illustrate the relevance of those aspects, consider the following. I can talk to you for 15 minutes of time, but if your attention is directed towards your phone, then you're not fully engaged with me. If you are paying attention during the time we have together but you haven't slept for 36 hours, then your alertness may not be 100\%. 

\ \\


% TODO
These all directly impact your 
\hyperref[sec:reputation]{reputation}.
%; see section~\ref{sec:reputation}.

% TODO
Enumerate \hyperref[sec:tropes]{tropes} to figure how to respond.


% https://graphthinking.blogspot.com/2021/10/why-i-dont-like-being-in-management-role.html
Solo work may be more emotionally rewarding due to fewer external constraints, but the cost is complexity and scope being limited to the skills of the individual. 

Working with others allows you to occasionally accomplish complex results beyond your own skills or your own bandwidth in spite of collaborators not being under your control. How? Through persuasion. 

The challenge of collaboration is to multiply productivity rather than merely sum the output of a set of individuals. 

Inside an organization, cooperation/coordination is not held together by internal contracts or even service level agreements. What holds the organization together? Force of will of participants. 

%TODO
two networks: formal and informal.