\chapter{Working with other Bureaucrats\label{sec:working-with-other-bureaucrats}}
\iftoggle{showbacktotoc}{{\footnotesize Back to the \hyperref[sec:toc]{Main Table of Contents}}}{}

\iftoggle{showbacktotoc}{\ \\}{}

\iftoggle{showminitoc}{\minitoc}{}

\textit{The previous chapter focused on you. When working with a team in an organization you are not the main character. This chapter provides advice for navigating an organization comprised of bureaucrats who are not like you.}

You were hired to do a job. That job relies on technical skills you've accumulated through education and experience. Now you're a member of 
 the team. You'll need to interact with other bureaucrats for various reasons:
delegating tasks, asking for help, offering help, seeking input, and offering input.
%\begin{itemize}
%    \item Delegation of tasks.
%    \item Asking for help.
%    \item Seeking input.
%    \item Offering help.
%    \item Offering input.
%\end{itemize}
While that generic observation sounds reasonable, in practice you encounter significant friction in getting things done. 
% https://graphthinking.blogspot.com/2016/07/if-everyone-is-doing-right-then-why-are.html
If everyone participating in a process thinks they are doing the right thing, then why are there suboptimal outcomes? Potential reasons include
\begin{itemize}
    \item Each participant in the process may have a limited view of other parts of the process. Each person talks only to adjacent participants.
    \item Each participant typically has a limited scope of responsibility. They don't need to know everything, which can lead to incorrect decision-making. Specialization can simultaneously improve efficiency and cause harm by narrowing scope.
    \item Each participant typically has a limited scope of authority. Even if the person knows more than  their role, they lack control over other aspects of the process.
    \item Participants typically have insufficient time, resources, and expertise to enact improvements. There may be a theoretical ``best outcome'' but practical resource constraints result in suboptimal outcomes.
    \item Participants in the process may have a common goal but use different methods for addressing the challenges. This results in friction even though the destination is agreed upon. 
\end{itemize}
Those are all potential constraints; they don't exclude being effective or improving the process. 
For example, you can learn what someone else's priorities are and why those are priorities. Then you can take action that accounts for other perspectives.

% https://graphthinking.blogspot.com/2021/10/why-i-dont-like-being-in-management-role.html
Solo work may be more emotionally rewarding due to fewer external constraints, but the cost is that task complexity and scope are limited to the skills of the individual. 


When you can't do everything yourself, you rely on your team and members of the broader organization. You can choose to treat this loss of independence within the organization (\href{https://en.wikipedia.org/wiki/Deindividuation}{deindividuation}) 
\index{Wikipedia!deindividuation@\href{https://en.wikipedia.org/wiki/Deindividuation}{deindividuation}}\iftoggle{WPinmargin}{\marginpar{$>$Wikipedia: deindividuation}}{}
as either sublimation (and feel suffocated) or an extension of your capacity and abilities (and feel empowered).

Working with others allows you to occasionally achieve complex results beyond your skills or your bandwidth despite  collaborators not being under your control. How? Through persuasion. 
The challenge of collaboration is to multiply productivity rather than merely sum the output of a set of individuals. 

Inside an organization, cooperation/coordination is not held together by internal contracts or even \href{https://en.wikipedia.org/wiki/Service-level_agreement}{service-level agreements}. 
\index{Wikipedia!service-level agreement@\href{https://en.wikipedia.org/wiki/Service-level_agreement}{service-level agreement}}%
\iftoggle{WPinmargin}{\marginpar{$>$Wikipedia: Service-level agreements}}{}
What keeps the organization together is the force of will of participants. 
You engage other bureaucrats because you see merit in doing so, and the same perspective applies to every other person in the organization.

\subsection*{Time, Attention, and Alterness}

Bureaucratic organizations are comprised of humans. Your intent to be an effective bureaucrat must account for this aspect.
% https://graphthinking.blogspot.com/2017/05/presence-with-attention-time-alertness.html
When you engage fellow bureaucrats as part of a process, you are investing your \gls{attention-time} in the relationship. Relationships take time, which is time not spent on other opportunities. In addition to your time (and theirs), the other two factors are your attention and alertness. To illustrate the relevance of those aspects, consider the following. I can talk to you for 15 minutes, but if your attention is directed toward your phone or computer then you're not fully engaged with me. If you are paying attention during the time we have together but haven't slept for 36 hours, your alertness may not be 100\%. 

When interacting with fellow bureaucrats, are you giving them your attention? Are you alert to their input? While engaging, determine if they are not attentive or alert. You can be curious and ask if deferring the interaction would allow them to give more of their attention. If you're not feeling alert, let the people you're interacting with know.


