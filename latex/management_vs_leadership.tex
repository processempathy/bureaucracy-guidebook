

\subsection*{Roles of Management versus Leadership}

\index{mantra!management isn't leadership}
Teams include managers and leaders. Those roles are not necessarily filled by the same person. By clarifying the distinction, you can determine what to expect from various members of the hierarchy. A common mistake is to expect leadership from managers. (In fact, anyone can lead!)

A manager's role involves time management, task tracking, employee evaluations, promotion, pay, requesting resources for team members, firing, and hiring. Sometimes the responsibilities of promotion, pay, and hiring are split into a supervisor role, leaving task tracking and time management to the roles of project manager or product manager.

A leader's role includes on coordinating vision and principles. Vision can be either a destination (a specific outcome) or a direction (an area of focus). The principles (or strategy) identify which behaviors of team members are expected or shunned. The vision and principles for the team do not have to originate from the leader -- other team members can contribute ideas. The leader's responsibility includes creating and renewing social consensus around the vision and principles. 

Another responsibility of team leaders is aligning the efforts of members. That involves identifying roles, responsibilities, and relationships. Roles get titles to label the responsibilities. Relationships are codified by an \gls{org chart} that indicates who reports what to whom. 

As a thought experiment to illustrate the necessity of management and leadership, consider an uncoordinated mob of people. How would a mob create complex machines like a car, common electronic devices like a computer, or large infrastucture like skyscrapers and bridges?

The complexity and size of \gls{decentralized bureaucracy} requires managers and leaders even though the access to shared resources is less easy to visualize. In the next section on friction the difficulties of coordinating bureaucrats is reviewed. The understanding of sources of friction is import for managers, leaders, and bureaucrats to understand as part of their process empathy.

