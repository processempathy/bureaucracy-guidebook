\section[Bureaucrat's View of Their Organization]{Bureaucrat's View of Their Organization\label{sec:alternative-views-from-within}\iftoggle{shortsectiontitle}{\sectionmark{Bureaucrat's View}}{}}
\iftoggle{shortsectiontitle}{\sectionmark{Bureaucrat's View}}{}

% LONG1 shows up in the TOC
% LONG2 is the section title


Bureaucracy 
\hyperref[sec:define-bureaucracy]{as defined in this book} 
\marginpar{Page~\pageref{sec:define-bureaucracy}.}%
%in section~\ref{sec:define-bureaucracy} 
is not the only way that bureaucrats perceive their environment. To understand their situation, bureaucrats may create a narrative involving themselves, their work, their subjects, their coworkers, their supervisors, and the people they manage. Complicating this story, the \hyperref[sec:motivations]{motives of an individual bureaucrat}
\marginpar{Page~\pageref{sec:motivations}.}%
for an activity vary.
%(see section~\ref{sec:motivations}). 
The variance of motivations is even more complicated when a request incurs more work, there's no deadline, and no reward. What is your incentive? Is it emotional approval? Relationship building? Social approval?

Most bureaucrats don't self-identify as bureaucrats even if they do recognize they operate in a bureaucratic organization. The common paradigm is, ``I'm an engineer and the company I work for is bureaucratic," or ``I'm a manager and my team is distracted by too much administrativia," or ``My coworker 
Paul is not like me --  he doesn't respond to my messages."\marginpar{Page~\pageref{sec:slowing-communication}.}%

As a consequence of not thinking about the bureaucratic system you operate within, a typical response is to develop heuristics that fit the recurring patterns you've experienced. 
%Unlike relying on an explanatory theory, heuristics evolve as you encounter novel situations. 
While that feels like learning, the heuristics you develop are specific to your motives and your experiences.

\ \\

The perspectives below are archetypal for bureaucrats who don't consider bureaucracy \hyperref[sec:define-bureaucracy]{as defined} in this book.
\marginpar{Page~\pageref{sec:define-bureaucracy}.}%
In practice, a practitioner's perspective might be a mixture of these views. The descriptions of bureaucracy below are both wrong and harmful. The source of the harm is  that if you use the framing to guide your actions you'll be less effective compared to process empathy framing.  The alternative framings are listed so they can be contrasted with Process Empathy. 

\ \\
\textbf{As a bureaucrat, what matters is what I can do with my skills and the resources I have access to.} \\
\textit{Assessment}: This person is task oriented. Results are what matters. The intricacies of bureaucracy are a distraction to getting the work done. 
% https://graphthinking.blogspot.com/2015/05/a-method-for-herding-cats.html
Communication for coordination is a distraction from work being done by the individual. 
This bureaucrat may have the perception that if they participate in the organization then they will be blamed when things go wrong.
The emotional reward for this person is doing or completing the task. This person is likely to say to their manager, ``Tell me what I need to do to be successful" rather than identify collaborations.

\ \\
\textbf{As a bureaucrat, what matters is how I feel.} \\
\textit{Assessment}: Your feelings are real. They have consequence, in that your emotions affect motivation and enthusiasm. However, a feelings-centric perspective may not be productive for you, your team, or the organization. Brainstorming can feel unpredictable. Challenging statements made by coworkers may feel intimidating. Talking about tough topics is emotionally difficult. 
Being effective means compromise and some people may not get everything they wanted. %Balancing those competing needs is challenging.

\ \\ 
\textbf{What matters is how others feel.}\\
\textit{Assessment}: Depending on the emotional state of those around you is unhealthy and can be unproductive. Working for the happiness or satisfaction of other people is risky -- they may not know what's best, or they may not have your interests in mind.

\ \\
\textbf{What matters are my immediate coworkers.}\\
An approach to avoiding thinking about bureaucracy is to characterize interactions with other people merely as personal relationships. That is easier than thinking of bureaucracy as the complex system that exceeds your local view.
% this is also stated on page 24 of~\cite{1991_Wilson}

This perspective can be positive (e.g., I collaborate with those around me) or negative (e.g., I compete with those around me).
In this scenario everything beyond the local scope is personified or ignored.  \\
\textit{Assessment}: Your relationships do matter. However, they are not all that matters. Missing from this view is the ability to explain what is happening outside the immediately observable realm. 
This ``just relationships'' view misses emergent phenomena and over-simplifies the situation. As a result your effectiveness is decreased.





\ \\

Just because you are a bureaucrat doesn't mean you have a well-informed understanding of bureaucracy. Regardless of which of the above perspectives is held, a bureaucrat experiences the difficulties of operating within an organization. A bureaucrat can rationalize to themselves why things don't work in their organization with stories like
\begin{itemize}
\item Other people are lazy and don't want to work.
\item Other people are inexperienced.
\item Other people don't care.
\end{itemize}
There are lazy people, inexperienced people, and people who don't care in any given organization. Those are not unique to bureaucracy and do not explain bureaucracy.

The bureaucrat who uses explanations like laziness, lack of experience, and lack of care applies them to people he or she hasn't directly interacted with.  The bureaucrat using these explanations may not realize other people could apply those same stories to the bureaucrat. 