\subsection{Promotion}


If your request costs me additional work and there's no deadline and there's no reward, what is my incentive?
Emotional approval? Relationship building? social approval


% https://graphthinking.blogspot.com/2021/04/notes-from-peter-principle-by-peter-and.html
Peter principal \cite{1970_Peter} has an element of truth, especially initially in a person's assumption of a new role. However, a person can learn their role and become competent. Also, the Peter principles relies on the simplification of a single dimension of intelligence. 
\href{https://en.wikipedia.org/wiki/Theory_of_multiple_intelligences}{Gardner's theory of multiple intelligences}


Promotion is a critical aspect of designing incentives for behavior. Promotion is central because there are a variety of motives -- for money, for status, for authority. 

promotion of individual (rather than team) results in hero culture

What is preventing innovation is lack of risk taking. Actual risk means potential for failure. fear of failure because culture avoids failure due to concerns of waste. Also, anyone who fails can use this argument, but that isn't the same as ``fail fast'' because what's critical is learning from failure and sharing that insight gained so other's don't need to repeat. 

two options for learning: your own mistakes, or the mistakes of others. "formal education" is about the latter
