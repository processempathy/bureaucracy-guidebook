\section{Design of Processes\label{sec:design-of-processes}}




Bureaucracy is the use of distributed knowledge and distributed decision-making for policies concerning access to shared resources. A set of \iftoggle{glossarysubstitutionworks}{\glspl{process}}{processes}
\iftoggle{glossaryinmargin}{\marginpar{[Glossary]}}{} is used by \iftoggle{glossarysubstitutionworks}{\glspl{bureaucrat}}{bureaucrats} to leverage specialization and improve throughput. 
    
Because bureaucrats are typically not formally trained in designing processes, ad hoc ideas reactive to the immediate situation and local constraints are used. A thoughtful process designer tries to account for exceptions \iftoggle{haspagenumbers}{ (page~\pageref{sec:exceptions-to-process})}{} 
and potential mistakes\iftoggle{haspagenumbers}{ (page~\pageref{sec:process-mistakes}).}{.}

% https://graphthinking.blogspot.com/2016/04/if-you-want-boulder-to-roll-place-it-at.html
The most effective process is a minimal process.
When designing a process, look for ways to minimize the work for the people involved. Minimizing work (both physical and mental) for the people involved means less sensitivity to entropy in a bureaucracy. Minimizing work requires significant situational awareness on the part of the designer. To minimize the requirements to achieve success, minimizing the effect on existing bureaucracy is vital.

Another question to ask when designing a bureaucratic process is whether a process is needed. Can the goal be accomplished by relying on ad hoc responses and leveraging relationships among bureaucrats?
% https://graphthinking.blogspot.com/2021/04/laffer-curve-and-minimum-viable.html
There is a Goldilocks situation for the number of processes in an organization:
\marginpar{$>>$ \href{https://en.wikipedia.org/wiki/Goldilocks_principle}{Goldilocks balance}}
\index{Goldilocks balance!number of processes}
\begin{itemize}
    \item Too few processes (all social relationships). New participants, lacking necessary relationships, find the team or organization chaotic. 
    \item Just the right amount -- a balance of process and social relationships, and knowing when to use which. Each participant  has a different view about what the right balance is based on what relationships are accessible to them, and how burdensome they find a process. 
    \item Too much process (not enough leveraging of social relationships). When this happens, bureaucrats can end up resorting to the use of social relationships to circumvent burdensome processes.
\end{itemize}
This optimization is like the \href{https://en.wikipedia.org/wiki/Laffer_curve}{Laffer curve} in economics.
\index{Wikipedia!\href{https://en.wikipedia.org/wiki/Laffer_curve}{Laffer curve}}

Minimal processes and subtle change may not be available.
The following sections describe why static process designs are the norm, the role of bureaucratic debt, and how to design for turnover of staff.
    
\subsection*{Static and Dynamic Process\label{sec:static-dynamic-processes}}

% https://graphthinking.blogspot.com/2017/04/static-versus-dynamic-processes.html

Change within an \gls{organization} \iftoggle{glossaryinmargin}{\marginpar{[Glossary]}}{}%
is to be expected since the external environment the organization exists in is not static. 
Sources of external change include improving technology, changes to the \gls{shared resource}, \iftoggle{glossaryinmargin}{\marginpar{[Glossary]}}{}%
or shifting expectations of subjects of the bureaucracy.
Change is also driven internally to the organization by \hyperref[sec:turnover]{turnover of staff}.%
\marginpar{See page~\pageref{sec:turnover}.}
Since change is expected, why are static processes that are not robust to change created in the first place? Because static processes are easier to design and appear initially to require less maintenance.

Creating robust processes that are dynamic takes more effort to create. First, the process must be documented so that it can be analyzed. What is expected to happen? Who are the stakeholders? These conditions are likely to change, making the process fragile. Second, document assumptions used in the process. If the assumptions are invalidated, then the process is broken and needs to be discarded or at least revised. 

A challenge is that even when the process is documented and assumptions enumerated, there may not be an incentive to check to see if revision is necessary. Measurements (which are costly and disruptive) need to be periodically taken to see if the assumptions are still applicable. To force periodic validation of assumptions, one approach is to use \href{https://en.wikipedia.org/wiki/Sunset_provision}{sunset provisions} -- automatic expiration dates set at the time of creation. 
\index{Wikipedia!sunset provisions@\href{https://en.wikipedia.org/wiki/Sunset_provision}{sunset provisions}}
\iftoggle{WPinmargin}{\marginpar{$>$Wikipedia: sunset provisions}}{}

A more quantitative approach (and even less frequently used) is to tie a process to a cost-benefit model. Enacting a process provides a benefit and comes at some cost. If the assumptions of the process can be tied to a cost-benefit model, then we can determine whether the process is worth enacting. Periodic measurements are needed to update the cost-benefit model and determine whether the process is effective.

Summarizing the steps for creating a robust process,
\begin{enumerate}
    %\item If a process already exists, document the process that is fragile.
    \item List assumptions used in the process. Who are the stakeholders, what are the goals, and what are the constraints?
    \item Relate the assumptions to a \href{https://en.wikipedia.org/wiki/Cost\%E2\%80\%93benefit_analysis}{cost-benefit model}.
    \index{Wikipedia!cost-benefit model@\href{https://en.wikipedia.org/wiki/Cost\%E2\%80\%93benefit_analysis}{cost-benefit model}}
    \iftoggle{WPinmargin}{\marginpar{$>$Wikipedia: cost-benefit model}}{}
    \item Determine the measurable parameters of the cost-benefit model. 
    \item Collect recurring measurements to verify the assumptions. 
    \item If the assumptions are broken, revise the process. 
\end{enumerate}
A robust process is just a fragile process with a feedback loop informed by ongoing measurements. Robust processes require extra work by bureaucrats compared to static processes. A static process shifts the burden to subjects. In this situation bureaucrats have externalized the burden.
In practice, ignoring exceptions and reacting to problems is common because then there's less work for the bureaucrats enacting the process. Process Empathy in this case is not just rationalizing suboptimal behavior, but empowers you to take action. 

\ \\

The above description of robust dynamic processes and fragile static processes characterizes workflows in isolation from the history of a team or organization. Typically processes are evolved from previous processes.  That evolution induces another source of bureaucratic friction. 

\subsection*{Decisions and Processes Create Bureaucratic Debt\label{sec:bureaucratic-debt}}

% https://graphthinking.blogspot.com/2017/09/bureaucratic-debt-and-what-to-do-about.html

Suppose a \gls{process} is enacted and later found to be ineffective. Some work is needed to revise the process and hopefully improve effectiveness (or an \hyperref[sec:exceptions-to-process]{exception} is needed).
%\ifsectionref
%; see section~\ref{sec:exceptions-to-process}).
%\fi
\iftoggle{glossarysubstitutionworks}{\Gls{bureaucratic debt}}{Bureaucratic debt}\footnote{Similar to the concept of \href{https://en.wikipedia.org/wiki/Technical_debt}{technical debt} in the creation and maintenance of software\iftoggle{printedonpaper}{; see Wikipedia entry}{}.
% 2023-11-15: the "\index" on the next line has to have a line break, or else pandoc's parser gets confused
\index{Wikipedia!\href{https://en.wikipedia.org/wiki/Technical_debt}{technical debt}}} is 
\iftoggle{glossaryinmargin}{\marginpar{[Glossary]}}{}  the work needed to change a process.
Bureaucratic debt is caused by choosing an easy solution now (with limited information or insufficient resources) instead of using an approach that would take longer to design and enact but be more robust.


Decisions made by \glspl{bureaucrat} occur in a resource-constrained environment.
Getting information (measurement) and analysis are costly in terms of money, time, skill, and labor.
Each decision made results in options that are not explored. These missed opportunities are associated with short-term versus long-term trade-offs of costs.

The \href{https://en.wikipedia.org/wiki/Opportunity_cost}{opportunity costs}
\index{Wikipedia!\href{https://en.wikipedia.org/wiki/Opportunity_cost}{opportunity cost}}
(options the organization doesn't take) alter which future decisions become available.

\ \\

The purpose of defining bureaucratic debt as a concept is to capture the work resulting from decisions that would otherwise be unaccounted for.
Once the concept of bureaucratic debt is understood it can be tracked.

To document bureaucratic debt, you need to record aspects of decisions as they are made:
\begin{itemize}
    \item What is the decision to be made?
    \item When was the decision  identified?
    \item When was the decision made?
    \item Who made the decision?
    \item What options were identified?
    \item Which option was chosen?
    \item Why was that option  chosen over the other options?
\end{itemize}
The purpose of documenting decisions is to enable both aversion to bad decisions and attraction to good decisions. That may sound strange, but the default of decision-makers is to apply the same behavior in future decisions. 
Without documenting decisions, there is no transparency, accountability, historical measure of progress, or ability to track dependencies. 

Creating a record of decisions is necessary but not sufficient. The documentation of decisions needs to be shared with stakeholders to enable accountability. This should occur as promptly as possible. 

Every bureaucrat exercises policies that apply to subjects, even if the subjects are other bureaucrats. What I'm describing above is beyond merely documenting what the processes and policies are for subjects. Documenting bureaucratic debt is for use internal to the team or organization.  

The scale of decision impact determines the level of documentation. ``Do I choose pencil or pen?" incurs negligible bureaucratic debt; therefore the documentation needed is also negligible. Projecting then consequence of decisions is a subjective prediction. 

%Similarities of tech debt and bureaucratic debt.
%In developing software, there are three artifacts: the software, documentation on how to use the software, and documentation on why to use the software. The two distinct types of documentation are typically combined in one document. Each of these three artifacts are independent. The ramification of this is that each artifact can be created independently, and it takes work to maintain synchronization of the artifacts. 



\subsection*{Design Processes for Turnover of Staff\label{sec:turnover}}

% https://graphthinking.blogspot.com/2020/02/design-for-turnover-rather-than-rely-on.html

When designing a \gls{process}, there are a few goals to optimize for: time-to-first-result, average latency, initial financial cost, total financial cost, flexibility to input conditions, throughput, and scalability. In theory all these factors should inform decision-making. An often neglected aspect that is harder to predict and harder to measure is the importance of employee turnover. 
On a long time scale, the turnover of \iftoggle{glossarysubstitutionworks}{\glspl{bureaucrat}}{bureaucrats} is 
a significant source of risk for any team or project. 

Besides the loss of knowledge associated with turnover, another complication is the change of assumptions when new people join an existing process. 
Processes are enacted differently than initially intended because the people implementing them are not the same people who came up with and designed them. One solution (rarely enacted) is to document the assumptions and reasoning for the design of a process. Having a written record enables bureaucrats who were not present at the time of conception to understand the purpose of the process. 

The conditions under which a process is created are not static -- requirements and resources change. 
Making processes resilient to change requires  bureaucrats to be educated beyond the requirements of the immediate task. The relevance of education is on-going: during onboarding of the bureaucrat, while carrying out the process, and as bureaucrats exit participation in the process. 

Providing training for a process is complicated by the variety of bureaucrats participating in a process.
That is why designing a process typically relies on roles -- participants are treated as interchangeable with other people who have similar skills. As a bureaucrat coming up with a novel process, accounting for differences in enthusiasm or communication among participants is difficult. 
Designing processes that are robust to turnover does not mean ignoring the unique talents of participants. 
To account for differences emphasize documentation that explains the how and why in training new participants. 
\marginpar{Actionable Advice}%
\index{actionable advice}


Onboarding new bureaucrats involves technical training, explanation of norms, learning the processes, and creating a professional network of coworkers. During this onboarding the new bureaucrat should be documenting their observations. Postponing the creation of documentation until the new person has experience results in a skewed and incomplete capture of the challenges.

Trained and experienced bureaucrats executing a process are responsible for coordinating with other bureaucrats. Bureaucrats should be cross-trained in other roles to gain familiarity with other parts of a process. 

Bureaucrats exiting the team or organization or changing roles have a responsibility to document their knowledge for team members. An exit interview can inform how the team improves. 

In each phase, documentation is the mechanism for spreading knowledge. Documenting the why (in addition to the how) is critical for the reader. How frequently  documentation is accessed should be measured to determine whether there was value in investing in documentation. 


\ \\

% TRANSITION to process_change_existing
The opportunity for you to create novel processes is likely to be rare if you are a bureaucrat in an old organization or team. A more frequent activity is updating and revising existing processes. The next section covers topics specific to changing a process. 