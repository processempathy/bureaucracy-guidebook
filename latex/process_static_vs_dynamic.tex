\subsection*{Static and Dynamic Process\label{sec:static-dynamic-processes}}

% https://graphthinking.blogspot.com/2017/04/static-versus-dynamic-processes.html

Change within an \gls{organization} \iftoggle{glossaryinmargin}{\marginpar{[Glossary]}}{} is to be expected since the external environment the organization exists in is not static. 
Sources of external change include improving technology,  changes to the \gls{shared resource}, \iftoggle{glossaryinmargin}{\marginpar{[Glossary]}}{} or shifting expectations of subjects of the bureaucracy.
Change is also driven internally to the organization by \hyperref[sec:turnover]{staff turnover}.
\marginpar{See page~\pageref{sec:turnover}.}
Since change is expected, why are static processes that are not robust to change created in the first place? Because static processes are easier to design and appear initially to require less maintenance.

Creating robust processes that are dynamic takes more effort to create. First, the process must be documented so that it can be analyzed. What is expected to happen? Who are the stakeholders? These conditions are likely to change, making the process fragile. Second, document assumptions used in the process. If the assumptions are invalidated, then the process is broken and needs to be discarded or at least revised. 

A challenge is that even when the process is documented and assumptions enumerated, there may not be an incentive to check to see if revision is necessary. Measurements (which are costly and disruptive) need to be periodically taken to see if the assumptions are still applicable. To force periodic validation of assumptions, one approach is to use \href{https://en.wikipedia.org/wiki/Sunset_provision}{sunset provisions}. 
\index{Wikipedia!\href{https://en.wikipedia.org/wiki/Sunset_provision}{sunset provisions}}
\iftoggle{WPinmargin}{\marginpar{[Wikipedia] sunset\\provisions}}{}

A more quantitative approach is to tie a process to a cost-benefit model. Implementing a process provides a benefit and comes at some cost. If the assumptions of the process can be tied to a cost-benefit model, then we can determine whether the process is worth implementing. Periodic measurements are needed to update the cost-benefit model and determine whether the process is effective.

Summarizing the steps for creating a robust process,
\begin{enumerate}
    %\item If a process already exists, document the process that is fragile.
    \item List assumptions used in the process. Who are the stakeholders, what are the goals, and what are the constraints?
    \item Relate the assumptions to a \href{https://en.wikipedia.org/wiki/Cost\%E2\%80\%93benefit_analysis}{cost-benefit model}.
    \index{Wikipedia!\href{https://en.wikipedia.org/wiki/Cost\%E2\%80\%93benefit_analysis}{cost-benefit model}}
    \iftoggle{WPinmargin}{\marginpar{[Wikipedia] cost-\\benefit model}}{}
    \item Determine the measurable parameters of the cost-benefit model. 
    \item Collect recurring measurements to verify the assumptions. 
    \item If the assumptions are broken, revise the process. 
\end{enumerate}
A robust process is just a fragile process with a feedback loop informed by ongoing measurements. Robust processes require extra work by bureaucrats compared to static processes. A static process shifts the burden to subjects -- bureaucrats have externalized the burden.
In practice, ignoring exceptions and reacting to problems is common because then there's less work for the bureaucrats enacting the process. 

\ \\

\noindent\hrulefill

\ \\

The above description of robust dynamic processes and fragile static processes characterizes workflows in isolation from the history of a team or organization. Typically processes are evolved from previous processes.  That evolution induces another source of bureaucratic friction. 