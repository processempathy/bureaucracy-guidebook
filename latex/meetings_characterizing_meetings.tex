\subsection*{Characterizing Meetings\label{sec:characterizing-meetings}}
% relevance of this section:
Characterizing meetings is critical to distinguishing which norms are applicable, and what people expect from the different formats. 


Meetings can be categorized as internal meetings, customer meetings, conferences, \hyperref[sec:meetings-one-on-one]{scheduled one-on-ones}\iftoggle{has page numbers}{ (see page~\pageref{sec:meetings-one-on-one})}{}, or
\hyperref[sec:walk-arounds]{impromptu walk-arounds}\iftoggle{haspagenumbers}{ (see page~\pageref{sec:walk-arounds})}{}. These are easy to label and participants agree on the labels.
%What is the purpose of a meeting?
% https://graphthinking.blogspot.com/2019/12/what-is-purpose-of-this-meeting.html
A more important framing is to evaluate the potential purposes of a meeting. The relevance of this approach is that sometimes participants in a meeting wouldn't agree about the purpose (if that question as raised) and sometimes the purpose of a meeting evolves during the meeting. 

Standard reasons bureaucrats have to create a meeting include:
\begin{itemize}
    \item To gather input from attendees.
    \item To make a pronouncement to attendees.
    \item To educate attendees. Formally labeled as a training course, but often makes an appearance in meetings that weren't initially intended to be educational. 
    \item To educate one person. Especially important if the personal lacking context is the decision-maker. 
    \item To signal interaction. Participants in these meetings can say they met with other stakeholders, but that doesn't imply value was generated. 
    \item To brainstorm ideas.
    \item To make progress towards an goal.
\end{itemize}
When the purpose is not explicitly stated, confusion arises. 
When multiple purposes occur in one meeting and the transition is not explicitly stated, confusion arises.
The reason for this confusion is that the assumptions and expectations and norms of each purpose are different. When the attendees don't know the purpose or the purpose shifts, the expected behaviors and roles are unclear. 

An attendee can ask what the purpose of the meeting is during the meeting but that is generally considered rude. An attendee can try to deduce the purpose of a meeting, but this takes time and attention and can result in the wrong conclusion. An attendee can try to set the purpose of the meeting during the meeting, but this can conflict with the intent of other attendees. 

A meeting's purpose can shift during a meeting. If done intentionally, the changes should be stated explicitly. Otherwise an attendee may continue to work under the previous set of expectations rather than the current norms. 

\ \\

% TODO: transition needed

% https://graphthinking.blogspot.com/2014/12/how-to-understand-meetings-at-work.html
Level of formality, start time (early or on time or late), 
end time (early or on time or late), utility, 
duration, number of attendees, number of speakers, and number of participants.

\ \\

% TODO: transition needed

Meetings involve people, either known or strangers.
Meetings involve information, either relevant or irrelevant. Relevant information is either new or related to previous work.
Meetings either have a leader or no leader (e.g., team brainstorming). If there's a leader, the leader may be disseminating information to participants or gathering information from attendees.


