\subsection{Why is bureaucracy so hard?}

When a person has a positive experience engaging with bureaucracy, positive attribution is made to the people involved. Or ease of a solution makes the bureaucracy less visible and the solution seems obvious. 
When a person has a negative experience with bureaucracy, complaints are about the incompetence of the people involved, or the incomprehensibleness of the system. Don't these bureaucrats know how to do their job? Why isn't the solution obvious? Why does this system not work for me?

\begin{quote}
 Amongst working-class Americans, government is now generally seen as being made up of two sorts of people: 'politicians,' who are blustering crooks and liars but can at least occasionally be voted out of office, and 'bureaucrats,' who are condescending elitists almost impossible to uproot.   
\end{quote}
\footnote{David Graeber}


\ \\


How much work is attributable to cover your ass? How much work due to ill-informed decisions? How much relevant work is carried out by insufficiently trained workers. 


There are many nuances not visible to external perspective:
\begin{itemize}
\item organization politics (personalities, resources, prioritization)
\item lack of unified voice
\item legacy policies to overcome/change/be consistent with
\end{itemize}

***********

Why bureaucracy is complicated : effective action benefits from relevant information. Information is from function which requires process which decompose to steps. Map people who take action to steps.
