\section{Subject's View of Bureaucracy\label{sec:subjects-view}}


When a person has a positive experience engaging with bureaucracy, positive attribution is made to the people involved. Or the ease of a solution makes bureaucracy less visible and the solution seems obvious. 
When a person has a negative experience with bureaucracy, complaints are about the incompetence of the people involved, or the incomprehensibleness of the system. Don't these bureaucrats know how to do their job? Why isn't the solution obvious? Why does this system not work for me? David Graeber summarized this view:
\begin{quote}
 Amongst working-class Americans, government is now generally seen as being made up of two sorts of people: `politicians,' who are blustering crooks and liars but can at least occasionally be voted out of office, and `bureaucrats,' who are condescending elitists almost impossible to uproot.   
\end{quote}

%\subsection{Sources of complexity}

The scale of bureaucracy (number of people in an organization) and the processes of an organization can seem disproportionate to the apparent complexity of the task. Typically when person (in the role of subject of bureaucracy) interacts with the bureaucratic organization the artifacts are very simple -- a check, a form, or a permit. The simplicity of the artifact does not correlate to the number of decisions made, the tracking of information, or the precautions taken. All these aspects are invisible to the subject.

As an example, consider when you go to the doctor and they mend your broken arm with a cast. That seems straightforward because all you see is the doctor putting a cast on. You don't get insight on the decisions they had to make. Why did they need 20 years of focused schooling to carry out a procedure that took 15 minutes?

Judging bureaucracy by the artifact visible to the individual subject under sells the complexity of the decision making necessary to take action. The contingencies that you were not exposed to because everything went well make the amount of investment from the bureaucrat appear wasteful.

As the subject of bureaucracy, you also lack the ability to distinguish how much work is attributable to the bureaucrat generating justifications for their actions (colloquially, \href{https://en.wikipedia.org/wiki/Cover_your_ass}{covering the bureaucrat's ass}). 

As the subject of bureaucracy, you typically can't distinguish when work is caused by a bureaucrat's ill-informed decisions. Is the person stupid, mistaken, or is there something you are not taking into account?
Sometimes the work is carried out by insufficiently trained bureaucrats, but you don't get to know whether you're working with an experienced and knowledgeable bureaucrat or a new untrained bureaucrat. 

As the subject of bureaucracy, you don't have visibility on the many nuances of an organization. The internal power struggles and organizational politics that depend on personalities, resources, prioritization are not clear to outsiders.
The inconsistency of an organization's policies may not be felt by bureaucrats in that organization. Each bureaucrat may have a different opinion, resulting in a lack of consistent guidance.
There may be legacy policies to overcome/change/be consistent with.

Effective action by a bureaucratic organization is complicated by the need for relevant information. Gathering, analyzing, and sharing that information consistently requires bureaucratic processes. Sometimes the organization lacks the staffing with relevant skills. 


For advice, listen to \href{https://www.npr.org/2022/03/16/1086915600/get-what-you-want-customer-service}{How to talk with customer service} (\href{https://www.npr.org/transcripts/1086915600}{transcript}).


