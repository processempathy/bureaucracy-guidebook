\section{Subject's View of Bureaucracy\label{sec:subjects-view}}

This section takes on the perspective of the subject of bureaucracy but is meant to be read by bureaucrats who want to improve their \gls{process empathy}. This book doesn't provide advice for subjects of bureaucracy.\footnote{For advice navigating bureaucracy, listen to National Public Radio's \href{https://www.npr.org/2022/03/16/1086915600/get-what-you-want-customer-service}{How to talk with customer service}~\cite{2022_LifeKit}.} 

\ \\

When you have a positive experience engaging with bureaucracy, your positive attribution is to the people involved. Or the ease of a solution makes bureaucracy less visible and the solution seems obvious. 
When you have a negative experience with bureaucracy, complaints are about the incompetence of the people involved or the incomprehensibleness of the system. Don't these bureaucrats know how to do their job? Why isn't the solution obvious? Why does this system not work for me? David Graeber summarized this view~\cite{2015_Graeber_regulation}:
% May have actually come from "The Utopia of Rules"
\begin{quote}
 ``Amongst working-class Americans, government is generally seen as being made up of two sorts of people: `politicians,' who are blustering crooks and liars but can at least occasionally be voted out of office, and `bureaucrats,' who are condescending elitists almost impossible to uproot."
\end{quote}


%\subsection{Sources of complexity}

The scale of bureaucracy (the number of people in an organization) and the processes of an organization can seem disproportionate to the complexity of the task. Typically when a person (being a subject of bureaucracy) interacts with the bureaucratic organization the artifacts are simple, like a form to fill out. The simplicity of the artifact does not correlate to the number of decisions made, the tracking of information, or the precautions taken by the organization. All these aspects are invisible to the subject because they are internal to the organization.

As an example, consider when you go to the doctor and they put your broken arm in a cast. That seems straightforward because all you see is the doctor putting a cast on. You don't get insight into the decisions they had to make. Why did they need 20 years of focused schooling to carry out a procedure that took 15 minutes?

Judging bureaucracy by the artifact visible to the individual subject undersells the complexity of the decision-making necessary to take action. The contingencies that you were not exposed to because everything went well make the amount of investment from the bureaucrat appear wasteful. Distinguishing essential bureaucracy from accidental or malicious bureaucracy is difficult.

As the subject of bureaucracy, you also lack the ability to distinguish how much work is attributable to the bureaucrat generating justifications for their actions (colloquially, \href{https://en.wikipedia.org/wiki/Cover_your_ass}{covering the bureaucrat's ass}). 
\index{Wikipedia!cover your ass@\href{https://en.wikipedia.org/wiki/Cover_your_ass}{cover your ass}}
\iftoggle{WPinmargin}{\marginpar{$>$Wikipedia: Cover your ass}}{}
These justifications are needed both within the organization and potentially for external stakeholders. Each bureaucrat's rationalization may not be reviewed, but it needs to be available for review later.

As the subject of bureaucracy, you typically can't distinguish when work is caused by a bureaucrat's ill-informed decisions. Is the person stupid, mistaken, or is there something you are not taking into account?
Sometimes the work is carried out by insufficiently trained bureaucrats, but you don't get to know whether you're working with an experienced and knowledgeable bureaucrat or a new untrained bureaucrat. 

As the subject of bureaucracy, you don't have visibility on the many nuances of an organization. The internal power struggles and organizational politics that depend on personalities, resources, and competing prioritization are not clear to outsiders.
The inconsistency of an organization's policies may not be felt by bureaucrats in that organization. Each bureaucrat may have a different opinion, resulting in a lack of consistent guidance.
There may be legacy policies in effect.

There is no transparency on either the input or output for subjects. For input, forms ask for what but do not provide the why. Similarly for output, decisions are made but without justification or reasoning provided.
The subject has no input on who gets to make the decisions, nor does the subject feel the decision maker represents them, nor is there any obligation of competence in the decision making.

%The experience of a subject outside the organization is not the same as the experience of a bureaucrat-as-subject within the organization. Both can complain to management. However, the hierarchical distance is significantly different.
%In contrast to the subject's expeience, the bureaucrat's experience is more focused on the policies associated with the task of allocating and managing the shared resource. Bureaucrats are executing task assigned by someone else's policy


% summary paragraph
Effective action by a bureaucratic organization is complicated by the need for relevant information. Gathering, analyzing, and sharing that information persistently requires bureaucratic processes. To further complicate the ideal process, sometimes the organization lacks the staffing with relevant skills. 


% Transition paragraph
The next section documents why bureaucracy is hard from the perspective of the bureaucrat. Even without getting into the specifics of a bureaucrat's role or the purpose of an organization, there are generic reasons that bureaucracy is a burdensome responsibility.