\subsection{Why is bureaucracy so hard?}

When a person has a positive experience engaging with bureaucracy, positive attribution is made to the people involved. Or ease of a solution makes the bureaucracy less visible and the solution seems obvious. 


When a person has a negative experience with bureaucracy, complaints are about the incompetence of the people involved, or the incomprehensibleness of the system. Don't these bureaucrats know how to do their job? Why isn't the solution obvious? Why does this system not work for me?


Sub-optimal solutions arise due to decision-makers being under-informed, unknowledgeable, and inexperienced. Another factor is the inability to gather data due to time constraints.



\ \\

On being productive in an organization:
How much work is attributable to cover your ass? How much work due to ill-formed decisions? How much relevant work is carried out by insufficiently trained workers. 

Because decisions by bureaucrats are subjective, there is significant risk of being wrong or being called up by others as being wrong. Therefore, a motive to cover your ass for decisions made

Leads to conservative decisions and risk aversion and decrease innovation


There are many nuances not visible to external perspective:
\begin{itemize}
    \item organization politics (personalities, resources, prioritization)
\item lack of unified voice
\item legacy policies to overcome/change/be consistent with
\end{itemize}

***********

Why bureaucracy is complicated : action requires information. Information is from function which requires process which decompose to steps. Map people who take action to steps.
