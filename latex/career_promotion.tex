\section{Promotion\label{sec:promotion}}

Bureaucrats get promoted for a variety of reasons: due to their technical competence, a role needs to be filled, to motivate retention, or for their managerial ability.  Not being promoted can feel like a slight against any of those reasons. 

In this section, folk wisdom illustrates the frustration bureaucrats have with promotion. The purpose of familiarizing yourself with these sentiments is to understand that the failures of promotion are not specific to the team you are on or the organization you are a member of. The deficiencies are generic to hierarchy. With this insight, you can choose a more emotionally healthy response to the situation. Process Empathy is intended to produce emotional resilience based on an improved understanding of the situation you are in.

\subsection*{Folk Wisdom about Promotion}

The 
\href{https://en.wikipedia.org/wiki/Dilbert_principle}{Dilbert principle}
\index{Wikipedia!Dilbert principle@\href{https://en.wikipedia.org/wiki/Dilbert_principle}{Dilbert principle}}%
\marginpar{$>>$ Folk Wisdom}%
\index{folk wisdom!Dilbert principle@\href{https://en.wikipedia.org/wiki/Dilbert_principle}{Dilbert principle}}%
says organizations
``Systematically promote incompetent employees to management to get them out of the workflow.''
A similar-in-spirit observation is 
\href{https://en.wikipedia.org/wiki/Putt\%27s_Law_and_the_Successful_Technocrat}{Putt's Law}
\index{Wikipedia!Putt's Law@\href{https://en.wikipedia.org/wiki/Putt\%27s_Law_and_the_Successful_Technocrat}{Putt's Law}}
\marginpar{$>>$ Folk Wisdom}%
\index{folk wisdom!Putt's Law@\href{https://en.wikipedia.org/wiki/Putt\%27s_Law_and_the_Successful_Technocrat}{Putt's Law}}%
which says,
``\sout{Technology}[Bureaucracy] is dominated by two types of people, those who understand what they do not manage and those who manage what they do not understand."
Those two observations are post-hoc rationalizations for managers lacking managerial skills and technical skills. 

For bureaucrats who don't seem effective in their role, try to determine whether there is a lack of training, lack of awareness, or lack of interest. Training can take the form of reading books or attending classes. Lack of awareness can be addressed with measurements of turnover rate and surveying team members. 


\href{https://en.wikipedia.org/wiki/Peter_principle}{Peter principle}~\cite{1970_Peter}:
\index{Wikipedia!Peter principle@\href{https://en.wikipedia.org/wiki/Peter_principle}{Peter principle}}%
\marginpar{$>>$ Folk Wisdom}%
\index{folk wisdom!Peter principle@\href{https://en.wikipedia.org/wiki/Peter_principle}{Peter principle}}%
``People in a hierarchy tend to rise to `a level of respective incompetence': employees are promoted based on their success in previous jobs until they reach a level at which they are no longer competent, as skills in one job do not necessarily translate to another.''
% https://graphthinking.blogspot.com/2021/03/a-tension-between-working-at-discount.html
This incompetence might manifest as recurring failure for a person in a role they are inadequate for. More likely, your incompetence will blind you to your failures.

To avoid falling victim to the Peter principle, do not seek promotion until you have a track record of performing at the next level. The downside is that you are working at a discounted rate since you are providing value beyond what is expected of your grade. (That's Dilemma~\ref{table:dilemma-personal-work-extra-or-work-as-expected}\iftoggle{haspagenumbers}{ on page~\pageref{table:dilemma-personal-work-extra-or-work-as-expected}.)}{.)} 


% https://graphthinking.blogspot.com/2021/04/notes-from-peter-principle-by-peter-and.html
The Peter principle has an element of truth, especially when a person starts a new role. It takes time to find your footing in a new job. However, you can learn their role and become competent. %Also, the Peter principle relies on the simplification of a single dimension of intelligence.  \href{https://en.wikipedia.org/wiki/Theory_of_multiple_intelligences}{Gardner's theory of multiple intelligences}.
%\index{Wikipedia!Theory of multiple intelligences@\href{https://en.wikipedia.org/wiki/Theory_of_multiple_intelligences}{Theory of multiple intelligences}}
%\iftoggle{WPinmargin}{\marginpar{$>$Wikipedia: Theory of multiple intelligences}}{}


\subsection*{Designing Incentives Determines Values}
Promotion is critical when designing incentives for behavior. Promotion is challenging to engineer because people a driven by various motives like money, status, and authority. 


Promotion typically focuses on the past success of a person. This de-emphasizes teamwork and lessons learned from failures. 
Promotion of a person (rather than the team) results in hero culture.


%Lowering the value of learning from failures increases risk aversion. 
The risk tolerance of a bureaucratic organization is driven in part by conflicts of interest in the promotion processes: an interest in success (more success is better) and avoidance of failure (organizations don't promote you for failing). This imbalance can lead participants to seek ownership of efforts that are likely to succeed, to seek easier successes, and to steal credit for outcomes created by other people. 

%\subsection*{Promotion and Risk}

The promotion process within an organization is coupled to the idea of innovation, even if your promotion is at odds with taking risks. 

