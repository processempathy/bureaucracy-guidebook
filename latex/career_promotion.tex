\section{Promotion\label{sec:promotion}}

Bureaucrats get promoted for multiple reasons: due to their technical competence (a bad reason) or their managerial ability (a good reason).  

\subsection*{Folk Wisdom about Promotion}

The 
\href{https://en.wikipedia.org/wiki/Dilbert_principle}{Dilbert principle}
\index{Wikipedia!\href{https://en.wikipedia.org/wiki/Dilbert_principle}{Dilbert principle}}
\marginpar{[Tag] Folk Wisdom} 
\index{folk wisdom!\href{https://en.wikipedia.org/wiki/Dilbert_principle}{Dilbert principle}}
says organizations
``Systematically promote incompetent employees to management to get them out of the workflow.''
That observation is a post-hoc rationalization for managers with no apparent management skills and a lack of technical skills. 

A similar-in-spirit observation is 
\href{https://en.wikipedia.org/wiki/Putt\%27s_Law_and_the_Successful_Technocrat}{Putt's Law}
\index{Wikipedia!\href{https://en.wikipedia.org/wiki/Putt\%27s_Law_and_the_Successful_Technocrat}{Putt's Law}}
\marginpar{[Tag] Folk Wisdom}
\index{folk wisdom!\href{https://en.wikipedia.org/wiki/Putt\%27s_Law_and_the_Successful_Technocrat}{Putt's Law}}
which says
``\sout{Technology}[Bureaucracy] is dominated by two types of people, those who understand what they do not manage and those who manage what they do not understand."


\href{https://en.wikipedia.org/wiki/Peter_principle}{Peter principle}:
\index{Wikipedia!\href{https://en.wikipedia.org/wiki/Peter_principle}{Peter principle}}
\marginpar{[Tag] Folk Wisdom} 
\index{folk wisdom!\href{https://en.wikipedia.org/wiki/Peter_principle}{Peter principle}}
``People in a hierarchy tend to rise to `a level of respective incompetence': employees are promoted based on their success in previous jobs until they reach a level at which they are no longer competent, as skills in one job do not necessarily translate to another.''

% https://graphthinking.blogspot.com/2021/03/a-tension-between-working-at-discount.html
This might manifest as recurring failure. More likely, your incompetence will blind you to your failures.

To avoid falling victim to the Peter principal, do not promote unless the employee has a track record of performing at the next promotion level. The downside is that the employee is working at a discounted rate since they are providing value beyond what is expected of their grade.


% https://graphthinking.blogspot.com/2021/04/notes-from-peter-principle-by-peter-and.html
The Peter principal~\cite{1970_Peter} has an element of truth, especially when a person starts a new role. However, a person can learn their role and become competent. Also, the Peter principles relies on the simplification of a single dimension of intelligence. 
See \href{https://en.wikipedia.org/wiki/Theory_of_multiple_intelligences}{Gardner's theory of multiple intelligences}.
\index{Wikipedia!\href{https://en.wikipedia.org/wiki/Theory_of_multiple_intelligences}{Theory of multiple intelligences}}


\subsection*{Promotion Selects Values}
Promotion is a critical aspect of designing incentives for behavior. Promotion is central because there are a variety of motives -- for money, for status, for authority. 


\subsection*{Promotion of Individuals}

Promotion typically focuses on the successes of a person. This deemphasizes teamwork and lessons learned from failures. 
Promotion of a person (rather than the team) results in hero culture.

\subsection*{Promotion and Risk}



% the following is already a dilemma
%There are two options for learning: your mistakes, or the mistakes of others. Formal education is about the latter.
