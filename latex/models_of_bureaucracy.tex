\section{Models of Bureaucracy that are Incomplete\label{sec:models-of-bureaucracy}}

Coming up with a holistic theory of bureaucracy is desirable but difficult. Bureaucracy exists in every society, so having an explanatory theory would help identify what aspects are essential and what is accidental. Having a theory of bureaucracy could help identify what should be improved and what should be discarded. An expectation for the existence of a theory stems from the repeated independent creation of bureaucracy in diverse societies in different time periods. 

Characterizing bureaucracy is difficult because organizations comprised of unique humans are aware of attempts to be characterized and respond to stories told about bureaucracy; see the \href{https://en.wikipedia.org/wiki/Hawthorne_effect}{Hawthorne effect}. 
\index{Wikipedia!\href{https://en.wikipedia.org/wiki/Hawthorne_effect}{Hawthorne effect}}
\marginpar{[Wikipedia] Hawthorne\\effect}
As soon as a claim about aspects that characterize bureaucracy is made, then an individual can respond to that claim by behaving in an opposing manner. Worse still for the theory, bureaucrats can coordinate amongst themselves to provide counterexamples. 

In the following section I outline a few conventional ways of modeling bureaucracy to point out the shortcomings of each model. The relevance to the practicing bureaucrat of familiarity with these models is so that you know the boundaries of each model. Awareness of the limitations of each model enables you to know when the model is an applicable story and when the model is not explanatory. 

\subsection*{Bureaucracy as a Machine}

\textit{Narrative}: Bureaucracy is a machine that has throughput and latency and dependencies and mechanisms. Bureaucrats are cogs in that machine.\\
\textit{Why this model feels true}: Bureaucracy can be complicated and feel mechanical with standardization, top-down dictates, and interlocking processes. There is a risk of \href{https://en.wikipedia.org/wiki/Deindividuation}{deindividuation} 
\index{Wikipedia!\href{https://en.wikipedia.org/wiki/Deindividuation}{deindividuation}}
\marginpar{[Wikipedia] deindividuation}
for bureaucrats. \\
\textit{What this model is missing}: Perceiving yourself as a cog in the machine implies a loss of agency. This is a \href{https://en.wikipedia.org/wiki/Self-fulfilling_prophecy}{self-fulfilling prophecy}
\index{Wikipedia!\href{https://en.wikipedia.org/wiki/Self-fulfilling_prophecy}{self-fulfilling prophecy}}
\marginpar{[Wikipedia] self-\\fulfilling prophecy}
-- if you think you don't have agency, then you may stop acting as though you have agency. 


\subsection*{Bureaucracy as an Economic model}

\textit{Narrative 1}: Bureaucracy is a collection of individual rational actors. \\
% https://graphthinking.blogspot.com/2019/05/cooperation-and-competition.html
\textit{Why this model feels true}: Individual bureaucrats cooperate or compete to promote their self-interests.
The culture of an organization is a result of individual self-interests of each bureaucrat.
Individuals make decisions that promote their self-interests. \\
\textit{What this model is missing}: This view is hard to distinguish from a marketplace. 

\ \\
\textit{Narrative 2}: Bureaucracy is comprised of competing special-interest groups. \\
\textit{Why this model feels true}: The observation of \href{https://en.wikipedia.org/wiki/Public_choice}{Public Choice theory} 
\index{Wikipedia!\href{https://en.wikipedia.org/wiki/Public_choice}{Public Choice theory}}
\marginpar{[Wikipedia] Public\\Choice theory}
is that a concentrated minority that stands to gain a disproportionate benefit will act in their self-interest. \\
\textit{What this model is missing}: In a generic bureaucratic organization it is not clear which team would be more concentrated than any other team, nor is it clear what the disproportionate benefit might be. There is a concentrated minority at the top of any hierarchy, and the top of the hierarchy does act in their self-interest, though this is not particular to bureaucracy.

Bureaucratic organizations may have specific teams with access to disproportionate benefits, but this book focuses on generic bureaucratic features. Effective bureaucrats are all alike; every bureaucrat is ineffective in their own way
\footnote{A modified version of the first sentence of Leo Tolstoy's novel ``Anna Karenina.''}.
% Anna Karenina is also referenced on page 142 of "Making of a Manager." 
% I came up with my adaptation before reading "Making of a Manager"



\ \\
\textit{Narrative 3}: A Bureaucracy is a subcategory of a Firm. \\
% https://graphthinking.blogspot.com/2017/09/market-friction-and-bureaucratic.html
Firms exist in a market because negotiating contracts and prices for every interaction is burdensome. 
% https://www.kellogg.northwestern.edu/faculty/hubbard/htm/research/ec174/lectures/3coase.htm
A bureaucracy could be considered as a type of firm that specializes at the organizational level in policy administration or resource management. \\
\textit{Assessment}: This is correct.

%Doesn't address small vs large companies, and doesn't distinguish between profit-oriented and non-profit and government. 


\subsection*{Bureaucracy as Emergent phenomenon}

\textit{Narrative}: There is a universality to bureaucracy in both the diversity of scenarios and persistence across time that hints at \href{https://en.wikipedia.org/wiki/Emergence}{emergence}.\\
\index{Wikipedia!\href{https://en.wikipedia.org/wiki/Emergence}{emergence}}
\marginpar{[Wikipedia] emergence}
\textit{Why this model feels true}: Bureaucracy as a macroscopic phenomenon is emergent when there are a sufficient number of people involved. The size of the organization is important because there is no longer dependence on individual relationships (i.e., a size above \href{https://en.wikipedia.org/wiki/Dunbar\%27s_number}{Dunbar's number}). 
\index{Wikipedia!\href{https://en.wikipedia.org/wiki/Dunbar\%27s_number}{Dunbar's number}}
\marginpar{[Wikipedia] Dunbar's\\numbers}
There are people in the organization that you don't know and therefore there is a lack of personal accountability. An organization is subdivided into teams recursively until there is local person-to-person accountability.  The underlying behaviors that enable emergence are bilateral interactions among humans and a lack of feedback mechanisms. 

At the scale of individual bureaucrats, every person is playing by different rules and has different goals and everything is changing -- both the individuals and the conditions. 
Above the threshold for emergence of bureaucracy there is scale-free behavior. The same patterns are observable at large organizations and extremely large organizations. A bureaucrat in one organization recognizes patterns of professional life experienced by bureaucrats at another organization. The local mechanisms bureaucrats employ to enable distributed decisions using distributed knowledge include meetings, processes, and communications. While local nuances differ, a generic pattern is apparent. 

The choices faced by an individual bureaucrat are interdependent with the choices made by other bureaucrats in their environment. There is a \href{https://en.wikipedia.org/wiki/Flocking_(behavior)}{flocking behavior} 
\index{Wikipedia!\href{https://en.wikipedia.org/wiki/Flocking_(behavior)}{flocking behavior}}
\marginpar{[Wikipedia] flocking\\behavior}
where my choices are informed by the choices of those around me. Unlike flocking of birds, the adjacency metric for bureaucrats is not necessarily spatial distance. Instead, visibility of the decisions and consequences inform adjacency.


The relevance of claiming bureaucracy is emergent is that there is behavior occurring at the macroscopic scale. Knowing  motives and actions of every individual bureaucrat at the microscopic level is not relevant. Treating organizations as complex and adaptive systems gives insight on how to work within the dynamic environment~\cite{2011_Eisenhardt}.


A colloquial interpretation of emergent behavior from complex phenomena is treating the system as an entity -- personification.
\marginpar{[Tag] Fallacy} 
When an organization is assigned behaviors~\cite{2002_Gall}, it is useful to remember that the organization is comprised of individual bureaucrats. 


% https://www.preposterousuniverse.com/podcast/2021/10/11/168-anil-seth-on-emergence-information-and-consciousness/
More formally, there are distinct categories of emergence~\cite{2002_Bedau, 2021_Carroll_168}. Nominal emergence names the phenomena of a thing being distinct from its constituents: a circle is emergent from a collection of points; a pile of sand emerges from grains of sand. Merely putting many bureaucrats in a room is insufficient to create bureaucracy; there's more to bureaucracy. 

Weak emergence occurs when there are phenomena that are independent of the underlying interactions. For example, \href{https://en.wikipedia.org/wiki/Glider_(Conway\%27s_Life)}{gliders} 
\index{Wikipedia!\href{https://en.wikipedia.org/wiki/Glider_(Conway\%27s_Life)}{gliders}}
in \href{https://en.wikipedia.org/wiki/Conway\%27s_Game_of_Life}{Conway's Game of Life}.
\index{Wikipedia!\href{https://en.wikipedia.org/wiki/Conway\%27s_Game_of_Life}{Conway's Game of Life}}
Weak emergence is measurable using \href{https://en.wikipedia.org/wiki/Granger_causality}{Granger causality} 
\index{Wikipedia!\href{https://en.wikipedia.org/wiki/Granger_causality}{Granger causality}}
or, equivalently\footnote{\href{https://arxiv.org/abs/0910.4514}{https://arxiv.org/abs/0910.4514}}, \href{https://en.wikipedia.org/wiki/Transfer_entropy}{transfer entropy} 
\index{Wikipedia!\href{https://en.wikipedia.org/wiki/Transfer_entropy}{transfer entropy}}
(information theory). I don't know how to apply these measures to bureaucratic organizations. 

\textit{What this model is missing}: The problem with treating an organization as an entity is that apparent behavior is counter-intuitive~\cite{2002_Gall}. Breaking the organization into individual people with motives helps clarify causes of observed behavior. 
%This merely improve the post hoc rationalization. 

In practice, bureaucracy is worse than emergent because the rules can be altered or ignored by  stakeholders. Bureaucracy is a \href{https://en.wikipedia.org/wiki/Wicked_problem}{wicked problem}~\cite{1973_Rittel} 
\index{Wikipedia!\href{https://en.wikipedia.org/wiki/Wicked_problem}{wicked problem}}
\marginpar{[Wikipedia] wicked\\problem}
which resists mathematical models. 

\subsection*{Bureaucracy in \href{https://en.wikipedia.org/wiki/Game_theory}{Game Theory}}
\index{Wikipedia!\href{https://en.wikipedia.org/wiki/Game_theory}{Game Theory}}
\textit{Narrative}: Bureaucracy is comprised of one or more \href{https://en.wikipedia.org/wiki/List_of_games_in_game_theory}{games} 
\index{Wikipedia!\href{https://en.wikipedia.org/wiki/List_of_games_in_game_theory}{list of games in game theory}}
played by bureaucrats. Which game is applicable depends on the specifics of the situation. \\
\textit{Why this model feels true}: Different bureaucrats have different motives in distinct situations. \\
\textit{What this model is missing}: While interactions among bureaucrats may be describable in terms of games, that doesn't provide an underlying motive for bureaucracy as an identifiable experience.  

Defining bureaucracy as distributed knowledge and distributed decision-making for the subjective management of access to shared resources does not fit as a \href{https://en.wikipedia.org/wiki/Coordination_game}{coordination game} 
\index{Wikipedia!\href{https://en.wikipedia.org/wiki/Coordination_game}{coordination game}}
or \href{https://en.wikipedia.org/wiki/Non-cooperative_game_theory}{competitive} game. 
\index{Wikipedia!\href{https://en.wikipedia.org/wiki/Non-cooperative_game_theory}{non-cooperative game theory}}
%Bureaucracy is self-modifying. 
Bureaucracy is in constant flux due to external conditions, externally imposed constraints, staff turn-over, internal dilemmas, and disagreements among individuals. 
% https://en.wikipedia.org/wiki/Evolutionary_game_theory

\subsection*{Bureaucracy as Evolutionary Outcome}

\textit{Narrative 1}: Biological, Genetic evolution -- Individual level. \\
There might be biological arguments for a genetic basis for bureaucracy. 
\textit{Why this model feels true}: For example, Zebra fish and Hyenas have a gene for dominance\footnote{``Society, demography and genetic structure in the spotted hyena'' (2012); doi:10.1111/j.1365-294X.2011.05240.x}. \\
%Breeding animals for aggressiveness?
\textit{What this model is missing}: ``Genes code for proteins, so there are no `genes for' phenotypes per se, including behavioral phenotypes."~\cite{2015_Lilienfeld}

Non-human animals make subjective decisions and do not get labeled as bureaucrats. Human make decisions that are not explained by reproductive fitness.

\ \\
\textit{Narrative 2}: Biological, Genetic evolution -- Group selection. \\
\textit{Why this model feels true}: \hyperref[sec:hierarchy-of-roles]{Hierarchy}
%(section~\ref{sec:hierarchy-of-roles}) 
is not unique to humans. Primates form into social hierarchies based on dominance over a shared resources like mates and food. \\
\textit{What this model is missing}: a gene for bureaucracy. A genetic model does not explain what behaviors an individual bureaucrat can take to be more effective. 

\ \\
\textit{Narrative 3}: \href{https://en.wikipedia.org/wiki/Memetics}{Memetic}
\index{Wikipedia!\href{https://en.wikipedia.org/wiki/Memetics}{Memetic}}
-- bureaucracy as method for coordination is a better idea than \href{https://en.wikipedia.org/wiki/Nepotism}{nepotism} 
\index{Wikipedia!\href{https://en.wikipedia.org/wiki/Nepotism}{nepotism}}
or religion. \\
\textit{What this model is missing}: Bureaucracy persists along with nepotism and religion, and bureaucracy occurs within religion. 

\subsection*{Bureaucracy as Product-focused Narrative}
\textit{Narrative}: Ignore the bureaucrats and instead focus on how a product progresses through a process.\\
\textit{Why this feels true}: Ignoring bureaucrats involved in processes simplifies the narrative (the product is the main character). 
\textit{Example}: \href{https://www.youtube.com/watch?v=OgVKvqTItto}{School House Rock: I'm Just A Bill}\\
\textit{What this model is missing}: Decision-makers involved the process exert subjective control. Outcomes depend on who participates. 

\subsection*{Bureaucracy as Subject-focused Narrative}
\textit{Narrative}: Ignore the bureaucrats and instead focus on the person subjected to bureaucracy. \\
\textit{Why this feels true}: The person experiencing bureaucracy as a subject is confused by ``why isn't this easier?''  \\
\textit{What this model is missing}: History of why the process exists and how it evolved (i.e., legacy), protection against malicious subjects, and protection against malicious bureaucrats. 


\subsection*{Bureaucracy as Psychological Phenomenon}

\textit{Narrative}: Bureaucracy as a pure power struggle. Or the interplay of individual pathologies. 
%Are you doing what's best for you, the group you're in, or everyone?
%Altruistic or reciprocal? Retaliation.
%The answer changes time the time and situation of situation and person to person.
%\href{https://en.wikipedia.org/wiki/Spiral_of_silence}{Spiral of silence}
% and
%\href{https://en.wikipedia.org/wiki/Social_proof}{social proof}

Organizations are composed of individuals with personalities, and inefficiency is attributable to a mashing together of distinct individuals with conflicting desires.
While this is true, it isn't complete. \\
\textit{What this model is missing}: Personality-focused narrative neglects the history of processes (legacy), protection against malicious subjects, and protection against malicious bureaucrats. Analysis that stops at personalities misses emergent phenomena. %Analysis that stops at personalities of individuals typically explains larger scale phenomena using personification of teams and organizations. 

\ \\

Each of the above models of bureaucracy has shortcomings. Knowing why each model is not a complete description helps you avoid the trap of thinking only in terms of a single model. 

The rest of this book ignores these partial characterizations of bureaucracy. 
Rather than take the external (\href{https://en.wikipedia.org/wiki/Emic_and_etic}{emic}) 
\index{Wikipedia!\href{https://en.wikipedia.org/wiki/Emic_and_etic}{emic and etic}}
view of bureaucracy, this book takes the internal (etic) perspective of a bureaucrat operating within an organization. 
The description of bureaucracy in this book is used to contextualize advice for how to be an effective bureaucrat. 