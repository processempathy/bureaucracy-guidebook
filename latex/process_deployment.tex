\section{Deployment of Processes: Policy Update\label{sec:deployment-of-processes}}

New processes arise organically or are issued as top-down edicts. Processes evolve to fit the constraints of available resources and skills of bureaucrats and subjects. These causes of consistent flux mean updates to policies and process are ongoing. Whether you are a bureaucrat administering a process, a subject burdened by a process, or a bureaucrat responsible for revising processes, you have input on changes.

\ \\

Transitioning from legacy processes to new processes impacts the bureaucrats involved. Changes have to overcome legacy investments. People can feel threatened by change of role or skills.  

\ \\

Bureaucrats, recognizing that bureaucracy stifles innovation, will occasionally invest in internal innovation efforts. Because innovation can come from unexpected sources, the process of innovation is opened to everyone. 

Incompetent people can innovate just as much as well-informed experts, but the output of inexperienced people is more likely to be disappointing, wasteful, or even harmful. 

% [there's no point in stating the obvious]
%Design processes to prevent dumb people from joining the process, detect them when they are in the process, and remove them when they do not change their behavior. Simultaneously, enable and detect good ideas and promote them and give them resources.
%Promote efficiency, 


% https://graphthinking.blogspot.com/2016/04/if-you-want-boulder-to-roll-place-it-at.html
When designing a process, look for ways to minimize the work for people involved. Minimizing work (both physical and mental) for the people involved means less sensitivity to entropy in a bureaucracy. Minimizing work requires significant situational awareness on the part of the designer. To minimize effect on existing bureaucracy, it's vital to minimize the requirements to achieve success.


Similar to product deployment in that similar dilemmas are faced.
TODO

Deploying a new process in a bureaucratic organization draws on tactics from 
\index{Wikipedia!\href{https://en.wikipedia.org/wiki/Product_management}{product management}}
\href{https://en.wikipedia.org/wiki/Product_management}{product management} and
\index{Wikipedia!\href{https://en.wikipedia.org/wiki/Project_management}{project management}}
\href{https://en.wikipedia.org/wiki/Project_management}{project management}. 

Deployment of processes and products need to account for 
normal users, power users, malicious users, and edge cases.

TODO
\href{https://en.m.wikipedia.org/wiki/The_Innovator's_Dilemma}{Innovator's Dilemma} applies within bureaucracies to policies and processes.
\index{Wikipedia!\href{https://en.m.wikipedia.org/wiki/The_Innovator's_Dilemma}{Innovator's Dilemma}}

TODO
\href{https://en.wikipedia.org/wiki/Diffusion_of_innovations}{Diffusion of Innovation}
\index{Wikipedia!\href{https://en.wikipedia.org/wiki/Diffusion_of_innovations}{Diffusion of Innovation}}

%\subsection*{external to the org}


\subsection*{Internal Product Development and Deployment\label{sec:internal-product}}

Teams in a bureaucratic organization 
\begin{itemize}
    \item consume from outside-the-organization
    \item consume from within-the-organization
    \item produce to outside-the-organization
    \item produce to within-the-organization
\end{itemize}

This section focuses on the relation between teams that produce to within-the-organization and teams that consume from within-the-organization. Tools or products that are created internally and consumed internally.

Feedback mechanisms and incentives in a non-profit monopoly. The claim of success that a team created a product that met all design requirements on-time may have no actual benefit to users. Or maybe a product that benefited internal customers was created and there were happy users, but the originating team has to quickly move onto the next project to create another success and thus has no attention to on-going support. Determining the metric of success is tricky. do you want to aim for highest average happiness of stakeholders, or are some more important?

TODO
captive users who have little leverage 


\subsection*{Slowing Deployment of Processes\label{sec:slow-deployment}}

Three tactics which inhibit deployment of changes within a bureaucratic organization are stonewalling, slow rolling, and red herrings. By learning this concepts you will better be able to identify and then respond to their use.

\index{responsiveness!stonewalling}
\Gls{stonewalling} is when the recipient of a request or question simply doesn't reply. There are legitimate reasons for a lack of response. The person may be busy and didn't see your message, or they did see your message but didn't have a chance to reply yet because a response to you is lower priority than other tasks they have. I'm unable to differentiate those from when the recipient doesn't want to enable me to proceed. They may disagree with my objective and see silence as less confrontational than explicit rejection. 

One way of circumventing stonewalling is to ask if the respondent is opposed to your idea. 
\marginpar{[Tag] Actionable Advice} 
\index{actionable advice}
Then a lack of response indicates no opposition. This tactic applies if you are confident the recipient will read or hear the message.

An unintentional source of stonewalling is when you ask on the wrong channel. Sending an email may result in what appears to be stonewalling if the person relies on chat messages or the phone. The solution for this 
\marginpar{[Tag] Actionable Advice} 
\index{actionable advice}
is for the person who only uses certain channels to explicitly indicate that. An automatic out-of-office email that says, ``Contact me by phone'' tells the sender the preferred channel.

If you need time to think or gather information before responding, 
\marginpar{[Tag] Actionable Advice} 
\index{actionable advice}
tell the person asking that you acknowledge their message and will follow up in more detail later (with a specific timeline). 

\index{responsiveness!slow rolling}
\Gls{slow rolling} is when you get a response to your request or question, but the response isn't helpful. There is a delay in the outcome because you have to iterate to get an answer. There are valid reasons for a slow roll and there are uncool reasons for a slow roll response. Perhaps the person wants to acknowledge your request but doesn't currently have time to provide a full explanation. The person may need to gather more information for the complete response. Or the person may understand your question and does not want to enable your progress. 

The reason for a slow roll should be made explicit by the respondent, 
\marginpar{[Tag] Actionable Advice}
\index{actionable advice}
and a timeline for a complete response is helpful. 

TODO: integrate, transition
\index{responsiveness!bikeshedding}
\gls{bikeshedding} is when the recipient of a question or request \href{https://en.wikipedia.org/wiki/Law_of_triviality}{focuses on unimportant details relative to the primary topic}. 
\index{Wikipedia!\href{https://en.wikipedia.org/wiki/Law_of_triviality}{Law of Triviality}}

\index{responsiveness!red herring}
A \gls{red herring} response is misleading, whether intentional or not. The respondent provides what looks like a reasonable answer but results in unproductive work. Occasionally there is a coincidental benefit of discovering something unexpected, but that wasn't the intent of the respondent. 


