\subsection{Hierarchy of Roles\label{sec:hierarchy_of_roles}}

A feature of bureaucracy is the \href{https://en.wikipedia.org/wiki/Organizational_structure}{structure of the organization}. The formal hierarchy enables delegation of responsibility for policies. The structure of an organization is dynamic, but at each point in time an organization typically has a defined set of roles. Each role is distinguished by different scopes of decision authority. 

Roles in an organization are defined by the boundaries of responsibility. The purpose of a role is to minimize conflict and reduce redundancy, allowing control. Clear responsibility enables effective bureaucracy. 


The two choices in how a hierarchy is shaped are 1) a supervisor oversees how many people and 2) a person has how many supervisors. Naively one might expect that an employee has one boss, but that is not a requirement. 
The more people a supervisor oversees, the flatter the organization. See \cite{2012_Valve} and \cite{1972_Joreen} for contrasting views on the shape of an organization's hierarchy.

Acting as part of a group means ceding part of your autonomy. Hierarchy is an additional layer of ceding responsibility and adding expectations about relationships.
The consequence of hierarchy in an organization means that as a member of the bureaucracy you do not have full autonomy -- otherwise you would not be a member of the hierarchy. At the same time you are not under strict control of the organization -- you still have some subjective decision making authority as a bureaucrat.

The person at the top of the hierarchy does not know everything. The person at the top of the hierarchy does not have input on every decision made in the organization. Some autonomy is retained by all members of the bureaucracy.

Independent of the defined roles and designated titles in an organization's hierarchy, there are a set of implicit roles and a separate social hierarchy of informal influencers and decision makers. Informal influencers in a bureaucracy usually have long relationships with the decision maker or relevant credentials or both. The credentials can be formal (e.g., a \href{https://en.wikipedia.org/wiki/Doctor_of_Philosophy}{PhD}) or informal (demonstrated success on a project). In either case, the decision maker is relying on another person's expertise. 

Another set of informal relations within an organization is mentors and mentees. These relations allow mentors to transmit institutional knowledge to mentees, and allows people in senior positions to access the novice perspective. 


\ \\

One consequence of hierarchy is a sense of fear felt by people who report to other people. This fear stems from the loss of control (less autonomy) that leaves the person feeling disempowered. 

For example, consider the following relationship. A person, Sue, is perceived to have power over another person, Amy, because Amy gave up some control to Sue. Amy not having control triggers the feeling of fear in Amy, regardless of how Sue behaves. 

If Sue is aware of the potential for this emotional experience, Sue can compensate for Amy's fear by being friendly and receptive towards Amy. Alternatively Sue may exploit or rely on the fear felt by subordinates. 


% Active bystander when the person doing wrong is in a position of authority
% PACT (Probe, Alert, Challenge, Take Action)
% https://mobile.twitter.com/GeorgetownABLE/status/1408498438203969541


% Mintzberg's Coordination Mechanisms
% https://www.youtube.com/watch?v=IZET8VjSifQ