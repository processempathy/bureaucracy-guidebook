\subsection*{Hierarchy of Roles\label{sec:hierarchy-of-roles}}

TODO: 
why hierarchy: Members of an organization gravitate towards hierarchy because it defines scope and assigns responsibility and obviates a need for building consensus.

Ideally sufficient depth and breath for decision making would be embodied in one person. That might not be possible in every situation. One way to resolve this is to identify distinct scopes of responsibility and then assign different members of an organization separate scopes for decision making. Within a decision making scope there may be more work than one person can handle, so a team is formed. That team may have some members focused on tactical work and other members focused on strategy and coordination. Hierarchy within an organization is the formalization of separate decision-making scopes and associated specialization. 

Partitioning knowledge and decision making enables complexity beyond what one person can accomplish and causes friction among members. An expert reporting to a manager knows things the manager does not, and the manager may have context that the expert lacks. Both bureaucrats (the expert and the manager) need to convey their respective nuance and seek out the holistic view.

A hierarchical organization with partitioned knowledge introduces a challenge: the order in which you share information with others matters. Your choices for who to first describe an idea to are your peers, your management, and your subordinates. 
\marginpar{[Tag] Trilemma} 
\index{trilemma!communication priority: peers, management, subordinates}
The people subordinate to you know more about the topic and are exposed to the consequences. Giving them a chance to vet the idea results in a more robust idea and validates their value in organization. Sharing your idea with management first allows your superiors to provide context you might not be aware off. Starting the conversation with your peers first indicates you value the relationship and decrease the risk of overlapping work.

\ \\

I've included hierarchy in the section on Fundamentals of Bureaucracy, but that does not imply that hierarchy is a required feature of bureaucracy. Hierarchies of bureaucrats are a common \href{https://en.wikipedia.org/wiki/Organizational_structure}{organizational structure} and  worth studying even if not essential to bureaucracy. The relevance of understanding hierarchy is to identify recurring behavior and patterns to leverage.
Organizations of bureaucrats can intentionally work against hierarchy, but the amount of effort needed to enable alternatives implies that hierarchy is a natural approach.

The benefits of formal hierarchy include improved capacity for the number of policy decision made, enabling consistency of decisions, and leveraging specialization of knowledge. 
Hierarchical decision making has costs: higher latency, inconsistency among bureaucrats, waste due to inefficiency, and 
\hyperref[sec:unavoidable-hazards]{many others}.
\ifsectionref
described in section~\ref{sec:unavoidable-hazards}. 
\fi
As another example of harm, hierarchy enables strategic ignorance. Bureaucrats in positions of power can deny having knowledge of improper activity. 
\footnote{L.~McGoey, ``The Unknowers: How Strategic Ignorance Rules the World" (2019)
% review: https://www.tandfonline.com/doi/abs/10.1080/19460171.2020.1768422?journalCode=rcps20
and 
L.~McGoey, ``The logic of strategic ignorance" (2012). DOI 
10.1111/j.1468-4446.2012.01424.x
}


The structure of an organization is dynamic, but at each point in time an organization typically has a defined set of roles. Each role is distinguished by different scopes of decision authority. 
Roles are often confused with titles. What matters is the role (scope of decisions) and who reports to whom. The names of teams can be similarly not descriptive.




Roles in an organization are defined by the boundaries of responsibility. The purpose of a role is to minimize conflict and reduce redundancy, allowing control. Clear responsibility enables effective bureaucracy. 


A conventional characterization of an organization's hierarchy involves two criteria: the depth and breadth of the org chart.
The more people a supervisor oversees, the flatter the organization -- that's the breadth of the organization. See the Valve handbook~\cite{2012_Valve} and Joreen's essay~\cite{1972_Joreen} for contrasting views on the shape of an organization's hierarchy. Depth of the hierarchy is how many layers there are.

A more practical view of an organization's hierarchy also involves two criteria. The two choices in how a hierarchy is shaped are 
1) how many people a supervisor oversees and 
2) how many supervisors a person has. 
Though you might naively expect that an employee has one boss, but that is \href{https://en.wikipedia.org/wiki/Matrix_management}{not a requirement}. A supervisor for a given topic may have many people reporting to them, and a bureaucrat with multiple roles may report to more than one supervisor.

\ \\

Acting as part of a group means ceding part of your autonomy. Hierarchy is an additional layer of ceding responsibility and adding expectations about relationships.
The consequence of hierarchy in an organization means that as a member of the bureaucracy you do not have full autonomy -- otherwise you would not be a member of the hierarchy. At the same time you are not under strict control of the organization -- you still have some subjective decision making authority as a bureaucrat.

The person at the top of the hierarchy does not know everything. The person at the top of the hierarchy does not have input on every decision made in the organization. Some autonomy is retained by all members of the bureaucracy.

Independent of the defined roles and titles in an organization's hierarchy, there are a set of implicit roles and a separate social hierarchy of informal influencers and decision makers. Informal influencers in a bureaucracy usually have long relationships with the decision maker or relevant credentials or both. The credentials can be formal (e.g., a \href{https://en.wikipedia.org/wiki/Doctor_of_Philosophy}{PhD}) or informal (demonstrated success on a project). In either case, the decision maker is relying on another person's expertise. 

Another set of informal relations within an organization is mentors and mentees. These relations allow mentors to share institutional knowledge to mentees, and allows people in senior positions to access the novice perspective. 


\ \\

One consequence of hierarchy is a sense of fear felt by people who report to other people. This fear stems from the loss of control (less autonomy) that leaves the person feeling disempowered. 

For example, consider the following relationship. A person, Sue, is perceived to have power over another person, Amy, because Amy gave up some control to Sue. Amy not having control triggers the feeling of fear in Amy, regardless of how Sue behaves. 

If Sue is aware of the potential for this emotional experience, Sue can compensate for Amy's fear by being friendly and receptive towards Amy. Alternatively Sue may exploit or rely on the fear felt by subordinates. 


% Active bystander when the person doing wrong is in a position of authority
% PACT (Probe, Alert, Challenge, Take Action)
% https://mobile.twitter.com/GeorgetownABLE/status/1408498438203969541


% Mintzberg's Coordination Mechanisms
% https://www.youtube.com/watch?v=IZET8VjSifQ

