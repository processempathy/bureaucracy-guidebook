\chapter{Communication within a Bureaucracy\label{sec:communication-within-bureaucracy}}
\iftoggle{showbacktotoc}{{\footnotesize Back to the \hyperref[sec:toc]{Main Table of Contents}}}{}
\iftoggle{showminitoc}{\minitoc}{}


\section{Communication is Critical for Bureaucracy}

\iftoggle{glossaryinmargin}{\marginpar{[Glossary]}}{}
\iftoggle{glossarysubstitutionworks}{\Gls{bureaucracy}}{Bureaucracy}
is a system of distributed knowledge and distributed decision-making. Distributed decision-making relies on effective communication. Bureaucrats communicate by writing and speaking; both methods convey incomplete information and are imprecise. The inability to communicate  comprehensively and precisely  means iteration is helpful when establishing a shared mental model of the situation and plan.


The dependence on iterative interactions means relationships matter for communication. 
\href{https://en.wikipedia.org/wiki/Metcalfe\%27s_law}{Metcalfe's law} 
\index{Wikipedia!\href{https://en.wikipedia.org/wiki/Metcalfe\%27s_law}{Metcalfe's law}}
says
\marginpar{$>>$ Folk Wisdom}
\index{folk wisdom!\href{https://en.wikipedia.org/wiki/Metcalfe\%27s_law}{Metcalfe's law}}
the value of an organization is proportional to the square of the number of people interacting in the organization. A team of 5 people is not just the aggregated skills of five individuals -- there's the synergy arising from ten bilateral relationships. Broadening your network of collaborators increases your potential effectiveness and, symmetrically, the reach of your fellow bureaucrats. 

% https://graphthinking.blogspot.com/2018/08/confusion-leads-to-confusion.html
Suppose that you don't invest in relationships with your fellow bureaucrats. Then you'll be less likely to know what is happening in your organization, and you won't understand why certain decisions were made. Your confusion might lead you to focus on your interests since that is what you have some control over and insight on. Being selfish doesn't leverage the potential synergy of collaboration, but at least you look busy. 

The analysis above applies to each person in the bureaucratic organization.
% A Nash equilibrium?
The consequence is that the organization's goals are not achieved when bureaucrats act independently. 
There is a way out of this: broadcast your intent to other people, be transparent with decisions, and share the results of your activities. That decreases the confusion other people may have about your actions and can improve everyone's effectiveness. You can make yourself and the organization more effective even when your investment in communication is not reciprocated.

\ \\

Information within a bureaucracy shapes the relationships among bureaucrats. You have a choice: you can decide to not share information, you can be transparent, or you can be vulnerable.  
Not sharing information can harm or protect relationships. 
Being transparent conveys ``here's what is happening." Being vulnerable expands transparency to explain ``here's why that's happening or what might happen."
Each of those options informs how your coworkers, peers, management, and team members interpret their perception of your intent. Your communication shapes your \hyperref[sec:reputation]{reputation}.
\marginpar{See page~\pageref{sec:reputation}.}


Another way communication alters bureaucratic organizations is when actions informed by locally-available information produce \hyperref[sec:failure-to-comm]{suboptimal results for participants}, 
\marginpar{See page~\pageref{sec:failure-to-comm}.}
as illustrated by the
\href{https://en.wikipedia.org/wiki/Prisoner\%27s\_dilemma}{Prisoner's dilemma}.
\index{Wikipedia!\href{https://en.wikipedia.org/wiki/Prisoner\%27s\_dilemma}{Prisoner's dilemma}}
Bureaucrats rarely have access to the holistic view of their organization. Bureaucrats have a narrow view of their team and interfaces to adjacent teams.
The concept of ``locally-available information'' applies to communication among bureaucrats that facilitates delegation of work, allocation of resources, relationship creation/maintenance, and carrying out processes. 

If having a holistic view yields better results, why not just do that instead of relying on local information? Each bureaucrat benefits from gathering information that shapes their actions, but communication has a cost: time spent building and verifying consensus delays action. There is an 
\href{https://en.wikipedia.org/wiki/Opportunity_cost}{opportunity cost}
\index{Wikipedia!\href{https://en.wikipedia.org/wiki/Opportunity_cost}{opportunity cost}}\iftoggle{WPinmargin}{\marginpar{[Wikipedia] opportunity\\cost}}{}
when you spend time talking with people.

% https://graphthinking.blogspot.com/2021/09/why-is-everything-so-hard-in-large.html


% TRANSITION to communication_reflects_thinking

\ \\

The explanations above should convince you that communication is critical for bureaucracy, and essential for being an effective bureaucrat. Communication is typically thought of as occurring between two or more people, but before we get there we'll focus on our \href{https://en.wikipedia.org/wiki/Intrapersonal_communication}{inner monologue}. 
\index{Wikipedia!\href{https://en.wikipedia.org/wiki/Intrapersonal_communication}{Intrapersonal communication}}
\iftoggle{WPinmargin}{\marginpar{[Wikipedia] Intrapersonal communication}}{}