\section{How to be an effective Bureaucrat\label{sec:effective_bureaucrat}}

So far I've described dilemmas in \S\ref{sec:dilemma_trilemma} and unavoidable hazards in \S\ref{sec:unavoidable_hazards}. Given this nuanced view of the complexity of bureaucracy, how can you leverage that insight?

The specific circumstances of the challenges you face as a bureaucrat depend on the individual people involved, what the purpose of the bureaucracy is, what technology is available for implementation of bureaucracy, and the resources (staffing, money, time). 

* Account for holistic view
* Concurrently work on 3 remedies
* Minimize external costs

Besides being a good project manager, and a good person, the relevance of knowing the paradoxes including dilemmas and unavoidable hazards, is that you should talk explicitly to your fellow bureaucrats about these specific topics in conversation. Not that the goal is to find consensus or agreement. But to find what the other person is thinking so that you can account for their processes
