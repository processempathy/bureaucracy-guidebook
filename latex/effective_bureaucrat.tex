\section{How to be an effective Bureaucrat\label{sec:effective_bureaucrat}}

So far I've described dilemmas in \S\ref{sec:dilemma_trilemma} and unavoidable hazards in \S\ref{sec:unavoidable_hazards}. Given this nuanced view of the complexity of bureaucracy, how can you leverage that insight?

good project manager, and a good person,

\begin{itemize}
\item you can share information with other stakeholders
\item you can seek information from other stakeholders
\item you can strive for and demonstrate transparency
\item apply consistent processes (rather than being reactionary and applying ad hoc responses)
\item hold others (and yourself) accountable 
\item account for varying incentives and reference experiences
\end{itemize}


Superpowers of a bureaucrat in a bureaucratic organization that facilitate cooperation and progress:
\begin{itemize}
    \item be more transparent
    \item make yourself accountable
    \item communicate more effectively
    \item Multi-tasking, or, more accurately, task switching. The switch among tasks is triggered when the current task encounters an externally-generated pause. \href{https://en.wikipedia.org/wiki/Pipeline_(computing)#Concept_and_motivation}{pipeline your activities}.
    \item Be prepared with a backlog of ideas (in writing) if someone shows up with resources.
    \item If you are dependent on someone else getting something done to enable your progress, \underline{presence creates priority}.
    \item Quickly replying to incoming requests. This allows other people's tasks to either be resolved or have a clear response about when progress can be expected. 
    \item intellectual empathy -- what is it; how to grow it. Theory of mind for thinking. shadow peers and bosses
    \item process empathy. Observe deviations and exceptions that cause processes to come into existence. 
    \item Avoid unintentionally dropping tasks or requests. This requires capturing incoming requests and then providing a response about prioritization and status. This matters because other participants in the organization should regard you as reliable in a positive sense. 
\end{itemize}

Bureaucracy induces emotional response in participants because things don't work they way each person wants. This can lead emotionally to frustration and then apathy. Understanding how things operate in a bureaucracy decreases the anger.

% Wandering the maze of bureaucratic processes as a subject.

Another emotional response to bureaucracy is a sense of powerlessness. 
\begin{quote}
Some third person decides your fate: this is the whole essence of bureaucracy.\footnote{``The Workers' Opposition'' by Alexandra Kollontai, 1921}
% https://alphahistory.com/russianrevolution/kollontai-on-soviet-bureaucracy-1921/
\end{quote}
That sense of powerlessness applies both to bureaucrats and to subjects of bureaucracy. 

The sense of powerlessness is somewhat valid, in that you are as a bureaucrat giving up some power compared to your ability to act individually. That is the trade for working with others

\ \\

A process feels bureaucratic when the subject is exposed to more than one step to get something done that if one person were doing it would be simpler. Consolidation from the subject's perspective decreases the apparent bureaucracy. The barrier to implementation is the necessary coordination among different stakeholders who receive no benefit from the coordination. Externalizing the coordination to the subject is what causes the sense of bureaucracy. 


\ \\

Don't rely on one person or one idea for your success. Don't spread yourself too thin. 

\ \\

what to do when you get stuck?
look up stream, look downstream, look to peers

\ \\

When you are asked to take on additional work, avoid responding with ``that's not my job." If the request is misguided and your perception is that is really is the responsibility of another person, ask if the requester is aware of that other person's responsibilities. If you are to really take on the work, get guidance on re-prioritizing and make sure the request is documented in writing. 

\ \\

If there's something you want to accomplish, strive for influence without authority instead of working to gain control over resources (e.g., through promotion). Avoid the following: ``In this organization X is important to me, but I can't do X right now because I don't have enough power in the org. So I'll get promoted and then do X."

\ \\



\textit{TODO}: \textbf{Learn the perspectives of those around you.}\\
The relevance of knowing the paradoxes including dilemmas and unavoidable hazards, is that you should talk explicitly to your fellow bureaucrats about these specific topics in conversation. Not that the goal is to find consensus or agreement. But to find what the other person is thinking so that you can account for their processes


\textit{TODO}: \textbf{Account for holistic view}\\
The specific circumstances of the challenges you face as a bureaucrat depend on the individual people involved, what the purpose of the bureaucracy is, what technology is available for implementation of bureaucracy, and the resources (staffing, money, time). 

\textit{TODO}: \textbf{Learn the history of the problem}\\
This goes beyond Chesterton's fence, which focuses about why the current approach is in placed. Learning the history of a problem means what has been tried before and failed? How did the previous iterations evolve into the current situation? Was the cause personalities, insufficient resources, inadequate technology? What's changed that enables this approach to be better? What do you know that prior attempts didn't?

\textit{TODO}: \textbf{Concurrently work on 3 remedies}\\


\textit{TODO}: \textbf{Minimize \href{https://en.wikipedia.org/wiki/Externality}{external costs}}\\
A solution that externalizes costs harms the greater organization and creates bureaucratic debt.


\textit{TODO}: \textbf{Exploit the flexibility of rules for the benefit of all parties.}\\
% https://graphthinking.blogspot.com/2019/07/winning-game.html
change the rules of the game and the objectives of the game such that every participant wins.

\textit{TODO}: \textbf{Align selfish interests with social interests.}\\
See also nudges. 
(Page 66 of \cite{2012_Schneier})

\textit{TODO}: \textbf{Find ways to rephrase negative complaints.}\\
% https://graphthinking.blogspot.com/2019/07/how-to-rephrase-negative-observations.html
Negative observation: "Logging into my computer takes a long time."\\
Positive statement and explanation of impact: "If the latency for logging into my computer were lower, I could make more progress on X."


Negative observation: "The service team I need support from doesn't offer a ticketing queue."\\
Positive statement: "If the service team I need support from offered a ticketing queue, I would be able to track the work done on my behalf."

\textit{TODO}: \textbf{Share lessons learned}\\