\section{An Effective Bureaucrat\label{sec:effective-bureaucrat}}

Elsewhere in this book I describe 
\marginpar{See page~\pageref{sec:dilemma-trilemma}.}
\hyperref[sec:dilemma-trilemma]{dilemmas} 
and 
\hyperref[sec:unavoidable-hazards]{unavoidable hazards}.
%CANTDO\marginpar{See page~\pageref{sec:unavoidable-hazards}.} 
You can leverage your improved understanding of complicated bureaucracy by changing your behavior. In this section I provide actionable suggestions. 

My baseline assumptions are that you are a good person, you have the relevant technical skills for your duties, and you are a good \href{https://en.wikipedia.org/wiki/Project_management}{project manager}.
\index{Wikipedia!\href{https://en.wikipedia.org/wiki/Project_management}{project management}}\iftoggle{WPinmargin}{\marginpar{$>$Wikipedia: project management}}{}

\subsection*{Characteristics of an Effective Bureaucrat}

Bureaucracy is a system of distributed knowledge and distributed decision-making, so coordination is critical. Here are characteristics to strive for. 
\begin{itemize}
    \item You communicate more effectively than anyone around you. This applies to verbal and written communication, \hyperref[sec:effective-presentations]{formal presentations} 
    \marginpar{See page~\pageref{sec:effective-presentations}.} 
    and \hyperref[sec:walk-arounds]{informal interactions}.
%CONFLICTING\marginpar{See page~\pageref{sec:walk-arounds}} \\
    \textit{Why this doesn't happen by default}: people plateau when there is no incentive to improve. What level of quality is good enough depends on the situation-specific risks and costs.
    \item You facilitate meetings so that productivity is noticeably improved. This means creating and sharing an agenda, providing rules on interaction (e.g., raising hands), taking notes, sharing notes, and following on with assigned tasks after the meeting.\\
    \textit{Why this doesn't happen by default}: facilitation is rarely taught explicitly. The work of facilitation is considered secondary to the discussion, even when the productivity of discussions is disastrous and demoralizing. 
    \item You take part in meetings, whether that means actively contributing or intentionally supporting other attendees. You leverage relationships with other attendees.\\ 
    \textit{Why this doesn't happen by default}: active participation requires effort and exposes you to more risk than remaining silent.
    \item You invest effort in written communication (emails, text-based chats, reports). Your writing empathizes with readers, captures relevant context, is concise, and is clearly worded.\\
    \textit{Why this doesn't happen by default}: writing-as-a-skill is rarely the focus of a bureaucrat's work. Writing is seen as an entry-level skill that doesn't require improvement. Finding examples of clear writing in bureaucratic organizations is challenging, so you may struggle to find role models. 
    \item You communicate verbally concisely and precisely. You listen and you teach. You seek shared definitions. \\
    \textit{Why this doesn't happen by default}: Without a feedback mechanism motivating urgency, bureaucrats tend to pontificate. 
    \item If you are in-person with colleagues, you walk around and talk with people one-on-one.  \\
    \textit{Why this doesn't happen by default}: bureaucrats interested in ``just doing the work'' view professional interactions as a distraction that decreases productivity. 
    \item You are 
\marginpar{See page~\pageref{sec:professional-vulnerability}.}
    \hyperref[sec:professional-vulnerability]{professionally vulnerable}. \\
    \textit{Why this doesn't happen by default}: Sharing stories of your personal experiences in the organization requires coherent narratives that the listener can learn from. That talent is not taught to all employees in most organizations. 
    \item In your role as bureaucrat you leverage project management skills: you have a vision, you make and share plans, all while building consensus with stakeholders. \\
    \textit{Why this doesn't happen by default}: Consensus requires knowing who is relevant to include and then investing in relationships that allow iterative feedback. That use of time is costly. The level of extroversion may not be attractive to every bureaucrat.
    \item You apply your negotiation skills~\cite{1982_Cohen} that improve your interactions and outcomes.\\
    \textit{Why this doesn't happen by default}: bureaucrats are not taught negotiation unless it is an explicitly named responsibility for their role. 
    \item You reply quickly to incoming requests, whether that means answering directly or acknowledging the request and providing a timeline for when you will answer. This allows other people's tasks to either be resolved or have a clear response about when progress can be expected. \\
    \textit{Why this doesn't happen by default}: The coping skills of responding to a large number of inbound requests either grows through experience, is learned by watching role models, or just doesn't happen.
    \item You strive for and demonstrate transparency. You share information with stakeholders. Transparency  enables coordination without interpersonal relationships.\\
    \textit{Why this doesn't happen by default}: Transparency endangers incumbents. Transparency requires an ongoing investment that doesn't contribute to addressing the primary issue. Transparency requires empathy with subjects.
\end{itemize}

\ \\

The superpowers of a bureaucrat that facilitate cooperation and progress in any organization include the following.
\begin{itemize}
\item You seek information from stakeholders without burdening them. When your requests are burdensome, you acknowledge that and seek ways to pay back (or forward) their investment.
\item You apply consistent processes (rather than being reactionary and applying ad hoc responses).
\item You hold others (and yourself) accountable for their actions. Accountability is created by clearly stating objectives and then measuring results.
\item You adapt to the varying incentives and reference experiences of those around you. Flexibility enables interactions with diverse coworkers. 
    \item Your effectively multi-task or, more accurately, switch tasks. The switch among tasks is triggered when the current task encounters an externally-generated pause. You can \href{https://en.wikipedia.org/wiki/Pipeline_(computing)\%23Concept_and_motivation}{work on a task while waiting for another task to finish}. 
    \index{Wikipedia!\href{https://en.wikipedia.org/wiki/Pipeline_(computing)\%23Concept_and_motivation}{pipeline (computing)}}
\iftoggle{WPinmargin}{\marginpar{$>$Wikipedia: pipeline (computing)}}{}
    \item You are prepared with a backlog of ideas (in writing) if someone shows up with resources.
    % https://graphthinking.blogspot.com/2016/12/life-lessons-i-learned-from-experience.html
    \item You know when to change communication channels (from text chats to phone calls to in-person). 
    \item If you are dependent on someone else getting something done to enable your progress, you can demonstrate priority by physically showing up -- \underline{presence creates priority}. 
    \index{mantra!presence creates priority}
    Being physically at a person's desk motivates that person to respond better than calling  or emailing them. Showing up where someone works and talking with them conveys how much priority you place on the actions of the person you're talking with.
    \item You have intellectual empathy -- theory of mind for thinking (whereas empathy refers to emotions). How to grow your intellectual empathy: shadow peers and bosses and coworkers and subordinates.
    \item You have \gls{process empathy}. You recognize the deviations and exceptions that cause processes to come into existence. 
    \item You avoid unintentionally neglecting tasks or requests. This requires capturing incoming requests and then providing a response about prioritization and status. This matters because other participants in the organization should regard you as reliable in a positive sense. 
    \item You focus on value delivery in relationships to a degree that exceeds the scope of your formal role.
\item You have altered your job description to fit the growth you're seeking.
\item You are willing to engage on a personal level and know stakeholders outside their professional role.
\item Each of your tasks has a customer, a deadline, and a deliverable artifact. You iterate towards a result. 
\end{itemize}

You occasionally ponder and discuss with other people introspective questions like
\begin{itemize}
    \item How can I be successful?
    \item What are the ways I could fail?
    \item How would the organization be characterized as successful?
    \item What are the ways the organization can fail?
\end{itemize}
% https://graphthinking.blogspot.com/2017/07/questions-to-ask-mentors.html
You find mentors and ask them questions like
\begin{itemize}
    \item What do you like most about your career? 
    \item Given a chance, what would you do differently?
    \item How do you manage work/life balance?
    \item What's the big challenge for our industry in the next two years?
    \item How would you tackle that challenge?
    \item What advice do you have for a young person starting in this industry? In this organization? 
    \item Are there mistakes to avoid?  
    \item How can I be successful?
    \item What books would you recommend reading?
\end{itemize}

\subsection*{Frames used by Effective Bureaucrats}

Being an effective bureaucrat is a mindset. I refer to these perspectives as framing. 

\ \\
\textit{Framing}: A bureaucrat can do more as part of an organization than by working alone. Being a member of an organization means the bureaucrat's identity is subsumed into service for the organization they are part of.\footnote{See Wikipedia entry on \href{https://en.wikipedia.org/wiki/Deindividuation}{Deindividuation} -- the loss of self-awareness in groups.
\index{Wikipedia!\href{https://en.wikipedia.org/wiki/Deindividuation}{deindividuation}}
%%%CANTDOINFOOTNOTE\marginpar{$>$Wikipedia: deindividuation}
} At the same time, bureaucracy enables the bureaucrat to amplify their presence by being part of a larger organization.  Sometimes the cost of being part of the organization exceeds the force multiplier of working together. 

\ \\
% https://graphthinking.blogspot.com/2019/05/definition-of-progress.html
\textit{Framing}: Measuring your personal growth in a bureaucracy is difficult due to the lack of \hyperref[sec:feedback-loop-and-ripples]{feedback loops}. One approach is to measure your capabilities for a specific task. If you can complete the task in less time and with fewer resources and with less effort, that's progress. If you can now complete a task that you previously wanted to but weren't able to, that's progress.


\ \\
\textit{Feelings}: Bureaucracy induces an emotional response in participants because things don't work the way each person wants. This can lead emotionally to frustration and then apathy. Understanding how things work in a bureaucracy can help decrease the anger.

% Wandering the maze of bureaucratic processes as a subject.

Another emotional response to bureaucracy is a sense of powerlessness. 
\begin{quote}
``Some third person decides your fate: this is the whole essence of bureaucracy~\cite{1921_Kollontai}."
\end{quote}
That sense of powerlessness applies both to bureaucrats and to subjects of bureaucracy. 

The sense of powerlessness is somewhat valid, in that you are as a bureaucrat giving up some power compared to your ability to act individually. That is the trade for working with other people.

\ \\
\textit{Feelings}: A process feels bureaucratic when there seems to be dissonance. The subject faces a multi-step task that they can imagine one person doing the same action in fewer steps and taking less time. The illusion of decreased bureaucracy is created by consolidation from the subject's perspective. The barrier to enacting consolidation is the necessary coordination among different stakeholders who receive no benefit from the consolidation. Externalizing the coordination to the subject is what causes the sense of bureaucracy. 

\subsection*{How to be an Effective Bureaucrat}

The following advice is specific to situations that you may not have yet encountered. 

\ \\
\textit{Do}: \textbf{Work on three tasks concurrently.} Don't rely on one person or one idea for your success. On the other end of the spectrum, don't spread yourself too thin.
\marginpar{$>>$ \href{https://en.wikipedia.org/wiki/Goldilocks_principle}{Goldilocks balance}}
\index{Goldilocks balance!number of tasks}
Working on a multitude of projects decreases risk of any one outcome failing but also decreases the amount of attention spent thinking about a specific challenge. 

The three tasks should be at off-set stages (early, mid-way, and nearing completion) rather than all being the same level of maturity.

\ \\
\textit{Do}: When you get stuck, use the following changes of perspective:  
\textbf{Look upstream, look downstream. Look to peers. Zoom in (narrower scope) and zoom out (broader scope)}. These are all ways of changing the context. That change may help you identify assumptions that are holding back progress.

\ \\
\textit{Do}: When you are asked to take on more work, \textbf{avoid responding with ``that's not my job."} If the request is misguided and your perception is that it really is the responsibility of another person, ask if the requester is aware of that other person's responsibilities. If you are to  take on the work, get guidance on re-prioritizing and ensure the request is documented in writing. 

\ \\
\textit{Do}: If there's something you want to do, \textbf{strive for influence without authority} instead of working to gain control over resources (e.g., through promotion). Avoid the following: ``In this organization X is important to me, but I can't do X right now because I don't have enough power in the organization. So I'll get promoted and then do X."

\ \\
\textit{Do}: \textbf{Learn the perspectives of those around you.}\\
The relevance of knowing the paradoxes including dilemmas and unavoidable hazards, is that you should talk explicitly to your fellow bureaucrats about these specific topics in conversation. Not that the goal is to find consensus or agreement. But to find what the other person is thinking so that you can account for their processes

\ \\
\textit{Do}: \textbf{Account for holistic view.}\\
The specific circumstances of the challenges you face as a bureaucrat depend on the individual people involved, what the purpose of the bureaucracy is, what technology is available for enacting bureaucracy, and the resources (staffing, money, time). 

\ \\
\textit{Do}: \textbf{Learn the history of the challenge.}\\
This goes beyond \href{https://en.wikipedia.org/wiki/G._K._Chesterton\%23Chesterton's_fence}{Chesterton's fence}
\index{Wikipedia!\href{https://en.wikipedia.org/wiki/G._K._Chesterton\%23Chesterton's_fence}{Chesterton's fence}}\iftoggle{WPinmargin}{\marginpar{$>$Wikipedia: Chesterton's fence}}{}
\iftoggle{haspagenumbers}{(see page~\pageref{concept:chestertons_fence}), }{,}
which focuses on why the current approach is in place. Learning the history of a problem means what has been tried before and failed. How did the previous iterations evolve into the current situation? Was the cause personalities, insufficient resources, inadequate technology? What's changed that enables this approach to be better? What do you know that prior attempts didn't?

\ \\
\textit{Do}: \textbf{For a given challenge, work on three remedies}.\\
One solution you're working on may fail, and another may be inadequate. 

\ \\
\textit{Do}: \textbf{Minimize \href{https://en.wikipedia.org/wiki/Externality}{imposing costs on other people}}.\\
\index{Wikipedia!\href{https://en.wikipedia.org/wiki/Externality}{externality}}\iftoggle{WPinmargin}{\marginpar{$>$Wikipedia: externality}}{}
A solution that externalizes costs harms the greater organization and creates bureaucratic debt.

\ \\
\textit{Do}: \textbf{Exploit the flexibility of rules for the benefit of all parties.}\\
% https://graphthinking.blogspot.com/2019/07/winning-game.html
change the rules of the game and the objectives of the game such that every participant wins.

\ \\
\textit{Do}: \textbf{Align selfish interests with social interests.}\\
See also \href{https://en.wikipedia.org/wiki/Nudge_theory}{nudges}
\index{Wikipedia!\href{https://en.wikipedia.org/wiki/Nudge_theory}{Nudge theory}}\iftoggle{WPinmargin}{\marginpar{$>$Wikipedia: Nudge theory}}{}
(page 66 of Schneier's Liars and Outliers~\cite{2012_Schneier}).

\ \\
\textit{Do}: \textbf{Find ways to rephrase negative complaints}.\\
% https://graphthinking.blogspot.com/2019/07/how-to-rephrase-negative-observations.html
Negative observation: ``Logging into my computer takes a long time."\\
Positive statement and explanation of impact: ``If the latency for logging into my computer were lower, I could make more progress on X."


Negative observation: ``The service team I need support from doesn't offer a ticketing queue."\\
Positive statement: ``If the service team I need support from offered a ticketing queue, I would be able to track the work done on my behalf."

\ \\
\textit{Do}: \textbf{Share lessons learned}.\\
This can build your network. People value vulnerability in others; you can make the first move. 