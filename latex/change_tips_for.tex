\subsection*{Creating Change in the Organization\label{sec:creating-change}}

If the organization you are in has no problems or challenges, this section can be skipped. For bureaucrats in organizations that do have issues, this part of the book provides points to ponder independent of the specific problem.

As a bureaucrat, you have unique insight into the challenges the organization faces, and you have unique leverage to alter the situation.  While you could proceed haphazardly, an effective bureaucrat has vision -- a story of what could be. This vision can be broken into goals -- specific, measurable outcomes. Each goal requires contingency plans for achieving the goal. Plans are marked by milestones that indicate whether the plan is proceeding successfully. 

Perspectives to consider when assessing change include what the situation is, what the situation could be, and what the situation looks like for different stakeholders. Each person has different information, a different way of responding to issues, and shifting intents.

People you depend on who have conflicting visions or no vision confound your ability to improve the organization. There are different views on whether something is a problem, distinct characterizations of an issue, and competing priorities.

A trade-off to consider is that having a n\"iche impact is easier than broad change (the \hyperref[table:dilemma-personal-scope-broad-vs-narrow]{Dilemma of Scope of Effect}). 
\marginpar{See page~\pageref{table:dilemma-personal-scope-broad-vs-narrow}.}%
There's also a trade-off of the quick fix versus more robust solutions (the \hyperref[table:dilemma-personal-quick-methodical]{Dilemma of Speed and Accuracy}).


%Determine social, political, and technical impediments. 

\index{list of tips!change within an organization}

\ \\
% https://graphthinking.blogspot.com/2018/08/potential-paths-when-faced-with.html
\textit{Suggestion}: Understand and empathize with people who fear change. \\
When faced with a challenge, the options are to take action or not take action. If you take action, the options are failure, success, or iteration. Action incurs risk due to uncertainty and costs work to create change.
The choice of not taking action is attractive if you are not suffering. Even if change would decrease suffering, delay minimizes work and there are other things to focus on.
%enables beneficial moves since the player can wait out other players


Fear of failure is justified if the cost of the failure is greater than the value of lessons learned.

Fear of iteration is a fear that the process might be stopped prematurely -- before perfection.

\ \\
\textit{Suggestion}: Before starting a new effort, check to see whether this has been tackled before.
\marginpar{$>>$ Actionable Advice}%
\index{actionable advice}%
Learn the history of the problem. Why hasn't this been solved?

\ \\
\textit{Suggestion}: Learn the folklore. Talk with your first- and second-order social network.
\marginpar{$>>$ Actionable Advice}
Ask your coworkers, and ask them who else you should talk with.
\index{actionable advice}

\ \\
% https://graphthinking.blogspot.com/2016/01/methodology-for-people-acting-as.html
\textit{Suggestion}: Use social recommendations by naming relevant individuals.\\
Leverage the trust already in an existing social network by starting with ``Person A recommended I talk to you about X."


\ \\
% https://graphthinking.blogspot.com/2016/01/methodology-for-people-acting-as.html
\textit{Suggestion}: Sit in on meetings, listen to topics, see who is talking, and see who is attending. After the meeting, talk to individuals about the meeting. Set up one-on-one informal discussions. Keep the first conversation brief - 10 or 15 minutes. Your body language should indicate engagement and interest. ``Who else would you recommend talking to?" is the last question in the first conversation.

\ \\
\textit{Suggestion}: Get feedback early before polishing. This iterative approach enables you to account for the concerns of stakeholders and decreases their surprise.
\marginpar{$>>$ Actionable Advice} 
\index{actionable advice}

\ \\
\textit{Suggestion}: Advertise the result. Don't rely on the change being sufficient for people to be aware the change happened.

\ \\
\textit{Suggestion}: Hear criticism and respond. Ignoring feedback harms relationships.

\ \\
\textit{Suggestion}: Leverage both social networks and bureaucratic processes. This requires building and maintaining relationships. 

\ \\
\textit{Suggestion}: Identify sources of power (hierarchical positions and titles, social influence, reputation and credibility, buzzphrases or popular paradigms) and leverage them.

\ \\
\textit{Suggestion}: Conduct your interactions with professional respect (for what the other person knows) and professional curiosity (for what you don't know). \\
Example: Getting approval from multiple overseers in different hierarchies is hard. Often different stakeholders have different objectives and incentives.



\ \\
Consensus doesn't mean everyone agrees on the problem, the remedy, the approach, or who's taking action. Consensus in a bureaucracy means people aren't going to resist the change.