\chapter{Interview questions}

Interviewing candidates for a bureaucratic position should involve evaluation of skills specific to bureaucracy. For example, draw out whether the candidate can communicate to facilitate coordination, work with experts from other domains, learn from self-reflection, has an ability to professionally disagree, and practices negotiation. 

\section{New Hire}

%\subsection{Questions for a Generic Bureaucrat}

\begin{itemize}
    \item Tell me about a time you had competing priorities. \\
    \textit{What to evaluate for}:
    \begin{itemize}
        \item Did the interviewee coordinate a response with other people? Teamwork is an essential skill in a bureaucratic organization.
        \item Did the interviewee clarify a misconception?
    \end{itemize}
    \item How does this position fit into your career goal?\\
    \textit{What to evaluate for}:
    \begin{itemize}
        \item Can the organization provide the candidate what they are seeking?
    \end{itemize}
    \item In previous jobs, how frequently did you check in with your supervisor (formally or informally)?\\
    \textit{What to evaluate for}:
    \begin{itemize}
        \item Does the candidate communicate with their supervisor? 
        \item Who initiates the discussion?
    \end{itemize}
    \item In previous jobs, how frequently did you check in with your coworkers (formally or informally)?\\
    \textit{What to evaluate for}:
    \begin{itemize}
        \item Does the candidate communicate with their peers? 
        \item Who initiates the discussion?
    \end{itemize}
    \item In previous jobs, in what situations did you turn to your boss for help? Or did your boss help you?\\
    \textit{What to evaluate for}:
    \begin{itemize}
        \item When the candidate's boundary is tested, how do they engage?
        \item Who initiates the discussion?
    \end{itemize}
    \item In previous jobs, in what situations did you turn to your coworkers for help? Or did your coworkers help you?\\
    \textit{What to evaluate for}:
    \begin{itemize}
        \item When the candidate's boundary is tested, how do they engage?
        \item Who initiates the discussion?
    \end{itemize}
    \item How do you engage with coworkers who know something work-related that you want to learn?
    \item Tell me about a time you were given the wrong scope (or an ill-defined scope) for a task. 
    \item In previous jobs, how did you evaluate the success and short-comings of your organization?
    \item What are the leading indicators you look for that the team you are on is headed in a bad direction?
    \item Tell me about a time you initiated a collaboration with a coworker to leverage their expertise?
    \item Tell me about a time you worked with someone who had a distinct perspective from yours. How did you collaborate?
    \item Tell me about a time when project requirements were unclear, ill-defined, or constantly shifting.
    \item Describe how you responded to a mistake you made at work.
    \item Tell me about a time the organization you were in changed and how you navigated that. 
    \item Tell me about a time when you, in your role as an expert, collaborated with people who didn't share your expertise. How does that compare with interacting with fellow experts?
    \item When you have learned a new topic in the past, what strategies did you use to balance understanding the theory and doing the practice?
    \item How have you assessed your progress when you were learning a new topic in the past?
    \item How have you structured your time when faced with a novel challenge?
    \item How do you distinguish between accidental complexity and essential complexity? \footnote{from Brook's \href{https://en.wikipedia.org/wiki/No_Silver_Bullet}{No Silver Bullet}}
\end{itemize}

%\subsection{Questions for a Generic Manager}

\clearpage
\section{Moving to a New Team}

Prior to leaving the team you're on, reflect on your current role:
\begin{itemize}
    \item What are my current care-abouts?
    \item What would I miss from my current role? Are there projects, tasks, responsibilities, or people that I will miss? If yes, can I change my current role to have more of what I like?
\end{itemize}
You can negotiate with your management and coworkers to address your needs. Sometimes other people are not aware of what you want. 

Once you are interviewing for a position you should be considering the interactions as a chance to learn more about the work and the people. When interviewing for a new position, ask the following questions\footnote{See also \href{https://www.themuse.com/advice/51-interview-questions-you-should-be-asking}{51 Great Questions to Ask in an Interview}}:
\begin{itemize}
    \item Is the team growing or is the focus shifting? \\
    (Teams hire to replace or grow. If replacing, why did the person leave? If growing, is it an increase in capacity or expansion of scope or change of focus?)
    \item What's the turn-over rate of the team I would be joining? Has anyone quit? Has anyone recently joined?
    \item How many members are their on the team I would be joining?\\
    (Is the team size stable, or does it fluctuate according to changes in tasking?)
    \item Would I be on a single team or would my work span multiple teams?
    \item How many different roles would a person in my position have? 
    
    \item What would my day-to-day routine be like?
    \item How many meetings would I have per day? Per week?

    \item What indicates success in my new team? 
    \item What indicates failure in my new team? 
    \item How am I evaluated as an employee?
    \item How is my team evaluated?

    \item Who do I report to? 
    \item How many bosses do I have?

    \item What balance of independence and collaboration is expected of me?
    \item Do I identify what to work on or is a task assigned to me? If a task is assigned, who assigns the work? (the customer or a manager or another team?)
    \item What is the typical duration of a task I would work on? \\
    (An hour or a day or a week or a month or a year.)
    \item How does my new team support my autonomy?
    \item What support does the organization provide a person in my role? (Training)
    \item In the past year, how quickly have new employees been trained in this position?
    \item What computer resources are available? Are other equipment relevant to the job? 

    \item How frequently would I interact with customers? 
    \item Who are my customers? What do my customers want?
    \item How many customers are there?
    \item How frequently would I interact with coworkers? 
    \item How frequently would I interact with boss(es)? 
    \item What are the goals for the organization? For the team?

    \item What is your favorite part of working in this organization?

    \item What are the technical challenges?
    \item How does the team document shared knowledge?
    \item Does the team have current onboarding materials to reference? Or is the process more social?
    \item What are the biggest challenges someone in this role faces? \\
    (Sometimes the challenges are political or personality rather than technical.)
    \item How diverse are the educational and experiential backgrounds of team members?
    \item What skills do other team members have?
    \item What gaps in skills does the team have that you are looking to fill?
    
    \item Where is the office I would be working in located?
    \item Is the location of the office expected to change in the next six to twelve months?
    \item Would I expected to travel?
    
    \item How many different tasks would I be expected to work on in a day? In a week?
    
    \item Do you expect the answers to any of the above questions to change in the next six to twelve months?

    \item What are the next steps in the hiring process?
    \item When will a hiring decision be made?
    \item When would I be starting in this position?
    
    % https://graphthinking.blogspot.com/2019/06/questions-to-ask-in-interview-of-your.html
    \item What are ethical issues or moral challenges should I expect to face if I were employed on this team in this organization?
    \item What are the known conflicts of interest that exist?

    \item Who is working against the progress of the team?
    
\end{itemize}

That's a long list of questions, and it doesn't even include questions specific to the role your interviewing for or the organization you're interviewing. Bringing a printed out list demonstrates thinking ahead and preparing. 

%\section{Exit Interview Questions}