In this appendix I provide summaries of various lengths that are intended to distill the most relevant points. These recaps can be used as previews or reminders but do not substitute for the full text.

\section*{One Sentence Summary of This Book}

You can be more effective in your role as a \gls{bureaucrat} by applying \gls{process empathy}.

\ \\

\section*{Two-Paragraph Summary of This Book}


\iftoggle{glossarysubstitutionworks}{\Gls{bureaucracy}}{Bureaucracy} is pervasive because it is crucial for society. Bureaucracy is defined as the set of subjective decisions made by individual participants in an organization. The decisions originate from the 
management of \iftoggle{glossarysubstitutionworks}{\glspl{shared resource}.}{shared resources.} 
The resources can be either tangible or in the form of expertise. Bureaucracy relies on
distributed knowledge and distributed decision-making.

\iftoggle{glossarysubstitutionworks}{\Gls{process empathy}}{Process empathy} 
is helpful in your role as a bureaucrat and when you are subject to bureaucracy. 
Process empathy derives from your understanding the \hyperref[sec:dilemma-trilemma]{dilemmas} and \hyperref[sec:unavoidable-hazards]{unavoidable hazards} of bureaucracy, recognizing internal \gls{motives} and external \gls{incentives} that individuals in each role have, and your ability to reconstruct the sequence of decisions that led to the current situation. 





