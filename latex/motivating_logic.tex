The motivating logic for this book is a sequence of claims:
\begin{enumerate}
    \item A community has limited resources that members desire. The community responds to that constraint by sharing resources.
    \item Managing access to a \gls{shared resource} 
    \iftoggle{glossaryinmargin}{\marginpar{[Glossary]}}{} requires making and enforcing policies that apply to community members. 
    \item When more than one person is responsible for making and enforcing policies, distributed knowledge and distributed decision-making is required. 
    \item 
\iftoggle{glossarysubstitutionworks}{\Gls{bureaucracy}}{Bureaucracy}
\iftoggle{glossaryinmargin}{\marginpar{[Glossary]}}{} is defined as multiple people managing access to 
\iftoggle{glossarysubstitutionworks}{\glspl{shared resource}.}{shared resources.} 
    \item Whether you're a \gls{bureaucrat} 
    \iftoggle{glossaryinmargin}{\marginpar{[Glossary]}}{} or a subject of bureaucracy, you can be more effective by applying \hyperref[sec:process-empathy]{process empathy}.
    %\marginpar{See page~\pageref{sec:process-empathy}.}
\end{enumerate}

Once you've familiarized yourself with the concepts in this book, using what you've learned is a two-step process. First, ask yourself whether the conditions for bureaucracy are present. If they are, then you can be more effective by applying process empathy.

The definition of bureaucracy is the basis for evaluating the scenario you are in. Illustrations of bureaucracy provided throughout this book demonstrate how you can grow your process empathy. 