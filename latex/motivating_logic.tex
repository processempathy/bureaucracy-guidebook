The motivating logic for this book is a sequence of claims:
\begin{enumerate}
    \item Society has limited \glspl{shared resource} and mitigates that constraint by sharing resources.
    \item Managing access to a shared resource requires making and enforcing policies that apply to community members. 
    \item When more than one person is responsible for making and enforcing policies, then distributed knowledge and distributed decision making are required. 
    \item \Gls{bureaucracy} is defined as multiple people managing access to \glspl{shared resource}. 
    \item Whether you're a \gls{bureaucrat} or a subject of bureaucracy, you can be more effective by applying \hyperref[sec:process-empathy]{process empathy}.
\end{enumerate}

Once you've familiarized yourself with the concepts in this book, using the what you've learned is a two step process. First, ask yourself whether the conditions for bureaucracy are present. If they are, then you can be more effective by applying process empathy.

The definition of bureaucracy and illustrations of it are the foundation on which you build process empathy. 