The motivating logic for this book is a sequence of claims:
\begin{enumerate}
    \item A community has limited resources that members desire. The members of the community respond to that constraint by sharing resources.
    \item Contention amongst community members regarding access to a \gls{shared resource} leads to a need to manage access.  
    \iftoggle{glossaryinmargin}{\marginpar{[Glossary]}}{}%
    Managing access requires making and enforcing policies that apply to community members. This may start as a consensus but often a single person is nominated because the cost of consensus is high. This is not yet bureaucracy because subjects can negotiate with the person managing access to the resource.
    \item Due to increased size or complexity, more than one person may be responsible for making and enforcing policies. Having multiple policymakers leads to inconsistent policies, so instead the workload is split into the roles of policymaker and policy execution by a bureaucrat. 
    \item 
\iftoggle{glossarysubstitutionworks}{\Gls{bureaucracy}}{Bureaucracy}
exists when people involved in managing access to 
\iftoggle{glossarysubstitutionworks}{\glspl{shared resource}}{shared resources} have the roles of policymaker, bureaucrat, and subject.
%While bureaucracy arises in response to  bureaucracy can exist without being assocaited with a shared resource.
    \item Managing access to shared resources can involve multiple bureaucrats operating together. In a large organization the use of distributed knowledge and distributed decision-making results in \gls{decentralized bureaucracy}.
    \item Whether you're a \gls{bureaucrat} 
    \iftoggle{glossaryinmargin}{\marginpar{[Glossary]}}{}%
    or a subject of bureaucracy, you can be more effective by applying \hyperref[sec:process-empathy]{process empathy}.
    %\marginpar{See page~\pageref{sec:process-empathy}.}
\end{enumerate}

Once you've familiarized yourself with the concepts in this book, using what you've learned is a two-step process. First, ask yourself whether the conditions for bureaucracy are present. If they are, then you can be more effective by applying process empathy.

The definition of bureaucracy is the basis for evaluating the scenario you are in. You may already have a sense of what bureaucracy is, but a description is provided
\iftoggle{haspagenumbers}{ on page~\pageref{sec:define-bureaucracy}}{} 
in the section~\hyperref[sec:define-bureaucracy]{``What is Bureaucracy?''} To supplement the definition, illustrations of bureaucracy provided throughout this book show how you can grow your process empathy -- the focus of the next section. 