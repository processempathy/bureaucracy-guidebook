\subsubsection*{Organization chart orientation
\label{sec:org-chart-orientation}}

A common method of describing relations within the bureaucracy is the organization chart (commonly the ``\gls{org chart}"). \iftoggle{glossaryinmargin}{\marginpar{[Glossary]}}{}%
Normally the Chief Executive Officer (CEO) is at the top of the chart, middle management is in the middle, and managed employees are at the bottom. See Figure~\ref{fig:org_chart_orientation_ceo-at-top}\iftoggle{haspagenumbers}{ on page~\pageref{fig:org_chart_orientation_ceo-at-top}.}{.}

Artifacts like org charts subtly convey an organization's culture. 
% What's the point of this section? Is there a consequence, or is this just an observation?
There are emotional connotations to alternative layouts. You can alter expected relations (culture and norms) by playing with the orientation of the org chart.
There is a risk of overanalyzing org chart orientation, so the exploration in this section is limited.

The point of thinking about org chart orientation is to frame how you perceive your chain of command, peers, and (if applicable) the bureaucrats you manage. Notice that the framing is embedded in the words -- prefixes super (over) and sub (under). 
These concepts inform what you expect from relations with supervisors and subordinates.
Do I seek support or direction and guidance from my supervisor? What do I expect from my supervisor? My peers? The people I oversee? What do I expect to provide them?

%\begin{itemize}
%\item 
%\end{itemize}

The relative orientation of the \href{https://en.wikipedia.org/wiki/Chief_executive_officer}{CEO} 
\index{Wikipedia!Chief Executive Officer@\href{https://en.wikipedia.org/wiki/Chief_executive_officer}{Chief Executive Officer}}\iftoggle{WPinmargin}{\marginpar{$>$Wikipedia: CEO}}{}
to the workers sets expectations for relations. 
Options for orientation are the conventional CEO at the top
(Figure~\ref{fig:org_chart_orientation_ceo-at-top}), 
CEO at the bottom (Figure~\ref{fig:org_chart_orientation_ceo-at-bottom}),
CEO on the right (Figure~\ref{fig:org_chart_orientation_ceo-leads}),
CEO on the left (Figure~\ref{fig:org_chart_orientation_ceo-follows}),
or the CEO at the center.\footnote{For example, the diagram on Wikipedia page for the 1930 \href{https://en.wikipedia.org/wiki/File:League_of_Nations_Organization.png}{League of Nations}.}
\index{Wikipedia!League of Nations diagram@\href{https://en.wikipedia.org/wiki/File:League_of_Nations_Organization.png}{League of Nations diagram}}

\begin{figure}
\begin{center}
\includegraphics[width=1\textwidth]{images/org-chart-orientation-ceo-at-top.pdf}
\end{center}
\caption{Standard orientation. The role with the most oversight and authority is at the top. Left-to-right ordering is usually intended to be irrelevant in this view, though reading left-to-right  order can implicitly emphasize relative importance.}
\label{fig:org_chart_orientation_ceo-at-top}
\end{figure}

\begin{figure}
\begin{center}
\includegraphics[width=1\textwidth]{images/org-chart-orientation-ceo-at-bottom.pdf}
\end{center}
\caption{Flipping the orientation of Figure~\ref{fig:org_chart_orientation_ceo-at-top} presents a more realistic view of the CEO's responsibility. The crushing burden of servant leadership is clear. Left-to-right ordering is intended to be irrelevant in this view.}
\label{fig:org_chart_orientation_ceo-at-bottom}
\end{figure}

\begin{figure}
\begin{center}
\includegraphics[width=0.7\textwidth]{images/org-chart-orientation-ceo-leads.pdf}
\end{center}
\caption{Conventionally time flows from left (old) to right (new), so in this graph the CEO leads the charge into the unknown. Is the CEO dragging workers forward, or are the workers pushing the CEO? The top-to-bottom order may be implicitly read as importance even if that wasn't the intent. }
\label{fig:org_chart_orientation_ceo-leads}
\end{figure}

\begin{figure}
\begin{center}
\includegraphics[width=0.7\textwidth]{images/org-chart-orientation-workers-lead.pdf}
\end{center}
\caption{The ``chariot view'' with the CEO in the chariot and the workers out front. Workers are in the future; the CEO is in the past operating on old information. As with Figure~\ref{fig:org_chart_orientation_ceo-leads}, top-to-bottom ordering can be read as importance. }
\label{fig:org_chart_orientation_ceo-follows}
\end{figure}

An implicit conclusion from Figures~\ref{fig:org_chart_orientation_ceo-at-top}, \ref{fig:org_chart_orientation_ceo-at-bottom}, \ref{fig:org_chart_orientation_ceo-leads}, and~\ref{fig:org_chart_orientation_ceo-follows} is that teams at the same level of the hierarchy are peers and should be treated similarly in terms of importance. Figure~\ref{fig:org_chart_wedding_cake_manufacturing} provides yet another perspective informed by which team provides services to other teams. Which teams are the customers informs the relationship dynamics.

\begin{figure}
\begin{center}
\includegraphics[width=0.8\textwidth]{images/org_chart_wedding_cake_dependencies_-_manufacturing.pdf}
\end{center}
\caption{An internal-customer-oriented view rather than a reporting-oriented view. The center of the bullseye is the team that generates the value that is the focus of the business or the organization, in this case a manufacturing company.
Teams in the outer rings support teams in the inner rings. The diagram is specific to an organization's domain. This visualization identifies which teams are the customers of which other teams in an organization. }
\label{fig:org_chart_wedding_cake_manufacturing}
\end{figure}



%extension of 
% \href{https://en.wikipedia.org/wiki/Conway\%27s_law}{Conway's law}: seating chart reflects org chart