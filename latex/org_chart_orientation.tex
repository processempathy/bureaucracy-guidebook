\subsubsection{Organization chart orientation
\label{org-chart-orientation}}

How an organization's culture is conveyed by artifacts like org charts is subtle and can be overanalyzed. 

A common method of describing relations within the bureaucracy is the organization chart (colloquially, the ``org chart"). Normally the CEO is at the top of the chart, middle management is in the middle, and managed employees are at the bottom. See Fig.~\ref{org_chart_orientation_ceo-at-top} 

% What's the point of this section? Is there a consequence, or is this just an observation?
There are emotional connotations to alternative layouts. Convey culture by playing with orientation of the relations.

The point of org chart orientation is to frame how you perceive your boss, peers, subordinates. Do I seek support or guidance from my boss? What do I expect from my boss, peers subordinates? What do I expect to provide them?

\begin{itemize}
\item CEO at the bottom -- Fig.~\ref{org_chart_orientation_ceo-at-bottom}
\item CEO on the left -- Fig.~\ref{org_chart_orientation_ceo-follows}
\item CEO on the right -- Fig.~\ref{org_chart_orientation_ceo-leads}
\item CEO as the center of a star\footnote{Example: \href{https://en.wikipedia.org/wiki/File:League_of_Nations_Organization.png}{League of Nations diagram}}
\end{itemize}


\begin{figure}
\includegraphics[width=1\textwidth]{images/org-chart-orientation-ceo-at-top.pdf}
\caption{Standard orientation. Role with most responsibility is at top. Left-right ordering is intended irrelevant in this view.}
\label{org_chart_orientation_ceo-at-top}
\end{figure}

\begin{figure}
\includegraphics[width=1\textwidth]{images/org-chart-orientation-ceo-at-bottom.pdf}
\caption{Flipping the orientation presents a more realistic burden on the CEO's responsibility. Left-right ordering is intended to be irrelevant in this view.}
\label{org_chart_orientation_ceo-at-bottom}
\end{figure}

\begin{figure}
\includegraphics[width=0.8\textwidth]{images/org-chart-orientation-ceo-leads.pdf}
\caption{Conventionally time flows from left (old) to right (new), so in this graph the CEO leads the charge into the unknown. The top-to-bottom ordering can be read as importance. }
\label{org_chart_orientation_ceo-leads}
\end{figure}

\begin{figure}
\includegraphics[width=0.8\textwidth]{images/org-chart-orientation-workers-lead.pdf}
\caption{The ``chariot view'' with the CEO in the chariot and the workers out front. As with Fig.~\ref{org_chart_orientation_ceo-leads}, top-to-bottom ordering can be read as importance. }
\label{org_chart_orientation_ceo-follows}
\end{figure}
