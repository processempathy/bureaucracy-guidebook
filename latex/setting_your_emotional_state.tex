\section{Setting Your Emotional State}

As a member of a bureaucratic organization, you may feel the frustration of bureaucracy, the happiness of success, the excitement of possibilities, or the fear of uncertainty of operating in a complicated environment. Your emotional state, whether set by your work or set by events outside the bureaucracy, impacts your productivity and the well-being of your coworkers. 

You may see emotions as getting in the way of productivity and creating problems, or perhaps you see emotions as unprofessional. This view is not shared by everyone, and there are real benefits to feelings like passion, enthusiasm, excitement, and joy. Yes, joy can be felt by bureaucrats. 

Expecting bureaucrats to apply cold, rational logic to decisions and interactions is unrealistic. Openness to engaging emotionally is more constructive than avoiding or suppressing emotional aspects of the job. 

One source of emotions inspired by bureaucracy is when your roles, responsibilities, and resources do not align. You want to be effective and feel frustrated. This frustration can infect your interactions with coworkers. 

Another source of emotions in a bureaucratic organization is when you see someone else making what appears to be a poor decision or policy. If you lack the ability to alter the decision or influence the policy, this can feel disheartening or demoralizing. Worse, these policies outside your control may impact you. 

A third set of emotions in bureaucratic interactions occurs when a group of bureaucrats with distinct perspectives, strong opinions, and diverse experiences faced with complex challenges have insufficient time to create consensus and select the optimal choice. Stress felt by each participant impacts the interaction. 

The relevance of these emotional states is that frustration and stress can decrease your capacity for effective communication. Happier bureaucrats can be more productive and collaborative. 

By identifying these emotional patterns, the situation can seem less personal. Your feelings are legitimate, but they are not unique to you. Anyone in that situation is likely to feel similarly. Realizing this helps build empathy instead of ignoring the feeling. 