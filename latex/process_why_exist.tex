\subsection{Why do processes exist?}


\subsubsection{Processes ensure consistent application of policy}

Processes are in response to allocate scarce resources. While doing that, also 
* limit in-the-moment bad behavior; the process should include guardrails to prevent harm. 
* limit the potential of things going bad in the future. 
* keep stakeholders informed

Processes attempt to address risk to the allocated resource and risk to the bureaucrats carrying out the process. The latter is why justification is necessary. 

    
Without oversight processes, \href{https://en.wikipedia.org/wiki/Tragedy_of_the_commons}{tragedy of commons} occurs and malicious actors dominate.

\href{https://en.wikipedia.org/wiki/Tragedy_of_the_commons}{Tragedy of the Commons} says that when there is a shared resource, someone will try to get away with behavior that is harmful to the organization.

Create processes for oversight/review/approval
Each process may be justifiable, but the aggregate can feel unreasonably burdensome to subjects and bureaucrats.


% https://advancedbiofuelsusa.info/rodney-hailey-sentenced-to-more-than-12-years-in-prison-for-selling-9-million-in-fraudulent-renewable-fuel-credits/



\subsubsection{Processes exist to Simplify}
Compared to an ad hoc approach, processes are intended to simplify what could otherwise be complicated moral decisions, coordination challenges, or financial assessments. %The simplification benefits the bureaucrat's workload; rarely are benefits to the subjects of the bureaucracy.  



A positive view is that specialization allows narrow focus and thus deeper understanding and skill. An alternative perspective is that specialization allows for dumbing down the role, thus enabling a cheaper workforce. In that view process allows dumb individuals to, as a group accomplish complicated things.  As a specific example, consider the process of designing and building a car. That complexity is feasible to undertake for a skilled and knowledgeable individual, but the cheaper approach is to hire individuals capable of installing the passenger-side doors in an assembly line.

Process development is driving by Complexity.

\subsubsection{Processes exist for Scaling Throughput}

Processes improve scalability. Cite the pin factory example

Ford factory throughput comparison

social processes work well for low-volume ad-hoc requests.

Processes work well for recurring high-volume requests

\subsubsection{Processes decrease reliance on Social Relations}

Relation between processes and social relations was identified by \cite{1943_Selznick}.

For a given complexity and given scale, there are people who are more social and less social and therefore desire more or less process.

There are people who want process and documentation and are confused as to how things are operating when those aren't present


social influence does not have visibility/transparency
In contrast, processes are easier to understand and to track

social influence is not antithetical to processes. There's always a mixture of the two


\subsubsection{Processes help new people}

New people are likely to arrive and ask what are the processes and where is the documentation?

People have been on a team for a long time say don't burden me with processes I just need to get things done using the relationships with the people I have.

In addition to not having formed a social network, there's a second reason new people seek processes.

Recently graduated people from school are used to the existence of formal processes at high school or university level. Therefore when they join a job, they expect a similar set of conditions for processes to exist and follow

onboarding for processes is easy -- folk lore or documentation

social onboarding: sitting with other teams (prisoner exchange) to build network.
--> Where you sit matters

\subsubsection{Processes exist as artifacts for Promotion}

Processes can arise organically (bottom up) or be created top-down

