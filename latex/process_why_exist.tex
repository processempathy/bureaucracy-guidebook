\section{Why do Processes Exist?}

Bureaucratic organizations manage scarce shared resources (tangible or expertise). Policies are decisions administered by bureaucrats. Processes inform the decision making of bureaucrats and result in access to shared resources. 

The following sections provide more detailed explanation for by organizations rely on processes. The reason this matters to you as a bureaucrat is when you design a process or try to revise an existing process, you should keep these aspects in mind. Similarly, as a person going through a process, the reasoning below explains why you are experiencing friction. 

\subsection*{Processes Ensure Consistent Application of Policy}

Organizations are designed to be insensitive to individual participants, both in terms of bureaucrats and subjects. Processes try to address risk to the allocated resource and risk to the bureaucrats carrying out the process.

The risk addressed by the use of processes is both in-the-moment bad behavior and the potential of things going bad in the future. Process act as guardrails to prevent harm to the bureaucrats and the shared resource. An example guardrail is keep stakeholders informed (justifications provided) so that intervention can be taken if needed. 

\href{https://en.wikipedia.org/wiki/Tragedy_of_the_commons}{Tragedy of the Commons} says that when there is a shared resource, someone will try to get away with behavior that is harmful to the organization.
Without oversight processes, \href{https://en.wikipedia.org/wiki/Tragedy_of_the_commons}{tragedy of commons} occurs and malicious actors dominate.


Create processes for oversight/review/approval
Each process may be justifiable, but the aggregate can feel unreasonably burdensome to subjects and bureaucrats.


% https://advancedbiofuelsusa.info/rodney-hailey-sentenced-to-more-than-12-years-in-prison-for-selling-9-million-in-fraudulent-renewable-fuel-credits/



\subsection*{Processes exist to Simplify}
Compared to an ad hoc approach, processes are intended to simplify what could otherwise be complicated moral decisions, coordination challenges, or financial assessments. %The simplification benefits the bureaucrat's workload; rarely are benefits to the subjects of the bureaucracy.  



A positive view is that specialization allows narrow focus and thus deeper understanding and skill. An alternative perspective is that specialization allows for dumbing down the role, thus enabling a cheaper workforce. In that view process allows dumb individuals to, as a group, do complicated things.  As a specific example, consider the process of designing and building a car. That complexity is feasible to undertake for a skilled and knowledgeable individual, but the cheaper approach is to hire individuals capable of installing the passenger-side doors in an assembly line.

% TODO
Process development is driving by increased complexity.

\subsection*{Processes Enable Increased Throughput}

Processes that leverage specialization can improve scalability. Adam Smith's \href{https://en.wikipedia.org/wiki/Business_process#Adam_Smith}{pin factory example} cites a productivity gain of 240x. Similarly, the introduction of moving assembly lines in Ford's car factory produced an 8x improvement of throughput. 

While social processes may work well for low-volume ad-hoc requests, formalized bureaucratic processes become necessary for recurring high-volume activities.

\subsection*{Processes Decrease Reliance on Social Relations}

Relation between processes and social relations was identified by Selznick in~\cite{1943_Selznick}.

For a given complexity and given scale, there are people who are more social and less social and therefore desire more or less process.

There are people who want process and documentation and are confused as to how things are operating when those aren't present

% TODO
social influence does not have visibility/transparency
In contrast, processes are easier to understand and to track

% TODO
social influence is not antithetical to processes. There's always a mixture of the two


\subsection*{Processes help new people}

New people are likely to arrive and ask what are the processes and where is the documentation?

People have been on a team for a long time say don't burden me with processes I just need to get things done using the relationships with the people I have.

Besides not having formed a social network, there's a second reason new people seek processes.

Recently graduated people from school are used to the existence of formal processes at high school or university level. Therefore when they join a job, they expect a similar set of conditions for processes to exist and follow

% TODO
onboarding for processes is easy -- folk lore or documentation

% TODO
social onboarding: sitting with other teams (prisoner exchange) to build network.
Where you sit matters.

\subsection*{Processes exist as Artifacts for Promotion}

Processes can arise organically (bottom up) or be created top-down. In either context, creating new processes counts as bureaucratic innovation. The cause of this confusion is the distinction of novelty from innovation. Real innovation incurs risk, whereas new processes are merely novel. Both cause change, but processes are typically designed to decrease risk. 

