\section{Why do Processes Exist?\label{sec:why-processes-exist}}


The following sections explain why organizations rely on \iftoggle{glossarysubstitutionworks}{\glspl{process}}{processes}. 
The reasoning matters to you in your role as a bureaucrat when you design a process or try to revise an existing process. Similarly, as a person going through a process, the reasoning below explains why you are experiencing \gls{process friction}. 

\subsection*{Processes Enable Consistent Application of Policy}

Organizations are intended to be insensitive to individual participants. That de-personalization applies to both bureaucrats and subjects. In practice, there is sensitivity to who the bureaucrat is and who the subject is. 

Processes address risk to the allocated resource and risk to the bureaucrats carrying out the process.
The use of processes addresses the risk of both current bad behavior and the potential of things going bad in the future. Process is the guardrail limiting harm to bureaucrats and the shared resource. 

One motive for bureaucracy is the \href{https://en.wikipedia.org/wiki/Tragedy_of_the_commons}{Tragedy of the commons}, 
\index{Wikipedia!\href{https://en.wikipedia.org/wiki/Tragedy_of_the_commons}{Tragedy of the Commons}}
which says that when there is a shared resource, someone will try to get away with behavior that is harmful to the community of users. Limiting harmful behavior can take the form of oversight processes. If oversight processes are not in place, malicious actors dominate.

Another example guardrail is keeping stakeholders informed (justifications) so that intervention can be taken if needed. 

Each process for oversight, review, or approval may be justifiable when evaluated in isolation, but the aggregate can feel unreasonably burdensome to both subjects and bureaucrats.


% https://advancedbiofuelsusa.info/rodney-hailey-sentenced-to-more-than-12-years-in-prison-for-selling-9-million-in-fraudulent-renewable-fuel-credits/



\subsection*{Processes Enable Simplification}
Compared to an ad hoc approach, processes are intended to simplify what could otherwise be complicated moral decisions, complex coordination challenges, or financial assessments. %The simplification benefits the bureaucrat's workload; rarely are benefits to the subjects of the bureaucracy.  


The relation between the complexity of a task and the skills of the bureaucratic workforce is another motive for the creation of processes.
A positive story is that specialization allows narrow focus and thus deeper understanding and skill. An alternative perspective is that specialization allows for dumbing down the role, thus enabling a cheaper workforce. See figure~\ref{fig:complexity-and-size} \iftoggle{haspagenumbers}{on page~\pageref{fig:complexity-and-size}}{}
for a visualization of the trade-off.

Applying a process allows dumb individuals to do complicated things. The collective talent exceeds that of any individual.  As a specific example, consider the process of designing and building a car. That complexity is feasible to undertake for a skilled and knowledgeable individual, but the cheaper approach is to hire individuals capable of installing the passenger-side doors in an assembly line.

% Potential story to inject: founders of AMD left Fairchild
% https://en.wikipedia.org/wiki/Advanced_Micro_Devices#History

\subsection*{Processes Enable Increased Throughput}

Processes that leverage specialization can improve scalability. Adam Smith's \href{https://en.wikipedia.org/wiki/Business_process#Adam_Smith}{pin factory example} cites a productivity gain of 240x.
\index{Wikipedia!\href{https://en.wikipedia.org/wiki/Business_process}{Business process}}
Similarly, the introduction of moving assembly lines in Ford's car factory produced an 8x improvement in throughput.\footnote{See the Wikipedia entry describing \href{https://en.wikipedia.org/wiki/Assembly_line\%2320th_century}{Ford's assembly line}.
\index{Wikipedia!\href{https://en.wikipedia.org/wiki/Assembly_line\%2320th_century}{Assembly line}}
} 


While social processes may work well for low-volume ad-hoc requests, formalized bureaucratic processes become necessary for recurring high-volume activities.

\subsection*{Processes Enable less Reliance on Social Relations}

The relation between bureaucratic processes and social bonds was noted by Selznick in his 1943 paper~\cite{1943_Selznick}. The needs of individual members of an organization may be at odds with (or at least not aligned with) the purpose of the organization. This dissonance is observed as the interplay between personal relationships and bureaucratic processes.

Generalizing claims about social bonds in an organization is unfounded. 
For a given complexity and given scale, there are people who are more social and less social and therefore desire more or less process.
There are bureaucrats who want documented processes and are confused as to how things are operating when those aren't present. 

The invisibility of social relationships limits bureaucrats not attuned to the importance of these bonds. 
Social influence among bureaucrats lacks transparency-- discoverability and documentation. 
In contrast, processes are easier to understand and track. The counter to social influence is the hierarchical  \iftoggle{glossaryinmargin}{\marginpar{[Glossary]}}{}
\gls{org chart}\iftoggle{haspagenumbers}{ described on page~\pageref{sec:org-chart-as-guide-and-lie}.}{.}
Social influence is not antithetical to processes. There's always a mixture of the two.


\subsection*{Processes Enable New Bureaucrats and Subjects}

Bureaucrats new to a team are likely to ask, ``What are the processes and where is the documentation?'' Bureaucrats who have been on a team for a long time say, ``Don't burden me with processes; I just need to get things done using the relationships with the people I have.'' One motive for processes is to help with onboarding team members who don't have relationships available.

Besides not having formed a social network, there's a second reason new people seek processes. New team members who recently graduated from school are used to the existence of formal processes at the high school or university level. Therefore when they join a job, they expect a similar set of conditions for processes to exist and follow.

Once this motive for processes is recognized, the relevance when onboarding new hires is apparent. New hires will need to discover the existing processes while they form social bonds. Discovering processes comes through oral folklore or written documentation. One technique to augment both discovery of processes  and the creation of relationships is to have new hires sit with experienced team members.
\marginpar{$>>$ Actionable Advice}
\index{actionable advice}
This is described in the section on  
\hyperref[sec:prisoner-exchange]{Prisoner exchange}\iftoggle{haspagenumbers}{ on page~\pageref{sec:prisoner-exchange}.}{.}
Where you sit matters to your effectiveness as a bureaucrat.

\subsection*{Processes Enable Promotion}

Processes can arise organically (bottom-up) or be created top-down. In either context, creating a new process counts as bureaucratic innovation. Process creation is confused as being worthy of promotion because of the lack of distinction between novelty and innovation. Real innovation incurs risk, whereas new processes are merely novel. Both cause change, but processes are typically designed to decrease risk. Rewarding decreased risk is reasonable, but it shouldn't be labeled as innovation.

