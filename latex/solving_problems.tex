\subsection*{How to Solve Problems}

%This list is somewhat in jest, somewhat as a source of bureaucratic patterns.

Not every bureaucrat responds to challenges the same way. Understanding the diversity of potential reactions (or inaction) before you encounter them decreases the surprise. 

% TODO: someone else at work wrote this. Get their permission
\begin{itemize}
    \item Solve the problem. This is the engineering approach.
    \item Solve a simpler problem. This is the Physics approach.
    \item Solve a more abstract problem. This is the Mathematical approach.
    \item Solve a different problem.
    \item Recognize and ignore the problem. 
    \item What problem?
    \item Assign someone else the problem.
    \item Redefine the problem so that it is already solved. ``That's not a problem; that's how it's supposed to work.''
    \item Point to the problem as a justification for more resources.
    \item Apply for resources (staff, training, money) to solve the problem.
    \item Give a talk about the problem.
    \item Write a paper about the problem.
    \item Teach a course on the problem.
    \item Talk to your therapist about the problem.
    \item Your manager declares the problem solved and gets a bonus.
    \item Reframe the problem as an opportunity. 
    \item Study the problem in the context of the broader system.
    \item Study the problem from other people's perspectives.
    \item Explore the impact of the problem on the broader system.
    \item Develop a set of activities for role-playing parts of the problem. 
    \item Solve part of the problem. Then another part.
    \item Research whether anyone else worked on the problem.
    \item Find other instances of the problem that were already solved.
    \item Find similar problems and identify the differences.
    \item Contract out the solution and select a bid.
    \item Explain why the problem does not need to be solved.
    \item Explain why the problem is beneficial. 
    \item Form a team and set a mission statement addressing the problem.
    \item Slow-roll the problem until you don't need to worry about the problem. 
\end{itemize}


Did you recognize any of those responses from your past experiences? 
%\marginpar{$>$ Reflect and Discuss}
Were any unexpected? 
Process Empathy expands beyond what you think should happen to the realm of possible responses by your fellow bureaucrats. 