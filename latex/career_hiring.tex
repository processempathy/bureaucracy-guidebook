
%See also 
% "Making of a Manager" has Chapter 7 (pages 161-187) on hiring



\section{Hiring into a Bureaucracy\label{sec:hiring}}


Hiring shapes the culture of an organization and determines what is feasible, both by the skills of those hired and how well new hires integrate with the existing organization. 

Hiring can be an expensive process, both in time and money spent by the organization to recruit and filter applicants, as well as the emotional investment of candidates. %, and the burden of reviewing applicants. 
% TODO: SO WHAT? What's the consequence?
As a candidate, recognize the risk the organization is taking. 
As a reviewer of candidates, hiring is an investment in the future of the organization. %Even if your bureaucratic role does not require an exclusive focus on hiring, there is value in studying this domain. 

%When hiring into an organization 
Organizations are not identical to the society they operate within. As a candidate you should expect a different set of norms. As a bureaucrat reviewing applicants you will see behaviors not consistent with the norms of your organization. 
Reasons for the difference in norms of society versus the norms of the organization include incentives and constraints specific to the organization. 
There is also a \href{https://en.wikipedia.org/wiki/Selection_bias}{selection bias} in
\index{Wikipedia!selection bias@\href{https://en.wikipedia.org/wiki/Selection_bias}{selection bias}}
the people who join a bureaucratic organization, and there is a selection bias to who stays in a bureaucratic environment. 
% TODO: SO WHAT? What's the consequence?
The people in a bureaucratic organization are unlikely to be perfectly representative of the society that the organization serves. Tactics that work inside a bureaucratic organization may not apply in society, and tactics that are useful outside an organization may not work on bureaucrats. 


% https://leadership.garden/onboarding-engineers/
% https://news.ycombinator.com/item?id=30810786

There are specific attributes that make a candidate more likely to be successful in a bureaucracy, regardless of the role they are being hired into. 
In addition to role-specific skills, hire candidates with the ability to reflect on 
thinking (\href{https://en.wikipedia.org/wiki/Metacognition}{metacognition}),
\index{Wikipedia!metacognition@\href{https://en.wikipedia.org/wiki/Metacognition}{metacognition}}\iftoggle{WPinmargin}{\marginpar{$>$Wikipedia: Metacognition}}{}
navigating complex social scenarios with competing interests, and intrinsic motivation are useful. If you're a candidate bureaucrat you can develop these traits. 

% https://graphthinking.blogspot.com/2021/04/screening-for-metacognition-in-job.html

% https://graphthinking.blogspot.com/2021/07/screening-for-intellectual-empathy-in.html

% https://graphthinking.blogspot.com/2021/04/questions-to-ask-interviewer-when.html

