\section{Not many Options for Organizations}
% https://graphthinking.blogspot.com/2021/07/patterns-anti-patterns-in-bureaucracy.html

When complaining about the ineptitude of organizations (or more specifically, leaders, managers, subordinates, and coworkers), consider the variables available to be modified. As a thought exercise, what would it take to rebuild the organization you are in from scratch? 
\marginpar{[Tag] Thought Exercise}

Organizations comprised of bureaucrats have fewer options for change than individual bureaucrats. The choices an individual bureaucrat faces are described in the section on 
\hyperref[sec:dilemma-trilemma]{dilemmas}
(page~\pageref{sec:dilemma-trilemma}).
In comparison, the choices faced by the designers of an organization are
\begin{itemize}
    \item Flatness of organizational hierarchy -- how many layers of oversight are there?
    \item Number of supervisors per employee. This informs the flatness and size of the organization.
    \item Organizations are segmented into teams, and there are not many options for segmentation: create a new team, merge existing teams, dissolve a team.
    \item Organizations structure recurring tasks into processes like hiring, promotion (pay or title), awards, compensation, recognition, professional training, and firing. Each of those have a set of design choices that inform the organization's culture.
\end{itemize}
In a government bureaucracy, the constraints of some incentives like pay and financial awards are set outside the organization. That further constrains the ability of bureaucrats to shape their organization's culture.

The organization may have policies and processes regarding hiring, promotion, training, and firing, but the decision may not be made by team managers rather than top-level management. 

Enumerating the choices is only relevant in the context of a desire to change. 
Individual bureaucrats can enact change, and subjects of bureaucracy can request improvement, but those are weak inputs. 
Organizations are accountable to their source of funding, not the subjects of bureaucracy or members of the organization. For example, the \href{https://en.wikipedia.org/wiki/Internal_Revenue_Service}{Internal Revenue Service (IRS)} 
\index{exemplar!Internal Revenue Service (IRS)}
is accountable to Congress, not the tax payer or IRS employees. 

\subsection*{Not many options for Teams}

Team managers might decide who gets hired and who gets promoted and who goes to what training and who gets fired. Though in some environments even that control is relegated to an external team. A team manager usually has decision making authority regarding tasks the team works on. 

Accountability in the context of teams comes from person-to-person interactions. These can be either lateral (sideways) or parent-child (top-down) or child-parent (bottom-up)~\cite{2014_Jorgensen}. Each of these three categories have associated constraints.

The upward child-parent (bottom-up) communication is either inadequate (too few updates, or not enough information, or insufficient context), relevant, or excessive. For example, a weekly or monthly report to multiple superiors may be inadequate. 

Finding the balance depends on the presenter knowing the individual audience members so that a tailored message is provided, and then adapting to the specifics of the situation. 
% tips on managing up: 
% https://svpg.com/managing-up/

The downward parent-child communication either is inadequate (no direction provided or imprecise direction provided), provides actionable vision, or micromanagement. Finding the right balance and specificity requires insight into the communication needs on both sides of the relationship. 

For lateral interactions among team members, the tension between cooperation and competition manifest in struggles over money, staffing, products (output), and resources (inputs). Here ``resources''  refers to constraints like access to data, control of data, technology resources, hardware, floor space, and expertise. 
%\begin{itemize}
%    \item money
%    \item staffing
%    \item prestige
%    \item products (output)
%    \item resources (inputs)
%    \begin{itemize}
%        \item access to or control of data
%        \item technology resources
%        \item hardware
%        \item floor space
%        \item expertise
%    \end{itemize}
%\end{itemize}


