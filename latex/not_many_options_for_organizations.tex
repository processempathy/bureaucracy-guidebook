\subsection{Not many options for organizations}
% https://graphthinking.blogspot.com/2021/07/patterns-anti-patterns-in-bureaucracy.html

Organizations of bureaucrats have fewer options than individual bureaucrats. The choices an individual bureaucrat faces are described in \S\ref{sec:dilemma_trilemma}.

At the organizational level, 
\begin{itemize}
    \item flatness of organizational hierarchy
    \item number of supervisors per employee
    \item create new team, merge teams, dissolve a team
    \item hiring process 
    \item promotion process 
    \item training process
    \item firing process
\end{itemize}
In a government bureaucracy, pay and financial incentives are typically set outside the organization.


Team managers decide who gets hired and who gets promoted and who goes to what training and who gets fired.

At the team level, interactions are either lateral (sideways) or parent-child (top-down) or child-parent (bottom-up).
\cite{2014_Jorgensen}

The child-parent bottom-up upward communication is either inadequate (too few updates, or not enough information, or insufficient context), relevant, or excessive. Finding the balance depends on the presenter knowing the individual audience members so that a tailored message is provided, and adapting to the specifics of the situation. A weekly or monthly report to multiple superiors is inadequate. 
% tips on managing up: 
% https://svpg.com/managing-up/

The parent-child downward communication either is inadequate (no direction), provides actionable vision, or micromanagement. 

For lateral interactions among teams, the tension between cooperation and competition manifest in struggles over
\begin{itemize}
    \item money
    \item staffing
    \item prestige
    \item products (output)
    \item resources (inputs)
    \begin{itemize}
        \item access to or control of data
        \item technology resources
        \item hardware
        \item floor space
    \end{itemize}
\end{itemize}
