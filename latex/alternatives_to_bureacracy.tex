\chapter[Alternatives to Bureaucracy Don't Work]{Alternatives to Bureaucracy Don't Work\label{sec:alternatives-to-bureaucracy}\iftoggle{shortsectiontitle}{\chaptermark{Alternatives to Bureaucracy}}{}}
\iftoggle{shortsectiontitle}{\chaptermark{Alternatives to Bureaucracy}}{}

% LONG1 shows up in the TOC
% LONG2 is the section title


\textit{This Appendix is academic and does not contain prescriptions for the practicing bureaucrat. This content may be of interest to the novice theoretician.}

A common refrain for participants in an organization is ``Can't I just do the work?'' with the implicit disinterest in coordination and administrivia. In this section we'll explore a few alternatives to bureaucracy. These thought experiments illustrate the na\"ive intent and the practical deficiencies of each alternative.

The models of bureaucracy below are all imprecise because they are analogies. The consequence of imprecision is wasted effort. Using a precise and accurate definition of bureaucracy enables improved effectiveness for the work you invest.


% https://graphthinking.blogspot.com/2021/02/how-to-have-efficient-bureaucracy.html

\subsection*{\textit{Alternative}: Efficient Bureaucracy}

In an ideal scenario with no inefficiency, everyone comes to the same conclusion when presented with the same information regarding management of shared resources. Efficient bureaucracy requires each person to know the skills of everyone else so that anyone can act as an expert in any field. 

In this scenario the 
\href{https://en.wikipedia.org/wiki/Overhead_(business)}{overhead cost} 
\index{Wikipedia!overhead, business@\href{https://en.wikipedia.org/wiki/Overhead_(business)}{overhead, (business)}}
of building consensus becomes unnecessary. There is no need to fight over resources (e.g., money, staffing) and no need to fight over what the direction of the organization should be. In this model of efficient bureaucracy based on everyone having the same view, a potential problem is the lack of diverse views needed to enable resilient organizations.

\ \\

While that idealized scenario is unrealistic, it points to how to improve bureaucratic efficiency. Each bureaucrat should have access to the same information (transparency). 
Each bureaucrat should apply the same decision-making process consistently (documented processes and policies).
Each bureaucrat should have the same incentives (fair policies).

The unrealistic model of efficient bureaucracy also points to why bureaucracy is inefficient in predictable ways:
\begin{itemize}
    \item Not everyone has the same information.
    \item Processes are applied inconsistently.
    \item \hyperref[sec:motivations]{Incentives vary} among bureaucrats.
    %(see section~\ref{sec:motivations}).
% https://graphthinking.blogspot.com/2017/03/what-slows-down-bureaucracy.html
    \item Bureaucrats and Subjects use imprecise language.
    \item Decision-makers use opinions and experience rather than collect data and analyze data.
    \item People prioritize putting out fires rather than attacking critical issues.
    \item People do not respond quickly (or at all) to communications (e.g., email, phone, meetings). See the section on \hyperref[sec:slowing-communication]{slow communication}.
    \marginpar{See page~\pageref{sec:slowing-communication}.}
    \item Participants are late to meetings.
    \item \hyperref[sec:scope-creep]{Scope creep} 
    \marginpar{See page~\pageref{sec:scope-creep}.}%
    for projects incurs unnecessary work.
    \item Planned scope and actual scope are mismatched  (due to staffing or skills).
    \item Teams work in silos and create redundancy.
% https://graphthinking.blogspot.com/2017/03/progress-in-spite-of-better-ways.html
    \item Bureaucrats and Subjects lie and employ other dark patterns (not addressed in this book).
    \item People make mistakes.
    \item Each person's reference experiences are unique.
\end{itemize}
By identifying why bureaucracy is inefficient you can actively work to remedy each of the above aspects. 

\ \\

Consider the following 
\href{https://en.wikipedia.org/wiki/Thought_experiment}{thought experiment}. 
\index{Wikipedia!thought experiment@\href{https://en.wikipedia.org/wiki/Thought_experiment}{thought experiment}}\iftoggle{WPinmargin}{\marginpar{$>$Wikipedia: thought experiment}}{}
%What if everybody in a bureaucracy were the same?
What if everybody in a bureaucracy had a different opinion? How would consensus be arrived at? Can an organization operate without having to agree on every decision? 
\marginpar{$>>$ Thought Exercise}



\subsection*{\textit{Alternative}: Perfect Bureaucracy}

The definition of ``perfect bureaucracy'' depends on who provides the perspective. Here I illustrate both the subject's view and the bureaucrat's view. 

From the perspective of the subject of a bureaucratic organization, perfect bureaucracy means many things: minimizing the time a subject waits on a decision, getting perfectly correct decisions, decisions being consistent across subjects (Dilemma~\ref{table:dilemma-subject-consistency-per-situation})
\marginpar{See page~\pageref{table:dilemma-subject-consistency-per-situation}.}%
and circumstances (Dilemma~\ref{table:dilemma-personal-policy-consistency-across-cases}), decisions take into account all relevant factors, and there is zero cost to the subject (Dilemma~\ref{table:dilemma-subject-transparency}). 
\marginpar{See page~\pageref{table:dilemma-subject-transparency}.}%
Any situation deviating from those expectations is a noticeable detriment to the subject, leading to a negative reference experience of bureaucracy. 

Never mind that the desires are unreasonable and conflicting. In a real (i.e., imperfect) bureaucracy, subjects have a negative or neutral experience. Positive interactions with bureaucracy are rare and are regarded as abnormal.

Perfect bureaucracy for a bureaucrat means all information is available, information is immediately available, and there is no moral ambiguity (i.e., each answer is objective). Trade-offs become trivial and the emotional toll of the work goes to zero. Then the bureaucrat can serve subjects (an emotionally rewarding prospect). 

This job in a perfect bureaucracy might feel hollow if all decisions are obvious. In a real (imperfect) bureaucracy, the typical experience is negative or neutral, punctuated by glimpses of satisfaction. 

\subsection*{\textit{Alternative}: Everyone does their own thing -- No Coordination, No Bureaucracy}
The scenario involving minimal bureaucracy is a single person working on a single task that does not last long (a few minutes), is relatively easy (cognitively, physically, and emotionally), and does not recur. In that situation, building consensus is irrelevant and no process is required. Even then, it is often the case that this simple task involves using shared limited resources --  essentials like water, air, and land. If your task involves use of those things, then how is fair use determined among consumers?

For simplistic tasks the concept of community-imposed limits to access 
\iftoggle{glossarysubstitutionworks}{\glspl{shared resource}}{shared resources}
\iftoggle{glossaryinmargin}{\marginpar{[Glossary]}}{} is not trivial. When there are no policies to constrain access, violence is used to determine allocation of shared resources.

Most of your actions occur beyond the limits of simplicity and thus incur some concept of \gls{process}
\iftoggle{glossaryinmargin}{\marginpar{[Glossary]}}{}%
(breaking a task into subtasks). Staying with the context of one person, a complex task can benefit from being broken into subtasks. Sometimes the order of the subtasks matters, so we need to track the dependencies. A recurring multi-step process with documentation is starting to have features of bureaucracy but lacks the need for consensus. 


If one person lacks the skills relevant to a multi-step process, they may engage another person to help. The interaction occurs on a spectrum from informal (anarchy) to formalized in a contract (\href{https://en.wikipedia.org/wiki/Libertarianism}{libertarian}).
\index{Wikipedia!libertarian@\href{https://en.wikipedia.org/wiki/Libertarianism}{libertarian}}\iftoggle{WPinmargin}{\marginpar{$>$Wikipedia: libertarian}}{}
If the parties working on the task fail to reach a consensus, what is the recourse? Choices include physical violence, threats, or involving a third party (e.g., a court with lawyers and judges). 


If a community wants to manage shared resources, how is that policy decided?  Building consensus is relevant, but what is the process for establishing consensus? 
\href{https://en.wikipedia.org/wiki/Nepotism}{Nepotism},
\index{Wikipedia!nepotism@\href{https://en.wikipedia.org/wiki/Nepotism}{nepotism}}\iftoggle{WPinmargin}{\marginpar{$>$Wikipedia: Nepotism}}{}
cultural norms, and 
\href{https://en.wikipedia.org/wiki/Religion}{religious practices}
\index{Wikipedia!religion@\href{https://en.wikipedia.org/wiki/Religion}{religion}}
served this role before the dominance of bureaucracy. 


\subsection*{\textit{Alternative}: Limit Bureaucracy to a Single Decider\label{sec:single-decider}}

Since bureaucracy is distributed knowledge and distributed decision-making, it could be replaced by centralized knowledge and centralized decision-making. If we are going to live in a society and coordinate shared resources, what if we had a single person deciding? How fast could that be? Good decisions are not instantaneous. To understand latency in decision making let's use the
\href{https://en.wikipedia.org/wiki/OODA_loop}{OODA loop} 
\index{Wikipedia!OODA loop@\href{https://en.wikipedia.org/wiki/OODA_loop}{OODA loop}}\iftoggle{WPinmargin}{\marginpar{$>$Wikipedia: OODA loop}}{}
model. 

OODA stands for Observe, Orient, Decide, Act. OODA applies to individuals as well as teams and organizations. 
% https://graphthinking.blogspot.com/2016/03/how-to-evolve-organization-community-or.html
The input for the OODA model is to observe and the output is to act. Observing and acting are measurable; the ``orienting" and ``deciding" phases are not as easy to measure. The ``orient" phase requires labeling data received in the ``observe" phase and connecting that information to what is already known.

How quickly could a single decision-maker apply the OODA loop for arbitrary questions about \iftoggle{glossarysubstitutionworks}{\glspl{shared resource}}{shared resources} accessed by a community? Three minutes is not much time to gather information and share it with the relevant people, but let's set that as a lower bound.
\marginpar{$>>$ Math}
If a single decision takes three minutes, then in a ten-hour work day that's a max of $(10*60)/3 = 200$ decisions per day. If this decider works 300 days out of the year, that gets us to 60,000 decisions per year. While 60,000 decisions sound like a lot, that limits how large the community could feasibly be, and the complexity of the decisions is limited. 

Everyone has to talk to this decider directly since there's no bureaucracy to gather or share the information. The diversity of questions regarding shared resources would be challenging to answer well.

Within the constraint of a single decider, we can't just automate everything because carrying out the automation requires staff to enact. 
\index{automation!challenges of}
Automation isn't free -- it requires creation and maintenance. Unless the same person making the decisions is also creating and maintaining the system, there will be multiple people in an organization.


\href{https://en.wikipedia.org/wiki/Monarchy}{Monarchies} 
\index{Wikipedia!monarchy@\href{https://en.wikipedia.org/wiki/Monarchy}{Monarchy}}\iftoggle{WPinmargin}{\marginpar{$>$Wikipedia: Monarchy}}{} 
and 
\href{https://en.wikipedia.org/wiki/Dictator}{dictatorships}
\index{Wikipedia!dictator@\href{https://en.wikipedia.org/wiki/Dictator}{dictator}}
at first glance seem to rely on a single decider. This is a simple model to understand, but addressing all the edge cases for a large society is difficult. A single decider doesn't scale for the number of decisions needed, so the decider then appoints bureaucrats to subjectively enforce policies. 

From this thought experiment we conclude that avoiding bureaucracy by relying on one person doesn't scale. More people are needed to develop and carry out policies. 


%\subsection*{Complexities of more than one Bureaucrat}

% https://graphthinking.blogspot.com/2019/07/first-principles-analysis-of-creating.html

If there's more than one person in an organization, then communication for coordination takes time. That's the ``orient'' phase of the OODA loop. Time spent orienting decreases the decision throughput of the organization.



\subsection*{\textit{Alternative}: Avoid Bureaucracy and Just use Common Sense}
\textit{Claim}: Everything would go smoothly if each person used common sense.
\marginpar{$>>$ Fallacy}

Common sense relies on your reference experiences, cultural norms, incentives, emotional state, and a lack of psychological defects. These aspects are particular to each person in an organization. This gives rise to the observation that 
``common sense is not so common." 
\marginpar{$>>$ Folk Wisdom}
\index{folk wisdom!common sense is not so common}

There are multiple origins of what gets considered to be common sense:
\begin{itemize}
\item Cultural norms. This is what I think everyone else does or thinks.
\item Personal \gls{reference experience}. \iftoggle{glossaryinmargin}{\marginpar{[Glossary]}}{} I've done it this way before.
\item A prescribed action seems obvious.
\end{itemize}

\ \\

Proponents of commonsense who work in existing hierarchical bureaucracies may advocate the removal of non-workers (i.e., management). That may sound like a worker's paradise, but then who coordinates activities when the number of people involved is more than any one member can track (above 
\href{https://en.wikipedia.org/wiki/Dunbar\%27s_number}{Dunbar's number} of about 150 people)?
\index{Wikipedia!Dunbar's number@\href{https://en.wikipedia.org/wiki/Dunbar\%27s_number}{Dunbar's number}}\iftoggle{WPinmargin}{\marginpar{$>$Wikipedia: Dunbar's number}}{}

\ \\

To illustrate the dissonance motivating common sense, consider the following. 
Changing a burnt-out light bulb at home takes me only a few minutes. I've done that many times. Why is that task so hard in an organization?

%\marginpar{$>>$ Story Time}
\index{story time!changing a lightbulb}
%\begin{storytime}{Changing a Lightbulb}
\begin{mdframed}[frametitle={Changing a Lightbulb},frametitlerule=true,frametitlealignment=\centering]
In a large organization comprised of specialized roles, an office worker sees a bulb is out. Rather than go to a nearby hardware store and buy a replacement, they notify their manager who notifies the building supervisor who submits a request to the maintenance team. 

The maintenance service desk team then schedules the repair. An order of 1000 replacement bulbs was made last year and there are some still available in storage. A maintenance team is assigned to the task and deployed to replace the bulb. 

First the maintenance team goes to the storage facility to get a ladder and multiple bulbs of multiple models. The team has to sign out the bulbs from storage so inventory can be tracked (because of prior incidents of theft). A team is needed because solo use of the ladder is prohibited (for safety, also born out of previous incidents). Multiple bulb models are needed because which model is required is unspecified in the service ticket. Having multiple bulbs available decreases the need to go back to storage. 

Once on site, the maintenance staff finds the bulb is a new type and needs to be ordered. Maintenance team notifies the supervisor. The building manager files a new maintenance request.
%\end{storytime}
\end{mdframed}

This is the efficiency of specialization of roles. If you don't like it, you could go to the hardware store, buy a bulb, rent a ladder, install the bulb, and return the ladder.

% https://graphthinking.blogspot.com/2021/09/why-is-everything-so-hard-in-large.html

\subsection*{\textit{Alternative}: Completely Avoid Bureaucracy}
The phrasing of avoidance is more precisely worded by replacing ``bureaucracy" with ``coordination of stakeholders." If you avoid coordination of stakeholders, you either are constrained to only work on tasks that involve one person, or you get random (uncoordinated) interactions. 

\subsection*{\textit{Alternative}: Minimize Bureaucracy}
Again, try replacing ``bureaucracy" with ``coordination of stakeholders." The goal of ``minimizing coordination" probably isn't the real objective. To be more precise, a specific objective might be ``minimize time spent executing the task" (which takes a lot of coordination before the task execution) or ``minimize the level of distraction to stakeholders" (chunk the coordination time, e.g., a meeting). Another strategy for minimizing bureaucracy is to reduce the number of stakeholders involved. For a given task complexity, this means having smarter people who have more skills. See figure~\ref{fig:complexity-and-size} for a quantitative illustration of the trade-off. 


\begin{figure}
\centering
\includegraphics[width=0.9\textwidth]{images/people-per-task-for-skill-level.pdf}
\caption{A task that one smart person can do might take two  people that are not as smart. This concept applies to any task size and any population of workers. In this diagram three levels of task complexity are shown. As task complexity increases, the size of the team needs to grow with intelligence held constant. The growth may be less if the team members are brilliant. Those brilliant people cost more and there are fewer of them available.}
\label{fig:complexity-and-size}
\end{figure}


\subsection*{\textit{Alternative}: Automation of Processes to Displace Bureaucracy\label{sec:automation}}

The role of automation is to make interactions more predictable, faster, and to handle more of them. Automation does not eliminate bureaucracy; automation obfuscates  subjective decisions and limits the ability to negotiate with decision-makers.
%Automation and computers merely obfuscate processes and make negotiation more challenging. 

Hoping that modern technology will eliminate or reduce bureaucracy is not a shortcut to progress. 
Even with faster decisions and fewer humans, there is still reliance on humans to make decisions and design processes.

% https://graphthinking.blogspot.com/2019/07/bureaucracy-is-social-process-executed.html
There are benefits to automation, and automation can be enacted to minimize harm to bureaucratic subjects. Automated systems can be made more transparent by including documentation about what is happening, why, and what's next.
\index{automation!benefits of}
\marginpar{$>>$ Actionable Advice}%
\index{actionable advice}%
Documentation helps  bureaucrats and subjects identify when assumptions made in the automation are incorrect. 


There are indicators of when to transition from a manual bureaucratic process to a more automated approach:\footnote{\href{https://xkcd.com/1319/}{https://xkcd.com/1319/} -- the risks of investing in automation.}
% https://graphthinking.blogspot.com/2019/07/cost-of-creating-and-supporting.html
\index{automation!when to}
\index{xkcd!xkcd.com/1319@\href{https://xkcd.com/1319/}{1319}}
\begin{itemize}
    \item The process to be automated is stable.
    \item The expected lifespan for the process -- how long the process will be needed -- is  longer than the time to enact the automation.
\item The process logic relies on objective evaluation criteria.  
\item The process is frequent.
\item The cost of automating (both the initial creation cost and the maintenance cost) is a savings over the manual implementation.
\end{itemize}


\subsection*{\textit{Alternative}: Market-based Approach}
% https://graphthinking.blogspot.com/2017/09/market-friction-and-bureaucratic.html

There is an alternative to bureaucracy that features a decentralized approach to complex tasks and avoids reliance on consensus: \href{https://en.wikipedia.org/wiki/Market_(economics)}{markets}.
\index{Wikipedia!market@\href{https://en.wikipedia.org/wiki/Market_(economics)}{markets}}

An oversimplified definition~\cite{2023_Kenton} is ``A market is a medium that allows buyers and sellers of a specific good or service to interact to facilitate an exchange." 
In practice, there is market friction~\cite{2021_Downey, 2011_Matouschek}: ``trading is always associated with certain costs or restraints."

The sources of friction in a market include
\begin{itemize}
    \item Commissions on trades.
    \item Taxes (needed to fund contract enforcement organizations).
    \item Uncertainty in hiring.
    \item Constraints on firing labor.
    \item Search cost in exploring the available opportunities.
    \item Risk uncertainty.
    \item Cost of creating contracts.
    \item Cost of insurance.
\end{itemize}
Market friction and bureaucratic inefficiency are similar~\cite{1937_Coase, 2010_economist}.


Complex tasks at a societal scale range from manufacturing advanced computer chips, to creating and distributing critical medicine, to immigration and border enforcement. 
If a society wants to carry out a complex task involving multiple people, then coordination of effort is required. The coordination can be explicit (in the form of a bureaucracy) or implicit (via a market).  Bureaucracy is one response to the complexity of a problem being solved.


\ \\

Regardless of whether a bureaucratic or market-based approach is used to mediate access to \iftoggle{glossarysubstitutionworks}{\glspl{shared resource}, }{shared resources, }%
% https://graphthinking.blogspot.com/2017/09/market-friction-and-bureaucratic.html
distributed knowledge and distributed decision-making are hindered by aspects like
%\begin{itemize}
limited bandwidth between people participating in the coordination,
non-zero latency of information between people,
the cost of getting data,
and
the cost of analysis of data.
%\end{itemize}
Due to these factors, suboptimal decisions get made. See  %section~\ref{sec:dilemma-trilemma} 
the discussion of 
\hyperref[sec:dilemma-trilemma]{dilemmas}
\marginpar{See page~\pageref{sec:dilemma-trilemma}.}%
for specific examples of trade-offs.

\ \\

The objective and quantified concept of money creates accountability that distinguishes commercial businesses from bureaucracy. Money as a metric is common to all participants within the business and with external stakeholders. 

As with government bureaucrats, commercial businesses have employees who make subjective decisions and enforce policies. Unlike government bureaucracy, in business there is a common metric for feedback: money. This distinction between business and government is not as clear as you might expect since the feedback mechanism does not apply to all members of a business. An effective commercial bureaucrat may rely on the success of other employees rather than a direct interaction with customers. Business employees may take action that is hard to quantify with respect to profits and losses.

Companies are motivated by financial profit, whereas bureaucracies like prisons, schools, hospitals, governments, and militaries  consume and spend money, but money isn't the goal. When faced with a decision, bureaucrats are not guided by which will generate more profit~\cite{2012_Wilson}.

\subsection*{\textit{Alternative}: Adhocracy instead of Bureaucracy}
% https://graphthinking.blogspot.com/2017/09/complex-tasks-necessitate-bureaucracies.html

\href{https://en.wikipedia.org/wiki/Adhocracy}{Adhocracy} 
\index{Wikipedia!adhocracy@\href{https://en.wikipedia.org/wiki/Adhocracy}{adhocracy}}%
\iftoggle{WPinmargin}{\marginpar{$>$Wikipedia: Adhocracy}}{}% 
(also called Tiger Teams)
\index{Wikipedia!tiger team@\href{https://en.wikipedia.org/wiki/Tiger_team}{Tiger team}}
has been proposed in reaction to the prevalence of bureaucracy in organizations. To enact Adhocracy a team of diverse experts is assembled to tackle a complex challenge of limited duration.
While this may be enough for short-duration tasks, if the challenge lasts more than a day or two there will be new issues beyond the original challenge:

\begin{itemize}
    \item What happens to the work that was previously being done by the members of this team?
    \item Who pays the salary for this labor?
    \item Who calculates the payroll?
    \item Who pays the rent for facilities used by the team?
    \item Who does the maintenance of facilities and equipment?
    \item Who cleans the facility?
    \item Does risk incurred mean insurance is needed?
\end{itemize}
To address these questions that are out-of-scope for the complex challenge, you can either use a market-based approach or build a bureaucracy. 

\subsection*{\textit{Alternative}: Consensus Algorithms}

In the description of a \hyperref[sec:single-decider]{single decider}
\marginpar{See page~\pageref{sec:single-decider}.}%
%section~\ref{sec:single-decider} 
the solution was to centralize decision-making and knowledge. The decentralized approaches are market-based or bureaucratic. Another approach to distributed decision-making is to use consensus algorithms. Modern algorithms like
\href{https://en.wikipedia.org/wiki/Paxos_(computer_science)}{Paxos} and
\index{Wikipedia!Paxos@\href{https://en.wikipedia.org/wiki/Paxos_(computer_science)}{Paxos algorithm}}%
\iftoggle{WPinmargin}{\marginpar{$>$Wikipedia: Paxos}}{}
\href{https://en.wikipedia.org/wiki/Raft_(algorithm)}{Raft}%
\index{Wikipedia!Raft@\href{https://en.wikipedia.org/wiki/Raft_(algorithm)}{Raft algorithm}}%
\footnote{For a tutorial on Raft see \href{https://raft.github.io/}{https://raft.github.io/}.} are widely used for various computer-based tasks. 

Although these algorithms are reliable even when the underlying mechanisms are faulty, the algorithms are not enacted using humans. Humans are unreliable, but more importantly humans game the rules and processes of systems rather than operate within the constraints. Also, relying on an algorithm neglects the feature that humans can adapt to unforeseen circumstances.

\subsection*{\textit{Alternative}: Elected Representatives}
Governments are composed of politicians and bureaucrats. (Government isn't the only place bureaucrats appear, but for this section we'll focus on government.)

The concept of political representatives is easier to understand. A politician is just one person acting on behalf of other people. Members of the community get a vote in who that representative is.
In contrast, the emergent behavior of bureaucracy is more challenging to understand: many people are involved (which inhibits creation of an explanatory narrative) and subjects of bureaucracy do not appoint the bureaucrats. 

% why not make the entire system out of politicians?
Suppose that instead of a bureaucracy all members of an organization charged with 
managing \iftoggle{glossarysubstitutionworks}{\glspl{shared resource}}{shared resources} were 
elected rather than being selected for their technical skills. This scenario eliminates one of \href{https://en.wikipedia.org/wiki/Bureaucracy\%23Max_Weber}{Weber's characteristics of bureaucracy}
\index{Wikipedia!bureaucracy@\href{https://en.wikipedia.org/wiki/Bureaucracy}{bureaucracy}}
-- competence for job appointments. 

In the United States there are bureaucratic positions that feature a mix of election versus appointment. For example, the method of selecting judges varies widely by state~\cite{Ballotpedia_judicial_selection}. In a 2017 survey, 63\% of more than 1000 judges favored appointment over being elected~\cite{2017_Johnson}.


Attorney Generals in the United States are similarly  selected by election and appointment~\cite{2022_Ballotpedia}. Positions that benefit from expertise (e.g., Attorney Generals, Coroners) sometimes lack minimum qualifications when selected by election. The more positions there are to vote on, the more challenging it is to have an informed electorate capable of selecting competing candidates.

\subsection*{\textit{Alternative}: Small Organizations}

Suppose you try to limit bureaucracy by imposing the constraint that teams or organizations be small.

Given three people in a team, options include:
\begin{itemize}
    \item Split the labor among the participants; each has the same workload and same tasks. Use consensus decision-making. Do not exploit skills unique to any member. (That last concept is an inefficiency.)
    \item Split the labor by specialization; each person becomes dependent on the other. Specialization enabled can improve effectiveness but also incurs coordination which decreases throughput.
    \item Make one person the manager to oversee the other two -- impose a hierarchy. This is a specific  specialization where one person is not directly involved in labor central to the purpose of the team.
\end{itemize}
Organizing members into teams (teams of teams) introduces new levels of coordination and competition.

\subsection*{\textit{Alternative}: Minimize Bureaucracy by Eliminating Guardrails}
%\subsection*{Defending against Malicious Participants}

Suppose  you are part of an organization that doesn't have oversight processes for finances and other resources your organization is responsible for. A small percentage of the population, say 1\%, will take advantage of the lack of oversight for their own gain. Those malicious actors can be either subjects of bureaucracy or bureaucrats within the organization. 
In his book \textit{Liars and Outliers} \href{https://en.wikipedia.org/wiki/Bruce_Schneier}{Bruce Schneier}
\index{Wikipedia!Schneier, Bruce@\href{https://en.wikipedia.org/wiki/Bruce_Schneier}{Schneier, Bruce}}\iftoggle{WPinmargin}{\marginpar{$>$Wikipedia: Bruce Schneier}}{}
calls these people defectors~\cite{2012_Schneier}.

Similarly, suppose a small percentage of the population, say 1\%, is stupid and makes mistakes. Without a review process, these mistakes will negatively impact your organization's finances and the resources your organization manages. Preventing, detecting, and correcting mistakes can be costly investments. 


% 2023-12-23:
% PREVIOUSLY THIS IS WHERE CHESTERTON'S FENCE WENT
% PRIOR TO THIS SECTION BEING REMOVED 

