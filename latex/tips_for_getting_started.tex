\section{Tips for Getting Started}

% https://graphthinking.blogspot.com/2019/11/formal-education-for-domain-exposure.html
You may have graduated from school with a degree in a specific subject. Then you go to work and expect the position you're in to apply the domain-specific education you have in your new role. Your work produces an outcome and gives meaning to the struggle of your prior education process (an investment of time, money, and effort). In practice, there can be a disconnect between your professional role and your educational training. The disconnect may feel dissatisfying when you are not practicing what you went to school for.

An alternative framing that is more helpful is that you went to school to learn how to learn. A degree represents your ability to learn and persist rather than a domain specialization. Then, when you get to work, you are presented with new opportunities to learn other domains. 

\subsection*{Onboarding into a Bureaucratic Organization}

The hiring process involves at least two people. Responsibility for successful integration relies on both people actively working together. This cooperative effort manifests in a few specific tactics. 

\ \\
%\begin{minipage}{\textwidth}
As the new hire, try to learn the names of those around you. %\\
\marginpar{$>>$ Actionable Advice}%
\index{actionable advice}%
As the hirer, make introductions. Not just the first time, but for the first week or two.
%\end{minipage}

\ \\
%\begin{minipage}{\textwidth}
As the new hire, learn the jargon of your new organization. %\\
As the hirer, explicitly expand jargon when used.
%\end{minipage}

\ \\
%\begin{minipage}{\textwidth}
As the new hire, seek one-on-one meetings to get feedback. %\\
As the hirer, offer one-on-one meetings to provide feedback.
%\end{minipage}

\ \\
%\begin{minipage}{\textwidth}
As the new hire, write documentation on processes as you experience them. You're in the best position to write down your observations because you are seeing things for the first time. %\\
As the hirer, provide documentation on processes. Your organization gets value from new hires faster when they are efficiently trained. 
%\end{minipage}

\subsection*{How to Orient Yourself}

Sometimes your role as a bureaucrat may be poorly defined. The lack of direction or objectives may feel disappointing or frustrating. This is especially the true for your first job having just left the structured environment of school. The positive perspective is that you can define your tasks and determine what would be helpful or interesting. 

Two mentalities that are useful throughout your career are enabling other people through collaboration and working yourself out of the job. Your goal is to facilitate success rather than being integral to the process. Enabling other people to be successful can apply to subjects, fellow bureaucrats, or management. Not everyone matters equally, so figuring out the priority of helping specific people informs your decision. 

Another tactic is to look around, look backward, and look forward. What are the challenges the team or organization faces? Then review previous attempts to address the challenges. Why did previous efforts fail to remedy the situation? Once you have a less na\"ive view and more context, what is your vision of the desired scenario? Without this vision for improvement, there is a danger of doing work that keeps you busy but yields no progress. 

The tactics of identifying challenges and learning the history of a challenge often rely on social interactions. A bureaucratic organization rarely maintains a searchable written record of decisions and failed efforts. If that content is available, it may lack the specificity available from a verbal narrative by people who were present. When investigating issues in a bureaucracy, end discovery-oriented conversations with the question, ``Who else should I talk to about this issue?'' 
\marginpar{$>>$ Actionable Advice}%
\index{actionable advice}%
Independent of the prior content or the quality of the preceding discussion, this question can be the entry point to an extensive second-order social network of expertise that is often separate from the hierarchical chain of command. If multiple  discussions with different people lead to the same person, then that person is key. 

Documenting your process as you proceed can help with self-reflection and your ability to measure progress. Written documentation of your status can be used as proactive updates to your management and provides a record when someone asks  why you took certain actions (rather than coming up with after-the-event rationalizations).

Bureaucratic challenges that do not violate the constraints of Nature (such as conservation of momentum, conservation of mass, the speed of light, \href{https://en.wikipedia.org/wiki/Uncertainty_principle}{Heisenberg's Uncertainty principle}) 
\index{Wikipedia!Heisenberg's Uncertainty principle@\href{https://en.wikipedia.org/wiki/Uncertainty_principle}{Heisenberg's Uncertainty principle}}
are solvable. There are typically political, social, personality, security, and technical aspects that need to be accounted for. Laws can be changed, regulations altered, people replaced, and norms updated. Constraints that appear to be logically inconsistent may be reframed to be resolved. Narrowing or broadening the scope may reset the context for a challenge.

\ \\

% TRANSITION

The tactics provided above can help you be an effective bureaucrat. Because there is variability to effectiveness there can be emotional highs and lows. The next section discusses the emotional aspect of bureaucracy.
