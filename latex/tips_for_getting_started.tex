\section{Tips for Getting Started}

% https://graphthinking.blogspot.com/2019/11/formal-education-for-domain-exposure.html
You may have graduated from school with a degree in a specific subject. Then you go to work and expect the position you're in to apply the domain-specific education you have at work. That work produces an outcome and gives meaning to the struggle of educational process (time, money, effort). When you are not allowed to practice what you went to school for the disconnect may feel dissatisfying.

An alternative attitude is that you went to school to learn how to learn. A degree represents your ability to learn and persist, rather than a domain specialization. Then, when you get to work, you are presented with new opportunities to learn other domains. 

\subsection*{Onboarding into a Bureaucratic Organization}

There are two parties in hiring. Responsibility for successful integration relies on both parties actively working together.

\ \\
As the new hire, make an effort to learn the names of those around you.\\
As the hirer, make introductions. Not just the first time, but for the first week or two.

\ \\
As the new hire, learn the jargon.\\
As the hirer, explicitly expand jargon.

\ \\
As the new hire, seek one-on-ones.\\
As the hirer, offer one-on-ones.

\ \\
As the new hire, write documentation on processes.\\
As the hirer, provide documentation on processes.

\subsection*{How to Orient Yourself}

Sometimes your role as a bureaucrat may be poorly defined. The lack of direction or objectives may be feel disappointing or frustrating. The positive perspective is that you get to define your tasks and determine what would be useful or interesting. 

Two mentalities that I find useful are enabling others and working myself out of the job. My goal is to facilitate success rather than being integral to the process. Enabling other people to be successful can apply to customers, fellow bureaucrats, or management. Not everyone matters equally, so figuring out the relevance of helping specific people impacts your decision. 

Another tactic I use is to look around, look backwards, and look forward. What are the challenges my team or my organization faces? Then review previous attempts to address the challenges. Why did previous efforts fail to remedy the situation? Once I have a less na\"ive view and more context, what is my vision of the desired scenario? Without this vision there is a danger of doing work that keeps me busy but yields no progress. 

The tactics of identifying challenges and learning the history of a challenge often relies on social interactions. A bureaucratic organization rarely maintains a searchable written record of decisions and failed efforts. If that content is available, it may lack the specificity available from oral narrative by people who were present. When investigating issues in a bureaucracy, I end discovery-oriented conversations with the question, ``Who else should I talk to about this issue?'' Independent of the content or the quality of the content, this question is the entry point to a large second-order social network of expertise that is often separate from the hierarchical chain of command. If multiple separate discussions lead to the same person, then that person is key. 

Documenting your process as you proceed helps both your self-reflection and your ability to measure progress. Written documentation of your status can be used as proactive updates to your management and provides a record when someone asks retroactively why you took certain actions.

Bureaucratic challenges that do not violate the constraints of Nature (such as conservation of momentum, conservation of mass, speed of light, the \href{https://en.wikipedia.org/wiki/Uncertainty_principle}{Heisenberg Uncertainty principle}) 
\index{Wikipedia!\href{https://en.wikipedia.org/wiki/Uncertainty_principle}{Heisenberg Uncertainty principle}}
are solvable. There are typically political, social, personality, security, and technical aspects that need to be accounted for. Laws can be changed, regulations altered, people replaced, and norms updated. Constraints that appear to be logically inconsistent may be re-framed to be resolved. Narrowing or broadening the scope may reset the context for a challenge.



