When a person has a positive experience engaging with bureaucracy, positive attribution is made to the people involved. Or ease of a solution makes the bureaucracy less visible and the solution seems obvious. 

When a person has a negative experience with bureaucracy, complaints are about the incompetence of the people involved, or the incomprehensibleness of the system. Don't these bureaucrats know how to do their job? Why isn't the solution obvious? Why does this system not work for me?

% Who this book is for

% from https://graphthinking.blogspot.com/2021/07/bureaucracy-book-outline.html
This book is for you if you are curious about bureaucracies, or you are thinking about working as a bureaucrat, or you are employed as a bureaucrat, or your job is shifting to be more bureaucratic. If you don't think of yourself as a bureaucrat, or if the term bureaucrat has negative connotations, I hope to change your mind on this vital topic. 


% What you should expect reading this book: 
The purpose of this book is to decrease surprise and arm you (both emotionally and intellectually) for the toil of being a bureaucrat. 

This book does not have a narrow focus on one topic like leadership, managing a team, being a team member, planning, time management, project management, advancing your career, or self-improvement. Some lessons may apply in those domains.

% What is the benefit of reading this book?
As a result of reading this book, you will be better able to recognize and navigate complex professional environments, both within your career and outside of work. The perspectives offered in this book can benefit you directly, whether by promotion of title or increase in pay; successful completion of a project; or through decreased stress of understanding how the world works.

There's harm in not recognizing yourself as a bureaucrat, as the role and responsibilities are distinct

Automation and computers will not eliminate or decrease bureaucracy. They merely obfuscate the processes and make negotiation more challenging. 

% my experience
% I wrote this book for a younger version of me.
 I was sufficiently self-aware when I first started my job in a large organization to recognize I didn't know much about working in that environment. Over the years I learned from my mistakes by reflecting on my (in)actions and the consequences. This approach has been an expensive education: mistakes delay progress and damage relationships.


% Caveats
Simplifying to "this interaction is characterized merely as human relations" is an easier perspective. However, that misses emergent phenomena. 

There's a risk of overanalysis. Sometimes a pipe is just a pipe. Avoiding conjecture about conspiracy and malice is a difficult boundary when insufficient information is available. 

I recognized the importance of navigating bureaucracy early in my career, but the insights were most clear when I entered middle management. 

My experiences cannot be generalized to every situation. Some of the observations here may be analogous to your context if you squint hard. 

Nothing in this book is domain specific, nothing is tied to engineering of products, and nothing is applicable solely in science research or policy development. While this material is intended to be timeless and generic, it is culturally specific to the USA. As a privileged white male, I did not encounter systemic hurdles in my career so there are blindspots not addressed in this book. 

% ? 
There are no alternatives to bureaucracy, so gaining skills in navigating bureaucracy are helpful. 

% Source of this content: 
This material is based on personal experience, reading published materials, and anecdotes from other people. No surveys were taken to support the claims made. No double blind experiments were conducted. 

% How the book should be read: 
Reading this book front-to-back is feasible. Each section is intended to be stand-alone. The book is intended to spark contemplation. 


% as per https://tex.stackexchange.com/q/393238/235813
\begin{flushright}
Ben Payne\\
\today\\
USA
\end{flushright}


