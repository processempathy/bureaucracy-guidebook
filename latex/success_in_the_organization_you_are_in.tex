\section[Success (and failure) in Your Organization]{Success (and failure) in Your Organization\label{sec:org-failure-and-success}\iftoggle{shortsectiontitle}{\sectionmark{Success and Failure}}{}}
\iftoggle{shortsectiontitle}{\sectionmark{Success and Failure}}{}

% LONG1 shows up in the TOC
% LONG2 is the section title

This section on how a team or organization conceptualizes failure is separate from %section~\ref{sec:learn-from-failure}
\hyperref[sec:learn-from-failure]{learning from failure} 
\marginpar{See page~\pageref{sec:learn-from-failure}.}%
as an individual bureaucrat. For your individual success you need to align with how your team defines success. 

Understanding the ways your organization (or team) succeeds or fails matters to you because it affects how you are evaluated and how you feel about your work. Success and failure should be measured against a quantitative objective that was agreed upon in advance by all stakeholders. While theoretically feasible, some organizations do not enact these agreements.


Bureaucratic organizations have varying definitions of success and failure because organizations are made up of individual bureaucrats with a multitude of motives and weak feedback mechanisms. There are a few macro-level patterns for organizations. Each pattern has a consequence on the individual bureaucrats operating within the organization. 

\ \\

A bureaucratic organization that provides infrastructure service (such as a public water utility, an electrical power company, or an organization's internal computer support) has a specific failure mode. When infrastructure is unavailable or degraded, that's a failure. Success means satisfactory operation at a minimal cost. An organization operating in this context is motivated to avoid failure. Success is merely a lack of failure, or an improved efficiency. An effort to improve processes or otherwise innovate will face the inertia of a culture organized around avoiding failure. 

\ \\

A bureaucratic organization that claims ``wins'' but no failures (because either more or different work is needed) is difficult to be productive in when the successful outcome has ill-defined value. The success can be counted (number of wins), but the relative importance of the success is unclear.

Examples of win-focused organizations include law enforcement, research teams, and teams with a responsibility for innovation. 

For an organization focused on ``wins,'' funding depends on how convincing these success narratives are. This biases the organization to focus on investments most likely to get more funding. When the funding for an organization does not come from consumers of products or policies created by the organization, there is a disconnect. The ``wins'' may not be helpful to the consumer.


\ \\

An organization that works to improve the productivity of other organizations may have neither successes nor failures. Money and time are invested with the hope that results are useful to someone else at some future time. If the organization didn't exist no outcome would be significantly affected. Any win might be attributable to either organization, and any failure can be attributed to either organization. As with no-failure organizations, success is focused on anecdotes and narratives. 

\ \\

Your ability to be effective requires identifying which of the above models your organization is described be. The foundation of your Process Empathy relies on recognizing the incentives of the organization you are in.
Understanding how your organization defines success and failure is vital to your emotional well-being as well as your ability to get promoted. The next section covers promotion.