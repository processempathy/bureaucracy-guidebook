\section{Success (and failure) in Your Organization\label{sec:org-failure-and-success}}

This section on how a team or organization conceptualizes failure is separate from %section~\ref{sec:learn-from-failure}
\hyperref[sec:learn-from-failure]{learning from failure} as an individual bureaucrat. 

Understanding the ways your organization succeeds or fails matters to you because it affects how you are evaluated and how you feel about your work. Success and failure are measured with respect to a quantitative objective. 


Bureaucratic organizations have varying definitions of success and failure because organizations are made up of individual bureaucrats with a multitude of motives and weak feedback mechanisms. There are a few macro-level patterns for organizations. Each pattern has affect on the individual bureaucrats operating within the organization. 

\ \\

A bureaucratic organization that provides infrastructure service (such as public water utility, electrical power company, or an organization's internal computer support) has a specific failure mode. When infrastructure is unavailable or degraded, that's a failure. Success means satisfactory operation at minimal cost. An organization operating in this context is motivated to avoid failure. Success is merely a lack of failure. An effort to improve processes or otherwise innovate will face the inertia of a culture organized around avoiding failure. 

\ \\

A bureaucratic organization that claims ``wins'' but no failures (because either more or different work needed) is difficult to be productive in when the successful outcome has ill-defined value. The success can be counted (number of wins), but the relative importance of the success is unclear.

Examples of a win-focused organization include law enforcement, research teams, and teams with a responsibility for innovation. 

The funding for an organization focused on ``wins'' depends on how well these successes can be sold as a narrative. This biases the organization to focus on investments most likely to get more funding. Have the funding organization is separate from the consumer of the ``wins'' yields outcomes not helpful to the consumer.

\ \\

% NOT INCLUDED -- need an example to make this more concrete
%A team that works to improve the productivity of other teams may have neither successes nor failures. Money and time are invested with the hope that results are useful to someone at some future time. If the team didn't exist no one would be significantly affected. 