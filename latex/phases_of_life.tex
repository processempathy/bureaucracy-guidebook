\section{Each Phase of Life involves Bureaucracy}
In life there are some standard milestones: birth, education, work, death. Each of these milestones involves engaging bureaucracy. 

In the conventional progression of life, relationships are specific to phases. Each phase is a different experience of bureaucracy, and different roles act as bureaucrat. 


Birth is marked by getting a name, registering with the state, initiating medical records. These tasks are administered by bureaucrats who manage the processes that implement policies. 

pre-education childhood: primarily family, possible community members, caretakers


The education process involves the community, typically as a public school, private school, or homeschooling. In any of those cases, the frontline bureaucrat is the teacher. 


The expectations of each phase of school (high school, undergraduate, graduate school) are distinct, and they are different from working in a large organization. 

Single choice decision examples: math problems, multiple choice tests
School doesn't prepare students for paradoxes

% https://graphthinking.blogspot.com/2013/02/all-little-things.html
Where you sit in the classroom matters. Being in the front means you will be exposed to fewer distractions, more likely to pay attention.


% https://graphthinking.blogspot.com/2012/09/how-to-not-be-average.html
How were you taught? Did you have any input on the method?
Who taught? Did you like them? Were they friendly, knowledgeable, and approachable?
How did you learn? What worked best? What didn't work, and why not?
What did you learn? What do you wish you had learned?

% https://graphthinking.blogspot.com/2012/09/how-to-not-be-average.html
What resources are available now for you to learn from? Do you like learning?
you have the freedom to pursue what ever intellectual endeavors you want

you get to choose the book you want. You don't even need to pick just one -- you can pick lots of books and figure out which author style best fits you. You can also choose the level of difficulty -- very basic and beginner level, or more advanced (depending on where you are, rather than where a class in school is supposed to be).

% https://graphthinking.blogspot.com/2011/09/which-skills-are-useful-after.html

25 years of school did not prepare me for addressing challenges at work. I learned how to approach technical issues and develop solutions. Problem and solution paradigm removed the negotiation and situation discovery steps. While in school I limited socializing. I've found that social skills and political savvy are at least as important for getting significant change implemented.  I've accepted this reality. If I had been aware of this gap earlier in my career, the projects I worked on that failed adoption by users may have been more successful

% https://graphthinking.blogspot.com/2018/07/the-difference-between-problems-at.html


primary education: family, friends, teachers


%\subsection{undergrad vs graduate}
 education process roles and expectations vary over time

college, graduate school: friends, teachers, advisors



%\subsection{military}
Less than 0.5\% of the United States population serves in the military. \footnote{source: \href{https://www.cfr.org/backgrounder/demographics-us-military}{Council on Foreign Relations}}

The military, with rigid hierarchy and defined protocols, is also a distinct experience. 



Within the employment phase, there are pairs start and end events which may apply: hiring or getting hired; firing, getting fired, or quitting. 

employment: managers above you, peer employees, people you manage are all members of an organization. 


Where do people encounter shared resources at work?
* bathroom. 
* conference rooms
* kitchen/fridge/microwave
* storage areas

Bathroom example: the bathroom smells sometimes so I'm a nice person and I bring a scented air freshener. Unbeknownst to me, that triggers an asthma or allergic reaction for my coworkers. Now a policy gets created so this mistake doesn't happen again. Signs are posted. 


healthcare. Doctors and nurses are bureaucrats; the hospital or clinic is the organization. 


Death can invoke both the medical system and the government. 





All of these roles and relationship involve subjectively administered policies within an organization. However, merely recognizing the role of bureaucrats is not a complete description of bureaucracy. 



Toilet paper at the hospital
If you go to a hospital, use the bathroom, and find there is no toilet paper, that would indicate a deficiency.
Someone didn't refill the toilet paper. Since the person who normally refills toilet paper wasn't also a user, they aren't directly impacted by the lack of toilet paper.
So someone needs to have a routine of checking bathrooms for toilet paper availability. Money spent checking to toilet paper is money not spent on the primary mission of the hospital.
Minimizing checking is important, so a feedback mechanism is instituted -- a phone number to text regarding bathroom status at the hospital.
