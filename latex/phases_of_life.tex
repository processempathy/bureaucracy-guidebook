\section[Each Phase of Life Involves Bureaucracy]{Each Phase of Life Involves Bureaucracy\label{sec:phase-of-life}\iftoggle{shortsectiontitle}{\sectionmark{Life Involves Bureaucracy}}{}}
\iftoggle{shortsectiontitle}{\sectionmark{Life Involves Bureaucracy}}{}

% LONG1 shows up in the TOC
% LONG2 is the section title


%\href{https://en.wikipedia.org/wiki/TL;DR}{TL;DR}: 
%\index{Wikipedia!\href{https://en.wikipedia.org/wiki/TL;DR}{TL;DR}}
\textit{Summary}: You have experience with bureaucracy, though you may not have framed the experiences as bureaucratic. Viewing relationships and roles through the lens of bureaucracy explains the limitations of your expectations.

\ \\

In each person's life there are standard milestones: birth, education, work, and death. 
%The relationships associated with each phase are distinct.
Each of these milestones and phases involves bureaucracy. Each phase is a different experience of bureaucracy because the bureaucratic roles change.

Bureaucracy is composed of three roles: the policy creator, the policy enforcer, and the subject (upon whom the policy is inflicted). 
\index{bureaucracy!roles}
These roles can be confounded by oversubscribing the label ``bureaucrat." Who is the bureaucrat and who is the subject depends on the relationship in a scenario. 

For example, a store manager creates a policy, a store clerk enforces the policy, 
\index{exemplar!store clerk}
and the policy is inflicted on the customer. Both the manager and the store clerk would be bureaucrats, while the customer is the subject. In a separate example, someone at corporate headquarters sets a policy, the store manager enforces it, and the policy is inflicted on the store clerk. Then the clerk was the subject of bureaucracy. 


\subsection*{Bureaucracy of Birth\label{sec:bureacracy-of-birth}}
Your birth was marked by getting a name, registering with the state, and initiating medical records. These tasks were administered by bureaucrats (doctors, nurses, and other hospital staff) on your behalf, and you were the subject of the bureaucracy. You had no autonomy or decision-making authority. 

\subsection*{Bureaucracy in Early Childhood\label{sec:bureaucracy-early-childhood}}
Before starting formal education, the bureaucracy of early childhood is inflicted primarily by family members setting and carrying out policies. The organization of bureaucrats is the family, with the shared resources being housing, food, and experience with survival. Other community members or caretakers may also be involved in carrying out the policies of taking care of you. Your decision-making authority as a subject in this bureaucracy was minimal. 

\subsection*{Bureaucrats at School\label{sec:bureaucracy-of-school}}
Once you started the formal education process, new bureaucrats got involved.  The community of bureaucrats could be a public school, a private school, or homeschooling. In any of those cases, the frontline bureaucrat is the teacher. You're not responsible for making policies that other people follow; you are still the subject of bureaucracy.


The expectations of each phase of school (high school, undergraduate, graduate school) are distinct, and they are different from working in a large organization. Your autonomy increases throughout the duration of school. %, and your ability to make policy that effects others grows. 
Your family and teachers are the bureaucrats. You start building informal organizations of friends, and you start to explore policies around social bonding.

Schooling sets a pattern that most students will fall into for the rest of their lives: you were handed a textbook and told to solve a set of problems. That pattern of taking direction can persist for a long time. However, you are not constrained to persist with that limitation. You have the autonomy to do more than what is required. You can find other textbooks that match your interests or are written from a different view. 

You get to choose the book you want, even if you don't get to choose the topic you will be evaluated on. You don't  need to pick just one reference book -- you can review lots of books and figure out which author style best fits you. You can also choose the level of difficulty -- basic and beginner level, or more advanced. You can choose what's best based on your understanding rather than defer to what a class in school is supposed to cover.

As a subject of the education bureaucracy, you can discover how you learn best. This extra effort requires self-reflection: How do you learn? What works best? What didn't work, and why not? What did you learn? What do you wish you had learned?

Another pattern that schooling relies on is single-question decisions with only one right answer. Examples include math problems and multiple-choice tests. Schooling tends to avoid setting up dilemmas or paradoxes for students. Academic problems in the education process are designed to be independent of the people involved or the history of the situation. 


% https://graphthinking.blogspot.com/2013/02/all-little-things.html
%Where you sit in the classroom matters. Being in the front means you will be exposed to fewer distractions, more likely to pay attention.


% https://graphthinking.blogspot.com/2012/09/how-to-not-be-average.html
%How were you taught? Did you have any input on the method?
%Who taught? Did you like them? Were they friendly, knowledgeable, and approachable?


% https://graphthinking.blogspot.com/2012/09/how-to-not-be-average.html
%What resources are available now for you to learn from? Do you like learning?
%you have the freedom to pursue what ever intellectual endeavors you want.


% https://graphthinking.blogspot.com/2011/09/which-skills-are-useful-after.html

% did not prepare me for addressing challenges at work. 
In my schooling I learned how to approach technical issues and develop solutions. That problem-solution paradigm neglects the crucial steps of discovering the problem, isolating the challenge, identifying stakeholders, learning the history of the challenge, and negotiating with stakeholders before trying to address the challenge. 

My schooling led me to emphasize academics over socializing. When I transitioned to professional work, I found social skills and political savvy useful when trying to change organizations and policies. 

Academic problems are intended to be solvable and answers submitted get evaluated. In contrast, working in a bureaucratic organization the challenges are ill-defined, there's no known solution, and the topic is sufficiently complex that you have to collaborate.

% https://graphthinking.blogspot.com/2018/07/the-difference-between-problems-at.html


%\subsection*{undergrad versus graduate}
% education process roles and expectations vary over time

%college, graduate school: friends, teachers, advisors



\subsection*{Military Service Bureaucracy\label{bureaucracy-of-military}}
\index{exemplar!military}
Somewhere between 13 percent~\cite{2023_cbpp} and 
20 percent~\cite{2019_Koshgarian} of 
the United States federal budget is spent on military, and 
less than 0.5\% of the United States population serves in the
military~\cite{2020_demo_mil_CFR}.
For those who serve, the military's rigid hierarchy and defined protocols are a distinct experience compared to school or work. Transitioning from military to civilian life can present a dissonance for service members used to the chain of command and clearly defined orders. 

While you are in the military the hierarchy and use of inefficient policies feel stifling. Once out of the military you may reflect fondly on the clarity of orders compared to the vagaries of social politics.  

\subsection*{Working in a Bureaucratic Organization\label{sec:bureaucracy-of-work}}
%Within the employment phase of life, there are pairs of events which may apply: hiring or getting hired; firing, getting fired, or quitting. 

%Employment: managers above you, peer employees, people you manage are all members of an organization. 


Organizations with communal workspaces have \iftoggle{glossarysubstitutionworks}{\glspl{shared resource}:}{shared resources:} bathrooms, conference rooms, storage areas, and kitchen areas with fridges and microwaves. Each of these incurs policies of use. Unlike being a student at school, you may find yourself responsible for developing and enforcing policy. 

As an example of workplace policies, consider the following scenario. 
%\marginpar{[Tag] Story Time}
\index{story time!bathroom smells}
%\begin{storytime}{The Bathroom Stinks}
\begin{mdframed}[frametitle={The Bathroom Stinks},frametitlerule=true,frametitlealignment=\centering]
The bathroom at work sometimes smells, so I'm a nice person and bring a scented air freshener. Unbeknownst to me, that triggers asthma or an allergic reaction in one of my coworkers. Because of this a policy gets created so this mistake doesn't happen again. 

Signs are posted. Violations are reported to management even if no one has a physical reaction to the air freshener.
%\end{storytime}
\end{mdframed}

Policies are often created in response to specific incidents. This intent can be helpful (promulgating lessons learned is efficient) or unhelpful (when policies are an overreaction). 

\subsection*{Healthcare and Death\label{sec:bureaucracy-of-death}}
Medical care alters our life, and a lot of money is spent on medical care: 25\% of the federal budget in the United States~\cite{2023_cbpp}. Understanding the bureaucracy of healthcare is outside the scope of this book, but your role in the bureaucracy is helpful to understand.

Doctors, nurses, and other staff are bureaucrats; you are the subject; the hospital or clinic is the organization. 

\ \\
To illustrate the bureaucracy of a large organization, consider the importance of toilet paper at a hospital. 
%\marginpar{[Tag] Story Time}
\index{story time!toilet paper at hospital}
%\begin{storytime}{Restocking Toilet Paper}
\begin{mdframed}[frametitle={Restocking Toilet Paper},frametitlerule=true,frametitlealignment=\centering]
If you go to a hospital, use the bathroom, and find there is no toilet paper, that would indicate a deficiency of the hospital.

The \textit{holistic view} is that someone didn't refill the toilet paper. Since the person who usually restocks toilet paper wasn't also a user, they aren't directly affected by the lack of toilet paper.

A routine is needed for checking the availability of toilet paper in bathrooms. 

The \textit{perspective of the purchasing manager} is that money spent checking the status of toilet paper is money not spent on the hospital's primary mission: improving the health of community members.

Minimizing checking of toilet paper is important for the organization's reputation, so a feedback mechanism is instituted: a phone number is posted in the bathroom so users can send a text regarding bathroom status at the hospital.

The \textit{perspective of the janitor} is that my routine used to be to go to each bathroom after normal business hours and refill toilet paper. Now I have to do that and be on-call when someone alerts management that service is needed. My responsibilities increased but my pay did not.

%\end{storytime}
\end{mdframed}

%\ \\

%Death can invoke both the medical system and the government. 

%\ \\


%All of these roles and relationships involve 

Bureaucracy is a set of subjectively administered policies within an organization. By recognizing the role of bureaucrats you can identify what is negotiable. 
