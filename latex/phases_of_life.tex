\section{Each Phase of Life involves Bureaucracy}
In each person's life there are standard milestones: birth, education, work, death. 
The relationships associated with each phase are distinct.
Each of these milestones and phases involves bureaucracy. Each phase is a different experience of bureaucracy because the bureaucratic roles change.

Bureaucracy is composed of three roles: the policy creator, the policy enforcer, and the subject (upon whom the policy is inflicted). 
\index{bureaucracy!roles}
These roles can be confounded by the label ``bureaucrat" because who is the bureaucrat and who is the subject depends on the relationship in a scenario. 

For example, a store manager creates a policy, a store clerk enforces the policy, 
\index{exemplar!store clerk}
and the policy is inflicted on the customer. Both the manager and the store clerk would be bureaucrats, while the customer is the subject. In a separate example, someone at corporate headquarters sets a policy, the store manager enforces it, and the policy is inflicted on the store clerk. Then the clerk was the subject of bureaucracy. 


\subsection*{Bureaucracy of Birth\label{sec:bureacracy-of-birth}}
Your birth was marked by getting a name, registering with the state, and initiating medical records. These tasks were administered by bureaucrats (doctors and nurses and other hospital staff) on your behalf, and you were the subject of the bureaucracy. You had no autonomy or decision making authority. 

\subsection*{Bureaucracy in Early Childhood\label{sec:bureaucracy-early-childhood}}
Before starting formal education, the bureaucracy of early childhood is inflicted primarily by family members setting and carrying out policies. The organization of bureaucrats is the family, with the shared resources being housing, food, and expertise in surviving. Other community members or caretakers may also be involved to carry out the policies of taking care of you. Your decision making authority was extremely limited. 

\subsection*{Bureaucrats at School\label{sec:bureaucracy-of-school}}
Once you started the formal education process, new bureaucrats got involved.  The community of bureaucrats could be a public school, a private school, or homeschooling. In any of those cases, the frontline bureaucrat is the teacher. You don't have responsibility for making policy that other people follow; you are still the subject of bureaucracy.


The expectations of each phase of school (high school, undergraduate, graduate school) are distinct, and they are different from working in a large organization. Your autonomy increases throughout the duration of school. %, and your ability to make policy that effects others grows. 
Your family and teachers are the bureaucrats. You start building informal organizations of friends, and you start to explore policies around social bonding.

Schooling sets a pattern that most students will fall into for the rest of their lives: you were handed a textbook and told to solve a set of problems. That pattern of taking direction can persist for a long time. However, you are not constrained to that limitation. You have the autonomy to do more than what is required. You can find other textbooks that match your interests or are written from a different view. 

You get to choose the book you want, even if you don't get to choose the topic you are going to be evaluated on. You don't even need to pick just one reference book -- you can pick lots of books and figure out which author style best fits you. You can also choose the level of difficulty -- basic and beginner level, or more advanced. You can choose what's best depending on where your understanding is, rather than depend on what a class in school is supposed to cover.

You can discover how you learn best. This extra effort requires self-reflection: How did you learn? What worked best? What didn't work, and why not? What did you learn? What do you wish you had learned?

Another pattern that schooling relies on is single choice decisions where there is only one right answer. Examples include math problems and multiple choice tests. Schooling tends to avoid setting up dilemmas or paradoxes for students. Academic problems in the education process are designed to be independent of the people involved or the history of the situation. 


% https://graphthinking.blogspot.com/2013/02/all-little-things.html
%Where you sit in the classroom matters. Being in the front means you will be exposed to fewer distractions, more likely to pay attention.


% https://graphthinking.blogspot.com/2012/09/how-to-not-be-average.html
%How were you taught? Did you have any input on the method?
%Who taught? Did you like them? Were they friendly, knowledgeable, and approachable?


% https://graphthinking.blogspot.com/2012/09/how-to-not-be-average.html
%What resources are available now for you to learn from? Do you like learning?
%you have the freedom to pursue what ever intellectual endeavors you want.


% https://graphthinking.blogspot.com/2011/09/which-skills-are-useful-after.html

% did not prepare me for addressing challenges at work. 
In my schooling I learned how to approach technical issues and develop solutions. That problem-solution paradigm neglects crucial steps of discovering the problem, isolating the challenge, identifying stakeholders, learning the history of the challenge, and negotiating with stakeholders before trying to address the challenge. 

My schooling led me to emphasize academics over socializing. When I transitioned to professional work, I found social skills and political savvy useful when trying to change organizations and policies. 

Academic problems are intended to be solvable and answers submitted get evaluated. In contrast, working in a bureaucratic organization the challenges are ill-defined, there's no known solution, and the topic is sufficiently complex that you have to collaborate.

% https://graphthinking.blogspot.com/2018/07/the-difference-between-problems-at.html


%\subsection*{undergrad vs graduate}
% education process roles and expectations vary over time

%college, graduate school: friends, teachers, advisors



\subsection*{Military Service Bureaucracy\label{bureaucracy-of-military}}
\index{exemplar!military}
Less than 0.5\% of the United States population serves in the military\footnote{source: \href{https://www.cfr.org/backgrounder/demographics-us-military}{Council on Foreign Relations}}. For those who do serve, the military's rigid hierarchy and defined protocols is a distinct experience compared to school or work. The transition from military to civilian life can present a dissonance for service members used to the chain of command and clearly defined orders. 

\subsection*{Working in a Bureaucratic Organization\label{sec:bureaucracy-of-work}}
%Within the employment phase of life, there are pairs of events which may apply: hiring or getting hired; firing, getting fired, or quitting. 

%Employment: managers above you, peer employees, people you manage are all members of an organization. 


Organizations that have communal work spaces have shared resources: bathrooms, conference rooms, kitchen areas with fridges and microwaves, storage areas. Each of these incur policies of use. Unlike being a student at school, you may find yourself responsible for developing and enforcing policy. 

\marginpar{[Tag] Story Time}
\index{story time!bathroom smells}
\begin{mdframed}
The bathroom at work smells sometimes so I'm a nice person and I bring a scented air freshener. Unbeknownst to me, that triggers an asthma or allergic reaction for my coworkers. Now a policy gets created so this mistake doesn't happen again. Signs are posted. Violations are reported to management even if no one has a physical reaction to the air freshener.
\end{mdframed}

\subsection*{Healthcare and Death\label{sec:bureaucracy-of-death}}
Doctors and nurses and other staff are bureaucrats; you are the subject; the hospital or clinic is the organization. 


\subsubsection*{Toilet paper at the hospital}

\marginpar{[Tag] Story Time}
\index{story time!toilet paper at hospital}
\begin{mdframed}
If you go to a hospital, use the bathroom, and find there is no toilet paper, that would indicate a deficiency of the hospital.

The \textit{holistic view} is that someone didn't refill the toilet paper. Since the person who normally refills toilet paper wasn't also a user, they aren't directly affected by the lack of toilet paper.

A routine is needed for checking availability of toilet paper in bathrooms. 

The \textit{perspective of the purchasing manager} is that money spent checking the status of toilet paper is money not spent on the primary mission of the hospital: improving the health of community members.

Minimizing checking of toilet paper is important for the reputation of the organization, so a feedback mechanism is instituted: a phone number is posted in the bathroom so users can send a text regarding bathroom status at the hospital.

The \textit{perspective of the janitor} is that my routine used to be to go to each bathroom after normal business hours and refill toilet papers. Now I have to do that and be on-call when someone alerts management that service is needed. My responsibilities increased but my pay did not.

\end{mdframed}

%\ \\

%Death can invoke both the medical system and the government. 

%\ \\


%All of these roles and relationships involve 

Bureaucracy is a set of subjectively administered policies within an organization. By recognizing the role of bureaucrats you are then able identify what is negotiable. 
