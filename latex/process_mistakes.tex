\section{Process Mistakes\label{sec:process-mistakes}}

Any process involving humans incurs mistakes.\footnote{See the Wikipedia entry for \href{https://en.wikipedia.org/wiki/Murphy\%27s_law}{Muphy's law}.
\index{Wikipedia!\string\href{https://en.wikipedia.org/wiki/Murphy\%27s_law}{Muphy's law}}
} The only way to drive the number of mistakes to zero is by doing nothing, but that isn't a practical way of managing shared resources. 
The purpose of pondering bureaucratic mistakes is to come up with ways to limit the consequence of mistakes to cause damage. 

\subsection*{Fixes are Difficult}
Imposing checks on a process as a way to  reduce mistakes creates more bureaucracy. The extra work can be in the form of redundant data collection, or more justification needed. 
Maintaining coordination of diverse activities in a bureaucracy requires that information propagate up the chain of command to a common authority. Some mistakes like duplication of effort can only be detected by checks made high up the chain of command.


\subsection*{Fixes Fail}
The path of information from the person with a problem to the person who can address the situation may pass through many people. 

The consequence is familiar to people who have played the \href{https://en.wikipedia.org/wiki/Chinese_whispers\%23Game}{game of telephone}.
\index{Wikipedia!\href{https://en.wikipedia.org/wiki/Chinese_whispers\%23Game}{game of telephone}}
%\marginpar{[Tag] Story Time}
\index{story time!game of telelphone}
%\begin{storytime}{Game of Telephone}
\begin{mdframed}[frametitle={Game of Telephone},frametitlerule=true,frametitlealignment=\centering]
The person has a problem and explains their problem to the helpdesk staff member. The helpdesk staff member hears the problem with incomplete context and from their own frame, then relays it to their manager, who talks to the manager of the engineering team, who delegates the responsibility to the engineer. 
%\end{storytime}
\end{mdframed}

This is a solvable problem: the engineer could talk with the customer. 
% from
% https://news.ycombinator.com/item?id=30480083

%TODO: transition here
At each hand-off there is a significant chance the information is unintentionally mangled. 

\subsection*{Where Process Mistakes come from}
Mistakes in a process can originate from the bureaucrat who inflicts the process, in the hand-off between participants, or from the \gls{subject} of the bureaucratic process. Typically there are more checks on subjects and fewer on bureaucrats. 

\subsection*{Choices about Mistakes}
As a process participant, you may cause a mistake or observe a mistake. Your options are to can ignore it, fix it (which incurs extra work), or report it. If you choose to report it, you can do so anonymously (which decreases the risk of harm to your reputation but also eliminates feedback) or with your name so that discussion can occur and you get feedback.

\ \\

\noindent\hrulefill

\ \\

% TRANSITION to next section: process_design_of
Pondering types of mistakes and causes of mistakes for processes is important if you want to revise a process or create a new process.