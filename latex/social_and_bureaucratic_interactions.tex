% https://graphthinking.blogspot.com/2020/05/invisible-bureaucracy.html

\section{Social and Bureaucratic interactions\label{sec:socializing}}

Interactions with other people in an organization are either a social interaction or a bureaucratic interaction. 

As examples of each of these,
\begin{itemize}
\item \textit{Social interaction example}: ``Did you see the game on TV last night? Our team really did well, right? I wanted to get tickets for the game but they were sold out."
\item \textit{Bureaucratic interaction example}: ``You'll need to get approval from Sue before presenting your idea to the board for their review. Then talk with Russ and get his thoughts about how to proceed."
\end{itemize}
Both social and bureaucratic interactions are vital to cohesion in an organization of people. 


Bureaucratic interaction can be broken into two subcategories: 
\gls{visible bureaucracy} 
\marginpar{[Tag] Glossary}
(procedures and processes are written down and can be discovered by stakeholders) and 
\gls{invisible bureaucracy} 
\marginpar{[Tag] Glossary}
(procedures and processes are known to some stakeholders and are conveyed verbally to some of the other stakeholders).

Invisible bureaucracy is akin to invisible domestic or relation work outside the professional environment. The work associated with emotional cohesion, logistics, planning, scheduling, and communicating is hard to quantify so it does not get counted.

Processes default to invisible bureaucracy because documentation and discoverability require work, and because some processes are embarrassingly inefficient. 

To make invisible bureaucracy visible, document the work.


The relevance of this jargon is to break down the components of an organization's ``culture" experienced by participants. The ratio of social relationship to visible bureaucracy to invisible bureaucracy is a characterization of the culture. There are norms associated with each of these three categories.

TODO: "there are norms" -- so what?