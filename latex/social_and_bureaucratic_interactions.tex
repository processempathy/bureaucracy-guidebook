% https://graphthinking.blogspot.com/2020/05/invisible-bureaucracy.html

\subsection*{Social and Bureaucratic interactions\label{sec:socializing}}

Change in a bureaucracy can apply to processes and people, but a more amorphous concept is changing the culture of a team or organization. What is meant by ``culture'' usually refers to norms -- the expectations of behavior that individuals hold to. That definition of culture is generic; what is meant within a bureaucratic context requires jargon for the specific expectations.

To evaluate expectations we start by introducing categories of interactions. 
Interactions among members of an organization are either a social interaction or a bureaucratic interaction. 

As examples of each of these,
\begin{itemize}
\item \textit{Social interaction example}: ``Did you see the game on TV last night? Our team really did well, right? I wanted to get tickets for the game but they were sold out."
\item \textit{Bureaucratic interaction example}: ``You'll need to get approval from Sue before presenting your idea to the board for their review. Then talk with Russ and get his thoughts about how to proceed."
\end{itemize}
Both social and bureaucratic interactions are vital to cohesion in an organization. 


Bureaucratic interaction can be broken into two subcategories: 
\gls{visible bureaucracy} 
\marginpar{[Glossary]}
(procedures and processes are written down and can be discovered by stakeholders) and 
\gls{invisible bureaucracy} 
\marginpar{[Glossary]}
(procedures and processes are known to some stakeholders and are conveyed verbally to some of the other stakeholders).

Invisible bureaucracy is akin to related topics outside the professional environment: invisible domestic work\footnote{Cleaning your living space, raising children, caring for pets; see~\cite{1987_Daniels}.} and invisible relationship work.\footnote{Consistent need to delegate, being curious without reciprocation.} The work associated with emotional cohesion, logistics, planning, scheduling, and communicating is hard to quantify so it does not get counted.


The relevance of this jargon is to break down the components of an organization's ``culture" experienced by participants.
When someone in the organization advocates for changing the culture, which expectations are they specifically referring to? Invisible bureaucracy is the hardest to alter because it is undocumented and not counted.
%The ratio of social relationship to visible bureaucracy to invisible bureaucracy is a characterization of the culture. There are norms associated with each of these three categories.

Processes default to invisible bureaucracy because creating and maintaining documentation requires work. Making the documentation discoverable requires work.
%, and because some processes are embarrassingly inefficient. 
To make invisible bureaucracy visible, document the work and enable other people to find the documentation.
