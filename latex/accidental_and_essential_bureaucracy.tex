\section[Bureaucracy as Accidental, Legacy, or Essential]{Bureaucracy as Accidental, Legacy, or Essential\label{sec:accidental}\iftoggle{shortsectiontitle}{\sectionmark{Accidental, Legacy, or Essential}}{}}
\iftoggle{shortsectiontitle}{\sectionmark{Accidental, Legacy, or Essential}}{}

% LONG1 shows up in the TOC
% LONG2 is the section title


\iftoggle{glossarysubstitutionworks}{\Gls{essential bureaucracy}}{Essential bureaucracy}
\iftoggle{glossaryinmargin}{\marginpar{[Glossary]}}{}%
is the minimum set of processes, staffing, and skills necessary to address the complexity of managing a community's access to a \gls{shared resource}.
\iftoggle{glossaryinmargin}{\marginpar{[Glossary]}}{}%
Achieving this minimum is tricky since optimization can be with respect to resilience to change, resilience to edge cases, staff turnover, speed experienced by subjects, financial cost, time spent by the organization, and the number of staff. Miss any one of those goals and the bureaucracy is deemed inefficient.

Undesirable bureaucracy is categorized as either accidental or legacy. Accidental bureaucracy arises when someone misunderstands what is needed or when skills of the bureaucrats involved are insufficient for the complexity a problem. Legacy bureaucracy occurs when the situation changes but the processes do not. 

Accidental and legacy bureaucracy accumulates within an organization. Optimization of improved efficiency is at odds with change which disrupts careers, relationships, and accumulated power. 

Resolving each of these suboptimal conditions may seem easy: have better knowledge of the problem, assign the right people to the problem, and change processes as problems evolve.  In practice, the easy resolution is challenging. 

Having enough knowledge is often infeasible, especially for complex problems at a large scale. Having the people with the right skills assumes that a pipeline of people with relevant talents exists and that people in the pipeline won't be poached to work on other challenges. Keeping up with evolving problems depends on having resources to change (beyond the maintenance baseline) and having a defined approach for changing the process. 