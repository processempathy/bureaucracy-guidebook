\section{Bureaucracy as Accidental, Legacy, or Essential}
\gls{essential bureaucracy} is the minimum processes and staffing and skills necessary to address the complexity of the problem space. Achieving this is tricky since the optimization can be with respect to resilience to change, resilience to edge cases, staff turn-over, speed experienced by consumer, financial cost, organization time, organization staffing level.

Undesirable bureaucracy is either accidental or legacy. Accidental bureaucracy arises when someone misunderstands what is needed, or when skills of the people involved are insufficient for the complexity a problem. Legacy bureaucracy occurs when the situation changes but the processes do not. Resolving each of these suboptimal conditions may seem easy: have better knowledge of the problem, assign the right people to the problem, and change processes as problems evolve. 

Having enough knowledge is often infeasible, especially for complex problems at large scale. Having the people with the right skills assumes a pipeline of relevant talents and that people in that pipeline won't be poached for other problem. Keeping up with evolving problems depends on having resources to change, and having a defined process for changing process. 