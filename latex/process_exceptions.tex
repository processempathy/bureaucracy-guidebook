\section{Exceptions to a Process\label{sec:exceptions-to-process}}


Process designers are usually motivated to address a specific issue associated with access to shared resources. When bureaucrats execute a process to support a policy, there may be cases that violate the designer's assumptions. Now the person with the exceptional case and the bureaucrat carrying out the policy have extra work to deal with the broken process. This is a source of \hyperref[sec:bureaucratic-debt]{bureaucratic debt}. 
\marginpar{See page~\pageref{sec:bureaucratic-debt}.}
%\ifsectionref
%(see section~\ref{sec:bureaucratic-debt} for details). 
%\fi

When the routine process is inadequate, there are routes for exceptions:
\begin{itemize}
    \item Change the process to address the exceptional case. This takes extra time and work for the subject and the bureaucrat.
    \item Ignore the process. Use relationships instead.
    \item Ignore the task. If the result wasn't crucial to your efforts and the work is too burdensome compared to alternatives, terminate the process.
    \item Seek an exception to the process. The exception can be recurring or one-time. The justification for the exception can be based on social capital of participants (title, reputation) or be a data-driven argument or some mixture of both.
\end{itemize}

The designer of a process may not be able to anticipate future circumstances, but expecting the capacity to be responsive to change is reasonable. The process designer should use sunset provisions and escape hatches.  
\marginpar{$>>$ Actionable Advice}
\index{actionable advice}
A sunset provision can be time-based (this policy expires after 3 years) or based on a threshold (this policy should be evaluated for renewal after 1000 cases). An escape hatch specifies the conditions under which the participants are excepted from the policy. A good escape hatch for a policy provides directions about what follow-on actions should be taken.

Another reason processes may be inadequate is due to urgency. A result that is needed quickly may not fit the timeline of sequential tasks for a process.

\subsection*{Time Sensitivity Exceptions}
Exceptions to a process may be required when a result is need quickly. In the medical setting there are categories of urgency; a task may be routine, priority, urgent, stat. 
Each of those categories is associated with a time bound, though the specific duration varies by medical domain and by location.\footnote{\href{http://docport.columbia-stmarys.org/EHR/PhysicianOrdersPriorityandDeptServiceHours.aspx}{Physician Orders Priority}\iftoggle{boundbook}{ on http://docport.columbia-stmarys.org}{} and \href{https://www.unitypoint.org/peoria/services-priority-definitions-and-critical-values.aspx}{Priority Definitions}\iftoggle{boundbook}{ from https://www.unitypoint.org}{}.}
% as of 2022-12-17, the docport.columbia-stmarys.org page is unavailable. The google cache and Wayback machine provide valid pages.

Common failure modes for processes include abuse of prioritization (everything is important) or abuse of urgency (everything is needed quickly). Either of these abuses need to be corrected to have a healthy bureaucracy, but typically the cost of labeling an effort as ``priority'' or ``urgent'' is low. 

To limit abuse of processes, interaction between the bureaucrats tasked with work and the people characterizing relative value is needed. This interaction is a negotiation of what work gets done when and in what order. Labeling every task as ``urgent'' is a failure to communicate the relative value of a set of tasks.


Process friction can arise from the separation of roles needed to support a process. Those separate roles can happen within a team of bureaucrats, or the process can span distinct teams. The difficulty imposed across teams within an organization is moderated by the existence of a common arbiter in the chain of command.  
% TRANSITION to next section: processes_two
Troubleshooting processes is even more complicated when processes span organizations.  