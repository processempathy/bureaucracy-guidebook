\section{Exceptions to a Process\label{sec:exceptions-to-process}}


Process designers a usually motivated to address a specific issue. When bureaucrats execute a process to support a policy, there may be cases that violate the designer's assumptions. Now the person with the exceptional case and the bureaucrat carrying out the policy have extra work to deal with the broken process. This is a source of bureaucratic debt (see section~\ref{sec:bureaucratic_debt} for details). 

When the routine process seem inadequate, there are routes for exceptions:
\begin{itemize}
    \item correct the process to address the exceptional case
    \item ignore the process
    \item ignore the task
    \item seek an exception to the process, recurring or one-time
    \begin{itemize}
        \item social capital (title, reputation)
        \item data driven argument 
    \end{itemize}
\end{itemize}

The designer of a process may not be able to anticipate future circumstances, but expecting the capacity to be responsive to change is reasonable. The process designer should use sunset provisions and escape hatches.  \marginpar{[Tag] Actionable Advice}
A sunset provision can be time-based (this policy expires after 3 years) or based on a threshold (this policy expires after 1000 cases). An escape hatch specifies the conditions under which the participants are excepted from the policy. 

\subsection*{Time Sensitivity of Exceptions}
Routine, priority, urgent, stat. 
Each is a time bound, varies by domain and by location.
\footnote{\href{http://docport.columbia-stmarys.org/EHR/PhysicianOrdersPriorityandDeptServiceHours.aspx}{http://docport.columbia-stmarys.org/EHR/PhysicianOrdersPriorityandDeptServiceHours.aspx}}