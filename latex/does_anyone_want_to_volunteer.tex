\section{Does Anyone Want to Volunteer?}

Bureaucracy relies on coordination to facilitate the management of shared resources. In the course of coordination, new tasks arise that have no one person responsible. The word volunteer has two meanings in this setting: contributing without seeking financial gain, and choosing to contribute of your own free will. The focus here is on the second meaning.

% https://graphthinking.blogspot.com/2020/06/what-to-ask-instead-of-does-anyone-want.html

When the facilitator of a meeting asks the group, ``Does anyone want to volunteer for this task?" the question is often met with silence. No one wants extra work that does not help them, even if there is benefit to the team or organization. 
The lack of response is an instance of the \href{https://en.wikipedia.org/wiki/Tragedy_of_the_commons}{tragedy of the commons}, 
\index{Wikipedia!tragedy of the commons@\href{https://en.wikipedia.org/wiki/Tragedy_of_the_commons}{tragedy of the commons}}
but in an anticipatory sense.
Instead of asking this poorly-framed question, the facilitator can get higher engagement with the following tactics:
\begin{enumerate}
    \item Ask each person in the meeting whether they are attending to passively observe or to participate.
    \item Ask each person willing to participate how much time they can invest. Responses could be zero hours, one hour (non-recurring), one hour per month, or something else.
    \item Ask each person what their goals in the interaction are. It is usually generic (``I want the group to succeed.") but it can be narrow, in which case that gives you something to focus on.
    \item Ask each person what they are good at and what skills they have. Are they good with personal interaction? Writing? Computers? Coordinating? Logistics? Fundraising? Making phone calls?
    \item From the inventory of tasks, are there any that fit both the skills and time? Can the task be scoped to fit the time? If there are multiple candidate tasks for a volunteer, let them pick the task. (If no existing task aligns with their skills, do not create work to be assigned.)
    \item If other people are working on the same task, put the volunteer in contact with the other people.
    \item Give the volunteer a deadline for the task. Your deadline should not be arbitrary -- it should be based on the task dependencies. 
    \item Confirm that the volunteer is willing to commit the time to complete the tasks by the deadline.
    \item Schedule a check-in with the volunteer before the task deadline to review progress.
    \item To create accountability among multiple volunteers, hold a group review to explain how each participant's work contributes to the goals and how work done by one person enables the next task done by someone else. (Enumerate the dependency graph; include deadlines.)
\end{enumerate}

As the facilitator looking to designate tasks to participants, the above tactics align the work with the skills of contributors. 

As a contributor, your obligations to the team are knowing how much time you have to contribute and being clear about the relative priorities of tasks. Explaining that the new task is lower priority than all your other obligations is a valid response. Your responsibility is to negotiate with the person looking to assign new work and the people who depend on the results of your current tasks. 