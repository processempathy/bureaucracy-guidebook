\subsection{Ideas for Innovation within a Bureaucracy\label{sec:innovation}}


The innovation lifecycle in a bureaucracy is
\begin{enumerate}
    \item You observe problems and challenges in your environment. This manifests as complaints from both the people directly harmed and observers who see inefficiency.
    \item You share ideas for innovation and gets feedback. Build a coalition of people willing to fight for the idea on your behalf
    \marginpar{[Tag] Actionable Advice}
    so that when you're not present, the idea is still proceeding towards implementation.  That puts the threshold at ``so important other people are willing to pause whatever they were working on and take up your cause.''
    \item Implementing these suggestions require either a change to existing processes or new processes or an investment of work. These changes may not succeed -- there's risk. Your idea could fail because it's not a good idea. It could also fail because someone doesn't like you, or the idea doesn't account for some dependency you weren't aware of, or it might conflict with other changes in progress.
    \item If you do decide to invest effort, the activity takes you away from your current work. Implementing the change might involve skills you don't have; learning those skills takes time. Carrying out the activity with new skills increases the likelihood of novice mistakes.
    \item If someone else implements the idea they get the credit for having done the work.
    \item If the idea saves money or time, there's typically no monetary reward. Recognition and benefit to your reputation isn't required as part of the change process. 
\end{enumerate}

There are many barriers in that lifecycle. The problem has to be observable to someone willing to invest effort in change. That person has to build a coalition of stakeholders. If the person isn't negatively harmed, that person may also lack clear benefit from resolving inefficiency. 

In addition to the work of implementing the change, there is an administrative overhead of documenting the reason for the change. 
Subjective decisions mean choices have to be defensible. 
The need for defensible justifications result in conservative decisions and risk aversion and a decrease of motive for innovation. 



These barriers lead to external observers to conclude something like the following simplification:
\begin{quote}
Bureaucracy destroys initiative. There is little that bureaucrats hate more than innovation, especially innovation that produces better results than the old routines.
\marginpar{[Tag] Folk wisdom}
Improvements always make those at the top of the heap look inept. Who enjoys appearing inept?\footnote{Frank Herbert (1987). ``Heretics of Dune'', page 201, Penguin}
% https://www.azquotes.com/quote/453163
\end{quote}

