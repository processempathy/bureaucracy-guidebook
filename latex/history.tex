\section*{History of Bureaucracy}

After making a few points I'm going to skip recounting the \href{https://en.wikipedia.org/wiki/Bureaucracy#History}{history of bureaucracy}.  % and merely cite other scholarly references in this section

First, bureaucracy has repeatedly arisen independently in a variety of cultures lasting for timescales that exceed the lifespan of one person\footnote{\href{https://www.youtube.com/watch?v=B_nsZlcC12g}{History of bureaucracy}}. That indicates the current situation is not a fluke or coincidence. There is some utility (or pathology) that is consistently recurring. 


Second, bureaucracy predates writing and language and even humans! Subjective policy enforcement in support of an organization arises in pre-human tribes, visible in groups of modern apes who have to manage access to share resources. 
% TODO: citation needed

Third, though bureaucracy is not new, the pervasiveness is. Prior to the industrial revolution the scale of both employment and government were small organizations limited by the speed of communication. For the past 100+ years the size of organizations (commercial and governmental and academia) have grown significantly. More people participate in more organizations that are more bureaucratic. The benefits are support for more complex products and processes. 

% TODO: did the scale of bureaucracy grow faster than the growth of human population?