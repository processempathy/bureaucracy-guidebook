\section{History of Bureaucracy\label{sec:history}}


Bureaucracy has repeatedly arisen independently in a variety of cultures
\footnote{Wikipedia entry on the \href{https://en.wikipedia.org/wiki/Bureaucracy\%23History}{history of bureaucracy}.}
lasting for timescales that exceed the lifespan of one person\footnote{YouTube video on the \href{https://www.youtube.com/watch?v=B_nsZlcC12g}{History of bureaucracy}.}. That indicates the current situation is not a fluke or coincidence. There is some utility (or pathology) that is consistently recurring. 


Bureaucracy predates writing and language and even humans! Subjective policy enforcement in support of an organization arises in pre-human tribes, visible in groups of modern apes who have to manage access to share resources~\cite{2016_Suchak}. 



Though bureaucracy is not new, the pervasiveness is. Prior to the industrial revolution the scale of both employment and government were small organizations limited by the speed of communication. For the past 100+ years the size of organizations (commercial and governmental and academia) have grown beyond \href{https://en.wikipedia.org/wiki/Dunbar\%27s_number}{Dunbar's number}. More people participate in more organizations that are more bureaucratic. The benefits are support for more complex products and processes. 

% claim: bureaucracy grow faster than the growth of human population?


