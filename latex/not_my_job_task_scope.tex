\section{Not My Job: Task Scope}

% https://graphthinking.blogspot.com/2021/05/not-my-job-task-scope-and-collaboration.html

If you ask someone for help on a task that benefits your team, that person might respond that the task is ``not my job" and explain that time spent on tasks like what you're asking for gets in the way of their progress. The emotionally engaging work is not that task.

This recurring pattern is sufficiently common that at least one domain has jargon for the concept.
`\href{https://www.urbandictionary.com/define.php?term=scut}{Scut}'\iftoggle{WPinmargin}{\marginpar{[Urbandictionary.com] scut}}{}
is  slang in the medical field for the non-clinical yet essential tasks that do not require a doctor's degree or expertise.
%This is different from 
Another label used for non-central tasks is administrivia (short for administrative tasks).
%as ``scut'' would include taking out trash. 

A mindset related to the ``not my job" view is the following.
``Once I realized someone else has the same problem, I stopped worrying about it.'' The separation of scope can be reasonable (no point in duplicating effort) but may cause harm (when there's no coordination). The ``not my job" refusal is merely about scope. 

The potential reasons for this reluctance to help include
\begin{itemize}
    \item Working on the task will not get them promoted.
    \item Their understanding of the scope of their job and their expertise does not include the task.
    \item The person doesn't know how to do the task and they don't want to learn.
\end{itemize}
In any case, progress as measured by the individual refusing to help is not aligned with the success of the organization. You can escalate the request up the chain of command to get a determination about whether the collaboration should happen.

\ \\

% TRANSITION

Asking for help and giving (or refusing to give) help is dependent on your reputation.
The next section discusses how you can manage your reputation in a bureaucratic context.