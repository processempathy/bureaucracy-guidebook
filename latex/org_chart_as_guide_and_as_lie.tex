\subsubsection*{Organizational chart as a Guide and a Lie\label{sec:org-chart-as-guide-and-lie}}

An \href{https://en.wikipedia.org/wiki/Organizational_chart}{organizational chart} 
\index{Wikipedia!\href{https://en.wikipedia.org/wiki/Organizational_chart}{organizational chart}}\iftoggle{WPinmargin}{\marginpar{$>$Wikipedia: Organizational chart}}{}
(hereafter an ``\gls{org chart}'') is a document that identifies formal roles and the formal relations among roles. An org chart is at best a snapshot in time, and more often aspirational than descriptive. Despite possible deficiencies, an org chart helps outsiders and newcomers understand the scope of responsibilities and interactions.\footnote{The organization chart hasn't always existed. The \href{https://en.wikipedia.org/wiki/George_Holt_Henshaw\%23First_organization_chart}{first known org chart} 
\index{Wikipedia!\href{https://en.wikipedia.org/wiki/George_Holt_Henshaw}{George Holt Henshaw}}
was created in the 1850s.} The other sense in which an org chart serves as a guide is to identify who in the organization is claiming which areas of responsibility. Aligning which people or teams to different divisions of labor informs members of the organization about who to talk to for specific topics.

Org charts are a lie because undocumented relationships can matter more than official roles. Org charts fail to capture the informal roles and network of relations that facilitate progress in any organization. Org charts document titles instead of describing roles.

Org charts foster a second separate lie by creating a sense of power dynamics based on visual orientation. For more on this issue see the discussion of~\hyperref[sec:org-chart-orientation]{org chart orientation}\iftoggle{haspagenumbers}{ on page~\pageref{sec:org-chart-orientation}}{}.
