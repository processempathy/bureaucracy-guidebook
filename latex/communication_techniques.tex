\subsection{Communication Tips}

Sometimes it's not a communication problem but a psychological deficit of personality.

%how to measure effectiveness: The waste or inefficiency in a bureaucracy is a measure of the lack of coordination or inconsistent decision making among the members

\subsubsection{Tip: Avoid questions that have a binary response\label{sec:yes_no_questions}}

Responding to a request with ``no'' is advantageous for the person replying to the question. There is less work required, less risk of failure, and better continuity. A common request I make is ``Can I have a copy of the data you're using?'' The person I'm asking is less disrupted if they refuse to share. 

A more constructive phrasing is ``I need information on X to work on Y, and I think you have information about X. How can you help me get information on X?'' By clarifying my intent, I allow the person with the data to provide options I may not have considered.

Similarly, when I'm being asked for information, I try to learn the person's intent motivating the question. Sometimes the requester doesn't actually know what to ask for. Instead of ``no'' I try to figure out how to enable the person to be successful. 

\subsubsection{Tip: Avoid Platitudes\label{sec:platitudes}}
% https://graphthinking.blogspot.com/2017/10/why-platitudes-are-used.html
\href{https://en.wikipedia.org/wiki/Platitude}{Platitudes} are \gls{thought terminating}; the statement feels true and is resistant to debate. Platitudes capture a feeling with sufficient accuracy, but with imprecise language. As a result, there's no specific action.

Because platitudes result in a conclusion, the conversation participants may feel more bonded. However, that bond is shallow.

Example platitudes to avoid:
% https://graphthinking.blogspot.com/2017/02/phrases-which-serve-as-thinking-stoppers.html
% https://graphthinking.blogspot.com/2017/10/a-list-of-platitudes.html
\begin{itemize}
    \item pick your battles
    \item Some things you can't explain
    \item Your time will come.
    \item You can be anything that you want to be
    \item I just want to get through this day
    \item It is what it is
    \item I'm just one person
    \item That's that
    \item Life's not perfect
    \item Life's not fair
    \item There's only so much you can do about it
    \item What is meant to be will be
    \item It is God's will
    \item It's part of God's plan
\end{itemize}

If your goal is to understand a concept or issue deeply, you need to use precise language.


\subsubsection{Tip: Eliminate crosstalk}
% https://graphthinking.blogspot.com/2017/10/crosstalk.html

Crosstalk is motivated by
limited time available to communicate
inspired by something someone else said
responding to something someone else said
Typically manifests as popcorn style stories based on experience. Intended as wisdom for self-validation by others in our community. 

Crosstalk indicates engagement and enthusiasm. 

Crosstalk is dependent on the level of aggressiveness of participants.

Crosstalk has four roles and a minimum of two people participating:
discussion facilitator
original speaker
interrupter
other meeting participants
During crosstalk, the discussion facilitator loses control of the interaction to the interrupter. The original speaker probably feels annoyed at being interrupted. The audience probably feels frustrated because where to focus has become unclear. This leads to a loss of focus. 

Bystander intervention for out-of-control meeting: raise your hand. This reverts focus back to the discussion facilitator. 

\subsubsection{Tip: Account for Warnock's dilemma}
% https://graphthinking.blogspot.com/2018/09/dealing-with-warnocks-dilemma-in.html
\href{https://en.wikipedia.org/wiki/Warnock\%27s_dilemma}{Warnock's dilemma}

\subsubsection{Tip: Action with the deadline}

% https://graphthinking.blogspot.com/2017/11/collected-wisdom.html
when asking someone for help or input, specify a deadline for their response. This helps the person prioritize


\subsubsection{Tip: ask-tell-ask}

Collaborating with fellow bureaucrats who have expertise in areas you do not requires extra work. There may be differences in the words used to describe certain situations, more precision in wording that you're used to, or thinking about situations in ways you are not familiar with. In that context, to bridge the differences you can ask, tell, ask\footnote{\href{https://cepc.ucsf.edu/sites/cepc.ucsf.edu/files/Curriculum_sample_14-0602.pdf}{``The 10 Building Blocks of Primary Care: `Ask Tell Ask' Sample Curriculum''} and \href{https://www.the-hospitalist.org/hospitalist/article/125126/qi-initiatives/ask-tell-ask-simple-technique-can-help-hospitalists}{Ask-Tell-Ask: Simple Technique}}. 

The first step is asking what the other person's perspective is on the topic. This helps establish the appropriate level of nuance is and can tell you how that person frames the issue. The second step is to tell the person what you want to say. The ``tell'' step should leverage what you learned from the first ``ask'' step. Use phrasing that is consistent with what you just learned from the other person. The third step is to ask the person what they heard from you. If they are unable to tell you, you may need to refine your delivery. To improve the likelihood of success keep the content in the second step short. 

The ask-tell-ask technique can be used iteratively in the same conversation, especially in discussion complex topics with a new collaborator. Using ask-tell-ask takes long than just telling but increases the effectiveness of the communication. You also get to learn more about the other person's perspective. 


\subsubsection{Tip: Initial responsiveness and status updates}
In a bureaucracy requiring approval, or soliciting input, sometimes waiting can provide value to the person doing the waiting. The request may be overcome by events, or the person asking may remind which indicates priority

\subsubsection{Tip: Read each email/memo/report to determine the purpose }
% https://graphthinking.blogspot.com/2021/03/read-each-email-to-determine-purpose.html

\textit{Problematic behavior}: scan the text, see if there is immediate action or response needed. If no action or response is needed, go to the next email. \\
 That does not work for emails that contain logistics associated with future events. 

Instead, consider possible intentions of the person writing the email. 

\textbf{Decision needed}.  Typically includes context. \\
\textit{Action}: if the team maintains a decision log in Confluence or wiki, update that.
Response is selection of a choice.

\textbf{Meeting logistics}.\\
\textit{Action}: create or update a calendar event
Response should restate the logistics (time/date/location/purpose) to confirm

\textbf{Brainstorming}\\
may provoke a response for building on an idea
"For your situational awareness, no action needed." Notification of activity by someone else. Or change in plans. 
If needed, a correction to the described direction might trigger a response or even a meeting.

\textbf{Reference} e.g. describing a process or business workflow. Or a citation.\\
\textit{Action}: Copy process documentation to Confluence or wiki. Copy citation to bibliography
Acknowledgement response thanking the sender for the update/clarification

\textbf{setting a formal policy or issuing an informal edict}\\
\textit{Action}: move the policy/edict documentation to Confluence or Wiki
Acknowledgement response needed only if the edit is aimed at just me or the group I am leading

\textbf{Question}\\
If this is a recurring question, move to a ``Frequently Asked Questions" page on Confluence or Wiki
Response needed that provides answer or seeks clarification


Here I'm using ``action" to refer to activities outside the email channel. 

If I read email to figure out the purpose of the email, that will help me determine what action and response are relevant. 

Whether I am the only recipient or on of many receivers can change the intent of the email, and whether I'm in the ``to" or ``cc" field matters. Unfortunately, ``to" versus ``cc" are not reliable indicators since email senders do not reliably conform to the expected use. 



This categorization of text within emails is a useful natural language processing challenge for machine learning. Gmail already does some of this with identifying meeting logistics, providing reminders to follow-up, and providing reply snippets. A browser plug-in that differentiates the various purposes of text could help readers determine relevant actions and responses. 



\subsubsection{Tip: Survey stakeholders}
% https://graphthinking.blogspot.com/2016/01/how-to-solve-and-not-solve-problems.html

Suppose you are a \href{http://www.peacecorps.gov/}{Peace Corps} worker in Africa. You show up and the village doesn't have easy access to clean water. Villagers walk a long ways in dangerous areas for dirty, unsafe water. This is a very obvious problem and all the villagers agree that they don't have good water and that this problem should be fixed.

Implementing the solution would take about a week - get the equipment to the village, drill a well, build a pump.

You could take additional time and involve the villagers in this project. They could participate in getting the equipment, which should lead to a sense of ownership.
But then when the equipment shows up, they don't take action to drill the well. If the well is drilled, it soon falls into disrepair and the villagers are back to doing things they way they used to. What happened?

The villagers don't see access to clean water as the most significant issue. You came in and imposed your view of what the problem is and how to fix it. When you impose your view of what the problem is, the solution won't be adopted by villagers because they don't prioritize it.
It is better to survey the community to see how they operate. What do they think the problems are?
Both leadership and the community members need to provide priorities.

This issue is exacerbated if you come to the village as a representative of a company providing wells. You are biased when you ask, ``Do you have any problems?"

Of course the villagers have water problems which could be fixed with better wells. However, when you get into the details of placing or improving a well, they lose interest. What the community really wants is free installation, zero maintenance, easy to use, and no operational costs. That would improve their life.

When you say there's cost (both initial investment of capital and then operations/maintenance) and a learning curve associated with the solution, then the user's interest wanes -- you are presenting another cost/benefit ratio for them to evaluate. Then they ask "Can we get by without the well?" Yes, they don't need the well -- they've survived without it.

Novel solutions (in this example, drilling a well and installing a pump) have have barriers to adoption. Two barriers are the current priorities of the community and the incumbent solution/processes.

If there are problems with higher priority, the community will delay implementing your solution. That's fine if the higher-ranked priorities are bounded, but they are often not. An example of this is the following:
Suppose a person has three tasks, and you introduce a solution which is a fourth task.
If the first task is ``go from point A to point B," then that task will eventually be eliminated and there will be three remaining.
If the second task is ``secure your village," that is an unbounded task. The person won't get to or won't prioritize your low-ranked task.

How will your solution impact their higher-ranked priorities?

\ \\

\textbf{Summary of what action should be carried out} \\

As the outsider, you should help the community enumerate and document all of the problems they identify. Then you can help enumerate and document how the problems are related (dependencies). Only then can you help the community identify and document the root causes.

If the solution you, the outsider, identified really is the root cause, then the community will arrive at that independently. If that is the case, then you can enable them to implement a solution which addresses the root causes. The community will then have a sense of ownership.
