\section*{One Page Summary of This Book}

\iftoggle{glossarysubstitutionworks}{\Gls{bureaucracy}}{Bureaucracy} is pervasive because it is crucial for society. Bureaucracy is defined as the set of subjective decisions made by individual participants in an organization. The decisions originate from the 
management of \iftoggle{glossarysubstitutionworks}{\glspl{shared resource}.}{shared resources.} 
The resources can be either tangible or in the form of expertise. Bureaucracy relies on
distributed knowledge and distributed decision-making.

There are common artifacts that stem from the coordinated decision-making that is central to bureaucracy: hierarchy (which enables specialization and authority) and communication (verbal and written, whether for informal or formal meetings). 
%TODO: transition needed
Feedback loops are a crucial feature for understanding decision-making in bureaucracy.

%TODO: transition needed
\iftoggle{glossarysubstitutionworks}{\Gls{process empathy}}{Process empathy} 
is helpful in your role as a bureaucrat and when you are subject to bureaucracy. 
Process empathy is based on your understanding the \hyperref[sec:dilemma-trilemma]{dilemmas} and \hyperref[sec:unavoidable-hazards]{unavoidable hazards} of bureaucracy, recognizing internal \gls{motives} and external \gls{incentives} that individuals in each role have, and your ability to reconstruct the sequence of decisions that led to the current situation. 


Why innovation is stifled by bureaucracy and what to do about it -- \hyperref[sec:extending-Heilmeier]{extended Heilmeier questions}.
\iftoggle{haspagenumbers}{page~\pageref{sec:extending-Heilmeier}}{}
%https://en.wikipedia.org/wiki/George_H._Heilmeier#Heilmeier's_Catechism

% TODO
Processes