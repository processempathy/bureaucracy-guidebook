\section*{One Page Summary of This Book}

\iftoggle{glossarysubstitutionworks}{\Gls{bureaucracy}}{Bureaucracy} is pervasive because it is crucial for society. Bureaucracy is defined as the set of subjective decisions made by individual participants in an organization. The decisions originate from the 
management of \iftoggle{glossarysubstitutionworks}{\glspl{shared resource}.}{shared resources.} 
The resources can be either tangible or in the form of expertise. Bureaucracy relies on
distributed knowledge and distributed decision-making.

There are common artifacts that stem from the coordinated decision-making that is central to bureaucracy: hierarchy (which enables specialization and authority) and communication (verbal and written, whether for informal or formal meetings). 
%TODO: transition needed
Feedback loops are a crucial feature for understanding decision-making in bureaucracy.



A policy is what to do in a certain context -- how to make a decision. A procedure is the steps of enacting the policy.

Policies and procedures are intended to help all participants, both bureaucrats and subjects. The value of procedures is that they are predictable. There's less cognitive load for bureaucrats and less emotional burden because ``I'm just following the rules.'' For subjects the value of procedures are that requirements are clear and can be deemed fair in application. There's no need to form relationships with the people administering the shared resources.

The negative value of procedures is in exceptional contexts and contexts where relationships already exist. Then policies and procedures get in the way.

Having both relationship-based and policy based paradigms and in effect at the same time is disruptive to each paradigm.  This inability of the two approaches (systematic versus ad hoc) to coexist is why bureaucracy is consistently in turmoil. Therefore an effective bureaucrat learns to take advantage of the dynamic change. A skilled bureaucrat:
forms and maintains relationships with fellow bureaucrats and subjects,
comprehends and leverages existing policies and procedures, and 
disrupts existing procedures with innovative policies.
The wise bureaucrat knows when to employ which approach. 

%TODO: transition needed
\iftoggle{glossarysubstitutionworks}{\Gls{process empathy}}{Process empathy} 
is helpful in your role as a bureaucrat and when you are subject to bureaucracy. 
Process empathy is based on your understanding the \hyperref[sec:dilemma-trilemma]{dilemmas} and \hyperref[sec:unavoidable-hazards]{unavoidable hazards} of bureaucracy, recognizing internal \gls{motives} and external \gls{incentives} that individuals in each role have, and your ability to reconstruct the sequence of decisions that led to the current situation. 


Innovation is stifled by bureaucracy%
\iftoggle{haspagenumbers}{ (see page~\pageref{sec:innovation})}{} because individual bureaucrats have incentive to adhere to existing policies and procedures as a way to decrease their risk. Innovation requires adapting policies to new contexts and discarding irrelevant policies. 
Use the \hyperref[sec:extending-Heilmeier]{extended Heilmeier questions}%
\iftoggle{haspagenumbers}{ (see page~\pageref{sec:extending-Heilmeier})}{} to interrogate the relevance of policies in a specific context.
%https://en.wikipedia.org/wiki/George_H._Heilmeier%23Heilmeier's_Catechism


Not all challenges you encounter as a bureaucrat in an organization are attributable to bureaucracy. The distribution of work in an organization lacking hierarchy can induce latency when not everyone has all relevant skills for tasks. Specialization of skills can improve task throughput when processes are stable, but specialization can hinder task throughput when task complexity requires coordination.

