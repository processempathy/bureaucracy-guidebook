\section{Why of Communication within a Bureaucracy Matters}

Taking action informed by locally-available information can produce \hyperref[sec:failure-to-comm]{suboptimal results for participants}, as illustrated by the
\href{https://en.wikipedia.org/wiki/Prisoner\%27s\_dilemma}{Prisoner's dilemma}.
\ifsectionref
(see section~\ref{sec:failure-to-comm}). 
\fi
Communication among bureaucrats facilitates delegation of work, allocation of resources, relationship creation/maintenance, and carrying out processes. Communication has a cost for participants: time spent building and verifying consensus delays action. There is an opportunity cost when time is spent talking with people.

% https://graphthinking.blogspot.com/2021/09/why-is-everything-so-hard-in-large.html



\subsection*{Communication Reflects and Shapes Your Thinking}

Can you turn complaints into impact statements? 
\marginpar{[Tag] Actionable Advice}

\textit{Negative observation}: ``Logging into my computer takes a long time.''\\
Positive statement with explanation of impact: ``If I were able to log into my computer more quickly, then I could accomplish more tasks.''

\textit{Negative observation}: ``The team I need support from doesn't provide a ticket tracking system.''\\
\textit{Positive statement with explanation of impact}: ``If the service team I need support from provided a ticket tracking system, then I would be able to know the status of my request.''

\textit{Negative observation}: ``When I submit a ticket for a support request to the team, I don't have visibility on the status of the request.''\\
\textit{Positive statement with explanation of impact}: ``If the service team I need support from provided visibility in the ticket tracking, then I could proceed with other tasks without worrying about the status of my request.''


Here's a technique to transform generalizations to action: 
\marginpar{[Tag] Actionable Advice}
\begin{enumerate}
    \item This organization does not like blueberry pie.
    \item No one in this organization likes blueberry pie.
    \item I don't know of anyone in the organization who likes blueberry pie.
    \item I like blueberry pie, how would I find someone else in the org who likes blueberry pie.
\end{enumerate}
In the first statement we personify an organization. Organizations cannot like or dislike tangible items, so this is a meaningless statement. The second statement is slightly more precise (individual people can have likes and dislikes) though still a generalization and therefore likely inaccurate. The source of the generalization might be your lack of relationships with every member of the organization. That leads to the fourth statement which is correct, precise, and curious. 