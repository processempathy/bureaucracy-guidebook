\subsection{Why of Communication within a Bureaucracy Matters}

Taking action informed by locally-available information can produce suboptimal results for participants, as illustrated by the \href{https://en.wikipedia.org/wiki/Prisoner\%27s\_dilemma}{Prisoner's dilemma} (see \S\ref{sec:failure-to-comm}). Communication among bureaucrats facilitates delegation of work, allocation of resources, relationship creation/maintenance, and carrying out processes. Communication has a cost for participants: time spent building and verifying consensus delays action. There is an opportunity cost when time is spent talking with people.

% https://graphthinking.blogspot.com/2021/09/why-is-everything-so-hard-in-large.html



\subsubsection{Communication Reflects and Shapes Your Thinking}

Negative observation: ``Logging into my computer takes a long time.''\\
Positive statement with explanation of impact: ``If I were able to log into my computer more quickly, then I could accomplish more tasks.''

Negative observation: ``The team I need support from doesn't provide a ticket tracking system.''\\
Positive statement with explanation of impact: ``If the service team I need support from provided a ticket tracking system, then I would be able to know the status of my request.''

Negative observation: ``When I submit a ticket for a support request to the team, I don't have visibility on the status of the request.''
Positive statement with explanation of impact: ``If the service team I need support from provided visibility in the ticket tracking, then I could proceed with other tasks without worrying about the status of my request.''
