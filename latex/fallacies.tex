\section{Bureaucratic Fallacies\label{sec:fallacies}}
\index{bureaucratic fallacy}

Discussions about bureaucracy by non-experts often rely on common conceptions that are \gls{thought-terminating}. Identifying these enables you to understand both why the fallacy is attractive during a discussion with fellow bureaucrats and how each idea is incomplete.

These fallacies may at first feel right but are misleading. In contrast, there are  
\hyperref[sec:unavoidable-hazards]{unavoidable hazards}
\iftoggle{haspagenumbers}{(see page~\pageref{sec:unavoidable-hazards})}{} that may feel bad but reveal underlying truths.

\ \\
\begin{samepage}
\textit{Bureaucratic fallacy}: \textbf{Bureaucracy is bad}. \\
\index{bureaucratic fallacy!bureaucracy is bad}
\textit{Why this feels true}: When a person subjected to bureaucracy has a negative experience, the easiest attribution is to the least understood aspect -- the bureaucracy.\\
\textit{What this is missing}: 
\iftoggle{glossarysubstitutionworks}{\Gls{bureaucracy}}{Bureaucracy}
 as defined in this guide is neither good nor bad. Bureaucracy is merely a way of managing  resources shared amongst a community. 
 \end{samepage}

\ \\
\begin{samepage}
\textit{Bureaucratic fallacy}: 
\textbf{There is no point in planning since everything (staffing, funding, purpose, scope) is always changing.}\\
\index{bureaucratic fallacy!no point in planning}
\textit{Why this feels true}: Change can feel disorienting, especially when it is unexpected. A change of the assumptions for a plan may make the plan less relevant. \\
\textit{What this is missing}: Preparing for change and thinking ahead about contingencies enables effective use of resources. Have a vision and work towards it while accounting for and adapting to change. This approach requires extra work, some of which will be left unused. See Dilemma \ref{table:dilemma-personal-emergencies-vs-ignore}\iftoggle{haspagenumbers}{ on page~\pageref{table:dilemma-personal-emergencies-vs-ignore}.}{.}
\end{samepage}

\ \\
\begin{samepage}
\textit{Bureaucratic fallacy}: \textbf{Bureaucracy is an aberration, a mistake, due to poor planning or incompetent participants}. \\
\index{bureaucratic fallacy!bureaucracy is a mistake}
\textit{Why this feels true}: Mistakes are made, poor or insufficient planning does happen, and some participants are incompetent.\\
\textit{What this is missing}: Bureaucracy occurs even if no mistakes are made, effort is spent on effective planning, and participants are competent. That's because, in the use of distributed knowledge and distributed decision-making, bureaucrats face \hyperref[sec:dilemma-trilemma]{dilemmas}.\marginpar{See page~\pageref{sec:dilemma-trilemma}.}%
%\ifsectionref
%(see section~\ref{sec:dilemma-trilemma}).
%\fi
\end{samepage}

\ \\
\begin{samepage}
\textit{Bureaucratic fallacy}: \textbf{Bureaucracy is inefficient}. \\
\index{bureaucratic fallacy!bureaucracy is inefficient}
\textit{Why this feels true}: Expressed by both subjects and bureaucrats who observe seemingly wasteful processes.\\
\textit{What this is missing}: If bureaucracy were truly inefficient (not allocating resources efficiently), then in a competitive environment it would be replaced by a more efficient approach. The key is to ask, ``Efficient with respect to what metric?'' The metric of money, time, number of people, stability, or robustness to perturbation?  Second, what would motivate improved efficiency? Without incentives, change is less likely. 
\end{samepage}

\ \\
\begin{samepage}
\textit{Bureaucratic fallacy}: \textbf{Bureaucracy is due to malfeasance.}\\
\index{bureaucratic fallacy!bureaucracy is due to malfeasance}
The specific number of malicious bureaucrats in variations on this fallacy ranges from ``all of the participants'' to ``just enough to be problematic.'' \\
\textit{Why this feels true}: There are bad actors present in any system comprised of humans. \\
\textit{What this is missing}: Most participants are earnestly trying to help make a positive contribution, even though that can be hard to see from the view of subjects or even other bureaucrats. Processes like isolation or promotion exist within bureaucracy to deal with malicious bureaucrats.
\end{samepage}

\ \\
\begin{samepage}
\textit{Bureaucratic fallacy}: \textbf{Bureaucracy is a sign of decay from within the organization.} \\
\index{bureaucratic fallacy!bureaucracy indicates decay}
\textit{Why this feels true}: Relationships within an organization have a half-life and require ongoing investment to renew. At the same time, new bureaucratic processes are constantly being developed by other bureaucrats. The number of processes increases as the organization ages. Bureaucracy seems to arise without effort and countering it takes effort.  \\
\textit{What this is missing}: Bureaucracy unavoidably emerges in every organization because coordination is required. The negative connotation of decay should be replaced with a sense of neutral evolution.
\end{samepage}

\ \\
\begin{samepage}
\textit{Bureaucratic Fallacy}: \textbf{If the response to a request I make can't be expedited, my request must not be important}.  \\
\index{bureaucratic fallacy!importance is measured by response latency}
\textit{Why this feels true}: Other people would show they care about what I am working on by prioritizing things I am dependent on.\\
\textit{What this is missing}: When everything gets prioritized, that's the same as nothing getting priority.
\end{samepage}

\ \\
\begin{samepage}
\textit{Bureaucratic Fallacy}: \textbf{The expected duration of a task is how long it would take one person to accomplish}.  \\
\index{bureaucratic fallacy!task duration for one person}
\textit{Why this feels true}: When I imagine carrying out a task, the default is a story with one character. \\
\textit{What this is missing}: This narrative fails to account for the overhead of interaction among participants and delays due to asynchronous dependencies. This is described well in Brook's \textit{Mythical Man-Month}~\cite{1975_brooks} and modeled 
in \hyperref[sec:work-distribution]{the appendix}%
\iftoggle{printedonpaper}{ on page~\pageref{sec:work-distribution}.}{.}
\end{samepage}

\ \\
% https://graphthinking.blogspot.com/2019/08/two-misleading-simplifications-when.html
\begin{samepage}
\textit{Bureaucratic Fallacy}: \textbf{When developing or altering policy, focus on the average or majority (to the exclusion of outliers)}. \\
\index{bureaucratic fallacy!ignore outliers}
\textit{Why this simplification is misleading}: Sometimes outliers are not just more of the same; they alter the outcome. During the transition from horses-for-transportation to cars, cars could initially have been considered outliers. 
\end{samepage}

\ \\
\begin{samepage}
\textit{Bureaucratic Fallacy}: \textbf{People learn from their mistakes}. \\
\index{bureaucratic fallacy!people learn from their mistakes}
\textit{Why this feels true}: There's an optimistic desire for this to be true. \\
\textit{What this is missing}: 
People repeat mistakes without noticing. People do not naturally reflect on their failings in a constructive way and then apply insights to future situations.
People \textit{can} learn from their mistakes. Doing so requires a low latency feedback loop and incentive to change. In bureaucracies feedback loops are weak so learning may not happen.
\end{samepage}

\ \\
\begin{samepage}
\textit{Bureaucratic Fallacy}: \textbf{Processes are serial}.\\
\index{bureaucratic fallacy!processes are serial}
A conventional approach to process design is a sequence of tasks. As an example, consider approval chains. \\
\textit{Why this feels true}: Serial processes are easier to understand. \\
\textit{What this is missing}: Some tasks that are independent can be carried out concurrently; see the section on \hyperref[sec:reducing-overhead]{reducing overhead}%
\iftoggle{haspagenumbers}{ on page~\pageref{sec:reducing-overhead}.}{.}
\end{samepage}

\ \\
\begin{samepage}
\textit{Bureaucratic Fallacy}: \textbf{Hard work creates results}.\\
\index{bureaucratic fallacy!hard work creates results}
\textit{Why this feels true}: Some results do require hard work. The alternative (results arise serendipitously or with little effort) is not inspiring. \\
\textit{What this is missing}: Hard work can be invested on wasteful effort. Don't confuse being busy with being productive. Sometimes insight is more useful than hard work. 
\end{samepage}

%\ \\

% NOT USEFUL
%\textit{Bureaucratic Fallacy}: \textbf{Motivations for bureaucrats are categorized as individualistic, tribal, organizational, societal, or humanity}.\\



\ \\
\begin{samepage}
\textit{Bureaucratic Fallacy}: 
\textbf{You cannot pay a little and get a lot}; see the \href{https://en.wikipedia.org/wiki/Common_law_of_business_balance}{Common law of business balance}. 
\index{Wikipedia!Common law of business balance@\href{https://en.wikipedia.org/wiki/Common_law_of_business_balance}{Common law of business balance}}
\marginpar{$>>$ Folk Wisdom}%
\index{folk wisdom!Common law of business balance@\href{https://en.wikipedia.org/wiki/Common_law_of_business_balance}{Common law of business balance}} \\
\index{bureaucratic fallacy!cannot pay a little and get a lot}
\textit{Why this feels true}: If small investments made a big difference we'd already be making the investment.\\
\textit{What this is missing}: This doesn't account for creative solutions and ignores \href{https://en.wikipedia.org/wiki/Nudge_theory}{nudge theory}
\index{Wikipedia!Nudge theory@\href{https://en.wikipedia.org/wiki/Nudge_theory}{Nudge theory}}
from behavioral economics. 
\end{samepage}

\ \\

When you are reasoning about bureaucratic systems, there may be a conclusion that is concise and feels explanatory. Then you should try to come up with counter-examples, either logically or from experience.  

Some fallacies are based on an expectation that other people should be more like what you imagine your best self to be. That mindset fails to account for your shortcomings and the diversity of other bureaucrats. 

