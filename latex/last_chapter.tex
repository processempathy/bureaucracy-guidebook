You could operate within a bureaucracy and entirely focus on ``just do the job.'' You can be more effective in your job by understanding the role of engaging with other people -- peer bureaucrats, supervisors, and subjects. %In this book a third option of accounting for emergent phenomena has been argued for. 

Learning bureaucracy as a skill doesn't mean you can ignore personalities of individuals. Bureaucracy as a skill is in addition to being a good person, being an effective member of a team, being a good project manager, having technical skills, etc. The distinction from those aspects is that a bureaucrat understands the complications and constraints of their environment and then can more effectively operate within those conditions.


\section{Skills of an effective bureaucrat}

The narrow scope of your role (which hopefully leverages your education/training) does not capture all relevant aspects of your job. 
\begin{itemize}
    \item You are able to facilitate meetings. This means agenda, providing rules on interaction (raising hands), taking notes, and follow-up.
    \item You participate in meetings, whether that means actively contributing or intentionally supporting other attendees. You leverage relationships with other attendees. 
    \item You have focused educational efforts on written communication (emails, text-based chats, reports). Your writing empathizes with readers, captures relevant context, is concise, and is clearly worded.
    \item In your role as bureaucrat you leverage project management skills: you have a vision, you make and share plans, all while building consensus with stakeholders.
    \item You have learned negotiation skills \cite{1982_Cohen} that improve your interactions and outcomes.
\end{itemize}

The attitude of the effective bureaucrat  is that of realistic optimism. A realistic optimist will occasionally be wrong but makes progress, whereas a pessimist will be right (because the prediction is self-fulfilling) but not make progress.
