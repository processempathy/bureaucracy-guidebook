\chapter{Skills of an Effective Bureaucrat\label{sec:last-chapter}}

You could operate within a bureaucracy and entirely focus on just doing the job you were hired for and ignore administrative distractions. Or you can be more effective in your job by understanding the role of engaging with other people -- peer bureaucrats, supervisors, and subjects. 

Learning bureaucracy as a skill doesn't mean you can ignore personalities of individuals. Bureaucracy as a skill separate from and in addition to being a good person, being an effective member of a team, being a good project manager, being a good product owner, skillful writing, excellent verbal communication skills, applying technical skills, etc. The distinction from those aspects is that as a bureaucrat you understand the complications and constraints of your environment and then can more effectively operate within those conditions.

If you are new to being a bureaucrat, then this book armed you with understanding your environment. On your first day of employment you are poised to frame incoming information constructively.
If you are an experienced bureaucrat, it is not too late to improve. Even on your last day of employment you can learn and be more effective.

In this book I described bureaucracy and provided options for action. If you are able to apply these generalized perspectives to your specific situation, you are applying the paradigm developed throughout this book. Concepts like learning the history of your situation, identifying and engaging stakeholders to learn their perspective. Brainstorming the incentives of the individuals involved, and listing what levers they have for action. What are the dilemmas? Are there feedback loops?


%\section{}

The narrow scope of your role (which hopefully leverages your education or training) does not capture all relevant aspects of your job. 
Based on the definition of \gls{bureaucracy}, the critical aspects of success are having knowledge, sharing your knowledge, leveraging the knowledge of others, effective communication, working well with other people, understanding the role of both processes and social influence, and how those interplay. 


The attitude of the effective bureaucrat  is that of realistic optimism. A realistic optimist will occasionally be wrong but makes progress, whereas a pessimist will be right (because the prediction is self-fulfilling) but not make progress.
