\subsection*{Fundamental: Written Communication\label{sec:written-communication}}

As with hierarchy, written communication is not required for bureaucracy. However, written communication is extremely common and can be helpful. %An artifact of interaction gains value beyond the content.

Paperwork,  \href{https://en.wikipedia.org/wiki/Red_tape}{red tape},
\index{Wikipedia!red tape@\href{https://en.wikipedia.org/wiki/Red_tape}{red tape}}
and modern electronic forms are closely associated with bureaucracy.
Digital reports, spreadsheets, and emails are the current technological artifacts of an organization's bureaucracy. A written record creates evidence about policies, justifications, and decisions regardless of format. %Existence of a record can be used for good or for harm.
Written records are burdensome to create and maintain and search, but they are not intrinsically good or bad. 

Becoming skilled at creating written records is crucial for being an effective bureaucrat. You may have some education on spelling, grammar, and composing essays, but the artifacts of bureaucracy have distinct best practices. For example, plagiarism can be good -- you're being consistent and efficient.  
Chapter~\ref{sec:communication-within-bureaucracy} 
\marginpar{See page~\pageref{sec:communication-within-bureaucracy}.}
provides advice on effective verbal and written communication in the context of a bureaucratic organization. 