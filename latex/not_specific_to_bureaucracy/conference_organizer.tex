\subsection{How to be a Successful Conference Organizer}

% https://graphthinking.blogspot.com/2011/11/how-to-organize-conference.html

% This content is 
% * a checklist 
% * contains no novel insight
% * is written based on my experience as an attendee (rather than some meaningful authority)

Plan to fail: When X fails, what is the backup?


Checklist for before the conference starts
\begin{itemize}
    \item determine theme
    \item check that dates do not conflict
    \item recruit speakers (experts in the field).
Determine keynote speaker
\item menu plan. 
Vegetarian and non-veg items (account for dietary restrictions)
\end{itemize}




Relevant parties:
\begin{itemize}
    \item participants. Attending to learn from the talks. What are current standards, what are the new developments?
    \begin{itemize}
        \item paying to attend, or receiving funding to attend. 
        \item special: session host
    \end{itemize}
    \item speakers. Giving talk to increase prestige, broadcast learned knowledge. Educate others.
    \begin{itemize}
        \item Volunteer or speaking pay. 
        \item special: keynote speaker
    \end{itemize}
    \item vendors. Advertise and sell product, gain new users, interact with existing customers. Pay for attendance, booth space. May get advertising space in program, on website.
    \item logistical support staff. Invisibly ensure a smooth conference. 
Paid or volunteer
\end{itemize}


location of conference

conference = attendees are also presenters

summer school = attendees and experts are distinct

Maximum daily schedule:
Wake, learn, coffee, learn, lunch, learn, snack, learn, supper

\textbf{Before the conference}\\

Scheduling the conference
Does the conference date conflict with the schedule for the audience? School, holidays, other conferences.

Finding good speakers
Need smart people who make good presentations on relavent topics.

Advertise
\begin{itemize}
\item Facebook group
\item Facebook event
\item Twitter updates
\item Youtube previews
\item Conference website
    \begin{itemize}
      \item Schedule of speakers, location of events
    \item Where to book hotel
    \item Transportation to/from conference (local bus or taxi)
    \end{itemize}
\end{itemize}
Minimum facilities
\begin{itemize}
\item bathrooms
\item presentation area
\item snack/conversation area
\item registration
\item speaker preparation/rest room
\end{itemize}

Catering must be professional. Must include liquids, especially coffee and at least one alter-
native.

Audio/Visual (AV) setup
for an audience of more than 20, a microphone is needed
Have a backup projector ready
Power standards
If participants will be using equipment from a country with a different electrical standard, be sure converters are available. Instead of each person using their own converter, use one converter and an outlet strip.


specify rules for the badge explicitly by printing them and including them with the badge
\begin{itemize}
\item badge is non-transferable
\item badge must be visible at all times
\item photo identification required upon request
\item tampering with badge invalidates usage
\item cost for badge replacement is \$XX
\end{itemize}
Other rules:
\begin{itemize}
\item all seating is first-come, first serve basis
\item no refunds
\item failure to comply with staff and security directions will result in badge revocation
\end{itemize}
offer coat check if weather is cold



Before participants arrive
Test wireless internet connection. Know the Network Administrator
Send an email with the following information
website for conference
transportation from airport to conference location
expected temperature range and weather type
dress style (business casual, formal)
If tutorials require computation, minimize dependence on remote computers. Network will be flaky, remote computer will be overwhelmed.

When participants arrive
Name tag, map, printed schedule

During the conference

Basics:
any medical problems must be addressed immediately. Health and safety of participants is primary concern
everyone getting food, sleep


For each presentation
Have water available for the speaker near the podium
Each speaker needs to have host. The host cuts off the speaker at the appropriate time. Otherwise the audience and next speaker are screwed by the overage. Experts like to talk about their work.

Increasing interaction
\begin{itemize}
    \item Informal coffee breaks. Allows for existing contacts to improve
    \item Each participant gives an "elevator talk." 1 minute, 1 slide (or with nothing).
    \item "Speed dating:" one-on-one interaction. Trade elevator speeches and business cards.
    \item "Think, pair, share." Give participants a scenario or question to think about. Then break into small groups (2 to 5) to discuss the topic. Finally, have groups share their results with other groups. Sets of groups can have different topic. For example, 30 people split into groups of 5, with three scenarios to discuss (two groups per scenario). If a problem is able to be broken into 3 parts (i.e., theory, experiment, marketing), then each set of two groups can work on a different part.
    \item If meals are included in the conference, force mingling among different social cliques. Instead of different disciplines at different tables, group people (assigned seating) by favorite color. See Birds of a feather
\end{itemize}

After the conference
Post presentations, conference materials online

Send thank you notes to everyone who helped
\begin{itemize}
    \item hotel/convention center
    \item attendees
    \item presenters
    \item organizing and support staff
\end{itemize}
Send surveys speakers and to audience to gauge success


prepare report to funding agency