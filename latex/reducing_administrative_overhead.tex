\section{How to Reduce Administrative Overhead\label{sec:reducing-overhead}}

You can read this list from a position of authority to take action, or you can read this list as an advocate for improvement. 

% https://graphthinking.blogspot.com/2018/08/how-to-decrease-bureacracy.html
In isolation, none of these ideas should be surprising. Enacting each idea requires ongoing investment. 
\begin{itemize}
    \item \textit{Tip}: Automate recurring decision processes. Automation can decrease work and decrease the risk of inconsistent outcomes.
\item \textit{Tip}: Where automation is infeasible, make customer advocates with end-to-end authority available. 
This decreases improves the customer's experience of processes.
\item \textit{Tip}: Make processes transparent to participants. 
This improves the customer's understanding of processes.
\item \textit{Tip}: Make information discoverable (e.g., via search engine) .
\item \textit{Tip}: Make information directly available, rather than mediated by a person.
\item \textit{Tip}: After an interaction is completed, summarize the steps and outcome for the participant. 
\item \textit{Tip}: When a process fails the needs of a participant, investigate the failure and improve the process\footnote{See the Wikipedia entry on \href{https://en.wikipedia.org/wiki/Continual_improvement_process}{Continual improvement process}}. 
\item \textit{Tip}: Make the goals and priorities of the organization clear to all stakeholders.
\item \textit{Tip}: Define measurable standards of performance, both for individual bureaucrats and for teams.
\item \textit{Tip}: Train bureaucrats how to engage participants effectively; these interactions determine the culture.
\item \textit{Tip}: Train bureaucrats by addressing their immediate problems (e.g., through mentorship).
\item \textit{Tip}: Make employment desirable to people who have desirable characteristics (e.g., educated candidates).
\item \textit{Tip}: Enhance accountability to peers (e.g., peer review of actions and outcomes).
\item \textit{Tip}: Make sure that incentives for organizations and individual people encourage improvement rather than maintenance of status quo.
\item \textit{Tip}: Decision making should be pushed down the hierarchy to the practitioner~\cite{2013_Marquet}.
\item \textit{Tip}: When decision making requires cross-organization interaction, form a team of practitioners.
\item \textit{Tip}: Bosses should share workload with their team to gain practical exposure to current challenges.
\item \textit{Tip}: Seek feedback from process participants; then provide status updates on the implementation of changes. This provides a feedback loop and is a form of accountability.

% https://graphthinking.blogspot.com/2018/08/sequential-approval-chains-are.html
\item \textit{Tip}: Replace sequential approval chains with a concurrent review process. Sequential approval reduces the number of ideas reaching later approvers by filtering good ideas as well as bad ideas. Approvers may perform independent (redundant) due diligence and end up disagreeing. 
\end{itemize}
