\section{How to Reduce Administrative Overhead\label{sec:reducing-overhead}}
\sectionmark{Reduce Overhead}

\textit{You can read this list from a position of authority to take action, or you can read this list as an advocate for improvement.}

% https://graphthinking.blogspot.com/2018/08/how-to-decrease-bureacracy.html
In isolation, none of the following suggestions should be surprising. However, enacting each idea requires ongoing investment. A champion is needed to actively promote and support each idea.
\index{list of tips!reduce administrative overhead}
\begin{itemize}
    \item \textit{Tip}: Automate recurring decision processes. Automation can decrease work and decrease the risk of inconsistent outcomes. Automation is a way of encoding bureaucratic processes that involve subjective decisions into repeatable steps.
    
    \textit{Why this doesn't happen}: Bureaucrats don't know how to automate processes, or there isn't infrastructure available to support automation. 

    \item \textit{Tip}: Where automation is infeasible for a process involving multiple bureaucrats, make advocates with end-to-end authority available to the subject. 
This improves the subject's experience of a process. A benefit to the organization is that the advocate might see opportunities for improvement not apparent to a bureaucrat responsible for one step in the process.

    \textit{Why this doesn't happen}: The organization may lack incentives to provide customer advocates. When there is a customer advocate, that person may not have the skills or time to exert end-to-end oversight of the process.

    \item \textit{Tip}: Make processes transparent to participants. 
This improves the subject's understanding of a process. Transparency can help with trustworthiness if the reasoning for actions is defensible.

    \textit{Why this doesn't happen}: Transparency requires technology that is not central to the process. Transparency costs money that could otherwise be spent on more central tasks in the organization. Transparency can provide data that is then used against the organization. 

    \item \textit{Tip}: Make information discoverable (e.g., via search engine).

    \textit{Why this doesn't happen}: Technology that isn't the core strength of the organization is a distraction from the central work and a cost (in terms of staff and money).
    
    \item \textit{Tip}: Make information directly available, rather than mediated by a person. 

    \textit{Why this doesn't happen}: Having to talk to a person reinforces the importance of that person. The data owner feels needed, and that can be emotionally rewarding. Whereas providing an API is less personal. 
    
    \item \textit{Tip}: After an interaction is completed, summarize the steps and outcome for the participant. 

    \textit{Why this doesn't happen}: Writing notes takes time away from accomplishing work. The written notes may not adequately capture nuances.
    
    \item \textit{Tip}: When a process fails the needs of a participant, investigate the failure and improve the process. Apply the practice of \href{https://en.wikipedia.org/wiki/Continual_improvement_process}{continual improvement}.    \index{Wikipedia!\href{https://en.wikipedia.org/wiki/Continual_improvement_process}{Continual improvement process}}

    \textit{Why this doesn't happen}: A way to detect failures is needed. Why would that exist? A goal of improvement sounds good, but what incentive does the bureaucrat have to drive improvement?
    
    \item \textit{Tip}: Make the goals and priorities of the organization clear to all stakeholders. This enables accountability by allowing a participant to point to dissonance.  

    \textit{Why this doesn't happen}: Sharing goals enables accountability. 
    
    \item \textit{Tip}: Define measurable standards of performance, both for individual bureaucrats and for teams. Again, accountability is a feedback loop that needs a starting point for identifying dissonance.

    \textit{Why this doesn't happen}: Carrying out measurements and evaluating the results takes work. 
    
    \item \textit{Tip}: Train bureaucrats on how to engage participants effectively; these interactions determine the culture of an organization. 

    \textit{Why this doesn't happen}: If training is pursued at all, justifying domain-specific technical training is easier. 
    
    \item \textit{Tip}: Train bureaucrats (e.g., formal education or through mentorship).

    \textit{Why this doesn't happen}: Formal training is a cost with an unquantified benefit. Mentorship doesn't result in promotion. The attention of experienced mentors is a zero-sum investment that takes away from time spent doing measurable work.
    
    \item \textit{Tip}: Make employment desirable to people who have desirable characteristics (e.g., educated candidates).

    \textit{Why this doesn't happen}: Hiring less expensive candidates is easier to justify. 
    
    \item \textit{Tip}: Enhance accountability to peers (e.g., peer review of actions and outcomes).

    \textit{Why this doesn't happen}: Accountability is relevant if improvement is desired and if you think your peers are in a position to provide useful feedback. 
    
    \item \textit{Tip}: Look for ways to encourage organizations and individual bureaucrats to improve and take risks rather than maintain the status quo.

    \textit{Why this doesn't happen}: Deviating from the status quo means taking risk. Specifically risk to incumbent power. 
    
    \item \textit{Tip}: Decisions should be pushed down the hierarchy to the practitioner. This is exemplified by Marquet's book \textit{Turn the Ship Around}~\cite{2013_Marquet}.\footnote{See Wikipedia entry on \href{https://en.wikipedia.org/wiki/Participative_decision-making_in_organizations}{Participative decision-making in organizations}.
    \index{Wikipedia!\string\href{https://en.wikipedia.org/wiki/Participative_decision-making_in_organizations}{Participative decision-making in organizations}}}

    \textit{Why this doesn't happen}: Power comes from decision-making. Pushing decisions down means letting go of power.
    
    \item \textit{Tip}: When making a decision requires cross-organization interaction, form a team of practitioners.

    \textit{Why this doesn't happen}: Relying on practitioners to coordinate takes management out of the critical path. 
    
    \item \textit{Tip}: Supervisors should share their workload with their team to gain practical exposure to current challenges.

    \textit{Why this doesn't happen}: The skills a supervisor has may not intersect with work being done by team members. 
    
    \item \textit{Tip}: Seek feedback from process participants, then provide status updates on the implementation of changes. This provides a feedback loop and is a form of accountability.

    \textit{Why this doesn't happen}: Gathering feedback is an opportunity cost for getting work done. If \href{https://en.wikipedia.org/wiki/Continual_improvement_process}{continual improvement}
    \index{Wikipedia!\href{https://en.wikipedia.org/wiki/Continual_improvement_process}{continual improvement}}
    isn't a basis for promotion of bureaucrats, there's no little motive to invest effort.
    
% https://graphthinking.blogspot.com/2018/08/sequential-approval-chains-are.html
    \item \textit{Tip}: Replace sequential approval chains with a concurrent review process. Sequential approval reduces the number of ideas reaching later approvers by filtering good ideas as well as bad ideas. Approvers may perform independent (redundant) due diligence and end up disagreeing. 
    
    \textit{Why this doesn't happen}: Concurrent approval has the risk of unnecessary work (someone says yes after someone else says no or asks for more justification) and likely conflict when two reviewers disagree. 
\end{itemize}
