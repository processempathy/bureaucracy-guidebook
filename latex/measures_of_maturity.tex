\section{Measuring Bureaucratic Maturity}
% https://graphthinking.blogspot.com/2021/07/three-measures-of-bureaucratic-maturity.html

Bureaucratic maturity of bureaucrats in an organization can be broken into three stages of behavior: complaining (acknowledges an issue but doesn't seek to resolve), seeing opportunity (``what if we..."), and finally enacting 
\href{https://en.wikipedia.org/wiki/Nudge_theory}{nudges} (ways to alter behaviors) or altering incentives.
\index{Wikipedia!Nudge theory@\href{https://en.wikipedia.org/wiki/Nudge_theory}{Nudge theory}}\iftoggle{WPinmargin}{\marginpar{$>$Wikipedia: Nudge theory}}{}

The first and most widespread behavior is to see a problem and then complain about the situation. This indicates an awareness of the environment but no creativity regarding what to do about the problems.

The second, less common behavior, is to see a problem and recognize the situation as an opportunity. This requires creativity and reframing. 

The third behavior is to see a problem and then alter the situation toward a vision. 
The vision could be in the form of a long-term (temporally distant) outcome, or the vision could be of immediate multi-party cooperation. 
Changes could be through either direct action or by influencing others. 


An individual bureaucrat can show one or more of these behaviors. 
That is, being capable of the third behavior still allows the person to complain about other topics. 

There is not a specific amount of experience within the organization needed to arrive at any of these three paradigms. A holistic understanding of the system helps.
