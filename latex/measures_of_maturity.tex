\section{Measuring Bureaucratic Maturity of Bureaucrats}
% https://graphthinking.blogspot.com/2021/07/three-measures-of-bureaucratic-maturity.html

Bureaucratic maturity of bureaucrats in an organization can be broken into three behaviors: 

The first and most wide spread behavior is to see a problem and then complain about the situation. This indicates awareness of the environment but no creativity regarding what to do about the problems.

The second, less common behavior, is to see a problem and recognize the situation as an opportunity. This requires creativity and reframing. 

The third behavior is to see a problem and then nudge the situation towards a vision. The nudges could be through either direct action or by influencing others. 



An individual bureaucrat can show one or more of these behaviors. That is, being capable of the third behavior does not mean the person lacks the ability to complain on other topics. 

There is not a specific amount of experience within the organization needed to arrive at any of these three paradigms. A holistic understanding of the system certainly helps.

The ``vision" of the third behavior could be in the form of a long-term (temporally distant) outcome, or the vision could be of immediate multi-party cooperation. 