\documentclass{book}

\usepackage{hyphenat} % http://www.ctex.org/documents/packages/special/hyphenat.pdf

\usepackage{setspace}\onehalfspacing\frenchspacing\flushbottom\sloppy

\usepackage{hyperref}
\hypersetup{
    colorlinks=true,
    linkcolor=blue,
    filecolor=magenta,      
    urlcolor=cyan,
    pdftitle={Overleaf Example},
    pdfpagemode=FullScreen,
    }

% https://en.wikibooks.org/wiki/LaTeX/Glossary says
% "\usepackage{glossaries} and \makeglossaries in your preamble (after \usepackage{hyperref} if present)"

% https://www.overleaf.com/learn/latex/Glossaries
\usepackage[toc]{glossaries}

\makeglossaries % The command must be before the first glossary entry.

% https://en.wikibooks.org/wiki/LaTeX/Glossary says
% "define any number of \newglossaryentry and \newacronym glossary and acronym entries in your preamble"
% see https://en.wikibooks.org/wiki/LaTeX/Glossary


\newglossaryentry{bureaucracy}{
    name=bureaucracy,
    description={definition here}
}

\newglossaryentry{visible bureaucracy}{
    name=visible bureaucracy,
    description={procedures and processes are written down and can be discovered by stakeholders}
}
\newglossaryentry{invisible bureaucracy}{
    name=invisible bureaucracy,
    description={procedures and processes are known to some stakeholders and are conveyed verbally to some of the other stakeholders.}
}

\newglossaryentry{process}{
name=process,
description={a task broken into a specified set of subtask dependencies.}
}

% https://graphthinking.blogspot.com/2021/07/bureaucracy-book-outline.html
\newglossaryentry{bureaucrat}{
    name=bureaucrat,
    description={a person responsible for subjective implementation of someone else's intent, with unquantifiable results. Examples of a bureaucrat role: teacher, police, government employee. Not bureaucrats: factory line worker, student}
}

\title{Bureaucracy Guidebook: How to be an effective bureaucrat}
\author{Ben Payne}
\date{\today}

\begin{document}

\maketitle
\frontmatter % the front of the book has roman numerals

\thispagestyle{empty}

Copyright \copyright 2022 Ben Payne

Creative Commons \href{https://creativecommons.org/licenses/by-nc-nd/4.0/}{Attribution-NonCommercial-NoDerivs}

CC BY-NC-ND
%\clearpage

\thispagestyle{empty}

Thank you to my coworkers. Our interactions helped me learn how to be a better bureaucrat.%\clearpage

\chapter*{Foreword}% * excludes from Contents)
When a person has a positive experience engaging with bureaucracy, positive attribution is made to the people involved. Or ease of a solution makes the bureaucracy less visible and the solution seems obvious. 

When a person has a negative experience with bureaucracy, complaints are about the incompetence of the people involved, or the incomprehensibleness of the system. Don't these bureaucrats know how to do their job? Why isn't the solution obvious? Why does this system not work for me?

% Who this book is for

% from https://graphthinking.blogspot.com/2021/07/bureaucracy-book-outline.html
This book is for you if you are curious about bureaucracies, or you are thinking about working as a bureaucrat, or you are employed as a bureaucrat, or your job is shifting to be more bureaucratic. If you don't think of yourself as a bureaucrat, or if the term bureaucrat has negative connotations, I hope to change your mind on this vital topic. 


% What you should expect reading this book: 
The purpose of this book is to decrease surprise and arm you (both emotionally and intellectually) for the toil of being a bureaucrat. 

This book does not have a narrow focus on one topic like leadership, managing a team, being a team member, planning, time management, project management, advancing your career, or self-improvement. Some lessons may apply in those domains.

% What is the benefit of reading this book?
As a result of reading this book, you will be better able to recognize and navigate complex professional environments, both within your career and outside of work. The perspectives offered in this book can benefit you directly, whether by promotion of title or increase in pay; successful completion of a project; or through decreased stress of understanding how the world works.

There's harm in not recognizing yourself as a bureaucrat, as the role and responsibilities are distinct

Automation and computers will not eliminate or decrease bureaucracy. They merely obfuscate the processes and make negotiation more challenging. 

% my experience
% I wrote this book for a younger version of me.
 I was sufficiently self-aware when I first started my job in a large organization to recognize I didn't know much about working in that environment. Over the years I learned from my mistakes by reflecting on my (in)actions and the consequences. This approach has been an expensive education: mistakes delay progress and damage relationships.


% Caveats
Simplifying to "this interaction is characterized merely as human relations" is an easier perspective. However, that misses emergent phenomena. 

There's a risk of overanalysis. Sometimes a pipe is just a pipe. Avoiding conjecture about conspiracy and malice is a difficult boundary when insufficient information is available. 

I recognized the importance of navigating bureaucracy early in my career, but the insights were most clear when I entered middle management. 

My experiences cannot be generalized to every situation. Some of the observations here may be analogous to your context if you squint hard. 

Nothing in this book is domain specific, nothing is tied to engineering of products, and nothing is applicable solely in science research or policy development. While this material is intended to be timeless and generic, it is culturally specific to the USA. As a privileged white male, I did not encounter systemic hurdles in my career so there are blindspots not addressed in this book. 

% ? 
There are no alternatives to bureaucracy, so gaining skills in navigating bureaucracy are helpful. 

% Source of this content: 
This material is based on personal experience, reading published materials, and anecdotes from other people. No surveys were taken to support the claims made. No double blind experiments were conducted. 

% How the book should be read: 
Reading this book front-to-back is feasible. Each section is intended to be stand-alone. The book is intended to spark contemplation. 


% as per https://tex.stackexchange.com/q/393238/235813
\begin{flushright}
Ben Payne\\
\today\\
USA
\end{flushright}


%\clearpage


\tableofcontents

\mainmatter % the main part of the book will have standard pages



\chapter{Introduction to Bureaucracy}
% essentials

\section{Fundamentals of Bureaucracy}

\subsection{What is bureaucracy?}
While you may know it when you see it or experience it, for this book a definition is useful. 
\gls{bureaucracy} is coordination of stakeholders. This concept is most visible for complex, long lasting, and recurring situations involving many people. The apparent friction can be lower when there are only a few people involved ("I'm just talking to my collaborator"), but there is a continuous gradient. 

\subsection{What does Bureaucracy imply for you?}
Bureaucracy is neither good nor bad. Bureaucracy is not tied to politics, or any specific institution (corporations, governments, academics). Bureaucracy is not defined to be efficient nor, does it have to be inefficient. Bureaucracy is not restricted to paperwork, or record keeping, or quantification, or gathering metrics. 

Bureaucracy is about delegation of control, communication, decision making, coordination, and processes. In that context, wouldn't it be useful to be skilled at bureaucracy? 

\subsection{Why does bureaucracy exist? Can't we just do the work?}

The minimal scenario to start from is to imagine a single person working on a single task that does not last long (a few minutes), is relatively easy (cognitively and physically and emotionally), and does not recur. Most of what you do occurs outside those limits and thus incurs some concept of \gls{process} (breaking a task into subtasks). Staying with the one-person constraint, a complex task can benefit from being broken into subtasks where order of the subtasks matters. 

% https://graphthinking.blogspot.com/2021/09/why-is-everything-so-hard-in-large.html

What if we completely avoided bureaucracy? Try replacing "bureaucracy" in that question with "coordination of stakeholders". If you avoid coordination of stakeholders, what you get is random interactions. 

What if we minimized bureaucracy? Again, try replacing "bureaucracy" in that question with "coordination of stakeholders". The goal of "minimizing coordination" probably isn't the real objective. To be more precise, a specific objective might be "minimize time spent executing the task" (which takes a lot of coordination prior to the task execution) or "minimize the level of distraction to stakeholders" (chunk the coordination time). Another strategy for minimizing bureaucracy is to reduce the number of stakeholders involved. For a given task complexity, this means having smarter people who have more skills. 


\subsection{Hierarchy of roles}

An \gls{organization} often (though not always) has a defined set of roles, and those roles have different amount of decision authority. 

\subsection{Org chart as a guide and a lie}

org chart identifies roles and the relations among roles. 

lie in the sense that undocumented relationships matter more than the roles

lie in the sense of orientation; see \ref{org-chart-orientation}

\subsection{Approval process}

\subsection{Meetings for coordination}

coordination and signaling

\subsection{Written communication}

Reports, memos, emails are artifacts of bureaucracy. They create evidence and can be used for good or bad. 



\chapter{Bureaucracies are made of Humans}
% unordered essays to be clustered latter

\subsubsection*{Organization chart orientation
\label{sec:org-chart-orientation}}

A common method of describing relations within the bureaucracy is the organization chart (commonly the ``\gls{org chart}"). \iftoggle{glossaryinmargin}{\marginpar{[Glossary]}}{}%
Normally the Chief Executive Officer (CEO) is at the top of the chart, middle management is in the middle, and managed employees are at the bottom. See Figure~\ref{fig:org_chart_orientation_ceo-at-top}\iftoggle{haspagenumbers}{ on page~\pageref{fig:org_chart_orientation_ceo-at-top}.}{.}

Artifacts like org charts subtly convey an organization's culture. 
% What's the point of this section? Is there a consequence, or is this just an observation?
There are emotional connotations to alternative layouts. You can alter expected relations (culture and norms) by playing with the orientation of the org chart.
Org chart orientation can be overanalyzed, so the exploration in this section is limited.

The point of thinking about org chart orientation is to frame how you perceive your supervisors, peers, and the bureaucrats you manage. Notice that the framing is embedded in the words -- prefixes super (over) and sub (under). 
These concepts inform what you expect from relations.
Do I seek support or direction and guidance from my boss? What do I expect from my boss? My peers? The people I oversee? What do I expect to provide them?

%\begin{itemize}
%\item 
%\end{itemize}

The relative orientation of the \href{https://en.wikipedia.org/wiki/Chief_executive_officer}{CEO} 
\index{Wikipedia!Chief Executive Officer@\href{https://en.wikipedia.org/wiki/Chief_executive_officer}{Chief Executive Officer}}\iftoggle{WPinmargin}{\marginpar{$>$Wikipedia: CEO}}{}
to the workers sets expectations for relations. 
Options for orientation are the conventional CEO at the top
(Figure~\ref{fig:org_chart_orientation_ceo-at-top}), 
CEO at the bottom (Figure~\ref{fig:org_chart_orientation_ceo-at-bottom}),
CEO on the right (Figure~\ref{fig:org_chart_orientation_ceo-leads}),
CEO on the left (Figure~\ref{fig:org_chart_orientation_ceo-follows}),
CEO as the center of a star 
(for example, the diagram for the \href{https://en.wikipedia.org/wiki/File:League_of_Nations_Organization.png}{League of Nations} in 1930.)
\index{Wikipedia!League of Nations diagram@\href{https://en.wikipedia.org/wiki/File:League_of_Nations_Organization.png}{League of Nations diagram}}

\begin{figure}
\begin{center}
\includegraphics[width=1\textwidth]{images/org-chart-orientation-ceo-at-top.pdf}
\end{center}
\caption{Standard orientation. The role with the most responsibility and authority is at the top. Left-right ordering is intended to be irrelevant in this view, though left-to-right reading order emphasizes importance.}
\label{fig:org_chart_orientation_ceo-at-top}
\end{figure}

\begin{figure}
\begin{center}
\includegraphics[width=1\textwidth]{images/org-chart-orientation-ceo-at-bottom.pdf}
\end{center}
\caption{Flipping the orientation presents a more realistic view of the CEO's responsibility. The crushing burden of servant leadership is clear. Left-right ordering is intended to be irrelevant in this view.}
\label{fig:org_chart_orientation_ceo-at-bottom}
\end{figure}

\begin{figure}
\begin{center}
\includegraphics[width=0.7\textwidth]{images/org-chart-orientation-ceo-leads.pdf}
\end{center}
\caption{Conventionally time flows from left (old) to right (new), so in this graph the CEO leads the charge into the unknown. Is the CEO dragging workers forward, or are the workers pushing the CEO? The top-to-bottom order can be read as importance. }
\label{fig:org_chart_orientation_ceo-leads}
\end{figure}

\begin{figure}
\begin{center}
\includegraphics[width=0.7\textwidth]{images/org-chart-orientation-workers-lead.pdf}
\end{center}
\caption{The ``chariot view'' with the CEO in the chariot and the workers out front. Workers are in the future; the CEO is in the past operating on old information. As with Figure~\ref{fig:org_chart_orientation_ceo-leads}, top-to-bottom ordering can be read as importance. }
\label{fig:org_chart_orientation_ceo-follows}
\end{figure}

\begin{figure}
\begin{center}
\includegraphics[width=0.8\textwidth]{images/org_chart_wedding_cake_dependencies_-_manufacturing.pdf}
\end{center}
\caption{An internal-customer-oriented view rather than a reporting-oriented view. The center of the bullseye is the team that generates the value that is the focus of the business or the organization.
The outer rings support teams that exist in the inner rings. The diagram is specific to an organization's domain. This visualization identifies which teams are the customers of which other teams in an organization.}
\label{fig:org_chart_wedding_cake_manufacturing}
\end{figure}



%extension of 
% \href{https://en.wikipedia.org/wiki/Conway\%27s_law}{Conway's law}: seating chart reflects org chart

% https://graphthinking.blogspot.com/2020/05/invisible-bureaucracy.html

\subsection*{Social and Bureaucratic interactions\label{sec:socializing}}

Change in a bureaucracy can apply to processes and people, but a more amorphous concept is changing the culture of a team or organization. What is meant by ``culture'' usually refers to norms -- the expectations of behavior that individuals hold to. That definition of culture is generic; what is meant within a bureaucratic context requires jargon for specific expectations.

To evaluate expectations, we start by introducing categories of interactions. 
Interactions among members of an organization are either a social interaction or a bureaucratic interaction. 

As examples of each of these,
\begin{itemize}
\item \textit{Social interaction example}: ``Did you see the game on TV last night? Our team did fantastic, right? I wanted to get tickets for the game, but they were sold out."
\item \textit{Bureaucratic interaction example}: ``You'll need to get approval from Sue before presenting your idea to the board for their review. Then talk with Russ and get his thoughts about how to proceed."
\end{itemize}
Both social and bureaucratic interactions are vital to cohesion in an organization. 


Bureaucratic interaction can be broken into two subcategories: 
\gls{visible bureaucracy} \iftoggle{glossaryinmargin}{\marginpar{[Glossary]}}{}%
(procedures and processes are written down and can be discovered by stakeholders) and 
\gls{invisible bureaucracy} \iftoggle{glossaryinmargin}{\marginpar{[Glossary]}}{}%
(procedures and processes are known to some stakeholders and are conveyed verbally to some of the other stakeholders).

Invisible bureaucracy is akin to related topics outside the professional environment: invisible domestic work\footnote{Cleaning your living space, raising children, caring for pets; see~\cite{1987_Daniels}.} and invisible relationship work.\footnote{Consistent need to delegate, being curious without reciprocation.} The work associated with emotional cohesion, logistics, planning, scheduling, and communicating is hard to quantify so it does not get counted.


The relevance of this jargon is to break down the components of an organization's ``culture" experienced by participants.
When someone in the organization advocates for changing the culture, which expectations are they specifically referring to? Invisible bureaucracy is the hardest to alter because it is undocumented and not counted.
%The ratio of social relationship to visible bureaucracy to invisible bureaucracy is a characterization of the culture. There are norms associated with each of these three categories.

Processes default to invisible bureaucracy because creating and maintaining documentation requires work. Making the documentation discoverable requires work.
%, and because some processes are embarrassingly inefficient. 
To make invisible bureaucracy visible, document the work and enable other people to find the documentation.


% https://graphthinking.blogspot.com/2021/02/how-to-have-efficient-bureaucracy.html

\section{Efficient Bureaucracy}

In an ideal scenario with no bureaucracy, everyone comes to the same conclusion when presented with the same information. Then the management process of building consensus becomes unnecessary. There is no need to fight over resources (money, staffing) and no need to fight over direction.

While that ideal scenario is not going to happen, it points to how to improve bureaucratic efficiency:
\begin{itemize}
\item each person has the same information. 
\item each person applies the same decision making process consistently
\item every person has the same incentives
\end{itemize}
The reason bureaucracy is inefficient is
\begin{itemize}
\item not everyone has the same information
\item processes are inconsistent
\item incentives vary
\end{itemize}
Also, add the issue that each person's reference experiences are unique. As a consequence, decision making is subjective. 

Can any action be taken to improve bureaucratic efficiency? Yes!
\begin{itemize}
\item you can share information with other stakeholders
\item you can seek information from other stakeholders
\item you can strive for and demonstrate transparency
\item apply consistent processes 
\item hold others (and yourself) accountable 
\item account for varying incentives and reference experiences
\end{itemize}


% https://graphthinking.blogspot.com/2021/04/laffer-curve-and-minimum-viable.html

The Laffer curve is a claim in economics that there is a relation between government tax rates and the revenue from taxes collected. The relation, based on Rolle's theorem, says that between a tax rate of 0% and 100%, there must be some amount of tax that corresponds to the maximum of revenue. 

While the mathematical statement may be provable, the use in economics seems hand-wavy. In this post, I'll extend that hand-waviness to a different domain: bureaucratic processes in organizations. The relation to the Laffer curve is that bureaucratic processes a tax on productivity. 

\clearpage

\printglossaries

\nocite{*} % causes LaTeX to include every entry in your .bib file.
\bibliographystyle{plain-annote}
\bibliography{biblio}

\end{document}
