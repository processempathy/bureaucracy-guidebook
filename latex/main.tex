\documentclass{book}
% to set font size use
% \documentclass[12pt]{book}

% https://www.overleaf.com/learn/latex/Tables
\usepackage{array}
\newcommand{\dilemmatablewidth}{5.5cm}


\usepackage{graphicx}

% https://www.quora.com/What-font-are-most-books-printed-in
% to set font, use
%\usepackage{ebgaramond}

\usepackage{hyphenat} % http://www.ctex.org/documents/packages/special/hyphenat.pdf

\usepackage{setspace}\onehalfspacing\frenchspacing\flushbottom\sloppy

% https://www.scivision.dev/include-svg-vector-latex/
%\usepackage{svg}
% see https://tex.stackexchange.com/questions/442077/is-it-possible-to-use-svg-images-with-overleaf

% https://tex.stackexchange.com/a/8459/235813
\usepackage[nottoc]{tocbibind}

\usepackage{hyperref}
\hypersetup{
    colorlinks=true,
    linkcolor=blue,
    filecolor=magenta,      
    urlcolor=cyan,
    pdftitle={Overleaf Example},
    pdfpagemode=FullScreen,
    }

% https://en.wikibooks.org/wiki/LaTeX/Glossary says
% "\usepackage{glossaries} and \makeglossaries in your preamble (after \usepackage{hyperref} if present)"

% https://www.overleaf.com/learn/latex/Glossaries
\usepackage[toc]{glossaries}
%\usepackage{glossaries-extra}

\makeglossaries % The command must be before the first glossary entry.

% https://en.wikibooks.org/wiki/LaTeX/Glossary says
% "define any number of \newglossaryentry and \newacronym glossary and acronym entries in your preamble"
% see https://en.wikibooks.org/wiki/LaTeX/Glossary


\newglossaryentry{bureaucracy}{
    name=bureaucracy,
    description={definition here}
}

\newglossaryentry{visible bureaucracy}{
    name=visible bureaucracy,
    description={procedures and processes are written down and can be discovered by stakeholders}
}
\newglossaryentry{invisible bureaucracy}{
    name=invisible bureaucracy,
    description={procedures and processes are known to some stakeholders and are conveyed verbally to some of the other stakeholders.}
}

\newglossaryentry{process}{
name=process,
description={a task broken into a specified set of subtask dependencies.}
}

% https://graphthinking.blogspot.com/2021/07/bureaucracy-book-outline.html
\newglossaryentry{bureaucrat}{
    name=bureaucrat,
    description={a person responsible for subjective implementation of someone else's intent, with unquantifiable results. Examples of a bureaucrat role: teacher, police, government employee. Not bureaucrats: factory line worker, student}
}

\title{How to be an Effective Bureaucrat\\
A Field Guide}
\author{Ben Payne}
\date{\today}

\begin{document}
\pagenumbering{alph} % Fixes problem with glossary links; see https://tex.stackexchange.com/questions/119527/glossary-backlink-points-to-wrong-page

\begin{titlepage}
\maketitle
\thispagestyle{empty}
\end{titlepage}
\newpage

%\thispagestyle{empty}
\frontmatter % the front of the book has roman numerals

%\pagenumbering{gobble}
\thispagestyle{empty}

Copyright \copyright 2022 Ben Payne

Creative Commons \href{https://creativecommons.org/licenses/by-nc-nd/4.0/}{Attribution-NonCommercial-NoDerivs}

CC BY-NC-ND
\clearpage
\thispagestyle{empty}

Thank you to my coworkers. Our interactions helped me learn how to be a better bureaucrat.%\clearpage
%\pagenumbering{roman}

\chapter*{Foreword}% * excludes from Contents)
When a person has a positive experience engaging with bureaucracy, positive attribution is made to the people involved. Or ease of a solution makes the bureaucracy less visible and the solution seems obvious. 

When a person has a negative experience with bureaucracy, complaints are about the incompetence of the people involved, or the incomprehensibleness of the system. Don't these bureaucrats know how to do their job? Why isn't the solution obvious? Why does this system not work for me?

% Who this book is for

% from https://graphthinking.blogspot.com/2021/07/bureaucracy-book-outline.html
This book is for you if you are curious about bureaucracies, or you are thinking about working as a bureaucrat, or you are employed as a bureaucrat, or your job is shifting to be more bureaucratic. If you don't think of yourself as a bureaucrat, or if the term bureaucrat has negative connotations, I hope to change your mind on this vital topic. 


% What you should expect reading this book: 
The purpose of this book is to decrease surprise and arm you (both emotionally and intellectually) for the toil of being a bureaucrat. 

This book does not have a narrow focus on one topic like leadership, managing a team, being a team member, planning, time management, project management, advancing your career, or self-improvement. Some lessons may apply in those domains.

% What is the benefit of reading this book?
As a result of reading this book, you will be better able to recognize and navigate complex professional environments, both within your career and outside of work. The perspectives offered in this book can benefit you directly, whether by promotion of title or increase in pay; successful completion of a project; or through decreased stress of understanding how the world works.

There's harm in not recognizing yourself as a bureaucrat, as the role and responsibilities are distinct

Automation and computers will not eliminate or decrease bureaucracy. They merely obfuscate the processes and make negotiation more challenging. 

% my experience
% I wrote this book for a younger version of me.
 I was sufficiently self-aware when I first started my job in a large organization to recognize I didn't know much about working in that environment. Over the years I learned from my mistakes by reflecting on my (in)actions and the consequences. This approach has been an expensive education: mistakes delay progress and damage relationships.


% Caveats
Simplifying to "this interaction is characterized merely as human relations" is an easier perspective. However, that misses emergent phenomena. 

There's a risk of overanalysis. Sometimes a pipe is just a pipe. Avoiding conjecture about conspiracy and malice is a difficult boundary when insufficient information is available. 

I recognized the importance of navigating bureaucracy early in my career, but the insights were most clear when I entered middle management. 

My experiences cannot be generalized to every situation. Some of the observations here may be analogous to your context if you squint hard. 

Nothing in this book is domain specific, nothing is tied to engineering of products, and nothing is applicable solely in science research or policy development. While this material is intended to be timeless and generic, it is culturally specific to the USA. As a privileged white male, I did not encounter systemic hurdles in my career so there are blindspots not addressed in this book. 

% ? 
There are no alternatives to bureaucracy, so gaining skills in navigating bureaucracy are helpful. 

% Source of this content: 
This material is based on personal experience, reading published materials, and anecdotes from other people. No surveys were taken to support the claims made. No double blind experiments were conducted. 

% How the book should be read: 
Reading this book front-to-back is feasible. Each section is intended to be stand-alone. The book is intended to spark contemplation. 


% as per https://tex.stackexchange.com/q/393238/235813
\begin{flushright}
Ben Payne\\
\today\\
USA
\end{flushright}


%\clearpage

\hypertarget{contents}{}
\tableofcontents

\mainmatter % the main part of the book will have standard pages

% https://tex.stackexchange.com/questions/2958/why-is-newpage-ignored-sometimes

\chapter{Introduction to Bureaucracy}
% essentials

  \section{What is bureaucracy?\label{sec:define_bureaucracy}}

Bureaucracy is the specialization of roles necessitated by scaling and complexity for the distribution of common resources or widespread policy

In the course of carrying out someone else's subjectively defined policy, you have to make your own subjective decisions in the execution and enforcement in the course of carrying out someone else's subjectively defined policy, you have to make your own subjective decisions in the execution and enforcement


While you may know it when you see it or experience it, for this book definitions are useful. There are three distinct roles in bureaucracy: policy creator, policy enforcer, and the person upon whom policy is inflicted.

The existence of bureaucracy is independent of an organization's purpose.


The policy creator is either a politician or a bureaucrat. 

A \gls{bureaucrat} is a person subjectively interpreting policies on behalf of an organization and has discretionary enforcement to facilitate coordination of stakeholders. 

Let's break that down piece-by-piece. First, ``subjective interpretation'' means there is a person making a decision about how to do something. The subjectivity arises from different reasons one might choose an option over a competing option.  ``Policies" is a set of actions in a given circumstance. ``An \gls{organization}" is the collection of people for who the policy is made. ``Discretionary enforcement'' means the person is choosing how to apply the policy in the specific circumstances. ``Facilitating coordination'' means bureaucracy is about getting multiple people (or sometimes a person at different instances in time) to work together. The ``stakeholders'' is a group of people who care about the application of the action in each circumstance.  That's still pretty dense, so the rest of the book is spent expanding on the nuances and implications of this definition.

Bureaucracy is neither good nor bad. Bureaucracy is not tied to politics, or any specific institution (corporations, governments, academics). Bureaucracy is not defined to be efficient nor, does it have to be inefficient. Bureaucracy is not restricted to paperwork, or record keeping, or quantification, or gathering metrics. 

Bureaucracy is about delegation of control, communication, decision making, coordination, and processes. Involves negotiation, primarily informal. 

An organization comprised of bureaucrats is a \gls{bureaucracy}. The definition of bureaucracy used in this book is independent of government. Nothing in this definition involves paperwork or an office building. Definitions that limit the concept of bureaucracy to specific contexts result in a decreased ability to describe complex large-scale human organizations. 

The protagonist within a \gls{bureaucracy} is the \gls{bureaucrat} -- the person who is a member of an organization and is responsible for subjective implementation of policy for the organization. The person that a bureaucrat's decisions are inflicted on a \gls{subject}.  Depending on context, a subject may be a student (when the bureaucrat is a teacher) or a subject may be a citizen if the bureaucrat is a police officer or government official. Sometimes a bureaucrat's decisions are inflicted on other bureaucrats-as-subjects, such as when a Chief of Police creates guidelines for police in their district, or when a senior diplomat sets policy for embassy employees. 

A critical aspect of bureaucracy is that everything is made up, specifically by other humans. The consequence is that everything is negotiable. You (in the role of either a subject or a bureaucrat) need to know who to negotiate with and how to negotiate the desired changes. The only actual rules are mathematical physics that describe nature. Everything else is either naturally occurring macroscopic emergent phenomena (e.g., chemistry, biology) or humans making up labels and norms. 

Bureaucracy arises when there is no common objectively quantifiable feedback mechanism for individual participants in the organization. This aspect is why governments, schools, and prisons are characterized as bureaucratic. The military doesn't rank soldiers by ``number of enemies killed'' and is bureaucratic. Even profit-driven commercial organizations are bureaucratic when the impacts of individual employees are not coupled to sales metrics. 

Profit-based feedback makes some roles in a business context slightly more predictable and understandable, though there are still trade-offs like long-term profit versus short-term profit and externalization of harm. 

The concept of bureaucracy is most visible for complex, long lasting, and recurring situations involving many people. The apparent friction can be lower when there are only a few people involved (``I'm just talking to my collaborator" or ``I'm just buying groceries from a clerk at the store'' or ``I'm using a website for a government service''), but there is a continuous gradient. 

There is the external resource (mail delivery for USPS, public safety for FBI, environment for EPA) and there are resources internal to the bureaucracy. The focus of this book is on internal resources. In that context, bureaucracy is for the disseminated responsibility for use of resources: attention, skill, expertise. Time, money, staffing are proxy measures.



A useful way to think about bureaucracy is as a system for distributed knowledge and distributed decision making. That is in contrast to easier-to-understand concepts like centralized knowledge and centralized decision making. A government run by dictatorship is easy to conceptualize compared to democracies because there is a central character around which a narrative can be formed. Similarly, telling stories about the \href{https://en.wikipedia.org/wiki/Chief_executive_officer}{CEO} of a company is much easier than capturing the thousands of interactions conducted by the many employees of that company. Linear story-telling with a small number of protagonists does not map well to the complexities of bureaucracy. 
% are there alternatives to Bureaucracy that accomplish the same non-centralized non-consensus approach to complexity?


% https://graphthinking.blogspot.com/2017/09/market-friction-and-bureaucratic.html
Distributed knowledge and distributed decision making are hindered by
\begin{itemize}
    \item limited bandwidth between people, specifically the bureaucrats involved
    \item non-zero latency of information between people, specifically the bureaucrats involved
    \item the cost of measurement (getting data)
    \item the cost of analysis of the data
    \item making decisions that are suboptimal
\end{itemize}


\subsection{Bureaucracy as an economics model}
Firms exist in a market because negotiating contracts and prices for every interaction is burdensome. 
% https://www.kellogg.northwestern.edu/faculty/hubbard/htm/research/ec174/lectures/3coase.htm

Doesn't address small vs large companies, and doesn't distinguish between profit-oriented and non-profit and government. 

\subsection{Bureaucracy as emergent phenomenon}
Bureaucracy as a set of many bilateral interactions may not need to invoke emergence. However, there's a universality that hints at emergence. 

Above the threshold for emergence, there is scale-free behavior. The same patterns are observable at large organizations and extremely large organizations.

All those choices faced by the individual are not independent choices with respect to other bureaucrats in their environment. There is a flocking behavior of my choices are informed by the choices of those around me. Not necessarily in space.

Everyone is playing by different rules and has different objectives and everything is dynamic (both individuals and the conditions). 

Bureaucracy as a macroscopic phenomenon is emergent at sufficient scale. The scale is important because there is no longer dependence on individual relationships (beyond \href{https://en.wikipedia.org/wiki/Dunbar\%27s_number}{Dunbar's number}. There are people in the organization that you don't know and for which there is no common accountability. An organization subdivided into team recursively until there is local person-to-person accountability.  

The local rules bureaucrats employ to enable distributed decisions using distributed knowledge is meetings, processes, and communications. 

The relevance of making a claim that something is emergent is that there is behavior occurring at the macroscopic scale, and Knowing that individual motives and actions of every player at the microscopic level is not relevant.

% https://www.preposterousuniverse.com/podcast/2021/10/11/168-anil-seth-on-emergence-information-and-consciousness/
What does ``emergent'' mean? Nominal emergence example: a circle is emergent from a collection of points. Weak emergence is measurable using \href{https://en.wikipedia.org/wiki/Granger_causality}{Granger causality} or, equivalently\footnote{https://arxiv.org/abs/0910.4514}, \href{https://en.wikipedia.org/wiki/Transfer_entropy}{transfer entropy} (information theory). 


\subsection{Bureaucracy in Game Theory}
Bureaucracy does not fit cleanly into game theory categories of cooperative or competitive.

Maybe all the interactions within a bureaucracy are a bunch of small games?

Bureaucracy is self-modifying. 

Bureaucracy is in constant flux due to external conditions, externally imposed constraints, staff turn-over, internal dilemmas, disagreements of individuals. 


\subsection{Bureaucracy resists characterization}
Actually, bureaucracy is worse than emergent - the system rules can be altered or ignored by the stakeholders. \href{https://en.wikipedia.org/wiki/Wicked_problem}{Wicked problem}. This is why coming up with a holistic theory of bureaucracy is difficult. 

As soon as a claim is made, then a group can respond to that claim by behaving in an opposing manner. 

\subsection{Money as a fitness function}
Commercial businesses have a different accountability -- money. Common across all participants within the organization, and common with external stakeholders. The goal of a company is to generate profit. Commercial businesses have people who make subjective decisions and enforce policies, but there is a common metric for feedback. The feedback mechanism is not perfect. Being a good commercial bureaucrat does not necessarily result in monetary success.

Prisons, schools, medical, government, military all consume and spend money, but money isn't the goal. When faced with a decision, choice is not guided by which will generate more profit. 


\subsection{Bureaucracy as evolutionary outcome}

Biological, Genetic -- individual level
Biological, Genetic -- Group selection
Memetic

\subsection{Bureaucracy as Psychological Phenomenon}

Are you doing what's best for you, the group you're in, or everyone?
Altruistic or reciprocal? Retaliation
The answer changes time the time and situation of situation and person to person

Just a mixture of pathologies?\newpage % section
  \subsection{Alternative views of bureaucracy within a bureaucratic organization\label{sec:alternative_views_from_within}}

Bureaucracy as I have defined it in \S\ref{sec:define_bureaucracy} is not the only way that bureaucrats perceive their environment. 

The perspectives below are archetypical; an individual's perspective might be a mixture of these views.

\ \\

\textbf{As a bureaucrat, what matters is what I can accomplish with my skills and the resources I have access to.} \\
\textit{Assessment}: This person is task oriented. Results are what matters. The intricacies of bureaucracy are a distraction to getting the work done. The emotional reward for this person is accomplishment of the task. This person is likely to say to their manager, "Tell me what I need to do to be successful" rather than identify collaborations.

\ \\

\textbf{As a bureaucrat, what matters is how I feel.} \\
\textit{Assessment}: Your feelings are real. They have consequence, in that your emotions impact motivation and enthusiasm. However, a feelings-centric perspective may not be productive for you or your team or the organization. Balancing those competing needs is challenging.

\ \\ 

\textbf{What matters is how others feel.}\\
Depending on the emotional state of those around you is unhealthy and can be unproductive. Working for the happiness or satisfaction of other people is risky -- they may not know what's best, or they may not have your interests in mind.

\ \\

\textbf{What matters is my immediate coworkers.}\\
In this scenario everything else is personified or ignored. This can be positive (I collaborate with those around me) or negative (I am in competition with those around me). \\
\textit{Assessment}: Your do relationships matter. However, they are not all that matters. Being able to explain what is happening outside the immediately observable realm is what is missing from this view. 


  \section{Models of Bureaucracy that are Incomplete\label{sec:models-of-bureaucracy}}

Coming up with a holistic theory of bureaucracy is desirable but difficult. Bureaucracy exists in every society, so having an explanatory theory would help identify what aspects are essential and what is accidental. Having a theory of bureaucracy could help identify what should be improved and what should be discarded. An expectation for the existence of a theory stems from the repeated independent creation of bureaucracy in diverse societies in different time periods. 

Characterizing bureaucracy is difficult because organizations comprised of unique humans are aware of attempts to be characterized and respond to stories told about bureaucracy; see the \href{https://en.wikipedia.org/wiki/Hawthorne_effect}{Hawthorne effect}. 
\index{Wikipedia!\href{https://en.wikipedia.org/wiki/Hawthorne_effect}{Hawthorne effect}}
\marginpar{[Wikipedia] Hawthorne\\effect}
As soon as a claim about aspects that characterize bureaucracy is made, then an individual can respond to that claim by behaving in an opposing manner. Worse still for the theory, bureaucrats can coordinate amongst themselves to provide counterexamples. 

In the following section I outline a few conventional ways of modeling bureaucracy to point out the shortcomings of each model. The relevance to the practicing bureaucrat of familiarity with these models is so that you know the boundaries of each model. Awareness of the limitations of each model enables you to know when the model is an applicable story and when the model is not explanatory. 

\subsection*{Bureaucracy as a Machine}

\textit{Narrative}: Bureaucracy is a machine that has throughput and latency and dependencies and mechanisms. Bureaucrats are cogs in that machine.\\
\textit{Why this model feels true}: Bureaucracy can be complicated and feel mechanical with standardization, top-down dictates, and interlocking processes. There is a risk of \href{https://en.wikipedia.org/wiki/Deindividuation}{deindividuation} 
\index{Wikipedia!\href{https://en.wikipedia.org/wiki/Deindividuation}{deindividuation}}
\marginpar{[Wikipedia] deindividuation}
for bureaucrats. \\
\textit{What this model is missing}: Perceiving yourself as a cog in the machine implies a loss of agency. This is a \href{https://en.wikipedia.org/wiki/Self-fulfilling_prophecy}{self-fulfilling prophecy}
\index{Wikipedia!\href{https://en.wikipedia.org/wiki/Self-fulfilling_prophecy}{self-fulfilling prophecy}}
\marginpar{[Wikipedia] self-\\fulfilling prophecy}
-- if you think you don't have agency, then you may stop acting as though you have agency. 


\subsection*{Bureaucracy as an Economic model}

\textit{Narrative 1}: Bureaucracy is a collection of individual rational actors. \\
% https://graphthinking.blogspot.com/2019/05/cooperation-and-competition.html
\textit{Why this model feels true}: Individual bureaucrats cooperate or compete to promote their own self-interests.
The culture of an organization is a result of individual self-interests of each bureaucrat.
Individuals make decisions that promote their own self-interests. \\
\textit{What this model is missing}: This view is hard to distinguish from a marketplace. 

\ \\
\textit{Narrative 2}: Bureaucracy is comprised of competing special-interest groups. \\
\textit{Why this model feels true}: The observation of \href{https://en.wikipedia.org/wiki/Public_choice}{Public Choice theory} 
\index{Wikipedia!\href{https://en.wikipedia.org/wiki/Public_choice}{Public Choice theory}}
\marginpar{[Wikipedia] Public\\Choice theory}
is that a concentrated minority that stands to gain a disproportionate benefit will act in their own self-interest. \\
\textit{What this model is missing}: In a generic bureaucratic organization it is not clear which team would be more concentrated than any other team, nor is it clear what the disproportionate benefit might be. There is a concentrated minority at the top of any hierarchy, and the top of the hierarchy does act in their own self-interest, though this is not particular to bureaucracy.

Bureaucratic organizations may have specific teams with access to disproportionate benefits, but this book focuses on generic bureaucratic features. Effective bureaucrats are all alike; every bureaucrat is ineffective in their own way
\footnote{A modified version of the first sentence of Leo Tolstoy's novel ``Anna Karenina.''}.
% Anna Karenina is also referenced on page 142 of "Making of a Manager." 
% I came up with my adaptation before reading "Making of a Manager"



\ \\
\textit{Narrative 3}: A Bureaucracy is a subcategory of a Firm. \\
% https://graphthinking.blogspot.com/2017/09/market-friction-and-bureaucratic.html
Firms exist in a market because negotiating contracts and prices for every interaction is burdensome. 
% https://www.kellogg.northwestern.edu/faculty/hubbard/htm/research/ec174/lectures/3coase.htm
A bureaucracy could be considered as a type of firm that specializes at the organizational level in policy administration or resource management. \\
\textit{Assessment}: This is correct.

%Doesn't address small vs large companies, and doesn't distinguish between profit-oriented and non-profit and government. 


\subsection*{Bureaucracy as Emergent phenomenon}

\textit{Narrative}: There is a universality to bureaucracy in both the diversity of scenarios and persistence across time that hints at \href{https://en.wikipedia.org/wiki/Emergence}{emergence}.\\
\index{Wikipedia!\href{https://en.wikipedia.org/wiki/Emergence}{emergence}}
\marginpar{[Wikipedia] emergence}
\textit{Why this model feels true}: Bureaucracy as a macroscopic phenomenon is emergent when there are a sufficient number of people involved. The size of the organization is important because there is no longer dependence on individual relationships (i.e., a size above \href{https://en.wikipedia.org/wiki/Dunbar\%27s_number}{Dunbar's number}). 
\index{Wikipedia!\href{https://en.wikipedia.org/wiki/Dunbar\%27s_number}{Dunbar's number}}
\marginpar{[Wikipedia] Dunbar's\\numbers}
There are people in the organization that you don't know and therefore there is a lack of personal accountability. An organization is subdivided into teams recursively until there is local person-to-person accountability.  The underlying behaviors that enable emergence are bilateral interactions among humans and a lack of feedback mechanisms. 

At the scale of individual bureaucrats, every person is playing by different rules and has different goals and everything is changing -- both the individuals and the conditions. 
Above the threshold for emergence of bureaucracy there is scale-free behavior. The same patterns are observable at large organizations and extremely large organizations. A bureaucrat in one organization recognizes patterns of professional life experienced by bureaucrats at another organization. The local mechanisms bureaucrats employ to enable distributed decisions using distributed knowledge include meetings, processes, and communications. While local nuances differ, a generic pattern is apparent. 

The choices faced by an individual bureaucrat are interdependent with the choices made by other bureaucrats in their environment. There is a \href{https://en.wikipedia.org/wiki/Flocking_(behavior)}{flocking behavior} 
\index{Wikipedia!\href{https://en.wikipedia.org/wiki/Flocking_(behavior)}{flocking behavior}}
\marginpar{[Wikipedia] flocking\\behavior}
where my choices are informed by the choices of those around me. Unlike flocking of birds, the adjacency metric for bureaucrats is not necessarily spatial distance. Instead, visibility of the decisions and consequences inform adjacency.


The relevance of claiming bureaucracy is emergent is that there is behavior occurring at the macroscopic scale. Knowing  motives and actions of every individual bureaucrat at the microscopic level is not relevant. Treating organizations as complex and adaptive systems gives insight on how to work within the dynamic environment~\cite{2011_Eisenhardt}.


A colloquial interpretation of emergent behavior from complex phenomena is treating the system as an entity -- personification.
\marginpar{[Tag] Fallacy} 
When an organization is assigned behaviors~\cite{2002_Gall}, it is useful to remember that the organization is comprised of individual bureaucrats. 


% https://www.preposterousuniverse.com/podcast/2021/10/11/168-anil-seth-on-emergence-information-and-consciousness/
More formally, there are distinct categories of emergence~\cite{2002_Bedau, 2021_Carroll_168}. Nominal emergence names the phenomena of a thing being distinct from its constituents: a circle is emergent from a collection of points; a pile of sand emerges from grains of sand. Merely putting many bureaucrats in a room is insufficient to create bureaucracy; there's more to bureaucracy. 

Weak emergence occurs when there are phenomena that are independent of the underlying interactions. For example, \href{https://en.wikipedia.org/wiki/Glider_(Conway\%27s_Life)}{gliders} 
\index{Wikipedia!\href{https://en.wikipedia.org/wiki/Glider_(Conway\%27s_Life)}{gliders}}
in \href{https://en.wikipedia.org/wiki/Conway\%27s_Game_of_Life}{Conway's Game of Life}.
\index{Wikipedia!\href{https://en.wikipedia.org/wiki/Conway\%27s_Game_of_Life}{Conway's Game of Life}}
Weak emergence is measurable using \href{https://en.wikipedia.org/wiki/Granger_causality}{Granger causality} 
\index{Wikipedia!\href{https://en.wikipedia.org/wiki/Granger_causality}{Granger causality}}
or, equivalently\footnote{\href{https://arxiv.org/abs/0910.4514}{https://arxiv.org/abs/0910.4514}}, \href{https://en.wikipedia.org/wiki/Transfer_entropy}{transfer entropy} 
\index{Wikipedia!\href{https://en.wikipedia.org/wiki/Transfer_entropy}{transfer entropy}}
(information theory). I don't know how to apply these measures to bureaucratic organizations. 

\textit{What this model is missing}: The problem with treating an organization as an entity is that apparent behavior is counter-intuitive~\cite{2002_Gall}. Breaking the organization into individual people with motives helps clarify causes of observed behavior. 
%This merely improve the post hoc rationalization. 

In practice, bureaucracy is worse than emergent because the rules can be altered or ignored by  stakeholders. Bureaucracy is a \href{https://en.wikipedia.org/wiki/Wicked_problem}{wicked problem}~\cite{1973_Rittel} 
\index{Wikipedia!\href{https://en.wikipedia.org/wiki/Wicked_problem}{wicked problem}}
\marginpar{[Wikipedia] wicked\\problem}
which resists mathematical models. 

\subsection*{Bureaucracy in \href{https://en.wikipedia.org/wiki/Game_theory}{Game Theory}}
\index{Wikipedia!\href{https://en.wikipedia.org/wiki/Game_theory}{Game Theory}}
\textit{Narrative}: Bureaucracy is comprised of one or more \href{https://en.wikipedia.org/wiki/List_of_games_in_game_theory}{games} 
\index{Wikipedia!\href{https://en.wikipedia.org/wiki/List_of_games_in_game_theory}{list of games in game theory}}
played by bureaucrats. Which game is applicable depends on the specifics of the situation. \\
\textit{Why this model feels true}: Different bureaucrats have different motives in distinct situations. \\
\textit{What this model is missing}: While interactions among bureaucrats may be describable in terms of games, that doesn't provide an underlying motive for bureaucracy as an identifiable experience.  

Defining bureaucracy as distributed knowledge and distributed decision-making for the subjective management of access to shared resources does not fit as a \href{https://en.wikipedia.org/wiki/Coordination_game}{coordination game} 
\index{Wikipedia!\href{https://en.wikipedia.org/wiki/Coordination_game}{coordination game}}
or \href{https://en.wikipedia.org/wiki/Non-cooperative_game_theory}{competitive} game. 
\index{Wikipedia!\href{https://en.wikipedia.org/wiki/Non-cooperative_game_theory}{non-cooperative game theory}}
%Bureaucracy is self-modifying. 
Bureaucracy is in constant flux due to external conditions, externally imposed constraints, staff turn-over, internal dilemmas, and disagreements among individuals. 
% https://en.wikipedia.org/wiki/Evolutionary_game_theory




\subsection*{Bureaucracy as Evolutionary Outcome}


\textit{Narrative 1}: Biological, Genetic evolution -- Individual level. \\
There might be biological arguments for a genetic basis for bureaucracy. 
\textit{Why this model feels true}: For example, Zebra fish and Hyenas have a gene for dominance\footnote{``Society, demography and genetic structure in the spotted hyena'' (2012); doi:10.1111/j.1365-294X.2011.05240.x}. \\
%Breeding animals for aggressiveness?
\textit{What this model is missing}: ``Genes code for proteins, so there are no `genes for' phenotypes per se, including behavioral phenotypes."~\cite{2015_Lilienfeld}

Non-human animals make subjective decisions and do not get labeled as bureaucrats. Human make decisions that are not explained by reproductive fitness.

\ \\
\textit{Narrative 2}: Biological, Genetic evolution -- Group selection. \\
\textit{Why this model feels true}: \hyperref[sec:hierarchy-of-roles]{Hierarchy}
%(section~\ref{sec:hierarchy-of-roles}) 
is not unique to humans. Primates form into social hierarchies based on dominance over a shared resources like mates and food. \\
\textit{What this model is missing}: a gene for bureaucracy. A genetic model does not explain what behaviors an individual bureaucrat can take to be more effective. 

\ \\
\textit{Narrative 3}: \href{https://en.wikipedia.org/wiki/Memetics}{Memetic}
\index{Wikipedia!\href{https://en.wikipedia.org/wiki/Memetics}{Memetic}}
-- bureaucracy as method for coordination is a better idea than \href{https://en.wikipedia.org/wiki/Nepotism}{nepotism} 
\index{Wikipedia!\href{https://en.wikipedia.org/wiki/Nepotism}{nepotism}}
or religion. \\
\textit{What this model is missing}: Bureaucracy persists along with nepotism and religion, and bureaucracy occurs within religion. 

\subsection*{Bureaucracy as Product-focused Narrative}
\textit{Narrative}: Ignore the bureaucrats and instead focus on how a product progresses through a process.\\
\textit{Why this feels true}: Ignoring bureaucrats involved in processes simplifies the narrative (the product is the main character). 
\textit{Example}: \href{https://www.youtube.com/watch?v=OgVKvqTItto}{School House Rock: I'm Just A Bill}\\
\textit{What this model is missing}: Decision-makers involved the process exert subjective control. Outcomes depend on who participates. 

\subsection*{Bureaucracy as Subject-focused Narrative}
\textit{Narrative}: Ignore the bureaucrats and instead focus on the person subjected to bureaucracy. \\
\textit{Why this feels true}: The person experiencing bureaucracy as a subject is confused by ``why isn't this easier?''  \\
\textit{What this model is missing}: History of why the process exists and how it evolved (i.e., legacy), protection against malicious subjects, and protection against malicious bureaucrats. 


\subsection*{Bureaucracy as Psychological Phenomenon}

\textit{Narrative}: Bureaucracy as a pure power struggle. Or the interplay of individual pathologies. 
%Are you doing what's best for you, the group you're in, or everyone?
%Altruistic or reciprocal? Retaliation.
%The answer changes time the time and situation of situation and person to person.
%\href{https://en.wikipedia.org/wiki/Spiral_of_silence}{Spiral of silence}
% and
%\href{https://en.wikipedia.org/wiki/Social_proof}{social proof}

Organizations are composed of individuals with personalities, and inefficiency is attributable to a mashing together of distinct individuals with conflicting desires.
While this is true, it isn't complete. \\
\textit{What this model is missing}: Personality-focused narrative neglects the history of processes (legacy), protection against malicious subjects, and protection against malicious bureaucrats. Analysis that stops at personalities misses emergent phenomena. %Analysis that stops at personalities of individuals typically explains larger scale phenomena using personification of teams and organizations. 

\ \\

Each of the above models of bureaucracy has shortcomings. Knowing why each model is not a complete description helps you avoid the trap of thinking only in terms of a single model. 

The rest of this book ignores these partial characterizations of bureaucracy. 
Rather than take the external (\href{https://en.wikipedia.org/wiki/Emic_and_etic}{emic}) 
\index{Wikipedia!\href{https://en.wikipedia.org/wiki/Emic_and_etic}{emic and etic}}
view of bureaucracy, this book takes the internal (etic) perspective of a bureaucrat operating within an organization. 
The description of bureaucracy in this book is used to contextualize advice for how to be an effective bureaucrat. \newpage % section

  \section{Fundamentals of Bureaucracy\label{fundamentals_of_b}}
  
  This section provides terminology and definition for a bureaucratic lens. Specifically bureaucracy is defined and labels for roles in bureaucracy are named. Structure in organizations is often characterized by hierarchy, and that hierarchy is described by an ``organization chart.'' Lastly, meetings and written communication are the way in which consensus among bureaucrats is established.

    % define the core concepts 
    \subsection*{Hierarchy of Roles\label{sec:hierarchy-of-roles}}


In an ideal situation, sufficient depth and breath for decision making would be embodied in one person. That might not be possible in every situation. One way to resolve this is to identify distinct scopes of responsibility and then assign different members of an organization separate scopes for decision making. Within a decision making scope there may be more work than one person can handle, so a team is formed. That team may have some members focused on tactical work and other members focused on strategy and coordination. Hierarchy within an organization is the formalization of separate decision-making scopes and associated specialization. 

Partitioning knowledge and decision making enables complex work beyond what one person can accomplish and causes friction among members. An expert reporting to a manager knows things the manager does not, and the manager may have context that the expert lacks. Both bureaucrats (the expert and the manager) need to convey their respective understanding and seek the holistic view.

Hierarchical decision making is one option for coordination among alternatives (like consensus), so why is hierarchy so common? Members of an organization gravitate towards hierarchy because it helps define task scope, assigns responsibility, and obviates a need for building consensus. Reaching consensus for every decision would take time and be more burdensome than appointing a person as the decision maker.

\ \\

A hierarchical organization with partitioned knowledge introduces a challenge: the order in which you share information with others matters. Your choices for who to first describe an idea to are your peers, your management, and your subordinates. 
\marginpar{[Tag] Trilemma} 
\index{trilemma!communication priority: peers, management, subordinates}
The people subordinate to you know more about the topic and are exposed to the consequences. Giving them a chance to vet the idea results in a more robust idea and validates their value in the organization. Alternatively, first sharing your idea with management  allows your superiors to provide context you might not be aware off. And choosing to first start the conversation with your peers indicates you value the relationship and decreases the risk of duplicating work.

\ \\

I've included hierarchy in the section on Fundamentals of Bureaucracy, but that does not imply that hierarchy is a required feature of bureaucracy. Hierarchies of bureaucrats are a common \href{https://en.wikipedia.org/wiki/Organizational_structure}{organizational structure} and are worth studying even if not essential to bureaucracy. The relevance of understanding hierarchy is to identify recurring behavior and patterns to leverage.
Organizations of bureaucrats can intentionally work against the use of hierarchy for decisions, but the amount of effort needed to enable alternatives results in hierarchy being a common approach.

The benefits of formal hierarchy include improved capacity for the number of policy decisions made, enabling consistency of decisions, and leveraging specialization of knowledge. 
Hierarchical decision making has costs: higher latency (compared to a signle decider), inconsistency among bureaucrats (dissemination isn't perfect), waste due to inefficiency, and 
\hyperref[sec:unavoidable-hazards]{many others}.
\ifsectionref
described in section~\ref{sec:unavoidable-hazards}. 
\fi
As another example of the potential for harm, hierarchy enables strategic ignorance. Bureaucrats in positions of power can deny having knowledge of improper activity\footnote{L.~McGoey, ``The Unknowers: How Strategic Ignorance Rules the World" (2019)
% review: https://www.tandfonline.com/doi/abs/10.1080/19460171.2020.1768422?journalCode=rcps20
and 
L.~McGoey, ``The logic of strategic ignorance" (2012). DOI 
10.1111/j.1468-4446.2012.01424.x
}. 



The structure of an organization is dynamic, but at each point in time an organization typically has a defined set of roles. Each role is distinguished by different scopes of decision authority. 
Roles are often confused with titles. What matters is the role (scope of decisions) and who reports to whom. The names of teams can be similarly not descriptive.




Roles in an organization are defined by the boundaries of responsibility. The purpose of a role is to minimize conflict, decrease the need for coordination, reduce redundancy, and allow for control of resources. Clear responsibility boundaries enables effective bureaucracy. 


A conventional characterization of an organization's hierarchy involves two criteria: the depth and breadth of the org chart.
The more people a supervisor oversees, the flatter the organization -- that's the breadth of the organization. The depth of the hierarchy is how many layers there are. See the Valve handbook~\cite{2012_Valve} and Joreen's essay~\cite{1972_Joreen} for contrasting views on the merits of an organization's hierarchy. 

A more practical view of an organization's hierarchy also involves two criteria. The two choices of how a hierarchy is shaped are 
1) how many people a supervisor oversees and 
2) how many supervisors a person has. 
Though you might naively expect that an employee has one boss, but that is \href{https://en.wikipedia.org/wiki/Matrix_management}{not a requirement}. A supervisor for a given topic may have many people reporting to them, and a bureaucrat with multiple roles may report to more than one supervisor.

\ \\

Acting as part of a group means ceding part of your autonomy. Hierarchy cedes more of your responsibility and adds expectations about relationships.
The consequence of hierarchy in an organization is that, as a member of the bureaucracy, you do not have full autonomy -- otherwise you would not be a member of the hierarchy. At the same time, you are not under strict control of the organization -- you still have some subjective decision making authority as a bureaucrat.

The person at the top of the hierarchy does not know everything. The person at the top of the hierarchy does not have input on every decision made in the organization. Some autonomy is retained by all members of the bureaucracy.

Independent of the defined roles and titles in an organization's hierarchy, there are a set of implicit roles and a separate social hierarchy of informal influencers and decision makers. Informal influencers in a bureaucracy usually have long relationships with the decision maker or relevant credentials or both. The credentials can be formal (e.g., a \href{https://en.wikipedia.org/wiki/Doctor_of_Philosophy}{PhD}) or informal (demonstrated success on a project). In either case, the decision maker is relying on another person's expertise. 

Another set of informal relations within an organization is mentors and mentees. These relations allow mentors to share institutional knowledge to mentees, and allows people in senior positions to access the novice perspective. 


\ \\

One consequence of hierarchy is a sense of fear felt by people who report to other people. This fear stems from the loss of control (less autonomy) that leaves the person feeling disempowered. 

For example, consider the following relationship. Sue, is perceived to have power over another person, Amy, because Amy gave up some control to Sue. Amy not having full control over decisions triggers the feeling of fear in Amy, regardless of how Sue behaves. Having complete responsibility for decisions also induces anxiety.

If Sue is aware of the potential for this emotional experience, Sue can compensate for Amy's fear by being friendly and receptive towards Amy. Alternatively Sue may exploit or rely on the fear felt by subordinates. Sue not noticing or accounting for Amy's fear does not invalidate Amy's emotional experience.


% Active bystander when the person doing wrong is in a position of authority
% PACT (Probe, Alert, Challenge, Take Action)
% https://mobile.twitter.com/GeorgetownABLE/status/1408498438203969541


% Mintzberg's Coordination Mechanisms
% https://www.youtube.com/watch?v=IZET8VjSifQ


    \subsubsection*{Organizational chart as a Guide and a Lie\label{sec:org-chart-as-guide-and-lie}}

An \href{https://en.wikipedia.org/wiki/Organizational_chart}{organizational chart} 
\index{Wikipedia!\href{https://en.wikipedia.org/wiki/Organizational_chart}{organizational chart}}\iftoggle{WPinmargin}{\marginpar{[Wikipedia] Organizational\\chart}}{}
(hereafter an ``\gls{org chart}'') identifies formal roles and the formal relations among roles. An org chart is at best a snapshot in time, and more often aspirational than descriptive. Despite possible deficiencies, an org chart helps outsiders and newcomers understand the scope of responsibilities and interactions.\footnote{The organization chart hasn't always existed. The \href{https://en.wikipedia.org/wiki/George_Holt_Henshaw\#First_organization_chart}{first known org chart} 
\index{Wikipedia!\href{https://en.wikipedia.org/wiki/George_Holt_Henshaw}{George Holt Henshaw}}
was created in the 1850s.}

Org charts are a lie because undocumented relationships can matter more than official roles. Org charts fail to capture the informal roles and network of relations that facilitate progress in any organization. Org charts document titles instead of describing roles.

Org charts foster a second separate lie by creating a sense of power dynamics based on visual orientation. For more on this issue see the discussion of~\hyperref[sec:org-chart-orientation]{org chart orientation}\iftoggle{haspagenumbers}{ on page~\pageref{sec:org-chart-orientation}}{}.

    \section{Not Many Options for Organizations}
% https://graphthinking.blogspot.com/2021/07/patterns-anti-patterns-in-bureaucracy.html

The purpose of an organization or a team is to obfuscate the bureaucracy of enacting policies for managing access to shared resources. The value of organizations and teams is that they are a simplifying assumption that depersonalize the function of inputs and outputs. Without organizations or teams you would be directly exposed to the complexity of knowing every person's title and responsibility. Organizations and teams are abstractions to hide complexity.

When complaining about the ineptitude of organizations (or more specifically, leaders, managers, bureaucrats you manage, and coworkers), consider the variables available to be modified. As a thought exercise, what would it take to rebuild the organization you are in from scratch? 
\marginpar{$>>$ Thought Exercise}

Listing the levers available for enacting change in an organization is intended to emphasize the importance of personal interactions within bureaucracy. When you get frustrated with inefficiency, consider the systemic changes available -- there aren't that many. Options for restructuring the organization are limited, so the importance of your creativity and informal influence is dominant in determining the effectiveness of bureaucracy. Those factors relevant to Process Empathy are separate from having the right staff and being able to have an adequate number of staff. 

Organizations comprised of bureaucrats have fewer options for change than individual bureaucrats. The choices an individual bureaucrat faces are described
\iftoggle{haspagenumbers}{on page~\pageref{sec:dilemma-trilemma}}{ }%
 in the section on 
\hyperref[sec:dilemma-trilemma]{dilemmas}.
In comparison, the choices faced by the designers of an organization include:
\begin{itemize}
    \item Flatness of organizational hierarchy -- how many layers of oversight are there? Another way of arriving at the same structure is to ask how many employees there are per supervisor.
    \item Hierarchy split by function (all the lawyers in one team, all the engineers in one team, all the sales people in one team) or by product (mixing experts to solve a customer's problem)?

% https://graphthinking.blogspot.com/2023/07/matrixed-organizations-best-practices.html
    
    \item Size of organization. Having more people allows for more specialization and requires more support. 
    
    \item Organizations create processes for recurring tasks like hiring, promotion (pay or title), awards, compensation, recognition, professional training, and firing. Each of those has a set of design choices that inform the organization's culture.
\end{itemize}
In a government bureaucracy, the constraints of some incentives like pay and financial awards are set outside the organization. That further constrains the ability of bureaucrats to shape their organization's culture.

The organization may have policies and processes regarding hiring, promotion, training, and firing, but the decision may be made by team managers rather than top-level managers. 

%Listing how an organization could change is relevant 
Individual bureaucrats can enact change, and subjects of bureaucracy can request improvement, but those are weak inputs. 
Organizations are accountable to their source of funding, not the subjects of bureaucracy or members of the organization. For example, the \href{https://en.wikipedia.org/wiki/Internal_Revenue_Service}{Internal Revenue Service (IRS)} 
\index{exemplar!Internal Revenue Service@\href{https://en.wikipedia.org/wiki/Internal_Revenue_Service}{Internal Revenue Service}}
\index{Wikipedia!Internal Revenue Service@\href{https://en.wikipedia.org/wiki/Internal_Revenue_Service}{Internal Revenue Service}}%
is accountable to Congress, not taxpayers or IRS employees. 



A flat organization (more employees per manager) means more people doing work instead of managing, but sufficiently large organizations are typically broken into teams to make coordination less chaotic. 
%TODO:~\cite{2015_Katzenbach} (Wisdom of Teams book).


Organizations are segmented into teams and there are not many options: create a new team, merge existing teams, or dissolve a team. 


Shuffling the structure of teams (referred to as re-organization) is a favorite hobby of executive leadership. The alignment of hierarchical structure with an organization's purpose is a change that is easy to point to as an accomplishment. 


For a given set of teams, the lateral interactions are competitive or cooperative. Coordination is required (or conflict will occur) for money, staffing, and resources. Examples of resources include access to or control of data, computer equipment, hardware, floor space in a building, prestige, and ownership of products.

% ==== hierarchy - expand/contract ====
% Ways to expand and contract hierarchy as the number of employees in the workforce grows and shrinks
% * Add a new organization below: person who was an employee is now a manager and hires a bunch of people
% * Add a new peer team,
% * Add a new parent oversight organization above the current one
% * Unflatten an organization by inserting a new layer of management


Not all changes have to be structural; there are choices to be made when operating within the existing hierarchy.


\subsection*{Not Many Options Within Teams}



Similar to the limited number of options available for shaping organizations, the structural choices faced by a team are constrained. 


The purpose of pointing out the variables that can be changed is to show that there aren't many. As a consequence of the limited options, relationships and effective communication are more significant for effective bureaucracy. 


Team managers might decide who gets hired, who gets promoted, who goes to what training, and who gets fired. Though in some environments even that control is relegated to an external team. A team manager usually has decision-making authority regarding tasks the team works on. 

% intra- and inter- team dynamics

Accountability in the context of teams comes from person-to-person interactions. These can be either lateral (sideways), parent-child (top-down), or child-parent (bottom-up)~\cite{2014_Jorgensen}. Each of these three categories has associated constraints.

The upward child-parent (bottom-up) communication is either inadequate (too few updates, not enough information, or insufficient context), relevant, or excessive. For example, a weekly or monthly report to multiple superiors may be inadequate. 

Finding the balance depends on the presenter knowing the individual audience members so that a tailored message is provided and then adapting to the specifics of the situation. 
% tips on managing up: 
% https://svpg.com/managing-up/

The downward parent-child communication either is inadequate (no direction provided or imprecise direction provided), provides actionable vision, or micromanagement. Finding the right balance and specificity requires insight into the communication needs on both sides of the relationship. The manager of a team member has a variety of tools available beyond setting tasks and deadlines. A manager can create an environment that provides psychological safety, promotion based on accomplishments, recognition of contributions, encouragement, and constructive feedback.

For interactions among team members at the same hierarchical level the conflict of interests between cooperation and competition manifests in struggles over money, staffing, products (output), and resources (inputs). Here ``resources''  refers to constraints like access to data, control of data, technology resources, hardware, floor space, and expertise. 


% https://graphthinking.blogspot.com/2021/07/patterns-anti-patterns-in-bureaucracy.html

%\begin{itemize}
%    \item money
%    \item staffing
%    \item prestige
%    \item products (output)
%    \item resources (inputs)
%    \begin{itemize}
%        \item access to or control of data
%        \item technology resources
%        \item hardware
%        \item floor space
%        \item expertise
%    \end{itemize}
%\end{itemize}


Understanding the options available for organizations, teams, and relationships informs your Process Empathy. There are constraints bureaucrats operate within and there are a limited number of ways to evolve out of a situation. By listing possible ways to change you can identify which options are best to invest in. 
    \subsection{Meetings for coordination\label{sec:meetings-for-coordination}}
In an organization comprised of more than one person, meetings are a necessary artifact for facilitating coordination. The coordination accomplished by a meeting can be explicit (verbal or written), or it can be indirect through signaling (who attended the meeting, when the meeting was held, where the meeting was held, how much notice was provided). 

Meetings are a vital aspect of coordination in a bureaucracy. Tips are in \S~\ref{well-run_meeting}.


% https://graphthinking.blogspot.com/2021/07/thought-terminating-concepts-in.html
\textit{No plan; I just do what you tell me.}\\
Employee: I just do what you tell me to do. With that approach, either I will be successful because I worked hard on what you directed, or I will fail because I was directed to do the wrong thing by you.\\
Manager: Is that how you want your career to go? I think you're smart, and I think you're capable of shaping your career.

\ \\

\textit{No point in making a plan}\\
Employee: There is no point in making a plan, because everything changes so frequently.\\
Manager: But the software has an end goal, right?\\

Employee: Yes, so then the plan is to get from where we are now to that end goal.\\
Manager: And there are no intermediary steps? Milestones?

Manager: Is it better to have no plans and just put up fires reactively, or to have a plan that is subject to change?

\ \\

\textit{What is a plan anyways?}\\
Manager: I think there is value in creating a goal, enumerating tasks that would support the goal, identifying the dependencies among the sub-tasks, and time-binning the dependencies with defined milestones and deliverables. That is my definition of a plan. And having that is more useful than merely reacting.

\ \\

\textit{Who's plan?}\\
Employee: I don't need to come up with that plan, you already have a plan. Just tell me what the plan is.\\

Manager: That's not as effective as coming up with independent plans and then resolving the differences. There's value in resolving the differences, Even though that will cost time and frustration and displace time to implement.
    \subsection{Written Communication}



Reports, memos, emails are artifacts of bureaucracy in an organization. A written record creates evidence about policies and decisions. Existence of a record can be used for good or for harm.


% https://graphthinking.blogspot.com/2020/09/identifying-and-eliminating.html % this isn't essential, just common
    \subsection{Decision making in a bureaucracy\label{sec:decision-making}}


Decision are not the only source of change in an organization. Occasionally events unfold without decisions being made. This might be because the decision makers are not informed, or there is an \href{https://en.wikipedia.org/wiki/Willful_blindness}{intentional neglect}. This section focuses on situations where bureaucrats recognize the need for a decision and want to make the best decision.

There are multiple types of decisions. 
A \gls{simple decision} has one correct or beneficial choice and one or more wrong or harmful choices. The work of decision making is then to gather information that identifies which is the correct or beneficial choice and select that option.

The best case scenario for any decision making is one person making a well-informed simple decision that has immediate consequence and the consequence is to the decision maker. Examples from elementary formal education include arithmetic math problems, multiple choice quizzes, spelling tests, and memorization tests. A bureaucrat's \href{https://en.wikipedia.org/wiki/Moral_injury}{moral injury} can come from decision making that involves multiple people, weak feedback loops, and complex decisions.

\subsubsection{Example Decision Method: Pareto Frontier\label{sec:pareto}}

A complex decision may have many choices, and there are might not be a best option. Then a \href{https://en.wikipedia.org/wiki/Pareto_front}{Pareto frontier} might exist where trade-offs can be made. 

As an example of a complex decision made by one person with immediate consequence and direct relevance to the decision maker, suppose you want to purchase a car. You car about only two aspects: fuel efficiency and cost. 

\begin{figure}[ht]
    \centering
    \includegraphics[width=1\textwidth]{images/pareto_frontier_car_options.pdf}
    \caption{Four cars, L, M, N, and P. The goal of the buyer is to spend less money and get better fuel efficiency. Choices not on the frontier should be avoided, but that doesn't yield a single result.}
    \label{fig:pareto_frontier_cars}
\end{figure}

Visualzing a Pareto frontier for two quantitative variables is easy, but typically decisions involve more factors. For example, evaluating the trade-off of three quantitative variables like speed, accuracy, and cost creates a surface. With more than three variables visualization are less useful, though the analysis technique still applies. 

Another constraint on using Pareto frontier analysis is that it works well when there are many options relative to the number of variables being optimized for. 
The assessment does not work as well when there are few choices relative to the number of variables. For example, suppose there are 10 choices of car and you want high fuel efficiency, sufficient cargo capacity, maximum number of passengers, stylish, low cost, low maintenance, good durability, and high resale value. 

For a set of quantitative variables, a Pareto frontier does not account for relative importance of different variables. Assign weights to each of these factors merely stretches one axis relative to the other axes. 

There are many possible decision making frameworks besides Pareto frontiers, but in practice a typical bureaucratic decision is ill-informed, has diffuse consequences, delayed impact, and does not affect the decision maker. In bureaucratic processes there is rarely a formal assessment of options. 
Decisions are rarely recorded. 
Even afterwards a decision can be difficult to evaluate for correctness because there are multiple stakeholders.


\subsubsection{Risks of Using Decision Frameworks}

Decision making frameworks can be attractive to bureaucrats intending to formalize processes (see \S\ref{sec:process}) and encourage predictability. There are potential risks worth being aware of. 

% https://graphthinking.blogspot.com/2019/01/political-decisions-versus-science.html
A decision is political when the basis is historical relationships, maintenance or creation of a relationship, or to enable future relationships. A decision is subjective when someone else faced with the same scenario would have come to a different conclusion.
A decision is quantitative when it is based on measurements. To avoid the appearance of subjective decision making or political decision making, a decision may be framed as ``data driven." 
% https://graphthinking.blogspot.com/2018/06/data-driven-decisions-versus-data.html
A good approach for data driven quantitative analysis involves coming up with a testable hypothesis, then performing experiments and collecting data to evaluate the hypothesis. More commonly, a decision is made, then data is gathered which supports the desired outcome. Forming an opinion and then looking for evidence to back the outcome yields suboptimal results for the organization.

Even if a bureaucrat is not intentionally biased towards an outcome, there are many ways to gather evidence and some approaches have biased sampling and produce biased results.

With valid and representative data measurement, decision makers can be led astray by poor modeling. A model may use inapplicable techniques, or may have implementation bugs.

For all the dangers of decision making methods, there are worse approaches that do not rely on measurement. People rely on history (if they are aware of it) and perpetuate bad ideas, or take action based on what is best for their career, or decide based on how to accumulate more power, or simply choose based on what someone else says to do.  


\subsubsection{Decision Making Delay}

What appears from the outside as ``organizational inertia'' is internal delay of decision making and the delay of dissemination. 
Delay comes from
\begin{itemize}
    \item It takes time for each decision maker to gather information, arrive at a decision, modify processes, disseminate their selection, justify their selection. 
    \item forcing a continuous variable into a discrete set of choices. Typically the number of choices is small. Discrete choices for a continuous variable is a loss of effectiveness.
    % https://dynomight.net/teaching/
    \item processes designed to account for cheaters and people with malicious intent, whether that means a malicious bureaucrat or malicious subject. 
\item Analysis paralysis, due to {insufficient information, too much information, which framing is unclear}
\item When other people who are needed to carry out the action push back, either in disagreement or seeking clarification. The fact that the organization is not profit driven is important because the justification for the action isn't quantitatively obvious. Therefore there's a higher burden for communication.
\end{itemize}


% The following characterization has no consequence
% Decision making by bureaucrats can be informal or formal, consensus-based or solo. 


\subsubsection{A Bureaucratic Decision involves many Decisions}

A decision is actually a collection of dependent choices. After recognizing the need for a decision, follow-on decisions include identifying the stakeholders (who to include in an impact analysis) and identifying options. Who is a stakeholder and what are the options are interrelated. Involving more people expands the number of options and the complexity. Additional choices associated with reducing uncertainty are how much time to spend on the decision, how much information to gather for the decision, whether the make the decision or push the decision to someone with more expertise, whether to push the decision to someone with more exposure to the consequences.

Most decisions you make as a bureaucrat do not have hard deadlines. Instead, there are trade-offs in allocation of your time. Sooner is preferable since the consequence of the decision benefits the organization and allows you to focus on other tasks, but delaying allows for more information gathering for a better informed decision. (See Dilemma~\ref{table:gather_data_lots-vs-little} and other related Dilemmas in that section.)


If a bureaucrat is going to rely on expert consultation (see \S\ref{sec:expertise} on expertise), the decision maker needs to be confident the expert is not of straying outside their area of expertise. For example, I don't rely on a botanist with many published papers to tell me how to change the oil in my car. 
Besides knowing their own limitations, the expert should be clear about whether their input is a factual summary, a predictive assessment, or a value judgement. This nuance complicates what you as a bureaucrat are interested in (what's the best choice?).


    \subsection*{Feedback loops and Ripples\label{sec:feedback-loop-and-ripples}}

%When there is a lack of input from customers or users, then someone has to decide.


A feedback loop exists when a decision-maker experiences the harms and benefits of their decision. Decisions that lack a feedback loop still have consequences for the decision-maker, in that potential future decisions are altered or limited \footnote{In~\cite{1983_Lipsky} Lipsky discusses feedback loops in chapter~4 on page~40.}.


Virtuous cycles and vicious cycles are rare in bureaucratic organizations because there are rarely mechanisms for feedback loops. Instead, there are \glspl{ripple} -- propagation of consequences for other people's schedules, altering what is possible for other people, and creating dependent tasks to be carried out by people other than the decision-maker.


The weak feedback mechanisms for a bureaucrat are reputation (subject to spin)
\index{reputation}
and rarely enforced retroactive accountability in the form of the question, ``What did you know and when did you know it?''
%Reputation is social.
Retroactive accountability depends on written records from meeting notes, emails, agendas, and reports. Reducing the potential for retroactive accountability is one motive for bureaucrats avoiding written records for decisions and policies.


Because of the lack of quantitative feedback loops for bureaucrats, there is significant interest in documenting justification, proactive monitoring, reporting, and retroactive assessment. Each of those activities creates more administrivia.


When there are multiple competing objectives among stakeholders in a 
\href{https://en.wikipedia.org/wiki/Zero-sum_game}{zero-sum}
\index{Wikipedia!\href{https://en.wikipedia.org/wiki/Zero-sum_game}{zero-sum game}}
\marginpar{[Wikipedia] zero-sum\\game}
use of resources, how can you determine what's best? In some domains there are feedback loops to guide progress. When feedback loops are weak or not present,  the most powerful stakeholder (potentially distinct from the biggest or loudest) will dominate. 


\ \\

%****************************************
% https://graphthinking.blogspot.com/2020/01/hierarchy-of-justification.html

Because feedback loops are weak for decision-makers, alternative mechanisms are needed in bureaucracy. A standard approach is the use of approvals and justifications. The strength of justification needed for a given action depends on your relationship with the approver and the potential ripples associated with the action. 

The amount of justification for an approval process should be just enough so that when something goes wrong there's a clear reason why. 

Ordering the quality of a justification
\index{decrease surprise!quality of justification}
\begin{enumerate}
    \item I have no explanation.
    \item This is my opinion.
    \item We've always done it that way (a \href{https://en.wikipedia.org/wiki/Cargo_cult}{cargo cult}).
    \index{Wikipedia!\href{https://en.wikipedia.org/wiki/Cargo_cult}{cargo cult}}
    \marginpar{[Wikipedia] Cargo\\cult}
    \item Based on my experience.
    \item I was told to do it this way.
    \item I think this is the best way (no reasoning, but a desire to optimize; optimization criterion undefined).
    \item This way is most effective because X (where X is not quantified, there is a desire to optimize, and there is an optimization criterion).
    \item This way is most effective because X (where X is quantified, there is a desire to optimize, and there is an optimization criterion).
    \item This way is most effective because X compared to other options (where X is quantified, there is a desire to optimize, and there is an optimization criterion).
\end{enumerate}

%****************************************

\ \\

% TODO: need a transition between topics


\subsubsection*{Special interest groups}

Within bureaucratic organizations special interest groups care about aspects of the shared resource central to the organization. 
Inefficiency can occur when there is a benefit to a small group and the cost is to a larger group.
% TODO? tie in with Public choice theory

The \href{https://en.wikipedia.org/wiki/Social_trap}{social trap}
\index{Wikipedia!\href{https://en.wikipedia.org/wiki/Social_trap}{social trap}}
\marginpar{[Wikipedia] social\\trap}
is ``a conflict of interest or perverse incentive where individuals or a group of people act to obtain short-term individual gains, which in the long run leads to a loss for the group as a whole."
The feedback loop for the diffused value relevant to the organization is weaker than the feedback loop for value to the interest group. 

\subsubsection*{Spending taxpayer dollars}

As an example of a weak feedback loop, consider the scenario of a government employee deciding how to spend government money.

%\marginpar{[Tag] Story Time; Math}
\index{story time!taxes and spending}
\begin{storytime}{Taxes and Spending}
Suppose you are a government bureaucrat and earn \$100,000 with a tax rate of 30\%. Then your tax money sent to the government is \$30,000.
How does that compare to what the government collects in taxes?

For the United States, ``in 2021 the government collected \$4.05 trillion in revenue."
\footnote{source: Government Revenue | \href{https://datalab.usaspending.gov/americas-finance-guide/revenue/}{U.S. Treasury Data Lab}}

That means your taxes of \$30,000 would be
30000/4050000000000 = 0.00000074\% of the tax base for the country.

If you are a federal government bureaucrat and you do not maximize the effectiveness of spending \$1,000,000 of government money, of that misallocated money only \$0.0074, or about one penny, was taxes you paid. The financial feedback loop is weak.

Some federal government bureaucrats earn slightly more, so the cost of wasting \$1,000,000 is more significant.
Federal pay is limited to about \$220,000\footnote{Wikipedia entry on \href{https://en.wikipedia.org/wiki/Executive\_Schedule}{Executive Schedule}
\index{Wikipedia!\href{https://en.wikipedia.org/wiki/Executive\_Schedule}{executive schedule}}
%%%CANTDOINFOOTNOT\marginpar{[Wikipedia] Executive\\schedule}
}, raising the feedback to 2 pennies.
\end{storytime}

The lack of feedback allows waste to go unfelt. There's no immediate consequence for the decision-maker.
%Waste is indistinguishable from not enough funding or insufficient skills.

\ \\

Example feedback loop:
\begin{center}
\begin{figure}[ht]
    \centering
    \includegraphics[width=0.8\textwidth]{images/feedback_loop_complexity_and_staffing}
    \caption{Increased complexity requires more staffing to enable specialization. More staff means more skills are available; under-utilized staff skills make room for more scope; more scope adds to complexity.}
    \label{fig:complexity_and_staff_growth}
\end{figure}
\end{center}

\subsubsection{Example feedback loop: Security agent making a fair decision}
% https://graphthinking.blogspot.com/2017/09/a-simple-illustration-of-bureaucracy.html
%\marginpar{[Tag] Story Time}
\index{story time!airport security}
\index{exemplar!\href{https://en.wikipedia.org/wiki/Transportation_Security_Administration}{Transportation Security Administration (TSA)}}
\begin{storytime}{Airport Security Line}
A coworker and I were going through passport control at an airport. 
A \href{https://en.wikipedia.org/wiki/Transportation_Security_Administration}{TSA officer} 
%\index{exemplar!\href{https://en.wikipedia.org/wiki/Transportation_Security_Administration}{Transportation Security Administration (TSA)}}
\index{Wikipedia!\href{https://en.wikipedia.org/wiki/Transportation_Security_Administration}{Transportation Security Administration}}
%CANTDOWITHINMDFRAMED\marginpar{[Wikipedia] TSA}
directing passengers into one of two lines. Both lines were long. The length of each line was not equal even though the entrance to the lines was at the same location. Each line terminated at a row of passport-checking agents. Each passport check took a minute. Passport-checking TSA officers operate independently and concurrently.


The TSA officer's perspective is that there are two long lines. Her procedure is two balance the two lines (for fairness). He does this by pointing people into one of the two lines, with her choice driven by which line appears to have room available.

My coworker and I enter the controlled area and are directed into lines by the TSA officer. The officer directs my coworker left and directs me to go right. The TSA officer completed her job.

My coworker in the left line finishes 5 minutes before me. This difference in completion time is frustrating for me.

Because the lines are not of equal length, balancing the start of the line is a suboptimal method. The consequence is that what seems fair to the TSA officer ends up not being fair for people going through the line. The TSA officer had incomplete information -- the lines are not of equal length. Because the TSA officer isn't exposed to the consequence of her approach, she doesn't get feedback on whether it is suboptimal or not.
\end{storytime}

Lesson: if the people making decisions do not experience the consequences of those decisions, then they have no incentive to improve decision-making.

The person in the longer line feels frustrated. The negative feeling is due to a sense of powerlessness, the situation is recurring, and a better solution is available.

The optimal solution in this situation is to have a single line feeding the multiple TSA passport checkers. A single line eliminates the need for the decision-maker but incurs work to change the status quo.


\subsubsection{Parking garage}
% source: 
% https://graphthinking.blogspot.com/2019/07/altering-feedback-loops-to-change.html

%\marginpar{[Tag] Story Time}
\index{story time!parking garage}
\begin{storytime}{Parking Garage}
Bob parks his car in a parking garage every day. 
The parking garage owner charges \$20 per day for people to park their car.

Bob recently found that one of the exit gates for the parking garage is broken. If Bob uses that gate to leave the parking garage, the gate does not function and Bob cannot exit. Then Bob has to call the parking gate operator to request an exception (which is granted) and Bob can then exit that gate, avoiding the \$20 per day charge.

This action (go to broken gate, request exception, avoid charge) has been repeated for a long time (months). Bob's motive is to avoid the \$20 parking charge; the cost is a mere phone call and a minor delay. This cheating behavior harms the parking garage owner's income. However, the parking gate operator, serving as intermediary, insulates the parking garage owner from interaction with the cheater. The cheating behavior is small enough that the parking garage owner may not notice.
\end{storytime}

Having an intermediary policy enforcer alleviates the need for the garage owner to interact with customers. If the incentives of stakeholders are not aligned then inefficiency goes unchecked. If the parking garage operator's profits are a fixed rate rather than tied to parking charges, then the operator lacks the motive to fix problems.  % subsection
  \newpage
  \section{History of Bureaucracy\label{sec:history}}


Bureaucracy has repeatedly arisen independently in various cultures
\footnote{See the Wikipedia entry on the \href{https://en.wikipedia.org/wiki/Bureaucracy\%23History}{history of bureaucracy}.
\index{Wikipedia!\href{https://en.wikipedia.org/wiki/Bureaucracy\%23History}{history of bureaucracy}}
}
lasting for timescales that exceed the lifespan of one person.\footnote{See the YouTube video on the \href{https://www.youtube.com/watch?v=B_nsZlcC12g}{History of bureaucracy}.} That indicates the current situation is not a fluke or coincidence. There is some utility (or pathology) that is consistently recurring. 


Bureaucracy predates writing and language and even humans! Subjective policy enforcement in support of an organization arises in pre-human tribes, visible in groups of modern apes who have to manage access to shared resources~\cite{2016_Suchak}. 



Though bureaucracy is not new, the pervasiveness is. Before the industrial revolution the scale of employment and government were small, with organizations limited by the speed of communication. For the past 100+ years the size of organizations (commercial, governmental, and academia) have grown beyond \href{https://en.wikipedia.org/wiki/Dunbar\%27s_number}{Dunbar's number} -- the number of human relationships you can maintain, about 150. \iftoggle{WPinmargin}{\marginpar{[Wikipedia] Dunbar's\\number}}{}
\index{Wikipedia!\href{https://en.wikipedia.org/wiki/Dunbar\%27s_number}{Dunbar's number}}
More people participate in more organizations that are more bureaucratic. Driving this increase is the support for more complex products and processes. 

% claim: bureaucracy grow faster than the growth of human population?


\newpage % section
  \section{Scope of bureaucracy}
  %\begin{flushright}
  %Back to the \hyperlink{contents}{Table of Contents}
  %\end{flushright}
    \subsection{Who is a bureaucrat?}

A cashier in a gas station is a bureaucrat. The ``policy'' might simply be ``take money from customer in exchange for items and gas,'' but the subjective application of that policy leaves a lot of room for the cashier to shape the customer's experience. Does the cashier greet the customer when the customer enters the store? Does the cashier look at the customer to acknowledge the customer? Smile? How quickly does the cashier engage the customer? Minor nuances that are left to the cashier in the execution of the store policy means there is room for subjective application of the policy. 

This same discretionary application of policy applies to commercial bureaucrats like sandwich makers, car salespeople, grocery clerks, retail clerks. Teachers, police, tax collectors, and other state works are government bureaucrats. 

% bureaucracy is not limited to white collard office workers
Factory line workers subjectively apply policies. Enforcement of quality standards is the most obvious area. Pacing of work is a negotiation.

Sometimes bureaucrats do not work directly with customers or citizens or products. Then the bureaucratic process is inflicted on fellow bureaucrats. In this scenario, a bureaucrat is subjectively applying a policy to other bureaucrats. 

Identifying yourself as a bureaucrat matters, both to the employee and to the business. The risk of not self-identifying as a bureaucrat is that you won't grasp how much control you have in implementing and enforcing policy. If you think of yourself as having to blindly follow rules, you will harm the people you are applying the rules to and you will harm the business/institution you are applying the rules for. Adapting policies to circumstances is the value of having judgement capacity. 

In a similar sense from the consumer/citizen perspective, if you don't think you are interacting with a bureaucracy, you won't perceive the opportunity to negotiate.  If you view rules as fixed and inflexible, you will harm your ability to make progress. If a rule was made by a human, then that rule is flexible. Who made the rule? Who enforces the rule? If you can talk to them, could they be convinced to make a modification or an exception?
 % subsection
    \subsection{Number of people in a bureaucracy}
Although bureaucracy can be present for one person, and bureaucracy is often apparent on teams (e.g., 3 to 20 people), this book primarily focuses on the situation of multiple teams comprising an organization. This might be a few hundred people (above \href{https://en.wikipedia.org/wiki/Dunbar's_number}{Dunbar's number}) up to millions of people. 

There are tens of companies that employ more than a million people\footnote{see \href{https://en.wikipedia.org/wiki/List_of_largest_employers}{Wikipedia's list of largest employers}}, including Walmart, Amazon, and McDonald's. Bureaucracy emerges at a far smaller scale, is scale invariant, and is generic across sectors. 

Small companies with a few people incur bureaucracy. Non-profit organizations encounter bureaucracy. The complexity of the tasks may be different, but the same scale-independent patterns can emerge because of a common factor: human behavior.

Size of bureaucracy scales with the complexity of the problem. 

Scaling a bureaucracy up is easier than scaling down. % subsection
    \subsection{Why does bureaucracy exist? Can't we just do the work?}

The short answer is that bureaucracy is a response to the complexity of a problem being solved. To see why that is, let's start simple and then increase the complexity. 

The minimal scenario to start from is to imagine a single person working on a single task that does not last long (a few minutes), is relatively easy (cognitively and physically and emotionally), and does not recur. In that situation, building consensus is irrelevant and no process is required. 

Most of what you do occurs outside those limits and thus incurs some concept of \gls{process} (breaking a task into subtasks). Staying with the one-person constraint, a complex task can benefit from being broken into subtasks. Sometimes the order of the subtasks matters, so we need to track the dependencies. A recurring multi-step process with documentation is starting to have features of bureaucracy, but lacks the need for consensus. 

If one person lacks the skills relevant to a multi-step process, they may engage another person to help. The interaction may be informal (anarchy) or formalized in a contract (\href{https://en.wikipedia.org/wiki/Libertarianism}{libertarian}). If the parties working on the task fail to reach consensus, what is the recourse? Options include physical violence, threats, or involving a third party (e.g., a court with lawyers and judges). 


The bureaucrat's identity is subsumed into service for the organization they are part of. At the same time, bureaucracy enables the bureaucrat to amplify their presence by being part of a larger organization. A bureaucrat can accomplish more as part of an organization than by working alone. Sometimes the cost of being part of the organization exceeds the force multiplier of working together. 

% https://graphthinking.blogspot.com/2021/09/why-is-everything-so-hard-in-large.html

What if we completely avoided bureaucracy? That question is better worded by replacing ``bureaucracy" with ``coordination of stakeholders". If you avoid coordination of stakeholders, you either are constrained to only work on tasks that involve one person, or you get is random (uncoordinated) interactions. 

What if we minimized bureaucracy? Again, try replacing ``bureaucracy" in that question with ``coordination of stakeholders". The goal of ``minimizing coordination" probably isn't the real objective. To be more precise, a specific objective might be ``minimize time spent executing the task" (which takes a lot of coordination prior to the task execution) or ``minimize the level of distraction to stakeholders" (chunk the coordination time). Another strategy for minimizing bureaucracy is to reduce the number of stakeholders involved. For a given task complexity, this means having smarter people who have more skills. 

\begin{figure}
\includegraphics[width=0.8\textwidth]{images/people-per-task-for-skill-level.pdf}
\caption{Three levels of task complexity are shown. As task complexity increases, the size of the team needs to grow. The growth may be less if the team members are brilliant. Those brilliant people cost more and there are fewer of them.}
\end{figure}
 % subsection
    \section{Subject's View of Bureaucracy\label{sec:subjects-view}}

This section takes on the perspective of the subject of bureaucracy but is meant to be read by bureaucrats who want to improve their \gls{process empathy}. This book doesn't provide advice for subjects of bureaucracy.\footnote{For advice navigating bureaucracy, listen to National Public Radio's \href{https://www.npr.org/2022/03/16/1086915600/get-what-you-want-customer-service}{How to talk with customer service}~\cite{2022_LifeKit}.} 

\ \\

When you have a positive experience engaging with bureaucracy, your positive attribution is to the people involved. Or the ease of a solution makes bureaucracy less visible and the solution seems obvious. 
When you have a negative experience with bureaucracy, complaints are about the incompetence of the people involved or the incomprehensibleness of the system. Don't these bureaucrats know how to do their job? Why isn't the solution obvious? Why does this system not work for me? David Graeber summarized this view:\footnote{\href{https://harpers.org/archive/2015/03/in-regulation-nation/}{In Regulation Nation. Harpers Magazine, 2015.}}
% May have actually come from "The Utopia of Rules"
\begin{quote}
 Amongst working-class Americans, government is generally seen as being made up of two sorts of people: `politicians,' who are blustering crooks and liars but can at least occasionally be voted out of office, and `bureaucrats,' who are condescending elitists almost impossible to uproot.   
\end{quote}


%\subsection{Sources of complexity}

The scale of bureaucracy (the number of people in an organization) and the processes of an organization can seem disproportionate to the complexity of the task. Typically when a person (being a subject of bureaucracy) interacts with the bureaucratic organization the artifacts are simple, like a form to fill out. The simplicity of the artifact does not correlate to the number of decisions made, the tracking of information, or the precautions taken by the organization. All these aspects are invisible to the subject because they are internal to the organization.

As an example, consider when you go to the doctor and they mend your broken arm with a cast. That seems straightforward because all you see is the doctor putting a cast on. You don't get insight into the decisions they had to make. Why did they need 20 years of focused schooling to carry out a procedure that took 15 minutes?

Judging bureaucracy by the artifact visible to the individual subject undersells the complexity of the decision-making necessary to take action. The contingencies that you were not exposed to because everything went well make the amount of investment from the bureaucrat appear wasteful.

As the subject of bureaucracy, you also lack the ability to distinguish how much work is attributable to the bureaucrat generating justifications for their actions (colloquially, \href{https://en.wikipedia.org/wiki/Cover_your_ass}{covering the bureaucrat's ass}). 
\index{Wikipedia!\href{https://en.wikipedia.org/wiki/Cover_your_ass}{cover your ass}}
\iftoggle{WPinmargin}{\marginpar{[Wikipedia] Cover\\your ass}}{}
These justifications are needed both within the organization and potentially for external stakeholders. Each bureaucrat's rationalization may not be reviewed, but it needs to be available for review later.

As the subject of bureaucracy, you typically can't distinguish when work is caused by a bureaucrat's ill-informed decisions. Is the person stupid, mistaken, or is there something you are not taking into account?
Sometimes the work is carried out by insufficiently trained bureaucrats, but you don't get to know whether you're working with an experienced and knowledgeable bureaucrat or a new untrained bureaucrat. 

As the subject of bureaucracy, you don't have visibility on the many nuances of an organization. The internal power struggles and organizational politics that depend on personalities, resources, and competing prioritization are not clear to outsiders.
The inconsistency of an organization's policies may not be felt by bureaucrats in that organization. Each bureaucrat may have a different opinion, resulting in a lack of consistent guidance.
There may be legacy policies in effect.

% summary paragraph
Effective action by a bureaucratic organization is complicated by the need for relevant information. Gathering, analyzing, and sharing that information persistently requires bureaucratic processes. To further complicate the ideal process, sometimes the organization lacks the staffing with relevant skills. 


% Transition paragraph
The next section documents why bureaucracy is hard from the perspective of the bureaucrat. Even without getting into the specifics of a bureaucrat's role or the purpose of an organization, there are generic reasons that bureaucracy is a burdensome responsibility.
  %\section{From Startup to Bureaucracy}

In a novel domain, people show up in order: nerds, then commoners, then jackasses


\chapter{Bureaucracy in general}
  \section{Each Phase of Life Involves Bureaucracy}

\href{https://en.wikipedia.org/wiki/TL;DR}{TL;DR}: 
\index{Wikipedia!\href{https://en.wikipedia.org/wiki/TL;DR}{TL;DR}}
You have experience with bureaucracy, though you may not have framed the experiences as bureaucratic. Viewing relationships and roles through the lens of bureaucracy explains the limitations of your expectations.

\ \\

In each person's life there are standard milestones: birth, education, work, death. 
%The relationships associated with each phase are distinct.
Each of these milestones and phases involves bureaucracy. Each phase is a different experience of bureaucracy because the bureaucratic roles change.

Bureaucracy is composed of three roles: the policy creator, the policy enforcer, and the subject (upon whom the policy is inflicted). 
\index{bureaucracy!roles}
These roles can be confounded by oversubscribing the label ``bureaucrat." Who is the bureaucrat and who is the subject depends on the relationship in a scenario. 

For example, a store manager creates a policy, a store clerk enforces the policy, 
\index{exemplar!store clerk}
and the policy is inflicted on the customer. Both the manager and the store clerk would be bureaucrats, while the customer is the subject. In a separate example, someone at corporate headquarters sets a policy, the store manager enforces it, and the policy is inflicted on the store clerk. Then the clerk was the subject of bureaucracy. 


\subsection*{Bureaucracy of Birth\label{sec:bureacracy-of-birth}}
Your birth was marked by getting a name, registering with the state, and initiating medical records. These tasks were administered by bureaucrats (doctors, nurses, and other hospital staff) on your behalf, and you were the subject of the bureaucracy. You had no autonomy or decision-making authority. 

\subsection*{Bureaucracy in Early Childhood\label{sec:bureaucracy-early-childhood}}
Before starting formal education, the bureaucracy of early childhood is inflicted primarily by family members setting and carrying out policies. The organization of bureaucrats is the family, with the shared resources being housing, food, and experience with survival. Other community members or caretakers may also be involved in carrying out the policies of taking care of you. Your decision-making authority as a subject in this bureaucracy was minimal. 

\subsection*{Bureaucrats at School\label{sec:bureaucracy-of-school}}
Once you started the formal education process, new bureaucrats got involved.  The community of bureaucrats could be a public school, a private school, or homeschooling. In any of those cases, the frontline bureaucrat is the teacher. You're not responsible for making policies that other people follow; you are still the subject of bureaucracy.


The expectations of each phase of school (high school, undergraduate, graduate school) are distinct, and they are different from working in a large organization. Your autonomy increases throughout the duration of school. %, and your ability to make policy that effects others grows. 
Your family and teachers are the bureaucrats. You start building informal organizations of friends, and you start to explore policies around social bonding.

Schooling sets a pattern that most students will fall into for the rest of their lives: you were handed a textbook and told to solve a set of problems. That pattern of taking direction can persist for a long time. However, you are not constrained to persist with that limitation. You have the autonomy to do more than what is required. You can find other textbooks that match your interests or are written from a different view. 

You get to choose the book you want, even if you don't get to choose the topic you will be evaluated on. You don't  need to pick just one reference book -- you can review lots of books and figure out which author style best fits you. You can also choose the level of difficulty -- basic and beginner level, or more advanced. You can choose what's best based on your understanding rather than defer to what a class in school is supposed to cover.

As a subject of the education bureaucracy, you can discover how you learn best. This extra effort requires self-reflection: How do you learn? What works best? What didn't work, and why not? What did you learn? What do you wish you had learned?

Another pattern that schooling relies on is single-question decisions with only one right answer. Examples include math problems and multiple-choice tests. Schooling tends to avoid setting up dilemmas or paradoxes for students. Academic problems in the education process are designed to be independent of the people involved or the history of the situation. 


% https://graphthinking.blogspot.com/2013/02/all-little-things.html
%Where you sit in the classroom matters. Being in the front means you will be exposed to fewer distractions, more likely to pay attention.


% https://graphthinking.blogspot.com/2012/09/how-to-not-be-average.html
%How were you taught? Did you have any input on the method?
%Who taught? Did you like them? Were they friendly, knowledgeable, and approachable?


% https://graphthinking.blogspot.com/2012/09/how-to-not-be-average.html
%What resources are available now for you to learn from? Do you like learning?
%you have the freedom to pursue what ever intellectual endeavors you want.


% https://graphthinking.blogspot.com/2011/09/which-skills-are-useful-after.html

% did not prepare me for addressing challenges at work. 
In my schooling I learned how to approach technical issues and develop solutions. That problem-solution paradigm neglects crucial steps of discovering the problem, isolating the challenge, identifying stakeholders, learning the history of the challenge, and negotiating with stakeholders before trying to address the challenge. 

My schooling led me to emphasize academics over socializing. When I transitioned to professional work, I found social skills and political savvy useful when trying to change organizations and policies. 

Academic problems are intended to be solvable and answers submitted get evaluated. In contrast, working in a bureaucratic organization the challenges are ill-defined, there's no known solution, and the topic is sufficiently complex that you have to collaborate.

% https://graphthinking.blogspot.com/2018/07/the-difference-between-problems-at.html


%\subsection*{undergrad vs graduate}
% education process roles and expectations vary over time

%college, graduate school: friends, teachers, advisors



\subsection*{Military Service Bureaucracy\label{bureaucracy-of-military}}
\index{exemplar!military}
Somewhere between 13 percent\footnote{\href{https://www.cbpp.org/research/federal-budget/where-do-our-federal-tax-dollars-go}{https://www.cbpp.org/research/federal-budget/where-do-our-federal-tax-dollars-go}} and 20 percent\footnote{\href{https://www.nationalpriorities.org/analysis/2019/tax-day-2019/where-your-tax-dollar-was-spent-2018/}{https://www.nationalpriorities.org/analysis/2019/tax-day-2019/where-your-tax-dollar-was-spent-2018/}} of the United States federal budget is spent on military, and 
less than 0.5\% of the United States population serves in the military.\footnote{\href{https://www.cfr.org/backgrounder/demographics-us-military}{Council on Foreign Relations}\iftoggle{boundbook}{, https://www.cfr.org/.}{.}}
For those who serve, the military's rigid hierarchy and defined protocols are a distinct experience compared to school or work. Transitioning from military to civilian life can present a dissonance for service members used to the chain of command and clearly defined orders. 

While you are in the military the hierarchy and use of inefficient policies feel stifling. Once out of the military you may reflect fondly on the clarity of orders compared to the vagaries of social politics.  

\subsection*{Working in a Bureaucratic Organization\label{sec:bureaucracy-of-work}}
%Within the employment phase of life, there are pairs of events which may apply: hiring or getting hired; firing, getting fired, or quitting. 

%Employment: managers above you, peer employees, people you manage are all members of an organization. 


Organizations with communal workspaces have \glspl{shared resource}: bathrooms, conference rooms, kitchen areas with fridges and microwaves, storage areas. Each of these incurs policies of use. Unlike being a student at school, you may find yourself responsible for developing and enforcing policy. 

As an example of workplace policies, consider the following scenario. 
%\marginpar{[Tag] Story Time}
\index{story time!bathroom smells}
%\begin{storytime}{The Bathroom Stinks}
\begin{mdframed}[frametitle={The Bathroom Stinks},frametitlerule=true,frametitlealignment=\centering]
The bathroom at work sometimes smells, so I'm a nice person and bring a scented air freshener. Unbeknownst to me, that triggers asthma or an allergic reaction in one of my coworkers. Because of this a policy gets created so this mistake doesn't happen again. 

Signs are posted. Violations are reported to management even if no one has a physical reaction to the air freshener.
%\end{storytime}
\end{mdframed}

Policies are often created in response to specific incidents. This intent can be helpful (promulgating lessons learned is efficient) or unhelpful (when policies are an overreaction). 

\subsection*{Healthcare and Death\label{sec:bureaucracy-of-death}}
Medical care alters our life, and a lot of money is spent on medical care: 25\% of the federal budget in the United States.\footnote{\href{https://www.cbpp.org/research/federal-budget/where-do-our-federal-tax-dollars-go}{https://www.cbpp.org/research/federal-budget/where-do-our-federal-tax-dollars-go}} Understanding the bureaucracy of healthcare is outside the scope of this book, but your role in the bureaucracy is helpful to understand.

Doctors, nurses, and other staff are bureaucrats; you are the subject; the hospital or clinic is the organization. 

\ \\
To illustrate bureaucracy of a large organization, consider the importance of toilet paper at a hospital. 
%\marginpar{[Tag] Story Time}
\index{story time!toilet paper at hospital}
%\begin{storytime}{Restocking Toilet Paper}
\begin{mdframed}[frametitle={Restocking Toilet Paper},frametitlerule=true,frametitlealignment=\centering]
If you go to a hospital, use the bathroom, and find there is no toilet paper, that would indicate a deficiency of the hospital.

The \textit{holistic view} is that someone didn't refill the toilet paper. Since the person who usually restocks toilet paper wasn't also a user, they aren't directly affected by the lack of toilet paper.

A routine is needed for checking the availability of toilet paper in bathrooms. 

The \textit{perspective of the purchasing manager} is that money spent checking the status of toilet paper is money not spent on the hospital's primary mission: improving the health of community members.

Minimizing checking of toilet paper is important for the organization's reputation, so a feedback mechanism is instituted: a phone number is posted in the bathroom so users can send a text regarding bathroom status at the hospital.

The \textit{perspective of the janitor} is that my routine used to be to go to each bathroom after normal business hours and refill toilet paper. Now I have to do that and be on-call when someone alerts management that service is needed. My responsibilities increased but my pay did not.

%\end{storytime}
\end{mdframed}

%\ \\

%Death can invoke both the medical system and the government. 

%\ \\


%All of these roles and relationships involve 

Bureaucracy is a set of subjectively administered policies within an organization. By recognizing the role of bureaucrats you can identify what is negotiable. 
 \clearpage
  \section{Avoiding Bureaucracy is Nearly Impossible}
\iftoggle{shortsectiontitle}{\sectionmark{Avoiding Bureaucracy}}{}

The only situation where bureaucracy might not exist is if you live entirely independently and have no interaction with other people. That means completely disengaging from society. Even then, personal routines are a self-imposed form of bureaucracy, with the roles of policy maker, bureaucrat, and subject collapsed to a single person -- you.

Self-sufficiency and autonomy are attractive alternatives to bureaucracy. The way participants in modern society strive for self-sufficiency is by denying their dependence on modern society. That's a relabeling of selfishness which feels better. 

For the rest of us who operate as members of  society, bureaucracy is necessary for our rights. You validate your name using paperwork, forms, and records. These artifacts are used, in cooperation with other people, to determine your claim of citizenship and associated rights. 
That's a subjective policy that \iftoggle{glossarysubstitutionworks}{\glspl{stakeholder}}{stakeholders} in society agree to. 


Bureaucracy is necessary because it is a response to the \iftoggle{WPinmargin}{\marginpar{$>$Wikipedia: Collective action problem}}{}%
\href{https://en.wikipedia.org/wiki/Collective_action_problem}{collective action problem} -- everyone would benefit from cooperation, but each person has to sacrifice their self-interests.  
\index{Wikipedia!collective action problem@\href{https://en.wikipedia.org/wiki/Collective_action_problem}{collective action problem}}
As long as humans form communities, we will address the challenge of \iftoggle{glossarysubstitutionworks}{\glspl{shared resource}}{shared resources}
 -- whether tangible (e.g., water, land, air) or intangible (e.g., expertise, information). 
That is why bureaucracy is culturally invariant and persistent across time.
Learning to be an effective bureaucrat improves your chances of success in society. 

The specific way society is constructed (democratic, authoritarian, dictatorship, anarchy) is irrelevant -- bureaucracy is still present. Even the libertarian approach of relying on contract enforcement implies some bureaucracy (e.g., forums for resolving contract disputes like a court system, enforcing decisions through violence). 


Not all bureaucracy is due to the state, nor is bureaucracy confined to companies. Parenting involves creating situation-specific requirements for children, with the organization being the family as mentioned in the section 
on \hyperref[sec:bureaucracy-early-childhood]{bureaucracy in childhood}.
\marginpar{See page~\pageref{sec:bureaucracy-early-childhood}.}%
Dress codes for sports teams are arbitrary standards. 
Store clerks are bureaucrats, as are website forum moderators.  Content moderation is the process of (inconsistently) enforcing arbitrary standards. This mindset even permeates individuals as internalized expectations of policy and enforcement when no one else is present. 

Recognizing instances of bureaucracy enables more skillful interaction, whether as a bureaucrat or subject. The rest of this section illustrates both the view of a person interacting with bureaucracy as a \gls{subject} and the perspective of bureaucrats working within organizations. 




 \clearpage
  %\section{Folk Wisdom}
While most of the entries on
 https://github.com/dwmkerr/hacker-laws
don't apply to bureaucracy, the list is useful to review. 

Folk wisdom is an attempt to explain bureaucratic features but ends up being \gls{thought terminating}.

\href{https://en.wikipedia.org/wiki/Murphy\%27s_law}{Murphy's law}

\href{https://en.wikipedia.org/wiki/Wiio\%27s_laws}{Wiio's laws}

\href{https://en.wikipedia.org/wiki/Hanlon\%27s_razor}{Hanlon's razor}

\href{https://en.wikipedia.org/wiki/Parkinson\%27s_law}{Parkinson's law}

\href{https://en.wikipedia.org/wiki/Putt\%27s_Law_and_the_Successful_Technocrat}{Putt's Law}

\href{https://en.wikipedia.org/wiki/Hick\%27s_law}{Hick's law}

\href{https://en.wikipedia.org/wiki/Allen_curve}{Allen curve}

\href{https://en.wikipedia.org/wiki/Peter_principle}{Peter principle}

\href{https://en.wikipedia.org/wiki/Dilbert_principle}{Dilbert principle}

\href{https://en.wikipedia.org/wiki/Law_of_triviality}{Law of Triviality}\clearpage
  \section{Bureaucracy as Accidental, Legacy, or Essential}
\gls{essential bureaucracy} is the minimum processes and staffing and skills necessary to address the complexity of the managing a community's access to a \gls{shared resource}. Achieving this minimum is tricky since the optimization can be with respect to resilience to change, resilience to edge cases, staff turn-over, speed experienced by consumer, financial cost, time spent by the organization, and number of staff. Miss any one of those and the bureaucracy is deemed inefficient.

Undesirable bureaucracy is categorized as either accidental or legacy. Accidental bureaucracy arises when someone misunderstands what is needed, or when skills of the bureaucrats involved are insufficient for the complexity a problem. Legacy bureaucracy occurs when the situation changes but the processes do not. Resolving each of these suboptimal conditions may seem easy: have better knowledge of the problem, assign the right people to the problem, and change processes as problems evolve. 

Having enough knowledge is often infeasible, especially for complex problems at large scale. Having the people with the right skills assumes that a pipeline of people with relevant talents exists and that people in the pipeline won't be poached to work on other challenges. Keeping up with evolving problems depends on having resources to change (beyond the maintenance baseline), and having a defined approach for changing the process. \clearpage
  \section{Bureaucratic Fallacies\label{sec:fallacies}}
\index{bureaucratic fallacy}

Discussions about bureaucracy by non-experts often rely on common conceptions that are \gls{thought-terminating}. Identifying these enables you to understand both why the fallacy is attractive during a discussion with fellow bureaucrats and how each idea is incomplete.

These fallacies may at first feel right but are in fact misleading. In contrast, there are  
\hyperref[sec:unavoidable-hazards]{unavoidable hazards} \iftoggle{haspagenumbers}{(see page~\pageref{sec:unavoidable-hazards})}{} that may feel bad but reveal underlying truths.

\ \\
\begin{samepage}
\textit{Bureaucratic fallacy}: \textbf{Bureaucracy is bad}. \\
\index{bureaucratic fallacy!bureaucracy is bad}
\textit{Why this feels true}: When a person subjected to bureaucracy has a negative experience, the easiest attribution is to the least-understood aspect -- the bureaucracy.\\
\textit{What this is missing}: 
\iftoggle{glossarysubstitutionworks}{\Gls{bureaucracy}}{Bureaucracy}
 as defined in this guide is neither good nor bad. Bureaucracy is merely a way of managing  resources shared amongst a community. 
 \end{samepage}

\ \\
\begin{samepage}
\textit{Bureaucratic fallacy}: 
\textbf{There is no point in planning ahead since everything (staffing, funding, purpose, scope) is always changing.}\\
\index{bureaucratic fallacy!no point in planning}
\textit{Why this feels true}: Change can feel disorienting, especially when it is unexpected. A change of the assumptions for a plan may make the plan less relevant. \\
\textit{What this is missing}: Preparing for change and thinking ahead about contingencies enables effective use of resources. Have a vision and work towards it while accounting for and adapting to change. This approach requires extra work, some of which will be left unused. See Dilemma \ref{table:dilemma-personal-emergencies-vs-ignore}\iftoggle{haspagenumbers}{ on page~\pageref{table:dilemma-personal-emergencies-vs-ignore}.}{.}
\end{samepage}

\ \\
\begin{samepage}
\textit{Bureaucratic fallacy}: \textbf{Bureaucracy is an aberration, a mistake, due to poor planning or incompetent participants}. \\
\index{bureaucratic fallacy!bureaucracy is a mistake}
\textit{Why this feels true}: Mistakes are made, poor or insufficient planning does happen, and some participants are incompetent.\\
\textit{What this is missing}: Bureaucracy occurs even if no mistakes are made, effort is spent on effective planning, and participants are competent. That's because, in the use of distributed knowledge and distributed decision-making, bureaucrats face \hyperref[sec:dilemma-trilemma]{dilemmas}.
\marginpar{See page~\pageref{sec:dilemma-trilemma}.}
%\ifsectionref
%(see section~\ref{sec:dilemma-trilemma}).
%\fi
\end{samepage}

\ \\
\begin{samepage}
\textit{Bureaucratic fallacy}: \textbf{Bureaucracy is inefficient}. \\
\index{bureaucratic fallacy!bureaucracy is inefficient}
\textit{Why this feels true}: Expressed by both subjects and bureaucrats who observe seemingly wasteful processes.\\
\textit{What this is missing}: If bureaucracy were truly inefficient (not allocating resources in the most efficient way), then in a competitive environment it would be replaced by a more efficient approach. The key is to ask, ``efficient with respect to what metric?'' The metric of money, or time, or number of people, or stability, or robustness to perturbation?  Second, what would motivate improved efficiency? Without incentives, change is less likely. 
\end{samepage}

\ \\
\begin{samepage}
\textit{Bureaucratic fallacy}: \textbf{Bureaucracy is due to malfeasance.}\\
\index{bureaucratic fallacy!bureaucracy is due to malfeasance}
The specific number of malicious bureaucrats in variations on this fallacy ranges from ``all of the participants'' to ``just enough to be problematic.'' \\
\textit{Why this feels true}: There are bad actors present in any system comprised of humans. \\
\textit{What this is missing}: Most participants are earnestly trying to help make a positive contribution, even though that can be hard to see from the view of subjects or even other bureaucrats. Processes like isolation or promotion exist within bureaucracy to deal with malicious bureaucrats.
\end{samepage}

\ \\
\begin{samepage}
\textit{Bureaucratic fallacy}: \textbf{Bureaucracy is a sign of decay from within the organization.} \\
\index{bureaucratic fallacy!bureaucracy indicates decay}
\textit{Why this feels true}: Relationships within an organization have a half-life and require ongoing investment to renew. At the same time, new bureaucratic processes are constantly being developed by other bureaucrats. The number of processes increases as the organization ages. Bureaucracy seems to arise without effort and countering it takes effort.  \\
\textit{What this is missing}: Bureaucracy unavoidably emerges in every organization because coordination is required. The negative connotation of decay should be replaced with a sense of neutral evolution.
\end{samepage}

\ \\
\begin{samepage}
\textit{Bureaucratic Fallacy}: \textbf{If the response to a request I make can't be expedited, my request must not be important}.  \\
\index{bureaucratic fallacy!importance is measured by response latency}
\textit{Why this feels true}: Other people would demonstrate they care about what I am working on by prioritizing things I am dependent on.\\
\textit{What this is missing}: When everything gets prioritized, that's the same as nothing getting priority.
\end{samepage}

\ \\
\begin{samepage}
\textit{Bureaucratic Fallacy}: \textbf{The expected duration of a task is how long it would take one person to accomplish}.  \\
\index{bureaucratic fallacy!task duration for one person}
\textit{Why this feels true}: When I imagine carrying out a task, the default is a story with one character. \\
\textit{What this is missing}: This narrative fails to account for the overhead of interaction among participants and delays due to asynchronous dependencies.~\cite{1975_brooks}
\end{samepage}

\ \\
% https://graphthinking.blogspot.com/2019/08/two-misleading-simplifications-when.html
\begin{samepage}
\textit{Bureaucratic Fallacy}: \textbf{When developing or altering policy, focus on the average or majority (to the exclusion of outliers)}. \\
\index{bureaucratic fallacy!ignore outliers}
\textit{Why this simplification is misleading}: Sometimes outliers are not just more of the same; they alter the outcome. During the transition from horses-for-transportation to cars, cars could initially have been considered outliers. 
\end{samepage}

\ \\
\begin{samepage}
\textit{Bureaucratic Fallacy}: \textbf{People learn from their mistakes}. \\
\index{bureaucratic fallacy!people learn from their mistakes}
\textit{Why this feels true}: There's an optimistic desire for this to be true. \\
\textit{What this is missing}: 
People repeat mistakes without noticing. People do not naturally reflect on their failings in a constructive way and then apply insights to future situations.
People \textit{can} learn from their mistakes. Doing so requires a low latency feedback loop and incentive to change. In bureaucracies feedback loops are weak so learning may not happen.
\end{samepage}

\ \\
\begin{samepage}
\textit{Bureaucratic Fallacy}: \textbf{Processes are serial}.\\
\index{bureaucratic fallacy!processes are serial}
A conventional approach to process design is a sequence of tasks. As an example, consider approval chains. \\
\textit{Why this feels true}: Serial processes are easier to understand. \\
\textit{What this is missing}: Some tasks that are independent can be carried out concurrently; see the section on \hyperref[sec:reducing-overhead]{reducing overhead}\iftoggle{haspagenumbers}{on page~\pageref{sec:reducing-overhead}.}{.}
\end{samepage}

\ \\
\begin{samepage}
\textit{Bureaucratic Fallacy}: \textbf{Hard work creates results}.\\
\index{bureaucratic fallacy!hard work creates results}
\textit{Why this feels true}: Some results do require hard work. The alternative (results arise serendipitously or with little effort) is not inspiring. \\
\textit{What this is missing}: Hard work can be invested on wasteful effort. Don't confuse being busy with being productive. Sometimes insight is more useful than hard work. 
\end{samepage}

%\ \\

% NOT USEFUL
%\textit{Bureaucratic Fallacy}: \textbf{Motivations for bureaucrats are categorized as individualistic, tribal, organizational, societal, or humanity}.\\



\ \\
\begin{samepage}
\textit{Bureaucratic Fallacy}: 
\textbf{You cannot pay a little and get a lot}; see the \href{https://en.wikipedia.org/wiki/Common_law_of_business_balance}{Common law of business balance}. 
\index{Wikipedia!\href{https://en.wikipedia.org/wiki/Common_law_of_business_balance}{Common law of business balance}}
\marginpar{$>>$ Folk Wisdom} 
\index{folk wisdom!\href{https://en.wikipedia.org/wiki/Common_law_of_business_balance}{Common law of business balance}} \\
\index{bureaucratic fallacy!cannot pay a little and get a lot}
\textit{Why this feels true}: If small investments made a big difference we'd already be making the investment.\\
\textit{What this is missing}: This doesn't account for creative solutions and ignores \href{https://en.wikipedia.org/wiki/Nudge_theory}{nudge theory}
\index{Wikipedia!\href{https://en.wikipedia.org/wiki/Nudge_theory}{nudge theory}}
from behavioral economics. 
\end{samepage}

\ \\

When you are reasoning about bureaucratic systems, there may be a conclusion that is concise and feels explanatory. Then you should try to come up with counter-examples, either logically or from experience.  

Some fallacies are based on an expectation that other people should be more like what you imagine your best self to be. That mindset fails to account for your  shortcomings and the diversity of other bureaucrats. 

 \clearpage
  \section{Bureaucrat's experience of Dilemmas\label{sec:dilemma_trilemma}}

% Not included here: 
% https://en.wikipedia.org/wiki/The_Innovator%27s_Dilemma
% because it is the etic view of change


When a decision has two viable options (neither being best), that presents a \href{https://en.wikipedia.org/wiki/Dilemma}{dilemma}. The name for a decision with three viable options is a \href{https://en.wikipedia.org/wiki/Trilemma}{trilemma}. This section describes the bureaucrat's experience of operating within an organization in terms of dilemmas and trilemmas. 
% https://en.wikipedia.org/wiki/Defeasible_reasoning

%Dilemma are not unique to bureaucracy. 
Dilemmas are a \href{https://en.wikipedia.org/wiki/Defeasible_reasoning}{simple way} of discussing decisions but are not the only relevant aspect of bureaucracy.
Dilemmas are merely a useful framing to highlight the following ideas:
\begin{itemize}
    \item You, an individual bureaucrat, face decisions that you may not have recognized. Failing to recognize a choice can lead to suboptimal results. The dilemmas below are generic to any bureaucratic process and are intended to stimulate your ability to identify decisions. 
    \item You face complex trade-offs in your role as a bureaucrat. Dilemmas are intended an entry point to more nuanced reflection that is specific to your situation. The dilemmas below are not exhaustive; this list is merely illustrative. 
    \item You can build \href{https://en.wikipedia.org/wiki/Theory_of_mind}{intellectual empathy} with fellow bureaucrats when you recognize they face the same dilemmas. These dilemmas are generic to the situation and the person facing the decisions. You now have a topic to discuss with them.
    \item You can be curious about the choice other bureaucrats select for a given dilemma. Everyone in the organization faces these decisions, so these dilemmas give you a topic of conversation to better understand each bureaucrat's view.
    \item You can identify and negotiate potential sources of friction. Other bureaucrats may arrive at different selections for a given dilemma, so recognizing this and discussing it can improve the effectiveness of all involved.
\end{itemize}


% claim
Dilemmas explain the inherent complexity of bureaucracy even when bureaucrats are honest and the purpose of the organization is clear.
% relevance
The point isn't that one should \href{https://en.wikipedia.org/wiki/False_dilemma}{select one of the two options}. The point of recognizing a dilemma is that it is a marker that there is a decision to be made.
% consequence
The action for the bureaucrat in response to this assessment is to identify nuances, enumerate alternatives, and talk with fellow bureaucrats about these decisions. Rather than seeking consensus, strive for comprehension of other people's perspective. That way you can navigate processes more effectively.




\subsection{How Dilemmas Arise in Bureaucracy}

Given a task, the simplest response for an individual is to take action and be done. 

If the task is more challenging, it may be necessary to first make a plan, then take action and be done. The source of the challenge may be task complexity, scale, number of people impacted, diversity of stakeholders, amount of time needed, how the current task shapes future decisions, number of collaborators, etc. Regardless of why the task is challenging, a dilemma has already arisen: how much time to spend planning versus doing. 

If the task is even more challenging, it may be useful to gather data for the plan, then make a plan, then take action and be done. A new dilemma arises: how much time to spend gathering data versus planning. The previous dilemma still exists -- planning versus doing. As an alternative to the dilemma framing, how much time should be allocated to three categories of activity?

But wait, how did the person in the first scenario (just do the task) know no plan was necessary? Either it was an explicit choice or they didn't perceive the need to decide about whether to plan. Someone else faced with that same task might choose differently, i.e., to make a plan. The task complexity is relative to the person's skills, experiences, expectations about stakeholders, and potential ramifications.

In this escalating sequence of increasingly challenging tasks, suppose the task involves you and another person. The overhead of coordination inflicts additional dilemmas. The examples of complexity arising from coordination are identified in the list of dilemmas below.

An independent source of friction is when people involved in the task don't agree on how challenging it is. The framing of the difficulty matters because it can lead different participants to distinct conclusions about how much time should be spent planning versus for carrying out the action. The choice of ``how complicated is the task?'' shapes the team dynamics and informs the need for hierarchical roles. 

\subsection{Folk Wisdom on Decision Making}

Bureaucrats, whether hired for their expertise or simply to provide labor, are rarely experts on decision making. There are multiple domains in which \href{https://en.wikipedia.org/wiki/Decision_theory}{decision making} is studied (e.g., \href{https://en.wikipedia.org/wiki/Rational_choice_theory}{economics}, \href{https://en.wikipedia.org/wiki/Game_theory}{mathematics}, \href{https://en.wikipedia.org/wiki/Decision-making}{psychology}), but practicing bureaucrats are more likely familiar with colloquialisms that feel descriptive. A few are provided here to give a sense of both the conciseness and the lack of action embedded in each meme.

\ \\
\href{https://en.wikipedia.org/wiki/Hick\%27s_law}{Hick's law}\marginpar{[Tag] Folk wisdom}: ``Increasing the number of choices will increase the decision time logarithmically.''

\ \\
\href{https://en.wikipedia.org/wiki/Hanlon\%27s_razor}{Hanlon's razor}\marginpar{[Tag] Folk wisdom}: ``Never attribute to malice that which is adequately explained by stupidity.''

\ \\
\href{https://en.wikipedia.org/wiki/Parkinson\%27s_law}{Parkinson's law}\marginpar{[Tag] Folk wisdom}: ``Work expands so as to fill the time available for its completion.''

\ \\
\href{https://en.wikipedia.org/wiki/Murphy\%27s_law}{Murphy's law}\marginpar{[Tag] Folk wisdom}: ``Anything that can go wrong will go wrong.''

\ \\
\href{https://en.wikipedia.org/wiki/Law_of_triviality}{Law of Triviality}\marginpar{[Tag] Folk wisdom}: ``People within an organization commonly or typically give disproportionate weight to trivial issues.''

\ \\

In comparison to the folk expressions above, dilemmas are intended to be accessible to practicing bureaucrats and trigger reflection and discussion. Effective action is enabled by improved understanding of the trade-offs faced within an organization. 

\subsection{Dilemmas are a poor framing}

Although the following concepts are presented as dilemmas and trilemmas, these are necessarily simplifications. For example, two ways to simplify situations into a dilemma are
\begin{itemize}
    \item Start with a single variable, e.g., ``how much data gathering", and force it into a binary choice ``more data gathering" versus ``less data gathering."
    
    % a reduction of a complicated situation to one variable. 
    \item Start with a complex trade-off space with many opportunities and reduce it to a \href{https://en.wikipedia.org/wiki/False_dilemma}{false dichotomy} of \href{https://en.wikipedia.org/wiki/Zero-sum_thinking}{zero-sum options}: ``more data gathering" vs ``more planning." The actual trade-space involves optimization of multiple objectives, like (maximize productivity) and (minimize risk) and (maximize quality) and (maximize employee satisfaction) and (minimize latency). 
    % https://bennorthrop.com/Essays/2022/code-ownership-stewardship-or-free-for-all.php
\end{itemize}
These over-simplifications neglect both the continuous nature of the trade-offs and the alternative creative approaches to a specific situation. 

%\subsection{Dilemmas as a Framing to Ease you into Complexity}

The choices described below in the simplified representation of dilemmas are intended as a starting point for introducing the decision space relevant to individual bureaucrats. Do not accept the dilemmas presented here as the final framing. Thinking in terms of a limited spectrum of opportunities neglects nuances that enable more creative approaches. 
Awareness of these dilemmas and trilemmas are intended to spark creative imagination about the nuances specific to your situation.

By recognizing the deficiencies of dilemmas, you can identify nuances specific to the situation you are in. Instead of responding to the recognition of a decision with ``should I do this or that?'' the better option is to assess the complexities and \href{https://en.wikipedia.org/wiki/Brainstorming}{brainstorm} multiple options. Talk with stakeholders and understand the history of the situation before making a choice.


Ponder theses dilemmas prior to the pressure of real-time decision making.  Recognize dilemmas and trilemmas and then avoid them by adapting to the local conditions and specific people available to help.

\subsection{Dilemmas are not Purely Intellectual}
Decision making is not a purely intellectual task; there is emotional stress induced by the process. Dilemmas create cognitive dissonance for the decision maker. Any selection is going to have downsides, and any compromise will be suboptimal. Those burdens weigh on deciders.

To counter this morale weight, talk with other people about decisions. Even if this does not alleviate the responsibility of deciding, discussion can help you arrive at new insights. 

\subsection{Additional Complications}
Adding to the difficulty and stress, dilemmas presented here are occur concurrently and continuously. The dilemmas are inter-dependent due to both to the common variables and the constrained resources.
Selecting an option for one dilemma alters the options available in other dilemmas.

Oscillation between approaches can be caused by change of management, accumulation of experience (dissatisfaction) with one solution, the desire for promotion (change as progress), or a desire for cost savings (efficiency). The rate of oscillation is an indicator of the half-life of \href{https://en.wikipedia.org/wiki/Institutional_memory}{institutional memory for an organization}.  

\ \\

The following two sections categorize dilemmas as personal policies (\S\ref{sec:personal_policy_dilemmas}) and policies regarding structure of the organization (\S\ref{sec:org_dilemma}). The personal policies apply to each bureaucrat in an organization, while the structural policies for an organization are faced by a subset of bureaucrats in the management role. 

\subsection{Personal Policy Dilemmas \label{sec:personal_policy_dilemmas}}

In practice, the following decisions are unordered and are constantly faced by the bureaucrat. As observed by Lindblom in \cite{1959_Lindblom}, this flurry of decisions contrasts to a regularized process that might be envisioned as optimal.


  
\begin{center}
\begin{table}[H] % ht
\begin{tabular}{ | m{\dilemmatablewidth}| m{\dilemmatablewidth} | } 
  \hline
  \textbf{Intervene before the deployment of a policy or process or product, perhaps lacking relevant context.} &
  \textbf{Wait with feedback until deployment.} \\
  \hline
  \textit{Cons}: Engaging prematurely betrays your awareness; future explorations by that team are made less visible. & 
  \textit{Cons}: The team wasted time and attention on something that wouldn't work or may even be harmful. \\
  \hline
\end{tabular}
\caption{As an outsider to a team responsible for a process/policy/product, suppose you learn of something prior to official deployment (i.e., you learn the internal musings of another team). This Dilemma of Early Intervention and is not unique to bureaucracies. There is folk wisdom on both sides: ``Stay in your own lane'' and ``Speak up when you see something wrong.'' See the related Dilemma of micromanaging, \ref{table:micromanaging}.}
\label{table:early-intervention}
\end{table}
\end{center}


\begin{center}
\begin{table}[H] % ht
\begin{tabular}{ | m{\dilemmatablewidth}| m{\dilemmatablewidth} | } 
  \hline
  \textbf{Review status of the work of other people early and often. Many milestones, check-ins, and updates.} &
  \textbf{Review status infrequently; just do the work.} \\
  \hline
  \textit{Description}: Micromanagement. & 
  \textit{Description}: Hands-off management style. \\
  \hline
  \textit{Cons}: Takes up your time and the people you're reviewing. & 
  \textit{Cons}: Team members are unsure how to proceed and don't know what the goal is. \\
  \hline
\end{tabular}
\caption{The dilemma of micromanaging peers and subordinates is not unique to bureaucratic organizations. The right blend of how much engagement by reviewers depends on the personalities and desires of each person.}
\label{table:micromanaging}
\end{table}
\end{center}


\begin{center}
\begin{table}[H] % ht
\begin{tabular}{ | m{\dilemmatablewidth}| m{\dilemmatablewidth} | } 
  \hline
  \textbf{Bureaucrat expects management to provide solutions -- just tell me members what to do.} & 
  \textbf{Bureaucrat dislikes managers micromanaging by telling people what to do.} \\ 
  \hline
  \textit{Cons}: Your manager may not have insight on what needs to be done. Or they may guide you in a less effective direction. &
  \textit{Cons}: No autonomy, unable to exploit your expertise and creativity. \\  
  \hline
\end{tabular}
\caption{This is the complement of \ref{table:micromanaging}. Nominally the manager helps identify the objectives and provides context and the subordinate figures out how to accomplish the objective, but who is responsible for what is negotiable in each relationship.
}
\label{table:solution_provider}
\end{table}
\end{center}


\begin{center}
\begin{table}[H] % ht
\begin{tabular}{ | m{\dilemmatablewidth}| m{\dilemmatablewidth} | } 
  \hline
  \textbf{Write everything down to \href{https://en.wikipedia.org/wiki/Cover_your_ass}{cover your ass}.} &
  \textbf{Don't record sensitive conversations that could be used against you or others.} \\
  \hline
  \textit{Cons}: Takes a lot of time and effort to accurately capture intent. Recording can be done poorly or be misconstrued.  & 
  \textit{Cons}: No written record to point to when someone changes their behavior. \\
  \hline
\end{tabular}
\caption{I personally write things down and share them with other people (e.g., this book), but there are costs and risks to investing in documentation. There are \href{https://en.wikipedia.org/wiki/Dark_pattern}{dark patterns} for this trade-off, like intentionally misquoting another person to bias the documentation in your favor, or only writing down the aspects of conversation that favor the outcome you are interested in.}
\label{table:notes_or_no_notes}
\end{table}
\end{center}


\begin{center}
\begin{table}[H] % ht
\begin{tabular}{ | m{\dilemmatablewidth}| m{\dilemmatablewidth} | } 
  \hline
  \textbf{Ponder what should or could be done.} &
  \textbf{Figure out how to accomplish the objective.}\\
  \hline
  \textit{Cons}: Less time for action. & 
  \textit{Cons}: Prematurely select an action that is suboptimal. \\
  \hline
\end{tabular}
\caption{Brainstorming is useful, as it considering the holistic situation. At some point that transitions to action, but when? This is a question of how much time to spend admiring the forest versus the trees. 
}
\label{table:forest-vs-trees}
\end{table}
\end{center}




\begin{center}
\begin{table}[H] % ht
\begin{tabular}{ | m{\dilemmatablewidth}| m{\dilemmatablewidth} | } 
  \hline
  \textbf{Allocate time for meetings to facilitate coordination.} &
  \textbf{Allocate time for action.} \\
  \hline
  \textit{Cons}: Less time for participants to implement ideas. & 
  \textit{Cons}: Results in uncoordinated activity which can be wasteful. \\
  \hline
\end{tabular}
\caption{Similar to \ref{table:forest-vs-trees}, but here the question is about coordination versus doing the work. The amount of coordination depends on how many stakeholders there are, how familiar the stakeholders are with the challenge, and whether the action is reversible when found to be incorrect.
}
\label{table:meetings-versus-work}
\end{table}
\end{center}


\begin{center}
\begin{table}[H] % ht
\begin{tabular}{ | m{\dilemmatablewidth}| m{\dilemmatablewidth} | } 
  \hline
  \textbf{Operate at the level you are being paid for.} &
  \textbf{Operate above the level that you are being paid for in order to be a promoted.} \\
  \hline
  \textit{Description}: Meet job requirements but nothing extra. &
  \textit{Description}: Exceed job requirements. \\
  \hline
  \textit{Cons}: Risk not being promoted. & 
  \textit{Cons}: Experience wage loss - Organization is getting free labor. \\
  \hline
\end{tabular}
\caption{Work above your pay grade (provide the organization extra labor and you get reduced pay) or at your pay grade (expected labor and pay)?
}
\label{table:work_extra_or_work_as_expected}
\end{table}
\end{center}


\begin{center}
\begin{table}[H] % ht
\begin{tabular}{ | m{\dilemmatablewidth}| m{\dilemmatablewidth} | } 
  \hline
  \textbf{Speak out/speak up if something is wrong or \href{https://en.wikipedia.org/wiki/Moral_injury}{offends you}.} &
  \textbf{Hold back comments and questions to minimize disruptions.} \\
  \hline
  \textit{Cons}: You could be missing context; you might look stupid. & 
  \textit{Cons}: You missed an opportunity to correct something; you missed an opportunity to get educated about a situation. \\
  \hline
\end{tabular}
\caption{There is conflicting folk wisdom on both sides of this dilemma: ``The squeaky wheel gets the greese" and ``The squeaky wheel gets replaced." How you raise the issue, with whom, and in what context all matter to either correcting the situation or getting better educated.
}
\label{table:speak-up-or-hold-back}
\end{table}
\end{center}


\begin{center}
\begin{table}[H] % ht
\begin{tabular}{ | m{\dilemmatablewidth}| m{\dilemmatablewidth} | } 
  \hline
  \textbf{Send bad news up the chain of command.} &
  \textbf{Minimize bad news up the chain of command.} \\
  \hline
  \textit{Pros}: You are a reliable source of news. &
  \textit{Pros}: You minimize the burden of managers. \\
  \hline
  \textit{Cons}: You are viewed as a source of problems. & 
  \textit{Cons}: Harmful events eventually catch up with the organization.  \\
  \hline
\end{tabular}
\caption{The canonical example is the \href{https://en.wikipedia.org/wiki/Space_Shuttle_Challenger_disaster}{Challenger disaster}.}
\label{table:bad-news-up-the-chain}
\end{table}
\end{center}



\begin{center}
\begin{table}[H] % ht
\begin{tabular}{ | m{\dilemmatablewidth}| m{\dilemmatablewidth} | } 
  \hline
  \textbf{Prepare for disasters and emergencies, invest in mitigation.} &
  \textbf{Wait for the specific problem to arise before responding.} \\
  \hline
  \textit{Pros}: Lessen the impact when bad things happen; decrease the number of problems from occurring in the first place. &
  \textit{Pros}: deal with the specifics of the scenario at that time and thus be better informed. \\
  \hline
  \textit{Cons}: fewer events evolve into emergencies because you're prepared, or the impact of disasters is lessened. Both make you look overly paranoid and wasteful. & 
  \textit{Cons}: Unexpected events result in worse outcomes.  \\
  \hline
\end{tabular}
\caption{The ``Preparation versus Cleanup'' Dilemma: How much to invest in contingency planning and preparedness.}
\label{table:emergencies-vs-ignore}
\end{table}
\end{center}

% https://graphthinking.blogspot.com/2019/08/two-misleading-simplifications-when.html
\begin{center}
\begin{table}[H] % ht
\begin{tabular}{ | m{\dilemmatablewidth}| m{\dilemmatablewidth} | } 
  \hline
  \textbf{Focus on the immediate problem.} &
  \textbf{Ponder the systemic issues and adjacent contexts.} \\
  \hline
  \textit{Description}: Focus on isolated problematic aspects and do worry about the interdependencies and feedback loops and stakeholder incentives. &
  \textit{Description}: Philosophical musings with a holistic view. \\
  \hline
  \textit{Cons}: Misses systemic issues, causes that exist outside the immediate scope, or issues that occur due to interacting processes. & 
  \textit{Cons}: Relies on knowledge of the wider system that you may have less awareness of. Less emphasis on getting things done. May reveal problems that you don't have authority to address. \\
  \hline
\end{tabular}
\caption{Scope of problem solving can be narrow or broad. Rather than limit your investigations to one or the other, flipping between the two on a recurring basis (but not too frequently) can help.
}
\label{table:focus-vs-systemic}
\end{table}
\end{center}



\begin{center}
\begin{table}[H] % ht
\begin{tabular}{ | m{\dilemmatablewidth}| m{\dilemmatablewidth} | } 
  \hline
  \textbf{Only let good ideas through as determined by a detailed review process of clearly specified plan.} &
  \textbf{Give resources to untested ideas.} \\
  \hline
  \textit{Pros}: Less waste of resources and time. Everyone has confidence in the investment. & 
  \textit{Pros}: High-risk/high-reward ideas that are disruptive can be implemented. \\
  \hline
  \textit{Cons}: Burdensome review process. & 
  \textit{Cons}: some ideas will fail. \\
  \hline
\end{tabular}
\caption{How much vetting should novel ideas get before implementation?
}
\label{table:idea-filtering}
\end{table}
\end{center}



% https://graphthinking.blogspot.com/2016/05/claim-innovation-is-either-disruptive.html
\begin{center}
\begin{table}[H] % ht
\begin{tabular}{ | m{\dilemmatablewidth}| m{\dilemmatablewidth} | } 
  \hline
  \textbf{Work on disruptive innovation.} &
  \textbf{Work on iterative (evolutionary) innovation.} \\
  \hline
  \textit{Description}: Start from scratch, aim for revolution, replace the legacy.  &
  \textit{Description}: Starts with change to existing solution, adjust the legacy path.  \\  
  \hline
  \textit{Pros}: High reward. & 
  \textit{Pros}: Low risk, low cost. \\
  \hline
  \textit{Cons}: High risk, high cost. & 
  \textit{Cons}: Low reward. \\
  \hline
\end{tabular}
\caption{Incremental change may not suffice. Disruption can be costly.
}
\label{table:disruptive-or-iterative}
\end{table}
\end{center}

\begin{center}
\begin{table}[H] % ht
\begin{tabular}{ | m{\dilemmatablewidth}| m{\dilemmatablewidth} | } 
  \hline
  \textbf{Innovate in a novel-to-your-team environment.} &
  \textbf{Innovate in your team's standard environment.} \\
  \hline
  \textit{Pros}: More likely to allows people to drop their expectations.  & 
  \textit{Pros}: Easy to operate in. \\
  \hline
  \textit{Cons}: Loses access to connections vital to actually create success. Use extra space, logistics of moving. & 
  \textit{Cons}: Allows conventional processes to take effect. People hold onto their assumptions. \\
  \hline
\end{tabular}
\caption{Where (physically, spatially) innovation takes place matters because the environment sets context for assumptions.
}
\label{table:where-to-innovate}
\end{table}
\end{center}

% https://graphthinking.blogspot.com/2016/06/innovation-in-open-versus-behind-curtain.html
\begin{center}
\begin{table}[H] % ht
\begin{tabular}{ | m{\dilemmatablewidth}| m{\dilemmatablewidth} | } 
  \hline
  \textbf{Work on innovation in the open.} &
  \textbf{Work on innovation in hiding.} \\
  \hline
  \textit{Pros}: More likely to be criticized. Criticism can be positive (as in an incubator setting).& 
  \textit{Pros}: Less drama -- the incumbent won't attack the innovation since they don't know about it. \\
  \hline
  \textit{Cons}: Negative criticism intended to harm -- an incumbent has reason to be defensive. The incumbent attacks the innovation before it is sufficiently developed or has time to build a user base. & 
  \textit{Cons}: Less opportunity for feedback. Project is easier to kill since the value is not advertised.\\
  \hline
\end{tabular}
\caption{How is innovation carried out within the organization?
}
\label{table:innovate-open-obscure}
\end{table}
\end{center}


\begin{center}
\begin{table}[H] % ht
\begin{tabular}{ | m{\dilemmatablewidth}| m{\dilemmatablewidth} | } 
  \hline
  \textbf{Seek recognition for your work.} &
  \textbf{Work in obscurity.} \\
  \hline
  \textit{Pros}: Helps with promotion. & 
  \textit{Pros}: Less distraction. \\
  \hline
  \textit{Cons}: Devalues the contributions of other people. & 
  \textit{Cons}: No one knows the value of your work and you won't get feedback. \\
  \hline
\end{tabular}
\caption{This dilemma is magnified when the task you work on is high risk or is resource intensive. 
}
\label{table:recognition-obscurity}
\end{table}
\end{center}

\begin{center}
\begin{table}[H] % ht
\begin{tabular}{ | m{\dilemmatablewidth}| m{\dilemmatablewidth} | } 
  \hline
  \textbf{Gather lots of data.} &
  \textbf{Gather minimal data.} \\
  \hline
  \textit{Description}: Gather lots of data for a well-informed decision. &
  \textit{Description}: Minimal information because decision maker knows what to do or outcome is irrelevant.  \\  
  \hline
  \textit{Cons}: High cost of gathering data (time, resources). \href{https://en.wikipedia.org/wiki/Opportunity_cost}{Opportunity costs}. & 
  \textit{Cons}: Lack of data results in decisions based on oversimplified assessment. \\
  \hline
\end{tabular}
\caption{How much data to gather for a decision. See Fig.~\ref{fig:data_collection_cost_uncertainty}. See also \S~\ref{sec:bureaucratic_debt} on bureaucratic debt.
%{\tiny Tag: Decision making.}
}
\label{table:gather_data_lots-vs-little}
\end{table}
\end{center}

\begin{figure}[H] % ht
        \centering
        \includegraphics[width=0.8\textwidth]{images/cost_and_uncertainty_for_data_collection}
        \caption{Collecting more data costs money and decrease uncertainty. See Dilemma~\ref{table:gather_data_lots-vs-little}.}
        \label{fig:data_collection_cost_uncertainty}
\end{figure}

The logistics of gathering data can be measured, but there are other subjective aspects to account for as well. Making a decision has an emotional toll on the decider due to the risk of failure. Also, decisions are made in a social context, with decision makers accounting for the ramifications on people they have relationships with. 


Gathering data (Dilemma~\ref{table:gather_data_lots-vs-little}) is distinct from planning (Dilemma~\ref{table:planning}). It is possible to do a lot of planning with only a little information gathered, and it is feasible to have lots of data and do no planning. 

\begin{center}
\begin{table}[H] % ht
\begin{tabular}{ | m{\dilemmatablewidth}| m{\dilemmatablewidth} | } 
  \hline
  \textbf{Extensive planning upfront (proactive)} & 
  \textbf{Iterative improvement of plans (reactive)} \\ 
  \hline
  \textit{Description}: Lots of time spent brainstorming potential scenarios and contingency options prior to taking action. & 
  \textit{Description}: Start taking action and use feedback to shape next actions. \\ 
  \hline
  \textit{Cons}: ``No plan survives contact with the enemy.'' & 
  \textit{Cons}: Less prepared. \\  
  \hline
\end{tabular}
\caption{How much time to invest in planning.
%{\tiny Tag: Decision making.}
}
\label{table:planning}
\end{table}
\end{center}



\begin{center}
\begin{table}[H]
\begin{tabular}{ | m{\dilemmatablewidth}| m{\dilemmatablewidth} | } 
  \hline
  \textbf{Involve people who disagree.} & 
  \textbf{Ignore people who disagree.} \\ 
  \hline
  \textit{Pros}: Get constructive feedback; account for factors you didn't consider; build a robust solution. & 
  \textit{Pros}: Save time by not interacting. \\  
  \hline
  \textit{Cons}: Results in a compromise or partial solution that minimizes aggregate unhappiness. & 
  \textit{Cons}: Miss a vital aspect you didn't consider. \\  
  \hline
\end{tabular}
\caption{Engagement with opposition to process or change.
%{\tiny Tag: Decision making.}
}
\label{table:opposition}
\end{table}
\end{center}

Making a decision imposes a bound on how much time is available for both gathering data and planning. Time is zero sum, so more time gathering data is less time planning. Similarly, the number of people available for data gathering and planning is bounded, and tasking people is a zero sum choice.

\begin{figure}[H] % ht
    \centering
    \includegraphics[width=0.8\textwidth]{images/planning_and_data_gathering.pdf}
    \caption{Planning (Dilemma~\ref{table:planning}) and data gathering (Dilemma~\ref{table:gather_data_lots-vs-little}) trade-off.}
    \label{fig:pareto_frontier}
\end{figure}

In practice, gathering data and planning rarely terminate -- they evolve.




When planning (Dilemma~\ref{table:planning}), aspects to consider include
%\begin{itemize}
%    \item 
the amount of risk seeking or tolerance (Dilemma~\ref{table:risk})
and
%    \item 
the intended scope of impact  (Dilemma~\ref{table:scope_broad-vs-narrow}).
%\end{itemize}

\begin{center}
\begin{table}[H] % ht
\begin{tabular}{ | m{\dilemmatablewidth}| m{\dilemmatablewidth} | } 
  \hline
  \textbf{Take on big risks and big rewards.} & 
  \textbf{Take on small risks and small rewards.} \\ 
  \hline
  \textit{Description}: High risk tolerance. &
  \textit{Description}: Low risk tolerance. \\
  \hline
  \textit{Pros}: Potential for failure and harm is significant. &
  \textit{Pros}: If any one investment fails, you can continue other efforts. \\
  \hline
  \textit{Cons}: Costly investment, longer feedback cycle. & 
  \textit{Cons}: Incremental can be slower. \\
  \hline
\end{tabular}
\caption{\href{https://en.wikipedia.org/wiki/Risk_assessment}{Risk tolerance}. 
%{\tiny Tag: Personal choice.}
}
\label{table:risk}
\end{table}
\end{center}

\ \\

\begin{center}
\begin{table}[H] % ht
\begin{tabular}{ | m{\dilemmatablewidth}| m{\dilemmatablewidth} | } 
  \hline
  \textbf{Broad scope of impact.} &
  \textbf{Narrow scope of impact.} \\
  \hline
  \textit{Description}: The consequence of the work has many stakeholders. &
  \textit{Description}: Small number of stakeholders. \\  
  \hline
  \textit{Pros}: Benefit more people. &
  \textit{Pros}: Niche impact means less dependencies on other people. \\
  \hline
  \textit{Cons}: Harder to get everyone in agreement. & 
  \textit{Cons}: Less visibility to the rest of the organization. \\
  \hline
\end{tabular}
\caption{Scope of impact of your work. 
%{\tiny Tag: Personal choice}
}
\end{table}
\label{table:scope_broad-vs-narrow}
\end{center}


Once data is gathered (Dilemma~\ref{table:gather_data_lots-vs-little}) and a plan is made (Dilemma~\ref{table:planning}), the result is disseminated. The choice on how to disseminate is Dilemma~\ref{table:consistency} and Dilemma~\ref{table:disseminate_one-by-one}.

\begin{center}
\begin{table}[H] % ht
\begin{tabular}{ | m{\dilemmatablewidth}| m{\dilemmatablewidth} | } 
  \hline
  \textbf{Guidance updated frequently; Incremental change.} & 
  \textbf{Consistent application of policy over time. Rules persist; then sudden drastic change.} \\ 
  \hline
  \textit{Pros}: Adapt policy to new information and changing conditions. &
  \textit{Pros}: Stability is easier to predict between regime changes.  \\
  \hline
  \textit{Cons}: More work needed. Accused of lacking stability. & 
  \textit{Cons}: Doesn't adapt as conditions change. Accused of being inflexible to evolving conditions. \\
  \hline
\end{tabular}
\caption{Consistency over time. Stability of rules; how change is implemented. Can also be characterized as when to tell other people: sooner or later (when firmer information is available) See \S~\ref{sec:static-dynamic_processes} for static versus dynamic processes.
%{\tiny Tag: Organization's culture. Tag: Personal choice.}
}
\label{table:consistency}
\end{table}
\end{center}

Deployment of products and deployment of policies face similar dilemmas. \href{https://en.wikipedia.org/wiki/Diffusion_of_innovations}{Diffusion of Innovation}

\begin{center}
\begin{table}[H] % ht
\begin{tabular}{ | m{\dilemmatablewidth}| m{\dilemmatablewidth} | } 
  \hline
  \textbf{Tell people one-by-one.} & 
  \textbf{Tell everyone at once.} \\ 
  \hline
  \textit{Pros}: One-on-one allows a freer response from audience. &
  \textit{Pros}: Save time for the speaker. \\
  \hline
  \textit{Cons}: Order matters for relationships. & 
  \textit{Cons}: Overwhelming feedback all at once. \\  
  \hline
\end{tabular}
\caption{How to disseminate information.
%{\tiny Tag: Personal choice.}
}
\label{table:disseminate_one-by-one}
\end{table}
\end{center}

Once a decision has been made, the decision is executed or enforced. How many rules are there (Dilemma~\ref{table:number_of_rules}) and
how strictly are the rules enforced (Dilemma~\ref{table:rule_strictness})?

\begin{center}
\begin{table}[H] % ht
\begin{tabular}{ | m{\dilemmatablewidth}| m{\dilemmatablewidth} | } 
  \hline
  \textbf{Enforce rules strictly.} & 
  \textbf{Lax rule enforcement.} \\ 
  \hline
  \textit{Pros}: Predictable. &
  \textit{Pros}: Bureaucrats feel empowered. \\
  \hline
  \textit{Cons}: Insensitive to nuance. & 
  \textit{Cons}: Tolerance for changing conditions or exceptional cases.  \\  
  \hline
\end{tabular}
\caption{Strictness of rules.
%{\tiny Tag: Organization's culture.}
}
\label{table:rule_strictness}
\end{table}
\end{center}

\begin{center}
\begin{table}[H] % ht
\begin{tabular}{ | m{\dilemmatablewidth}| m{\dilemmatablewidth} | } 
  \hline
  \textbf{If it's not against the rules, it must be okay.} & 
  \textbf{I can only do what is allowed by the rules and nothing more.} \\ 
  \hline
  \textit{Pros}: Autonomy &
  \textit{Pros}:  \\
  \hline
  \textit{Cons}: . & 
  \textit{Cons}: .  \\  
  \hline
\end{tabular}
\caption{Adherence to rules.
}
\label{table:rule_adherence}
\end{table}
\end{center}





\begin{center}
\begin{table}[H] % ht
\begin{tabular}{ | m{\dilemmatablewidth}| m{\dilemmatablewidth} | } 
  \hline
  \textbf{Control via rules.} & \textbf{Freedom/autonomy/agility.} \\ 
  \hline
  \textit{Description}: High number of rules to cover a variety of situations. & 
  \textit{Description}: Low number of rules to enable flexibility. \\ 
  \hline
  \textit{Cons}: The more rules that exist the more likely it is that someone will find a way to exploit them to their own advantage. & 
  \textit{Cons}: The fewer rules that exist the more likely it is that someone will try to get away with something bad. \\  
  \hline
\end{tabular}
\caption{Number of rules.
%{\tiny Tag: Organization's culture}
}
\label{table:number_of_rules}
\end{table}
\end{center}
Alternative approach: guidance derived from principles that can be adapted to specific situations. That has the problem of requiring good knowledge of the situation and wise judgement.

\ \\

\begin{center}
\begin{table}[H] % ht
\begin{tabular}{ | m{\dilemmatablewidth}| m{\dilemmatablewidth} | } 
  \hline
  \textbf{I can only do what is mandated by the organization.} & 
  \textbf{I can do anything that's not illegal.} \\ 
  \hline
  \textit{Cons}:  &
  \textit{Cons}:  \\  
  \hline
\end{tabular}
\caption{The scope of your actions bound by mandates and legality, but the way you interpret that is subjective. 
}
\label{table:legality}
\end{table}
\end{center}


\begin{center}
\begin{table}[H] % ht
\begin{tabular}{ | m{\dilemmatablewidth}| m{\dilemmatablewidth} | } 
  \hline
  \textbf{Quickly complete tasks.} & 
  \textbf{Methodically complete tasks.} \\ 
  \hline
  \textit{Description}: implementing a solution quickly to address urgent needs. &
  \textit{Description}: Methodical well-planned design and execution yield robust solutions. \\
  \hline
  \textit{Pros}: Rapid solution. &
  \textit{Pros}: More like to get the solution right. \\
  \hline
  \textit{Cons}: Risk of quick task is that the result is ineffective, inefficient, or wrong. &
  \textit{Cons}: \href{https://en.wikipedia.org/wiki/Opportunity_cost}{opportunity cost} \\  
  \hline
\end{tabular}
\caption{Speed versus accuracy of task completion.
}
\label{table:quick-methodical}
\end{table}
\end{center}

\ \\

\begin{center}
\begin{table}[H] % ht
\begin{tabular}{ | m{\dilemmatablewidth}| m{\dilemmatablewidth} | } 
  \hline
  \textbf{Push people to work really hard.} & 
  \textbf{Create a comfortable work environment.} \\ 
  \hline
  \textit{Cons}: Burn out and leave. & 
  \textit{Cons}: Lower instantaneous productivity. \\  
  \hline
\end{tabular}
\caption{Policy enforcement rate
}
\label{table:rate-of-work}
\end{table}
\end{center}

\ \\

% https://bennorthrop.com/Essays/2022/code-ownership-stewardship-or-free-for-all.php
\begin{center}
\begin{table}[H] % ht
\begin{tabular}{ | m{\dilemmatablewidth}| m{\dilemmatablewidth} | } 
  \hline
  \textbf{One person or team owns an area of responsibility.} & 
  \textbf{Anyone take on any task.} \\ 
  \hline
  \textit{Cons}: Staffing capacity may not be as flexible as varying workload. & 
  \textit{Cons}: Not everyone is skilled at everything. \\  
  \hline
\end{tabular}
\caption{Swimlanes, task boundaries.
}
\label{table:swimlanes}
\end{table}
\end{center}


\ \\

\begin{center}
\begin{table}[H] % ht
\begin{tabular}{ | m{\dilemmatablewidth}| m{\dilemmatablewidth} | } 
  \hline
  \textbf{Seek out experienced collaborators.} & 
  \textbf{Work with less experienced people.} \\ 
  \hline
  \textit{Pros}: Quicker to get something done. &
  \textit{Pros}: Less set in their ways and open to more novelty. \\  
  \hline
  \textit{Cons}: Experienced people who are good are probably busy. &
  \textit{Cons}: Slower progress. \\  
  \hline
\end{tabular}
\caption{People with experience are useful but less accessible.
}
\label{table:experience}
\end{table}
\end{center}


\ \\

\begin{center}
\begin{table}[H] % ht
\begin{tabular}{ | m{\dilemmatablewidth}| m{\dilemmatablewidth} | } 
  \hline
  \textbf{Say yes to new opportunities.} & 
  \textbf{Say no to new opportunities.} \\ 
  \hline
  \textit{Pros}: Positive attitude, collaborative. &
  \textit{Pros}: Able to prioritize and focus. \\
  \hline
  \textit{Cons}: Fail to complete tasks. &
  \textit{Cons}: Not a team player. \\  
  \hline
\end{tabular}
\caption{Acceptance or rejection of additional work.
}
\label{table:new-opportunties}
\end{table}
\end{center}

\ \\

\begin{center}
\begin{table}[H] % ht
\begin{tabular}{ | m{\dilemmatablewidth}| m{\dilemmatablewidth} | } 
  \hline
  \textbf{Share less data.} &
  \textbf{Share more data.} \\
%  \hline
%  \textit{Description}:  &
%  \textit{Description}:  \\  
  \hline
  \textit{Pros}: Restricting data access saves money for the data owner and improves flexibility.&
  \textit{Pros}: Sharing data improves transparency and accountability. \\
  \hline
  \textit{Cons}: Other people (inside and outside the organization) are unable to extract maximum value from data & 
  \textit{Cons}: Sharing data cost resources (people, money, time) \\
  \hline
\end{tabular}
\caption{How much data to share.
%{\tiny Tag: Personal choice.}
}
\label{table:data_share-vs-hide}
\end{table}
\end{center}

\ \\

\begin{center}
\begin{table}[H] % ht
\begin{tabular}{ | m{\dilemmatablewidth}| m{\dilemmatablewidth} | } 
  \hline
  \textbf{Compete for resources.} &
  \textbf{Cooperate for productivity.} \\
  \hline
  \textit{Description}: individuals compete for attention and promotion; teams compete for money and staffing resources &
  \textit{Description}: cooperation improves productivity \\  
  \hline
  \textit{Cons}: Fail to synergize skills resources & 
  \textit{Cons}: Not clear who to assign responsibility for success or failure \\
  \hline
\end{tabular}
\caption{Cooperate or Compete -- applies to teams and to individuals. 
%{\tiny Tag: Personal choice.}
}
\label{table:cooperate-vs-compete}
\end{table}
\end{center}

\ \\

\begin{center}
\begin{table}[H] % ht
\begin{tabular}{ | m{\dilemmatablewidth}| m{\dilemmatablewidth} | }
  \hline
  \textbf{Consistent application of policy across cases.} &
  \textbf{Adapt policy to specific cases.} \\
  \hline
  \textit{Description}: Maximize broad applicability; minimize exceptions. &
  \textit{Description}: Demonstrate flexibility for unique scenarios. \\  
  \hline
  \textit{Cons}: Less sensitive to the nuances of a specific situation. & 
  \textit{Cons}: Takes more work. More likely to be accused of bias. \\
  \hline
\end{tabular}
\caption{Case consistency vs adaptability.
%{\tiny Tag: Personal choice.}
}
\label{table:policy_consistency_across_cases}
\end{table}
\end{center}

When change to policies is desired, there are options on how to advocate for change -- Dilemma~\ref{table:how_to_change}.

\begin{center}
\begin{table}[H] % ht
\begin{tabular}{ | m{\dilemmatablewidth}| m{\dilemmatablewidth} | } 
  \hline
% https://graphthinking.blogspot.com/2019/07/not-too-loose-not-too-tight-determining.html
  \textbf{Adhere strictly to the scope of your role.} & 
  \textbf{Stray outside (or outright ignore) the scope of your role.} \\ 
  \hline
  \textit{Description}: Inflexible to novelty. Specialization of tasking. & 
  \textit{Description}: Lack of structure. Generalization. \\ 
  \hline
  \textit{Cons}: efficiencies of cooperation and specialization would not occur. & 
  \textit{Cons}: Deadlock condition arises due to a scheduling constraint -- no one can proceed because everyone is waiting on everyone else. Responsibilities are unclear when no ones scope is clear. \\  
  \hline
\end{tabular}
\caption{Both strict adherence to role scope and ignoring scope can decrease an organization's productivity. 
What happens when a person deviates from their role?
How are people who do not conform identified? Are they confronted?
}
\label{table:scope_of_activity}
\end{table}
\end{center}

\begin{center}
\begin{table}[H] % ht
\begin{tabular}{ | m{\dilemmatablewidth}| m{\dilemmatablewidth} | } 
  \hline
  \textbf{Delegate; share work with other people.} & 
  \textbf{Work alone; don't rely on other people.} \\ 
  \hline
  \textit{Cons}: Your success is dependent on other people. & 
  \textit{Cons}: Can't accomplish as much on your own. \\  
  \hline
\end{tabular}
\caption{Sharing work can improve productivity and build relationships but also incurs risks to reputation and success.
}
\label{table:delegate-or-not}
\end{table}
\end{center}

\begin{center}
\begin{table}[H] % ht
\begin{tabular}{ | m{\dilemmatablewidth}| m{\dilemmatablewidth} | } 
  \hline
  \textbf{Many small tasks or objectives.} & 
  \textbf{Fewer big tasks or objectives.} \\ 
  \hline
  \textit{Cons}: Enables a fail fast approach. & 
  \textit{Cons}: Less overhead to manage. \\  
  \hline
\end{tabular}
\caption{Chunk size.
}
\label{table:chunk_size}
\end{table}
\end{center}

\begin{center}
\begin{table}[H] % ht
\begin{tabular}{ | m{\dilemmatablewidth}| m{\dilemmatablewidth} | } 
  \hline
  \textbf{Task with many external dependencies.} & 
  \textbf{Task with few external dependencies.} \\ 
  \hline
  \textit{Cons}: Risk of failing because of a failed dependency. & 
  \textit{Cons}: Have to develop everything yourself; waste of resources due to redundancy. \\  
  \hline
\end{tabular}
\caption{External dependencies can enable broader scope. See also the Dilemma of Delgation, \ref{table:delegate-or-not}.
}
\label{table:number_of_external dependencies}
\end{table}
\end{center}

\begin{center}
\begin{table}[H] % ht
\begin{tabular}{ | m{\dilemmatablewidth}| m{\dilemmatablewidth} | } 
  \hline
  \textbf{Focused on one role.} & 
  \textbf{Have multiple roles.} \\ 
  \hline
  \textit{Cons}: If a role does not consume 40 hours per week, you'll be idle. & 
  \textit{Cons}: Context switches between roles and delayed responses. \\  
  \hline
\end{tabular}
\caption{The right number of roles for a bureaucrat depends on personality and tasking. 
}
\label{table:number_of_roles}
\end{table}
\end{center}


\begin{center}
\begin{table}[H] % ht
\begin{tabular}{ | m{\dilemmatablewidth}| m{\dilemmatablewidth} | } 
  \hline
  \textbf{Dissent is welcome and discussed freely.} & 
  \textbf{Dissent is suppressed.} \\ 
  \hline
  \textit{Cons}: Can be disruptive to normal operations. Distracts from the task. & 
  \textit{Cons}: Limits novel ideas from spreading. Harms morale. \\  
  \hline
\end{tabular}
\caption{Dissent is caused by dissatisfaction with people or processes. 
}
\label{table:how_dissent_is_responded_to}
\end{table}
\end{center}


\begin{center}
\begin{table}[H] % ht
\begin{tabular}{ | m{\dilemmatablewidth}| m{\dilemmatablewidth} | } 
  \hline
  \textbf{Do share lessons learned.} & 
  \textbf{Don't share lessons learned.} \\ 
  \hline
  \textit{Pros}: Honesty, accountability, self-awareness, and self-reflection. & 
  \textit{Pros}: Look competent, even when making mistakes. \\  
  \hline
  \textit{Cons}: Looks weak, unprofessional. & 
  \textit{Cons}: Limit the growth of bureaucrats in the organization. \\  
  \hline
\end{tabular}
\caption{Sharing lessons learned may seem reasonable unless you want to maintain a pristine reputation. 
}
\label{table:sharing_lessons_learned}
\end{table}
\end{center}

% https://graphthinking.blogspot.com/2019/07/vulnerability-of-organizations-in.html
\begin{center}
\begin{table}[H] % ht
\begin{tabular}{ | m{\dilemmatablewidth}| m{\dilemmatablewidth} | } 
  \hline
  \textbf{Share lessons learned about yourself.} & 
  \textbf{Share lessons learned from observing others.} \\ 
  \hline
  \textit{Cons}: Potentially look stupid. & 
  \textit{Cons}: Potentially hurts their reputation. \\  
  \hline
\end{tabular}
\caption{When sharing lessons learned (option 1 in \ref{table:sharing_lessons_learned}), the lessons do not have to be about you. 
}
\label{table:share_lessons_learned}
\end{table}
\end{center}

\begin{center}
\begin{table}[H] % ht
\begin{tabular}{ | m{\dilemmatablewidth}| m{\dilemmatablewidth} | } 
  \hline
  \textbf{Learn lessons from your own mistakes.} & 
  \textbf{Learn lessons from others (formal training).} \\ 
  \hline
  \textit{Cons}: Potentially look stupid; waste resources discovering what others already know. & 
  \textit{Cons}: Formal training may overemphasize irrelevant or impractical concepts. \\  
  \hline
\end{tabular}
\caption{How much formal training to invest in before learning by doing?
}
\label{table:lessons_learned_source}
\end{table}
\end{center}



\begin{center}
\begin{table}[H] % ht
\begin{tabular}{ | m{\dilemmatablewidth}| m{\dilemmatablewidth} | } 
  \hline
  \textbf{Build a small coalition of interested parties.} & 
  \textbf{Build a large base of support and get everyone on board.} \\ 
  \hline
  \textit{Cons}: May not be representative of all stakeholders. & 
  \textit{Cons}: Takes time away from the work. Many people may disagree or be disinterested. \\  
  \hline
\end{tabular}
\caption{A coalition can provide morale support but takes time to build.
%{\tiny Tag: }
}
\label{table:how_to_change}
\end{table}
\end{center}

\subsection{Dilemmas of Policy for an Organization's Structure\label{sec:org_dilemma}}

The constraints a decision maker faces are informed by the person's environment. Dilemmas \ref{table:people-per-supervisor} through \ref{table:market-vs-monopoly} shape the experience of bureaucrats in an organization.

\begin{center}
\begin{table}[H] % ht
\begin{tabular}{ | m{\dilemmatablewidth}| m{\dilemmatablewidth} | } 
  \hline
  \textbf{Flatter hierarchical organization.} &
  \textbf{More layers of hierarchy.} \\ 
  \hline
  \textit{Description}: More people managed per supervisor. & 
  \textit{Description}: Fewer people managed per supervisor. \\ 
  \hline
  \textit{Cons}: Less feedback/attention per employee. & 
  \textit{Cons}: Fewer people doing work. \\  
  \hline
\end{tabular}
\caption{Shape of hierarchical organization.
%{\tiny Tag: Organization's culture}
}
\label{table:people-per-supervisor}
\end{table}
\end{center}

\ \\

\begin{center}
\begin{table}[H] % ht
\begin{tabular}{ | m{\dilemmatablewidth}| m{\dilemmatablewidth} | } 
  \hline
  \textbf{Staffing: good coverage.} &
  \textbf{Staffing: minimal coverage.} \\
  \hline
  \textit{Description}: Sufficient staff. &
  \textit{Description}: As small of staff as possible. \\  
  \hline
  \textit{Pros}: Cover all edge cases; resilient to changing demands. &
  \textit{Pros}: Less expensive. \\
  \hline
  \textit{Cons}: Slack resources; sometimes inefficient. Increased communication needed. & 
  \textit{Cons}: Fragile when requirements change or workload increases. If one person departs and there's no redundancy, capacity and capability are harmed.  \\
  \hline
\end{tabular}
\caption{Size of team or organization.
%{\tiny Tag: Design of organization.}
}
\label{table:staff_many-vs-few}
\end{table}
\end{center}


\ \\

\begin{center}
\begin{table}[H] % ht
\begin{tabular}{ | m{\dilemmatablewidth}| m{\dilemmatablewidth} | } 
  \hline
  \textbf{In-house services for non-central activities.} &
  \textbf{External dependencies for non-central activities.} \\
%  \hline
%  \textit{Description}:  &
%  \textit{Description}:  \\  
  \hline
  \textit{Pros}: More control. &
  \textit{Pros}: Easier to replace. \\
  \hline
  \textit{Cons}: Expands scope of responsibilities. & 
  \textit{Cons}: Less understanding of problem.  \\
  \hline
\end{tabular}
\caption{Services that are necessary but not central.
%{\tiny Tag: Design of organization.}
}
\label{table:inhouse-vs-external}
\end{table}
\end{center}

Table~\ref{table:inhouse-vs-external} isn't unique to bureaucracy. It applies to businesses in a market and to governments in a global environment. 

\ \\

\begin{center}
\begin{table}[H] % ht
\begin{tabular}{ | m{\dilemmatablewidth}| m{\dilemmatablewidth} | } 
  \hline
  \textbf{Centralized services.} &
  \textbf{Locally distributed services.} \\
  \hline
  \textit{Description}: A single provider of services for the organization; typically top-down mandate. &
  \textit{Description}: Each team has a local service provider; typically bottom-up organic result. \\  
  \hline
  \textit{Pros}: Cheaper. Enables coordination. &
  \textit{Pros}: Quicker response. 
  Enables innovation. 
  Accounts for local deviations from the norm. \\
  \hline
  \textit{Cons}: Less sensitive to local issues. Less responsive. Longer delays. Single point of failure.  & 
  \textit{Cons}: Uneven quality of service. Inconsistent strategies and policies. \\
  \hline
\end{tabular}
\caption{Centralization of services. Oscillation (see Fig.~\ref{fig:central-vs-distributed}) indicates neither solution is optimal.
%{\tiny Tag: Design of organization.}
}
\label{table:central-vs-distributed}
\end{table}
\end{center}

\begin{figure}[H] % ht
    \centering
    \includegraphics[width=0.8\textwidth]{images/dilemma_centralization-vs-distributed.pdf}
    \caption{Dipole oscillation. See Dilemma~\ref{table:central-vs-distributed}. Migrating to the opposite paradigm gives people in charge a chance to show their responsiveness to the needs of participants. The rate of oscillation is a measure of institutional memory half-life.}
    \label{fig:central-vs-distributed}
\end{figure}

Centralization is often carried out for the purposes of cost efficiency. The cost savings are due to de-duplication and having less slack. Both of those ``savings'' are a decrease of redundancy, which has a cost when there are unexpected fluctuations in need. 

Centralization is an intentional monopolization, with the corresponding decrease in choices. 
Centralization (de-localization) can decrease the value assigned to feedback from people using the service because personal relations are de-valued. 

The weaker feedback, lack of redundancy, and decreased emphasis on relationships motivates the creation of local services. 

\ \\

\begin{center}
\begin{table}[H] % ht
\begin{tabular}{ | m{\dilemmatablewidth}| m{\dilemmatablewidth} | } 
  \hline
  \textbf{Decision making lower in a hierarchy} &
  \textbf{Decision making higher in a hierarchy} \\
  \hline
  \textit{Description}: Push decisions down to empower employees. &
  \textit{Description}: Escalate every decision to management. \\  
  \hline
  \textit{Pros}: better information &
  \textit{Pros}: better scope \\
  \hline
  \textit{Cons}: more inconsistency & 
  \textit{Cons}: Decision maker has less skin in the game and may be less well informed. Employees are disempowered. \\
  \hline
\end{tabular}
\caption{Where decisions get made in hierarchical organization.
%{\tiny Tag: Design of organization.}
}
\label{table:decisions_low-vs-high}
\end{table}
\end{center}

\ \\

\begin{center}
\begin{table}[H] % ht
\begin{tabular}{ | m{\dilemmatablewidth}| m{\dilemmatablewidth} | } 
  \hline
  \textbf{Redundant services in a market} &
  \textbf{Monopoly service provider} \\
  \hline
  \textit{Description}: using a market model within the organization &
  \textit{Description}:  \\  
  \hline
  \textit{Pros}: enable customers to choose the best service &
  \textit{Pros}: efficiency of a single service \\
  \hline
  \textit{Cons}: redundancy & 
  \textit{Cons}: might not meet the needs of all customers \\
  \hline
\end{tabular}
\caption{Services within an organization. See also Fig.~\ref{fig:market-vs-monopoly}.
%{\tiny Tag: Design of organization.}
}
\label{table:market-vs-monopoly}
\end{table}
\end{center}


\begin{figure}[H] % ht
    \centering
    \includegraphics[width=0.8\textwidth]{images/dilemma_market_vs_monopoly.pdf}
    \caption{Dipole oscillation: solution A exists but doesn't meet my needs. Rather than tweak A, re-invent solution B which mostly overlaps with A but has independent development and support. See also Dilemma~\ref{table:market-vs-monopoly}.}
    \label{fig:market-vs-monopoly}
\end{figure}


% https://graphthinking.blogspot.com/2021/04/laffer-curve-and-minimum-viable.html
There is a Goldilocks situation for amount of processes in an organization:
\begin{itemize}
    \item Too few processes (all social relationships).
    \item Just the right amount -- a balance process and social, and when to use which is known.
    \item Too much process (not enough leveraging of social relationships)
\end{itemize}
This optimization is similar to the \href{https://en.wikipedia.org/wiki/Laffer_curve}{Laffer curve} in economics.

\ \\

% https://graphthinking.blogspot.com/2021/12/hierarchical-organization-trilemma.html
Trilemma: \textbf{Do you work for your team, your manager, or yourself?}
Being a member of a team that operates within a hierarchy is tough. One reason is the question of who you are working for. The trilemma is whether you work for yourself, work for your supervisor, or work for your team.  Ideally you can find ways to do all three, but that is not always the case. 

Members of the team should work collaboratively, but there is a potential counter-force: accountability to the supervisor. Because each team member is accountable to their supervisor(s), that motivates the action of the individual. The team does not actually have autonomy -- they are accountable to the boss.

In the approach ``team members are directed by their supervisor," the synergy of the team is neglected and the supervisor becomes a bottleneck (for decision making and for creativity and for planning).

The third approach is for a person to ignore their team and their supervisor. This might enable quicker progress, at the risk of going in an unhelpful direction or not leveraging skills of coworkers. 

\ \\

Trilemma:
\textbf{Seek less budget, same, or larger budget.} Less budget is needed if you improved efficiency, but then the proportional power within the organization is decreased. A larger budget may be due to inefficiency or growth. Keeping the same budget indicates no promotions are relevant (although a steady state could result from a combination of growth and improved efficency). 

\ \\

A trilemma applicable to many situations is that options are \textbf{fast, inexpensive, good; choose two}. (This is the \href{https://en.wikipedia.org/wiki/Project_management_triangle}{Project management triangle}.) \\
In other words, the options are
\begin{itemize}
    \item Good and fast is expensive (i.e., requires lots of resources).
    \item Good and inexpensive takes a long time (i.e., a clever solution).
    \item Fast and inexpensive will be low quality.
\end{itemize}

\subsubsection{Consequences of the Dilemma-based Framing}

The dilemmas listed above are numerous but not exhaustive. Even if each bureaucrat considers the same choices (which doesn't necessarily happen), not everyone makes the same selection. One response might be to defer to someone higher in the hierarchy to coordinate. While this would ensure consistency, this would be \href{https://en.wikipedia.org/wiki/Micromanagement}{micromanagement}. The people in the hierarchy above the person facing the choice don't have exposure to the problem. Choices are delegated to leverage expertise. 

\ \\

When you are facing these dilemmas and trilemmas\marginpar{[Tag] Actionable Advice} there are constructive responses that can improve your effectiveness. Improvement benefits your results, your reputation, and the organization. 
\begin{itemize}
    \item Collect quantitative data for each variable. Quantitative arguments can augment qualitative stories. 
    \item Construct the Pareto frontier to identify non-optimal choices that can be eliminated.
    \item Instead of assessing variables in isolation, assess consequences in the context of workflows and impacts on stakeholders.
    \item Discuss subjective decisions with stakeholders so that potential disagreements can be negotiated instead of creating friction.
\end{itemize}
 

 \clearpage
  \section{Unavoidable Hazards of Bureaucracy\label{sec:unavoidable-hazards}}

In a bureaucracy there are certain challenges that cannot be avoided. The value of recognizing them is to understand that what you're experiencing isn't an anomaly. The problem isn't unique to you, your circumstances, your coworkers, or the organization. The cause is the combination of all those factors.

\ \\

\begin{samepage}
\textbf{Separation of responsibility and accountability}. \\
Each bureaucrat in an organization has responsibilities associated with their role. The ability to complete the tasks associated the role are not wholly within the scope of the bureaucrat's control. Even if there is a desire for action, the action might not be immediately feasible because of a dependence on another person or process. 
\end{samepage}

\ \\

\begin{samepage}
\textbf{Appreciation for being an effective bureaucrat is rare.}\\
If you do your job well, at best no one will notice.

``No one [thanks teachers] for policing cheating. Not the cheaters, not the honest students who feel inconvenienced and mistrusted, and certainly not the school [administrators] who have to process academic dishonesty paperwork.''\footnote{https://dynomight.net/teaching/}
\end{samepage}

The number of thank you cards sent to \href{https://www.fda.gov/}{Food and Drug Administration} meat inspectors, \href{https://www.osha.gov/}{Occupational Safety and Health Administration} regulators, \href{https://www.fcc.gov/}{Federal Communications Commission}, \href{https://www.ftc.gov/}{Federal Trade Commission}, and \href{https://www.sec.gov/}{Securities and Exchange Commission} is likely small. 
% TODO: ask each agency how many thank you letters they receive

There are counter examples in public service bureaucracy. 
Law enforcement is thanked when there is a victim of a crime. The military is held in high regard. 

\ \\

\begin{samepage}
\textbf{Lack of accountability to fellow bureaucrats who rely on you.}\\
You are more likely to feel accountable to your supervisor. Performing for your supervisor and explaining your value matters more than the work done to support coworkers.  This also applies when you depend on other people -- they aren't accountable to you.
\end{samepage}

\ \\

\textbf{Decision-makers are under-informed and inexperienced.}\\
You can make a decision with insufficient information, and there is rarely an expectation of expertise or experience. 
Gathering information takes time and is thus burdensome.
Having experience requires getting experience.

\ \\

\textbf{Gathering data for decisions takes resources and expertise.}\\
When there is a desire to gather data, and there is data available to be gathered, an investment of time is necessary. Well-informed decisions take time, experience, or both.

\ \\

\textbf{Defining success is subjective and dynamic}. \\
Who defines success and for which audience in a bureaucratic organization is subjective because of the lack of feedback mechanisms. Consequences may not be immediately obvious. Worse, how success is defined can be changed at any time -- there's no need for consistency. 

\ \\

\textbf{Change threatens incumbents}. \\
Change of plans, roles, tasks, resources, flatness of org, scope, technology.

\ \\

\textbf{\href{https://en.wikipedia.org/wiki/Diffusion_of_responsibility}{Diffusion of responsibility} in the bureaucracy} \\
A specific task needs to be completed, and action requires involvement of multiple collaborators. 

Once responsibility is assigned, the new risk is that of blame.

\ \\

\textbf{Diffusion of blame in the bureaucracy}. \\
Example: if something goes wrong, who is at fault: the creator of a policy or the person tasked with executing the policy?

\ \\

\textbf{High latency feedback}. \\
See 
%section~\ref{sec:slow_deployment} 
\hyperref[sec:slow-deployment]{slow deployment of processes} and 
%section~\ref{sec:decision-delay} 
 \hyperref[sec:decision-delay]{decision delay}.

\ \\

\textbf{Weak feedback}. \\
Suppose my bureaucratic task depends on a service provided by you, a fellow bureaucrat. If your computer isn't working, your ability to provide the service is blocked by a dependency on a computer repair person. You might feel relieved -- you can relax while you wait on the computer repair and you don't have to do work providing the service to me. You might feel anxious -- you're unable to provide the service until the computer is fixed. Regardless of how you feel about the situation, my task is blocked. If neither you nor I have priority relative to the computer repair person's other tasks, then we wait. This question of priority could be resolved hierarchically (for which there is limited attention bandwidth) or socially (dependent on someone having a relationship with the repair person). 

The delay for my task may be sufficiently small that raising the issue through the hierarchy or socially is not a good use of time.

\ \\

\textbf{The person making the rules that you follow doesn't actually know what they're doing}. \\
Your choices then include
\begin{itemize}
    \item Follow the rules that are not correct. This harms your productivity and morale. 
\item You can violate the rules and be more effective. This puts you at risk for sanctions. 
\item You can work the change the rules. Then you are not doing the work that's needed.
\end{itemize}


\ \\

\textbf{You rarely get to pick who is on the team}. \\
When a task requires collaboration, there is rarely a choice of who you get to work with. 

\ \\

% TODO
\textbf{Rarely able to alter team membership.}

\ \\

\textbf{Fear of the unknown.}\\
Current suffering is tolerable compared to the uncertainty of change, especially when the suffering isn't felt by the decision maker.

By identifying this fear in yourself or those you collaborate with, you can discuss specific concerns and work to address them.
\marginpar{[Tag] Actionable Advice} 

\ \\

\textbf{Fear of change.} \\
Change disrupts the status quo, putting accumulated power at risk and altering relationships. 

An emotional basis for decision making (or avoidance of decision making) may or may not be rational. Discussing the fear with fellow bureaucrats can ease the burden. 

\ \\

\textbf{Fear of conflict.} Refers to conflict of ideas, not physical conflict. Conflict of ideas is not personal, though the consequences may have impacts on careers for stakeholders. \\
Professional disagreement is to be expected in a bureaucratic organization.

Feeling uncomfortable is distinct from feeling unsafe. Communicating while feeling discomfort is a useful skill, rather than avoidance. You can demonstrate courage by talking about your sense of discomfort. 

\ \\

\textbf{Inadequate resources: staffing, time, money.}\\
The resources you have to address a challenge may not match the complexity of the issue.

\ \\

\textbf{Outcomes for the team are ill-defined and constantly shifting.}\\

\ \\

\textbf{The reward for good work is more work.}\\
When a coworker doesn't do their fair share, then the productive employee shoulders more burden. The deficient worker has no incentive to improve

\ \\

\textbf{The organization has a lack of vision; or has vision but no plan; or has vision and a plan but no consensus; or has vision and a plan and consensus but inadequate resources.}

\ \\

% TODO
\textbf{Progress depends on subjective decision making.}\\
Rarely is the optimal path deterministic. 

\ \\

\textbf{Easier to ask for big money than small money}\\
Processes are scale invariant -- Regardless of whether you're asking for \$1000 or \$100,000, the process is the same. 

\ \\

\textbf{Flux of people and processes.} \\
staff turn-over, changing conditions, changing timelines, change of vision, and the need to be promoted. in a bureaucracy, consistency doesn't yield promotion.

\ \\

\textbf{Why are there so many rules?}\\
To address edge cases and malicious or dumb people.

\ \\

\textbf{So much paperwork/forms.}
Why is Red tape endemic to bureaucracy?\\
Paperwork as a form of coordination in processes to facilitate decentralized decision making. 

One of the motives for paperwork is to catch people who are misrepresenting. Documentation makes prosecution more straightforward.

\ \\

\textbf{Everything is slower.}\\
What is meant is "slower than desired" or "slower than I imagined in my simplified model"

One reason progress is slower than expected is because there aren't as many hours available as imagined.


% What's the point of this? Is there a consequence for the reader?

In a bureaucracy the actual time spent working is less than the number of hours you get paid for. Breaks during work, vacation from work, holidays, sick leave. 

% https://rescuetime.wpengine.com/work-life-balance-study-2019/

\begin{figure}[H]
    \centering
    \includegraphics[width=0.8\textwidth]{images/hours_per_activity_per_employed_year}
    \caption{Hours of ``work'' per year when accounting for the rest of life. Assumes 5 weeks of vacation, 2 days of sick leave, and 11 holidays.}
    % footnotes in caption is not recommended; see https://texfaq.org/FAQ-ftncapt
    % however it can be done; see https://stackoverflow.com/questions/67621322/footnote-in-caption-of-figure-on-latex
    \label{fig:hours_per_year}
\end{figure}

Fig~\ref{fig:hours_per_year}
\footnote{\href{https://docs.google.com/spreadsheets/d/1ZaOZZXWkEzX4fFltUdlR4A6ENrAXnkzTW4YrjA4tDO8/edit?usp=sharing}{source for calculations}}

\ \\

\textbf{Bureaucracy inhibits creativity.}\\
See 
%section~\ref{sec:innovation} for 
notes on \hyperref[sec:innovation]{innovation}.

\ \\

% https://graphthinking.blogspot.com/2017/06/measuring-growth-of-bureaucracy.html
\textbf{Bureaucracy grows.}\\
The causes of growth include
scope creep, an increased number of people participating in the process, an increased workload, or the ratio of workers to managers decreases. 

You can measure the growth of bureaucracy using metrics like the latency in addressing a request and 
the number of people involved in addressing a request.  

\textbf{Increasing the size of a bureaucracy is easier than cutting it down.}\\
\ \\

% https://graphthinking.blogspot.com/2017/04/growth-of-bureaucracy.html
\textbf{The larger a hierarchy, the ratio of workers to managers gets worse}. \\

As complexity and workload increase, the number of staff needed increase. To facilitate coordination, managers arise. The amount of work being done in a bureaucracy can be small compared to the number of participants.

In figure~\ref{fig:growth_of_bureaucracy}a, the ratio of workers to participants is 3:3=1. When a manager is added, the ratio is 3:4 = 0.75 (a lower value means less work per employee is being accomplished). Adding a second team with dedicated management puts the ratio at 6:9=0.66. Finally, with a dedicated administrative team (e.g., payroll, hiring, facilities), the ratio is 6:13=0.46.

Adding more people to a hierarchical organization results in diminishing returns for time spent on the central work since bureaucrats invest time maintaining the hierarchy and administrative processes.

    \begin{figure}
        \centering
        \includegraphics[width=1\textwidth]{images/growth-of-bureaucracy.pdf}
        \caption{Growth of an organization as complexity and workload increase. a: three technical workers (green dots); b: administrative functions are delegated to an administrative assistant or manager (blue square); c: as complexity and scale grows, a new technical team (orange dots) is brought in which has their own manager; d: administrative work is delegated to a dedicated team of non-technical staff (purple diamonds).}
        \label{fig:growth_of_bureaucracy}
    \end{figure}

\ \\

\textbf{Each person in a bureaucracy has multiple roles}.\\
To limit the expansion of the bureaucracy. The multiple roles come from multiple relationships necessary to span an organization larger than \href{https://en.wikipedia.org/wiki/Dunbar\%27s_number}{Dunbar's number}. 
\ \\

\textbf{Everything is more complicated}. \\
Bureaucracy administers access to resources, whether tangible items (water, air) or expertise. There's friction for tangible resources because the easiest solution is you get the resource (without an intermediary). There's friction for expertise because the expert understands things you do not. 

\ \\

\textbf{Bureaucracy is inefficient and wasteful.}\\
The inefficiency of an organization is not just due to a breakdown of communication among its many members. Different individuals in different roles faced distinct incentives and have a different view of what problem is most relevant. There is a mixture of competition and cooperation internally for resources.


Can you explain a Pareto frontier for multiple objectives like speed, cost, accuracy?

Because decisions by bureaucrats are subjective, there is significant risk of being wrong or being called out by others as being wrong. Therefore, a motive to cover your ass for decisions made. Unnecessary work is carried out. 

Different participants have different motives, and the aggregation needed for coordination is inefficient regardless of what metric efficiency is measured against.

Another example: Inefficiency of changing the requirements on a project partway through. If an objective quantitative measure were available, the return on investment could be determined. 

Efficiency is typically assessed from the perspective of a serial process -- a single worker could accomplish this task faster, so why involve 10 people and get a slower, more burdensome result? The Mythical Man Month manifests \href{https://en.wikipedia.org/wiki/Amdahl\%27s_law}{Amdahl's law}. Dividing a task among 10 people does not make the task 10x quicker. It's not valid to expect that a task which takes 1 person 1 hour to perfectly scale so 10 people should be able to accomplish 10 results in 1 hour.

Inefficiency of process is similar. I need to file a request to replace the lightbulb in my office. Compared to the serial ``I'll just replace the lightbulb myself by running to the hardware store and purchasing one."

The excess resources characterized as waste are similarly driven by the serial-based assessment of what one person would require to carry out the task. 

\ \\

% https://graphthinking.blogspot.com/2020/07/scope-creep-is-experienced-differently.html
\textbf{Scope creep.} \\
The customer wants more, and the exploration of what's possible is exciting (a positive experience).

The creator hears more work. This implies a few trade-off options, all of which are negative for one or both parties.
\begin{itemize}
    \item sticking with the original terms (telling the customer "no", which is negative for both parties)
    \item re-negotiation for additional compensation (a burden to both parties)
    \item the programmer doing more for the same pay, which means less money per effort (yielding a less happy programmer)
    \item decreasing existing efforts to fit the additional new requirements (yielding a less happy customer)
    \item even if the programmer is getting paid by the hour, additional work means the end product will be delayed to accommodate additional features (yielding a less happy customer)
\end{itemize}

\ \\

% https://graphthinking.blogspot.com/2020/09/why-migrating-from-current-to-new.html
\textbf{Migrating technologies.} \\
The person implementing the transition has to be educated in both the old and new technology. 
the legacy code has to be migrated to the new implementation
convincing stakeholders; may require synchronization
difficulty scales with the number of stakeholders  \clearpage
  \section{An Effective Bureaucrat\label{sec:effective-bureaucrat}}

You can leverage your improved understanding of bureaucracy by changing your behavior. In this section I provide actionable suggestions. This section provides prescriptions, whereas other sections like the  
\marginpar{See page~\pageref{sec:dilemma-trilemma}.}%
\hyperref[sec:dilemma-trilemma]{dilemmas} 
and 
\hyperref[sec:unavoidable-hazards]{unavoidable hazards} are descriptive. If you only have the prescriptions (and skip the descriptions) you might get confused and frustrated as to why other bureaucrats are not applying best practices. 
%CANTDO\marginpar{See page~\pageref{sec:unavoidable-hazards}.} 


My baseline assumptions are that you are a good person, you have the relevant technical skills for your duties, and you are a good \href{https://en.wikipedia.org/wiki/Project_management}{project manager}.
\index{Wikipedia!project management@\href{https://en.wikipedia.org/wiki/Project_management}{project management}}\iftoggle{WPinmargin}{\marginpar{$>$Wikipedia: project management}}{}
Enacting Process Empathy is in addition to those elements. 

\subsection*{Characteristics of an Effective Bureaucrat}

Bureaucracy is a system of distributed knowledge and distributed decision-making, so coordination is critical. Here are characteristics to strive for. 
\begin{itemize}
    \item You communicate more effectively than anyone around you. This applies to verbal and written communication, \hyperref[sec:effective-presentations]{formal presentations} 
    \marginpar{See page~\pageref{sec:effective-presentations}.}% 
    and \hyperref[sec:walk-arounds]{informal interactions}. \\
%CONFLICTING\marginpar{See page~\pageref{sec:walk-arounds}} \\
    \textit{Why this doesn't happen by default}: people plateau when there is no incentive to improve. What level of quality is good enough depends on the situation-specific risks and costs.
    \item You facilitate meetings so that productivity is noticeably improved. This means creating and sharing an agenda, providing rules on interaction (e.g., raising hands), taking notes, sharing notes, and following on with assigned tasks after the meeting.\\
    \textit{Why this doesn't happen by default}: facilitation is rarely taught explicitly. The work of facilitation is considered secondary to the discussion, even when the productivity of discussions is disastrous and demoralizing. 
    \item You take part in meetings, whether that means actively contributing or intentionally supporting other attendees. You leverage relationships with other attendees.\\ 
    \textit{Why this doesn't happen by default}: active participation requires effort and exposes you to more risk than remaining silent.
    \item You invest effort in written communication (emails, text-based chats, reports). Your writing empathizes with readers, captures relevant context, is concise, and is clearly worded.\\
    \textit{Why this doesn't happen by default}: writing-as-a-skill is rarely the focus of a bureaucrat's work. Writing is seen as an entry-level skill that doesn't require improvement. Finding examples of clear writing in bureaucratic organizations is challenging, so you may struggle to find role models. 
    \item You communicate verbally concisely and precisely. You listen and you teach. You seek shared definitions. \\
    \textit{Why this doesn't happen by default}: Without a feedback mechanism motivating urgency, bureaucrats tend to pontificate. 
    \item If you are in-person with colleagues, you walk around and talk with people one-on-one.  \\
    \textit{Why this doesn't happen by default}: bureaucrats interested in ``just doing the work'' view professional interactions as a distraction that decreases productivity. 
    \item You are 
\marginpar{See page~\pageref{sec:professional-vulnerability}.}%
    \hyperref[sec:professional-vulnerability]{professionally vulnerable}. \\
    \textit{Why this doesn't happen by default}: Sharing stories of your personal experiences in the organization requires coherent narratives that the listener can learn from. That talent is not taught to all employees in most organizations. 
    \item In your role as bureaucrat you leverage project management skills: you have a vision, you make and share plans, all while building consensus with stakeholders. \\
    \textit{Why this doesn't happen by default}: Consensus requires knowing who is relevant to include and then investing in relationships that allow iterative feedback. That use of time is costly. The level of extroversion may not be attractive to every bureaucrat.
    \item You apply your negotiation skills~\cite{1982_Cohen} that improve your interactions and outcomes.\\
    \textit{Why this doesn't happen by default}: bureaucrats are not taught negotiation unless it is an explicitly named responsibility for their role. 
    \item You reply quickly to incoming requests, whether that means answering directly or acknowledging the request and providing a timeline for when you will answer. This allows other people's tasks to either be resolved or have a clear response about when progress can be expected. \\
    \textit{Why this doesn't happen by default}: The coping skills of responding to a large number of inbound requests either grows through experience, is learned by watching role models, or just doesn't happen.
    \item You strive for and demonstrate transparency. You share information with stakeholders. Transparency  enables coordination without interpersonal relationships.\\
    \textit{Why this doesn't happen by default}: Transparency endangers incumbents. Transparency requires an ongoing investment that doesn't contribute to addressing the primary issue. Transparency requires empathy with subjects.
\end{itemize}

\ \\

The superpowers of a bureaucrat that facilitate cooperation and progress in any organization include the following.
\begin{itemize}
\item You seek information from stakeholders without burdening them. When your requests are burdensome, you acknowledge that and seek ways to pay back (or forward) their investment.
\item You apply consistent processes (rather than being reactionary and applying ad hoc responses).
\item You hold others (and yourself) accountable for their actions. Accountability is created by clearly stating objectives and then measuring results.
\item You adapt to the varying incentives and reference experiences of those around you. Flexibility enables interactions with diverse coworkers. 
    \item You effectively multi-task or, more accurately, switch tasks. The switch among tasks is triggered when the current task encounters an externally-generated pause. You can \href{https://en.wikipedia.org/wiki/Pipeline_(computing)\%23Concept_and_motivation}{work on a task while waiting for another task to finish}. 
    \index{Wikipedia!pipeline computing@\href{https://en.wikipedia.org/wiki/Pipeline_(computing)\%23Concept_and_motivation}{pipeline (computing)}}
\iftoggle{WPinmargin}{\marginpar{$>$Wikipedia: pipeline (computing)}}{}
    \item You are prepared with a backlog of ideas (in writing) if someone shows up with resources.
    % https://graphthinking.blogspot.com/2016/12/life-lessons-i-learned-from-experience.html
    \item You know when to change communication channels (from text chats to phone calls to in-person). 
    \item If you are dependent on someone else getting something done to enable your progress, you can demonstrate priority by physically showing up -- \underline{presence creates priority}. 
    \index{mantra!presence creates priority}
    Being physically at a person's desk motivates that person to respond better than calling  or emailing them. Showing up where someone works and talking with them conveys how much priority you place on the actions of the person you're talking with.
    \item You have intellectual empathy -- theory of mind for thinking (whereas empathy refers to emotions). How to grow your intellectual empathy: shadow peers and bosses and coworkers and subordinates.
    \item You have \gls{process empathy}. You recognize the deviations and exceptions that cause processes to come into existence. 
    \item You avoid unintentionally neglecting tasks or requests. This requires capturing incoming requests and then providing a response about prioritization and status. This matters because other participants in the organization should regard you as reliable in a positive sense. 
    \item You focus on value delivery in relationships to a degree that exceeds the scope of your formal role.
\item You have altered your job description to fit the growth you're seeking.
\item You are willing to engage on a personal level and know stakeholders outside their professional role.
\item Each of your tasks has a customer, a deadline, and a deliverable artifact. You iterate towards a result. 
\end{itemize}

You occasionally ponder and discuss with other people introspective questions like
\begin{itemize}
    \item How can I be successful?
    \item What are the ways I could fail?
    \item How would the organization be characterized as successful?
    \item What are the ways the organization can fail?
\end{itemize}
% https://graphthinking.blogspot.com/2017/07/questions-to-ask-mentors.html
You find mentors and ask them questions like
\begin{itemize}
    \item What do you like most about your career? 
    \item Given a chance, what would you do differently?
    \item How do you manage work/life balance?
    \item What's the big challenge for our industry in the next two years?
    \item How would you tackle that challenge?
    \item What advice do you have for a young person starting in this industry? In this organization? 
    \item Are there mistakes to avoid?  
    \item How can I be successful?
    \item What books would you recommend reading?
\end{itemize}

Although you strive to enact Process Empathy, do not expect that of other people. 

\subsection*{Frames used by Effective Bureaucrats}

Being an effective bureaucrat is a mindset. I refer to these perspectives as framing. 

\ \\
\textit{Framing}: A bureaucrat can do more as part of an organization than by working alone. Being a member of an organization means the bureaucrat's identity is subsumed into service for the organization they are part of.\footnote{See Wikipedia entry on \href{https://en.wikipedia.org/wiki/Deindividuation}{Deindividuation} -- the loss of self-awareness in groups.
\index{Wikipedia!deindividuation@\string\href{https://en.wikipedia.org/wiki/Deindividuation}{deindividuation}}
%%%CANTDOINFOOTNOTE\marginpar{$>$Wikipedia: deindividuation}
} At the same time, bureaucracy enables the bureaucrat to amplify their presence by being part of a larger organization.  Sometimes the cost of being part of the organization exceeds the force multiplier of working together. 

\ \\
% https://graphthinking.blogspot.com/2019/05/definition-of-progress.html
\textit{Framing}: Measuring your personal growth in a bureaucracy is difficult due to the lack of \hyperref[sec:feedback-loop-and-ripples]{feedback loops}. One approach is to measure your capabilities for a specific task. If you can complete the task in less time, with fewer resources, and with less effort, that's progress. If you can now complete a task that you previously wanted to but weren't able to, that's progress.


\ \\
\textit{Feelings}: Bureaucracy induces an emotional response in participants because things don't work the way each person wants. This can lead emotionally to frustration and then apathy. Understanding how things work in a bureaucracy can help decrease the anger.

% Wandering the maze of bureaucratic processes as a subject.

Another emotional response to bureaucracy is a sense of powerlessness. 
\begin{quote}
``Some third person decides your fate: this is the whole essence of bureaucracy~\cite{1921_Kollontai}."
\end{quote}
That sense of powerlessness applies both to bureaucrats and to subjects of bureaucracy. 

The sense of powerlessness is somewhat valid, in that you are as a bureaucrat giving up some power compared to your ability to act individually. That is the trade for working with other people.

\ \\
\textit{Feelings}: A process feels bureaucratic when there seems to be dissonance. The subject faces a multi-step task that they can imagine one person doing the same action in fewer steps and taking less time. The illusion of decreased bureaucracy is created by consolidation from the subject's perspective. The barrier to enacting consolidation is the necessary coordination among different stakeholders who receive no benefit from the consolidation. Externalizing the coordination to the subject is what causes the sense of bureaucracy. 

\subsection*{How to be an Effective Bureaucrat}

The following advice is specific to situations that you may not have yet encountered. 

\ \\
\textit{Do}: \textbf{Work on three tasks concurrently.} Don't rely on one person or one idea for your success. On the other end of the spectrum, don't spread yourself too thin.
\marginpar{$>>$ \href{https://en.wikipedia.org/wiki/Goldilocks_principle}{Goldilocks balance}}%
\index{Goldilocks balance!number of tasks}%
Working on a multitude of projects decreases risk of any one outcome failing but also decreases the amount of attention spent thinking about a specific challenge. 

The three tasks should be at off-set stages (early, mid-way, and nearing completion) rather than all being the same level of maturity.

\ \\
\textit{Do}: When you get stuck, use the following changes of perspective:  
\textbf{Look upstream, look downstream. Look to peers. Zoom in (narrower scope) and zoom out (broader scope)}. These are all ways of changing the context. That change may help you identify assumptions that are holding back progress.

\ \\
\textit{Do}: When you are asked to take on more work, \textbf{avoid responding with ``That's not my job."} If the request is misguided and your perception is that another person has the responsibility, ask if the requester is aware of that other person's responsibilities. If you are to take on the work, get guidance on re-prioritizing and ensure the request is documented in writing. 

\ \\
\textit{Do}: If there's something you want to do, \textbf{strive for influence without authority} instead of working to gain control over resources (e.g., through promotion). Avoid the following: ``In this organization X is important to me, but I can't do X right now because I don't have enough power in the organization. So I'll get promoted and then do X."

\ \\
\textit{Do}: \textbf{Learn the perspectives of those around you.}\\
The relevance of knowing the paradoxes including dilemmas and unavoidable hazards, is that you should talk explicitly to your fellow bureaucrats about these specific topics in conversation. Not that the goal is to find consensus or agreement. But to find what the other person is thinking so that you can account for their processes

\ \\
\textit{Do}: \textbf{Account for holistic view.}\\
The specific circumstances of the challenges you face as a bureaucrat depend on the individual people involved, what the purpose of the bureaucracy is, what technology is available for enacting bureaucracy, and the resources (staffing, money, time). 

\ \\
\textit{Do}: \textbf{Learn the history of the challenge.}\\
This goes beyond
\href{https://en.wikipedia.org/wiki/G._K._Chesterton\%23Chesterton's_fence}{Chesterton's fence}\iftoggle{haspagenumbers}{ (see page~\pageref{concept:chestertons_fence}), }{,}%
\index{Wikipedia!Chesterton's fence@\href{https://en.wikipedia.org/wiki/G._K._Chesterton\%23Chesterton\%27s_fence}{Chesterton's fence}}%
\iftoggle{WPinmargin}{\marginpar{$>$Wikipedia: Chesterton's fence}}{}%
which focuses on why the current approach is in place. Learning the history of a problem means what has been tried before and failed. How did the previous iterations evolve into the current situation? Was the cause personalities, insufficient resources, inadequate technology? What's changed that enables this approach to be better? What do you know that prior attempts didn't?

\ \\
\textit{Do}: \textbf{For a given challenge, work on three remedies}.\\
One solution you're working on may fail, and another may be inadequate. 

\ \\
\textit{Do}: \textbf{Minimize \href{https://en.wikipedia.org/wiki/Externality}{imposing costs on other people}}.\\
\index{Wikipedia!externality@\href{https://en.wikipedia.org/wiki/Externality}{externality}}\iftoggle{WPinmargin}{\marginpar{$>$Wikipedia: externality}}{}
A solution that externalizes costs harms the greater organization and creates bureaucratic debt.

\ \\
\textit{Do}: \textbf{Exploit the flexibility of rules for the benefit of all parties.}\\
% https://graphthinking.blogspot.com/2019/07/winning-game.html
change the rules of the game and the objectives of the game such that every participant wins.

\ \\
\textit{Do}: \textbf{Align selfish interests with social interests.}\\
See also \href{https://en.wikipedia.org/wiki/Nudge_theory}{nudges}
\index{Wikipedia!Nudge theory@\href{https://en.wikipedia.org/wiki/Nudge_theory}{Nudge theory}}\iftoggle{WPinmargin}{\marginpar{$>$Wikipedia: Nudge theory}}{}
(page 66 of Schneier's Liars and Outliers~\cite{2012_Schneier}).

\ \\
\textit{Do}: \textbf{Find ways to rephrase negative complaints}.\\
% https://graphthinking.blogspot.com/2019/07/how-to-rephrase-negative-observations.html
Negative observation: ``Logging into my computer takes a long time."\\
Positive statement and explanation of impact: ``If the latency for logging into my computer were lower, I could make more progress on X."


Negative observation: ``The service team I need support from doesn't offer a ticketing queue."\\
Positive statement: ``If the service team I need support from offered a ticketing queue, I would be able to track the work done on my behalf."

\ \\
\textit{Do}: \textbf{Share lessons learned}.\\
This can build your network. People value vulnerability in others; you can make the first move.  \clearpage

\chapter{Bureaucracies are made of Humans\label{b_made_of_humans}}
% instead of presenting specific lessons, here I'm going to present insider perspectives on the fundamentals described in previous chapters. 

This chapter provides an insider's perspective of working in a bureaucracy. What are all the aspects to consider?
% unordered essays to be clustered later

  \section{Getting started in a Bureaucracy}
    This section focuses on a bureaucrat's initial exposure to bureaucracy. 
  
    \subsection{What to Read\label{sec:to_read}}

The content of this book is independent of the specific role you as a bureaucrat will be playing. I assume you are striving to be a good person, and that you have relevant education/experience for the position in the organization.

The reading lists 
\href{https://github.com/LappleApple/awesome-leading-and-managing}{Leading and managing}, 
\href{https://github.com/kdeldycke/awesome-engineering-team-management}{Engineering team management}, 
\href{https://github.com/pdfernhout/High-Performance-Organizations-Reading-List}{High Performance Organizations}, and 
\href{https://github.com/ankitjaininfo/awesome-managers}{Managers}
are outside the scope of a guidebook for bureaucracy. 
%Bureaucracy involves hierarchy, so understanding the mechanics and nuances of hierarchical roles is useful. 


%Books on management might not seem applicable if you are not already in a management role.
Reading management books is helpful regardless of whether you are a manager, want to become a manager, or if you never want to become a manager. In all three cases, gaining awareness of best practices and alternative approaches benefits you as a bureaucrat and the people you coordinate with -- bosses, peer bureaucrats, subordinates, and subjects.

    \subsection{Hiring into a Bureaucracy}
Regardless of the specifics of the job, there are specific attributes that make a candidate more likely to be successful in a bureaucracy. 

In addition to role-specific skills, hire for \href{https://en.wikipedia.org/wiki/Metacognition}{metacognition}, social skills, and intrinsic motivation.

% https://graphthinking.blogspot.com/2021/04/screening-for-metacognition-in-job.html

% https://graphthinking.blogspot.com/2021/07/screening-for-intellectual-empathy-in.html

% https://graphthinking.blogspot.com/2021/04/questions-to-ask-interviewer-when.html


    \subsection{Promotion}

Promotion is a critical aspect of designing incentives for behavior. Promotion is central because there are a variety of motives -- for money, for status, for authority. 

promotion of individual (rather than team) results in hero culture

What is preventing innovation is lack of risk taking. Actual risk means potential for failure. fear of failure because culture avoids failure due to concerns of waste. Also, anyone who fails can use this argument, but that isn't the same as ``fail fast'' because what's critical is learning from failure and sharing that insight gained so other's don't need to repeat. 

two options for learning: your own mistakes, or the mistakes of others. "formal education" is about the latter

    \subsubsection*{Organization chart orientation
\label{sec:org-chart-orientation}}

A common method of describing relations within the bureaucracy is the organization chart (commonly the ``\gls{org chart}"). \iftoggle{glossaryinmargin}{\marginpar{[Glossary]}}{}%
Normally the Chief Executive Officer (CEO) is at the top of the chart, middle management is in the middle, and managed employees are at the bottom. See Figure~\ref{fig:org_chart_orientation_ceo-at-top}\iftoggle{haspagenumbers}{ on page~\pageref{fig:org_chart_orientation_ceo-at-top}.}{.}

Artifacts like org charts subtly convey an organization's culture. 
% What's the point of this section? Is there a consequence, or is this just an observation?
There are emotional connotations to alternative layouts. You can alter expected relations (culture and norms) by playing with the orientation of the org chart.
Org chart orientation can be overanalyzed, so the exploration in this section is limited.

The point of thinking about org chart orientation is to frame how you perceive your supervisors, peers, and the bureaucrats you manage. Notice that the framing is embedded in the words -- prefixes super (over) and sub (under). 
These concepts inform what you expect from relations.
Do I seek support or direction and guidance from my boss? What do I expect from my boss? My peers? The people I oversee? What do I expect to provide them?

%\begin{itemize}
%\item 
%\end{itemize}

The relative orientation of the \href{https://en.wikipedia.org/wiki/Chief_executive_officer}{CEO} 
\index{Wikipedia!Chief Executive Officer@\href{https://en.wikipedia.org/wiki/Chief_executive_officer}{Chief Executive Officer}}\iftoggle{WPinmargin}{\marginpar{$>$Wikipedia: CEO}}{}
to the workers sets expectations for relations. 
Options for orientation are the conventional CEO at the top
(Figure~\ref{fig:org_chart_orientation_ceo-at-top}), 
CEO at the bottom (Figure~\ref{fig:org_chart_orientation_ceo-at-bottom}),
CEO on the right (Figure~\ref{fig:org_chart_orientation_ceo-leads}),
CEO on the left (Figure~\ref{fig:org_chart_orientation_ceo-follows}),
CEO as the center of a star 
(for example, the diagram for the \href{https://en.wikipedia.org/wiki/File:League_of_Nations_Organization.png}{League of Nations} in 1930.)
\index{Wikipedia!League of Nations diagram@\href{https://en.wikipedia.org/wiki/File:League_of_Nations_Organization.png}{League of Nations diagram}}

\begin{figure}
\begin{center}
\includegraphics[width=1\textwidth]{images/org-chart-orientation-ceo-at-top.pdf}
\end{center}
\caption{Standard orientation. The role with the most responsibility and authority is at the top. Left-right ordering is intended to be irrelevant in this view, though left-to-right reading order emphasizes importance.}
\label{fig:org_chart_orientation_ceo-at-top}
\end{figure}

\begin{figure}
\begin{center}
\includegraphics[width=1\textwidth]{images/org-chart-orientation-ceo-at-bottom.pdf}
\end{center}
\caption{Flipping the orientation presents a more realistic view of the CEO's responsibility. The crushing burden of servant leadership is clear. Left-right ordering is intended to be irrelevant in this view.}
\label{fig:org_chart_orientation_ceo-at-bottom}
\end{figure}

\begin{figure}
\begin{center}
\includegraphics[width=0.7\textwidth]{images/org-chart-orientation-ceo-leads.pdf}
\end{center}
\caption{Conventionally time flows from left (old) to right (new), so in this graph the CEO leads the charge into the unknown. Is the CEO dragging workers forward, or are the workers pushing the CEO? The top-to-bottom order can be read as importance. }
\label{fig:org_chart_orientation_ceo-leads}
\end{figure}

\begin{figure}
\begin{center}
\includegraphics[width=0.7\textwidth]{images/org-chart-orientation-workers-lead.pdf}
\end{center}
\caption{The ``chariot view'' with the CEO in the chariot and the workers out front. Workers are in the future; the CEO is in the past operating on old information. As with Figure~\ref{fig:org_chart_orientation_ceo-leads}, top-to-bottom ordering can be read as importance. }
\label{fig:org_chart_orientation_ceo-follows}
\end{figure}

\begin{figure}
\begin{center}
\includegraphics[width=0.8\textwidth]{images/org_chart_wedding_cake_dependencies_-_manufacturing.pdf}
\end{center}
\caption{An internal-customer-oriented view rather than a reporting-oriented view. The center of the bullseye is the team that generates the value that is the focus of the business or the organization.
The outer rings support teams that exist in the inner rings. The diagram is specific to an organization's domain. This visualization identifies which teams are the customers of which other teams in an organization.}
\label{fig:org_chart_wedding_cake_manufacturing}
\end{figure}



%extension of 
% \href{https://en.wikipedia.org/wiki/Conway\%27s_law}{Conway's law}: seating chart reflects org chart
    \subsection{Professional Training}

There is often a disconnect between the results formal education and the specific needs of the organization. Professional training is intended to fill that gap.  
  \newpage
  \section{You are an Individual in an Organization}
  
    This section focuses on the individual bureaucrat operating within an organization. 
    
    There is no neutral member of a bureaucracy. Either you are contributing to the organization, or you are sapping resources from the organization. This is because the organization operates in a zero-sum game for resource allocation.
    \subsection{Ideas for Innovation within a Bureaucracy\label{sec:innovation}}


The innovation lifecycle in a bureaucracy is
\begin{enumerate}
    \item You observe problems and challenges in your environment. This manifests as complaints from both the people directly harmed and observers who see inefficiency.
    \item You share ideas for innovation and gets feedback. Build a coalition of people willing to fight for the idea on your behalf
    \marginpar{[Tag] Actionable Advice}
    so that when you're not present, the idea is still proceeding towards implementation.  That puts the threshold at ``so important other people are willing to pause whatever they were working on and take up your cause.''
    \item Implementing these suggestions require either a change to existing processes or new processes or an investment of work. These changes may not succeed -- there's risk. Your idea could fail because it's not a good idea. It could also fail because someone doesn't like you, or the idea doesn't account for some dependency you weren't aware of, or it might conflict with other changes in progress.
    \item If you do decide to invest effort, the activity takes you away from your current work. Implementing the change might involve skills you don't have; learning those skills takes time. Carrying out the activity with new skills increases the likelihood of novice mistakes.
    \item If someone else implements the idea they get the credit for having done the work.
    \item If the idea saves money or time, there's typically no monetary reward. Recognition and benefit to your reputation isn't required as part of the change process. 
\end{enumerate}

There are many barriers in that lifecycle. The problem has to be observable to someone willing to invest effort in change. That person has to build a coalition of stakeholders. If the person isn't negatively harmed, that person may also lack clear benefit from resolving inefficiency. 

In addition to the work of implementing the change, there is an administrative overhead of documenting the reason for the change. 
Subjective decisions mean choices have to be defensible. 
The need for defensible justifications result in conservative decisions and risk aversion and a decrease of motive for innovation. 



These barriers lead to external observers to conclude something like the following simplification:
\begin{quote}
Bureaucracy destroys initiative. There is little that bureaucrats hate more than innovation, especially innovation that produces better results than the old routines.
\marginpar{[Tag] Folk wisdom}
Improvements always make those at the top of the heap look inept. Who enjoys appearing inept?\footnote{Frank Herbert (1987). ``Heretics of Dune'', page 201, Penguin}
% https://www.azquotes.com/quote/453163
\end{quote}


    \subsection{Motivation of Bureaucrats\label{sec:motivations}}

Appreciating the diverse motives of the bureaucrats you work with is instructive. If you expect everyone to have the same motives as you then you will be surprised by the friction created by diverse motives. 

Motivations of participants are rarely ``how can I make the organization more successful" or even ``how can I sell/produce more product"? Usually motivation is based on personal success in various manifestations, which leads to emergent phenomena which appears confounding to observers outside the bureaucracy. 


% see https://en.wikipedia.org/wiki/Social_influence

Each bureaucrat has a motive, even the bureaucrats who do nothing. 
% https://graphthinking.blogspot.com/2020/02/there-is-no-idle-status-for-paid.html
In an organization where you are a paid bureaucrat, you are either actively working for improvement of the organization, or your existence is parasitic to the organization. There is no ``idle" status for paid employees in an organization with limited resources.

As an example motive for a bureaucrat, I want to avoid being too efficient such that I eliminate the need for my job, and not so inefficient that the organization fails and I lose my job. Increasing the efficiency of bureaucracy is good for the organization and the outcomes, but can be harmful to the bureaucrat's career.

Career stability within an organization is a benefit, and it can be leveraged to take more risk. However, it typically manifests as inaction by an employee. There's no harm to the employee in not taking action. If an employee doesn't do anything, nothing bad will happen to that employee. Career stability decreases extrinsic motivation.


Example motivations for bureaucrats: 
stability (aka job security, the comfort of a routine),
money (current pay or future earnings), 
travel, 
problem solving, 
status, 
exerting power or control, 
credibility of being associated with the organization (if the organization has a positive reputation), 
logistical convenience (``the office is near where I lived''), 
service to people the organization serves.



The consequence of diverse motives is that expecting bureaucratic organizations to be logical, fair, consistent, and efficient is unreasonable even when every participant wants those features. Each bureaucrat thinks, ``I am logical, fair, consistent, and efficient.'' Therefore each bureaucrat expects other bureaucrats to meet those same (unrealistic) standards. Next, anthropomorphize the team or organization and expect the group to meet those standards. 

Even if each bureaucrat were logical, fair, consistent, and efficient (they are not, and neither are you), each person has a different motivation. Each person wants to accomplish something different using their unique skills and referencing their own experiences. Compounding the confusion, each bureaucrat has to coordinate using communication that has latency and limited bandwidth and isn't precise.

An expectation that bureaucracy feels illogical, unfair, inconsistent, and inefficient is a useful baseline. Working against bureaucratic entropy yields improvements even though perfection is inaccessible.
    % management isn't leadership
    \subsection{Creating change in the organization\label{sec:creating_change}}

If the organization you are a participant in has no problems or challenges, this subsection can be skipped. For bureaucrats in organizations that do have structural problems, this part of the guidebook provides points to ponder independent of the specific problem.

As a bureaucrat, you have unique insight on the problems the organization faces, and you have unique leverage to alter the situation.  While you could proceed haphazardly, an effective bureaucrat has vision, goals that break down the vision, plans on how to achieve each goal, and milestones which indicate whether the plan is proceeding. 

Perspectives to consider when assessing change include what the situation is, what the situation could be, and what the situation looks like from different stakeholders.

Confounding your ability to improve the organization, there are people around you who have conflicting visions or no vision. There are different views on whether something is actually a problem, different prioritizations, and different approaches to addressing problems.

A trade-off to consider is that having niche impact is easier than broad change. There's also a trade-off of the quick fix versus more robust solutions.

For a given structural problem in an organization, options include technical solutions, changing policy, or changing cultural norms of participants.

determine social/political/technical impediments. 

\textit{Tip}: Before starting a new effort, check to see whether this has been tackled before.\marginpar{[Tag] Actionable Advice}
Learn the history of the problem. Why hasn't this been solved?

\textit{Tip}: Query your first and second order social network.\marginpar{[Tag] Actionable Advice}

\textit{Tip}: get feedback early \marginpar{[Tag] Actionable Advice} before polishing

\textit{Tip}: advertise the result.

\textit{Tip}: hear criticism and respond

\textit{Tip}: leverage both social networks and bureaucratic processes. 

\textit{Tip}: Professional respect (for what the other person knows) and professional curiosity (for what you don't know) \\
Example: Getting approval from multiple overseers in different hierarchies is hard. Often different objectives and incentives.

% https://graphthinking.blogspot.com/2016/01/methodology-for-people-acting-as.html
\textit{Tip}: Use social recommendations to identify relevant individuals.
Leverage the trust already in their social network by starting with ``Person A recommended I talk to you about X."

% https://graphthinking.blogspot.com/2016/01/methodology-for-people-acting-as.html
\textit{Tip}: Sit in on meetings, listen to topics, see who is talking, who is attending. After the meeting talk to individuals about the meeting. Set up one-on-one informal discussions. Keep the first conversation  brief - 10 or 15 minutes. Your body language should indicate engagement and interest. ``Who else would you recommend talking to?" is the last question in the first conversation.


\textit{Tip}: Consensus doesn't mean everyone agrees on the problem, the remedy, or the approach, or who's taking the action. Consensus in a bureaucracy means people aren't going to resist the change.
    \section{Measuring Bureaucratic Maturity of Bureaucrats}
% https://graphthinking.blogspot.com/2021/07/three-measures-of-bureaucratic-maturity.html

Bureaucratic maturity of bureaucrats in an organization can be broken into three behaviors: 

The first and most wide spread behavior is to see a problem and then complain about the situation. This indicates awareness of the environment but no creativity regarding what to do about the problems.

The second, less common behavior, is to see a problem and recognize the situation as an opportunity. This requires creativity and reframing. 

The third behavior is to see a problem and then nudge the situation towards a vision. The nudges could be through either direct action or by influencing others. 



An individual bureaucrat can show one or more of these behaviors. That is, being capable of the third behavior does not mean the person lacks the ability to complain on other topics. 

There is not a specific amount of experience within the organization needed to arrive at any of these three paradigms. A holistic understanding of the system certainly helps.

The ``vision" of the third behavior could be in the form of a long-term (temporally distant) outcome, or the vision could be of immediate multi-party cooperation. 
  \newpage
  \section{Working with other Bureaucrats\label{sec:working-with-other-bureaucrats}}

Delegation of tasks
Asking for help
Seeking input
Offering to help
Offering input
Understand what someone else is priorities are and why those are priorities

These all directly impact your reputation; see \S~\ref{sec:reputation}.


enumerate tropes to figure how to respond

% https://graphthinking.blogspot.com/2021/10/why-i-dont-like-being-in-management-role.html
Solo work may be more emotionally rewarding due to fewer external constraints, but the cost is complexity and scope being limited to the skills of the individual. 

Working with others allows you to occasionally accomplish complex results beyond your own skills or your own bandwidth in spite of collaborators not being under your control. How? Through persuasion. 

The challenge of collaboration is to multiply productivity rather than merely sum the output of a set of individuals. 

inside an organization, cooperation/coordination is not held together by internal contracts or even service level agreements. What holds the organization together? Force of will of participants. 

two networks: formal an informal % section
    \section{Not my Job: Task Scope}

% https://graphthinking.blogspot.com/2021/05/not-my-job-task-scope-and-collaboration.html

If I ask someone for help on a task that benefits our team, that person might respond that the task is ``not my job" and time spent on tasks like that ``gets in the way" of their progress. The interesting (emotionally rewarding) work is not that task.

`\href{https://www.urbandictionary.com/define.php?term=scut}{Scut}' is medical slang for the non-clinical yet essential tasks that do not require a doctor's degree or expertise.
This is different from administrivia (administrative tasks).
%as ``scut'' would include taking out trash. 

% TODO: expand on
``Once I realized someone else has the same problem, I stopped worrying about it.'' Same as ``not my job."

The potential reasons for this reluctance to help include
\begin{itemize}
    \item Working on the task will not get them promoted.
    \item Their understanding of the scope of their job and their expertise does not include the task.
    \item The person don't know how to do the task and they don't want to learn.
\end{itemize}
In any case, progress for the individual is not aligned with the success of the team.
    \subsection{Building, Managing, and Spending Reputation\label{sec:reputation}}

Your personal reputation within the organization dramatically impacts your effectiveness.

Organizations have reputations externally. 
Internal-to-the-org there is cultural norms. 
% https://graphthinking.blogspot.com/2021/01/why-active-shaping-of-culture-is.html


% https://graphthinking.blogspot.com/2018/05/my-evolving-view-on-role-of-my.html


Building reputation through multiple small wins or larger risk on bigger bets


Relation between Reputation and Brand and Political Capital? Same thing?

% the following article is useless
% https://www.indeed.com/career-advice/career-development/build-a-reputation
% since it reduces to "be a good person"

How does an individual create and accumulate political capital? What does political capital mean with respect to teams?

% https://graphthinking.blogspot.com/2021/09/notes-from-class-on-being-politically.html

Reputation matters for influence. How other people perceive you impacts what you can accomplish and when people seek out your help or input.

Your reputation is actively changing based on your activities and communication -- both your communication and the stories others tell about you.

Neglecting to manage your reputation means you lose input to the stories others tell about you. Active management of your reputation requires engaging people and generating evidence. 

Reputation is set whenever and where ever you are observed, or artifacts are associated with you. What you wear, when you show up, how your emails appear, body posture in meetings. 


Reputation is perception of the person the bureaucrat is engaging with.  What does that person think of you?

Ideally that would be a function of their technical skill, ability to collaborate with other people, the strength of their network, creativity. None of this matters if the person you're engaging with doesn't know those things. 

Based on your reputation, what trust does that person have? 

To spend reputation is to bend the rules. 

Spending reputation means taking risk that involve other people

Build reputation by doing useful things that are visible to other people

    % https://graphthinking.blogspot.com/2021/07/patterns-anti-patterns-in-bureaucracy.html

% intra- and inter- team dynamics

\subsection{Teams are Subdivisions of an Organization}

\cite{2015_Katzenbach}

In the context of altering teams, there are a few major levers available: create new team, merge teams, dissolve a team. 


For a given set of teams, the lateral interactions are competitive or cooperative. Coordination is required (or conflict will occur) for money, staffing, and resources. Examples of resources include access to or control of data, computer equipment, hardware, floor space, prestige, products (output).

\subsubsection{Roles of Management versus Leadership}

Teams include managers and leaders. Those roles are not necessarily the same person. 

A manager's role involves time management, task tracking, employee evaluations, promotion, pay, requesting resources for team members, and hiring. 

A leader's role focuses on coordinating vision and principles. The vision and principles do not have to originate from the leader -- other team members can contribute ideas. The leader's responsibility is primarily social consensus. 
    \subsection{Leveraging Expertise}

Experts are cautious to include caveats and limitations

Translation process is necessary from "here's what's known in general" + "situation specific facts" to "plan of (in)action" and what's realistic with what confidence. 

Experts straying outside their expertise are dangerous because the boundary may not be clear to the audience.

non-experts think and speak less precisely


        \section{Does Anyone Want to Volunteer?}

% https://graphthinking.blogspot.com/2020/06/what-to-ask-instead-of-does-anyone-want.html

The facilitator of a meeting asking, ``Does anyone want to volunteer for this task?" to a group will most often be met with silence. 
The lack of engagement can be addressed by not asking for volunteers. The outcome can be higher engagement with the following tactics:
\begin{enumerate}
    \item Ask each person whether they are attending to observe or to volunteer.
    \item Ask each person how much time they are willing to volunteer. Responses could be zero house, one hour (non-recurring), one hour per month, or something else.
    \item Ask each person what their goals in the interaction are. It is usually generic (``I want the group to succeed.") but it can be narrow, in which case that gives you something to focus on.
    \item Ask each person what they are good at and what skills they have. Are they good with personal interaction? Writing? Computers? Coordinating? Logistics? Fundraising? Making phone calls?
    \item From the inventory of tasks, are there any that fit both the skills and time? Can the task be scoped to fit the time? If there are multiple candidate tasks for a volunteer, let them pick the task. (If no existing task aligns with their skills, do not create work to be assigned.)
    \item If there are other volunteers working on the same tasks, put them in contact with the other participants.
    \item Give the volunteer a deadline for the task. Your deadline should not be arbitrary -- it should be based on the task dependencies. 
    \item Confirm that the volunteer is willing to commit the time to complete the tasks by the deadline.
    \item Schedule a check-in with the volunteer before the task deadline to review progress.
    \item To create accountability among multiple volunteers, hold a group review to explain how each participant's work contributes to the goals and how work done by one person enables the next task done by someone else. (Enumerate the dependency graph; include deadlines.)
\end{enumerate}

  \section{Internal product development and deployment\label{sec:internal_product}}

Teams in a bureaucratic organization 
\begin{itemize}
    \item consume from outside-the-organization
    \item consume from within-the-organization
    \item produce to outside-the-organization
    \item produce to within-the-organization
\end{itemize}

This section focuses on the relation between teams that produce to within-the-organization and teams that consume from within-the-organization. Tools or products that are created internally and consumed internally.

Feedback mechanisms and incentives in a non-profit monopoly. The claim of success that a team created a product that met all design requirements on-time may have no actual benefit to users. Or maybe a product that benefited internal customers was created and there were happy users, but the originating team has to quickly move onto the next project to create another success and thus has no attention to on-going support. Determining the metric of success is tricky. do you want to aim for highest average happiness of stakeholders, or are some more important?

captive users who have little leverage 




  \newpage
  \section{Processes within a Bureaucracy}


    \subsection{Tactic: Approval, Forgiveness, Opposition}
% https://graphthinking.blogspot.com/2017/10/flipping-approval-mentatlity.html

A common task is consensus regarding action or expending resources. There are distinct options about how to get that consensus:
\begin{itemize}
    \item Seek approval. Incurs both providing justification and waiting.
    \item Ask forgiveness. Often viewed as being in contrast to seeking approval. 
    \item Solicit opposition. 
\end{itemize}
The best way to proceed depends on the personalities of the people involved in building consensus and their relationships. 

Most organization default to approval processes. Each new idea needs to be signed off as approved by a sequential list of bureaucrats. The sequence (not concurrent) process may be known in advance, or it may be ad hoc if the request is novel.

Relying on approval is harmful to innovation because sign-off by each bureaucrat is interpeted as ``I am 100\% in agreement with this.'' Each stakeholder has to bless innovation and tie their reputation to the outcome.

The ``I won't stop this'' is a more useful paradigm. With the consensus process language changed to "I won't stop this," then the bureaucrat can avoid taking responsibility for the idea and therefore are not tying their reputation to the result.

% https://news.ycombinator.com/item?id=15407757 % subsection
    \subsection{Static and Dynamic Process}

% https://graphthinking.blogspot.com/2017/04/static-versus-dynamic-processes.html

change is expected. What's the point of establishing static processes that are not robust to change?
So you can point to an effort and you don't look totally neglectful.

How to establish processes that are dynamic?
1) Document business logic; this is fragile to change.
2) enumerate assumptions used in the logic
3) relate the assumptions to a cost/benefit model
4) establish measurements that inform the cost/benefit model.

Then when change happens the variables of the cost/benefit get updated by measurement, the cost/benefit model alters the assumptions. Therefore the business logic needs to be updated.
    \subsection{Bureaucratic Debt}

% https://graphthinking.blogspot.com/2017/09/bureaucratic-debt-and-what-to-do-about.html

Suppose a process is implemented now, and later found to be ineffective. Some work is needed to revise the process and hopefully improve effectiveness. \gls{bureaucratic debt} is the cost of that work needed to change a process. The bureaucratic debt is caused by choosing an easy solution now (with limited information or insufficient resources) instead of using a better approach that would take longer to design and implement.

The purpose of defining this concept is to capture the otherwise unaccounted work resulting from decisions.


Decisions occur in a resource constrained environment (e.g., insufficient time, money, labor). Each decision made results in options that are not explored. Some of these missed opportunities are associated with short-term versus long-term trade-offs of costs.

These opportunity costs (what the organization doesn't do) alters which future decisions become available.

Getting information (measurement) and analysis are costly in terms of money, time, skill, and labor.

Once the concept of bureaucratic debt is recognized, the question is how to track it.

To document bureaucratic debt, we need to capture decisions as they are made:
\begin{itemize}
    \item what is the decision to be made?
    \item when the decision was identified?
    \item when the decision was made?
    \item who made the decision?
    \item what options were identified?
    \item which option was chosen?
    \item what that option was chosen over the other options
\end{itemize}
The purpose of documenting decisions is to enable both aversion to bad decisions and attraction to good decisions. Without documenting decisions, there is no transparency, accountability, or historical ability to track dependencies. 

The documentation of decisions needs to be disseminated to stakeholders. This should occur as promptly as possible. 

The scale of decision impact determines the level of documentation. "Do I choose pencil or pen?" incurs negligible bureaucratic debt; therefore the documentation needed is also negligible. Projecting impact of decisions is a subjective prediction. 

Similarities of technical debt and bureaucratic debt.
In developing software, there are three artifacts: the software, documentation on how to use the software, and documentation on why to use the software. The two distinct types of documentation are typically combined in one document. Each of these three artifacts are independent. The ramification of this is that each artifact can be created independently, and it takes work to maintain synchronization of the artifacts. 

Similarly for a bureaucracy, there is the processes and policies which get applied to customers, the documentation of what those processes and policies are, and guidance on when the processes and policies should be applied. As with technical debt, these three aspects are independent. 
 %subsection
    \subsection{Design processes for Turnover of Staff}

% https://graphthinking.blogspot.com/2020/02/design-for-turnover-rather-than-rely-on.html

When designing a process, there are a few goals to optimize for: time-to-first-result, average latency, financial cost, flexibility to input conditions, scalability. 
    %% https://graphthinking.blogspot.com/2021/04/laffer-curve-and-minimum-viable.html

The Laffer curve is a claim in economics that there is a relation between government tax rates and the revenue from taxes collected. The relation, based on Rolle's theorem, says that between a tax rate of 0% and 100%, there must be some amount of tax that corresponds to the maximum of revenue. 

While the mathematical statement may be provable, the use in economics seems hand-wavy. In this post, I'll extend that hand-waviness to a different domain: bureaucratic processes in organizations. The relation to the Laffer curve is that bureaucratic processes a tax on productivity. 
  \newpage
  \section{Communication within a Bureaucracy}
  I don't want to talk specifically about paper memos versus email versus phone calls versus video chat versus Skype versus slack. 
The relevant attributes are synchronous versus asynchronous. Searchable or not searchable. In that context, there might be some interesting bureaucratic specific things to think about
    \subsection{Role of Communication within a Bureaucracy}

Communication facilitates coordination of processes and allocation of resources within an organization. 

% https://graphthinking.blogspot.com/2021/09/why-is-everything-so-hard-in-large.html

Coordination among participants is one way to break the \href{https://en.wikipedia.org/wiki/Prisoner\%27s\_dilemma}{Prisoner's dilemma} of unaligned incentives. In practice, communication takes times and skill; not everyone is willing to invest in communication that is seen as not ``doing the work". Additionally, accounting for the \\href{https://en.wikipedia.org/wiki/Allen\_curve}{Allen curve} takes effort. Lastly, the time needed to arrive through consensus at an optimal approach for a given situation may exceed time available for solving the problem.
 
    \section{Failure to Communicate\label{sec:failure-to-comm}}

Communication is critical in bureaucracy because bureaucracy is a system of distributed knowledge and distributed decision making. When less effective communication occurs, individual bureaucrats are less able to rely on the knowledge of other experts, and they have to make decisions with less consensus. 

\subsection*{Slowing Communication\label{sec:slowing-communication}}

In an ideal scenario there would be no delay associated with communication -- you would get the information you need when you need it. In practice, there are various causes for why your progress is blocked when you depend on other people. 

Tactics that delay communication within a bureaucratic organization are stonewalling, slow-rolling, bikeshedding, and red herrings. By learning these concepts you will be better able to identify and then respond to their use.

\index{responsiveness!stonewalling}
\iftoggle{glossaryinmargin}{\marginpar{[Glossary]}}{}
\iftoggle{glossarysubstitutionworks}{\Gls{stonewalling}}{Stonewalling} 
is when the recipient of a request or question  doesn't reply. There may be \hyperref[sec:email-responsiveness]{legitimate reasons for the lack of response}.
\iftoggle{haspagenumbers}{\marginpar{See page~\pageref{sec:email-responsiveness}.}}{}%
The person may be busy and didn't see your message, or they did see your message but didn't have a chance to reply yet because a response to you is lower priority than other tasks they have. You cannot differentiate those reasonable causes from when the recipient doesn't want to enable you to proceed. They may disagree with your objective and see silence as \href{https://en.wikipedia.org/wiki/Passive-aggressive_behavior}{less confrontational}
\index{Wikipedia!passive-aggressive behavior@\href{https://en.wikipedia.org/wiki/Passive-aggressive_behavior}{passive-aggressive behavior}}\iftoggle{WPinmargin}{\marginpar{$>$Wikipedia: Passive-aggressive behavior}}{}
than explicit rejection. 

One way of circumventing stonewalling is to ask if the respondent is opposed to your idea. 
\marginpar{$>>$ Actionable Advice}%
\index{actionable advice}%
Then a lack of response indicates no opposition. This tactic applies if you are confident the recipient will read or hear the message.

An unintentional source of stonewalling is when you ask on the wrong channel. Sending an email may result in what appears to be stonewalling if the person relies on chat messages or the phone. The solution for this
\marginpar{$>>$ Actionable Advice}%
\index{actionable advice}%
is feasible, but action is required by the person who doesn't respond. The person who only uses certain channels should explicitly indicate that. An automatic out-of-office email that says, ``Contact me by phone'' tells the sender the \hyperref[sec:communication-preferences]{preferred channel}. 
\marginpar{See page~\pageref{sec:communication-preferences}.}

From the view of the person doing the stonewalling, if you need time to think or gather information before responding, 
\marginpar{$>>$ Actionable Advice}%
\index{actionable advice}%
tell the person who sent a request that you acknowledge their message and will follow up in more detail later (with a specific timeline). While better than no response, this leads to the next challenge.

\index{responsiveness!slow-rolling}
\iftoggle{glossaryinmargin}{\marginpar{[Glossary]}}{}
\iftoggle{glossarysubstitutionworks}{\Gls{slow-rolling}}{Slow-rolling} 
is when you get a response to your request or question, but the response isn't helpful. Progress is delayed because you have to iterate to get an answer. There are valid reasons for a slow-roll and there are uncool reasons for a slow-roll response. Perhaps the person wants to acknowledge your request but doesn't currently have time to provide a full explanation. The person may need to gather more information for the complete response. Or the person is \href{https://en.wikipedia.org/wiki/Passive-aggressive_behavior}{passive-aggressive}
\index{Wikipedia!passive-aggressive behavior@\href{https://en.wikipedia.org/wiki/Passive-aggressive_behavior}{passive-aggressive behavior}}
\iftoggle{WPinmargin}{\marginpar{$>$Wikipedia: Passive-aggressive behavior}}{}%
and may understand your question but does not want to enable your progress. 

The reason for a slow-roll should be made explicit by the respondent, 
\marginpar{$>>$ Actionable Advice}%
\index{actionable advice}%
and a timeline for a complete response is helpful. 


\label{concept:bikeshedding}
\index{responsiveness!bikeshedding}
\iftoggle{glossaryinmargin}{\marginpar{[Glossary]}}{}
\iftoggle{glossarysubstitutionworks}{\Gls{bikeshedding}}{Bikeshedding} 
is when the recipient of a question or request 
\href{https://en.wikipedia.org/wiki/Law_of_triviality}{focuses on unimportant details relative to the primary topic}. 
\index{Wikipedia!Law of triviality@\href{https://en.wikipedia.org/wiki/Law_of_triviality}{Law of triviality}}\iftoggle{WPinmargin}{\marginpar{$>$Wikipedia: Law of Triviality]}}{}%
Whether this behavior is intentional or not, the best response is to refocus the conversation on the core issue. The amount of time allocated for various topics should be proportional to their consequence. 

\index{responsiveness!red herring}
A \gls{red herring}\iftoggle{glossaryinmargin}{\marginpar{[Glossary]}}{}
response is misleading, whether intentional or not. The respondent provides what looks like a reasonable answer but results in unproductive work. Occasionally there is a coincidental benefit of discovering something unexpected, but that wasn't the respondent's intent. 


Even though bikeshedding and a red herring can be a \href{https://en.wikipedia.org/wiki/Dark_pattern}{dark pattern},% 
\index{Wikipedia!dark pattern@\href{https://en.wikipedia.org/wiki/Dark_pattern}{dark pattern}}
that may not be the intent of the speaker or author. Perception matters more than intent.
\marginpar{$>>$ Mantra}%
\index{mantra!perception matters more than intent}%
Process Empathy applies both when you are sharing information (how could the information be perceived?) and when you are receiving information (what was the author's intent?).

\subsection*{Decreased Effectiveness in Communication and Some Remedies}

\textit{Challenge}: The \href{https://en.wikipedia.org/wiki/Allen_curve}{Allen curve} 
\index{Wikipedia!\href{https://en.wikipedia.org/wiki/Allen_curve}{Allen curve}}
is 
\marginpar{[Tag] Folk Wisdom}
\index{folk wisdom!\href{https://en.wikipedia.org/wiki/Allen_curve}{Allen curve}}
an ``exponential drop in frequency of communication between engineers as the distance between them increases.'

\index{mantra!presence creates priority}
Presence creates priority - go to their desk. I once needed some data from a coworker. After trying email and phone calls, I ended up flying across the country. Once I arrived the person was able to provide the data in a few hours.

Merely sitting next to a coworker, even with no official purpose of interaction, results in spontaneous informal discussions. See the discussion of 
\hyperref[sec:prisoner-exchange]{Prisoner exchange} on 
page~\pageref{sec:prisoner-exchange}.

Take advantage of the \href{https://en.wikipedia.org/wiki/Allen_curve}{Allen curve} 
\index{Wikipedia!\href{https://en.wikipedia.org/wiki/Allen_curve}{Allen curve}}
by implementing the Inverse Conway Maneuver. If you know what structure is needed for a product, then design the placement of your team to reflect that.

\ \\
\textit{Challenge}: \href{https://en.wikipedia.org/wiki/Wiio\%27s_laws}{Wiio's law}: 
\index{Wikipedia!\href{https://en.wikipedia.org/wiki/Wiio\%27s_laws}{Wiio's law}}
\marginpar{[Tag] Folk Wisdom}
\index{folk wisdom!\href{https://en.wikipedia.org/wiki/Wiio\%27s_laws}{Wiio's laws}}
``Communication usually fails, except by accident.''\\
This pessimistic take is similar to \href{https://en.wikipedia.org/wiki/Murphy\%27s_law}{Murphy's law}
\index{Wikipedia!\href{https://en.wikipedia.org/wiki/Murphy\%27s_law}{Murphy's law}}
and is indicative of the level of ongoing investment needed for effective communication. 

\ \\
\textit{Challenge}: Periodic status reports sent up the chain of command get sanitized so that only good news is shared. This impedes risk analysis. \\
If your reports are getting sanitized, as for a copy of the sanitized version. If you are the person consolidating and aggregating reports, aim for conciseness rather than good news. 

\ \\
\textit{Challenge}: Decisions by bureaucrats high in the \href{https://en.wikipedia.org/wiki/Command_hierarchy}{chain of command}
\index{Wikipedia!\href{https://en.wikipedia.org/wiki/Command_hierarchy}{command hierarchy}}
are not pushed down the chain. \\
You can request management provide a summary of their activities.

\ \\
% Role of assumptions 
\textit{Challenge}: To assume makes an ass out of you and me, 
\marginpar{[Tag] Folk Wisdom}
\index{folk wisdom!To assume makes an ass out of you and me}
yet assumptions are necessary to making progress in communication.\\ 
This dissonance can be addressed by looking for sources of difference and then talking about them. For example, when I talk with someone for the first time I ask what their educational background it. If they have a different degree than mine, I can tune my language to their academic training. I can make my story more relatable. 

Another technique for detecting differences is to ask about the person's previous experience. What did they work on previously in this organization? What were their jobs before joining this organization? This backstory can provide context for decisions that need to be made in the current context. 

\subsection*{Leveling up Your Communication}

There are levels of enlightenment for bureaucrats. If you know the progression, you can step up to the next level more easily.
\begin{enumerate}
    \item I feel bad.
    \item (complaint) I can't do what I want.
    \item (complaint) I can't do what I want in the way I want.
    \item (complaint) There is a problem.
    \item I have a solution.
    \item I have an implementation.
    \item I tried but my solution didn't work.
    \item (burnout) Life sucks but I get a pay check.
    \item I quit (in hopes of being more successful somewhere else).
\end{enumerate}


\subsection*{Theory of What Could be Done and Why the Theories are Impractical}

% https://graphthinking.blogspot.com/2021/09/why-is-everything-so-hard-in-large.html 

To break the 
\href{https://en.wikipedia.org/wiki/Prisoner\%27s\_dilemma}{Prisoner's dilemma}, 
\index{Wikipedia!\href{https://en.wikipedia.org/wiki/Prisoner\%27s\_dilemma}{Prisoner's dilemma}}
options are 
\begin{itemize}
    \item Expose all participants to the consequence of outcomes. In practice this feels unfair to each participant because the outcome is partially attributable to other people involved in the process. Dividing responsibility limits exposure to consequences.
    \item Have all participants communicate. In practice communication takes time and skill. Not everyone is willing to invest since communication is not seen as ``doing the work.'' Accounting for the \href{https://en.wikipedia.org/wiki/Allen\_curve}{Allen curve}
    \index{Wikipedia!\href{https://en.wikipedia.org/wiki/Allen\_curve}{Allen curve}}
    takes effort. The time needed to arrive through consensus at an optimal approach for a given situation may exceed time available for solving the problem.
    \item Limit everything to what can be accomplished by one person. This hero-based approach is limited to the attention-bandwidth of the person and their skills. As the complexity increases the necessary skills increase and the number of candidate heros decreases. Large organizations accomplish complicated tasks by leveraging diverse skillsets of teams of bureaucrats.

\end{itemize}

%  How does communication among individuals fail?

 
 
    \subsection{Communication Preferences}
% why this matters: source of friction among bureaucrats
% benefit to reader: understanding why people are different

There are typically multiple communication channels available between bureaucrats in an organization. Phone, email, chat, in-person, voicemail, websites, video calls, calendars, message boards. 

Do you know which channel is preferred for what purpose? Can you list your own preferences, and the preferences of other stakeholders?

Do you keep your calendar up-to-date? Or is your calendar irrelevant? Is your calendar visible to your coworkers?

Text chat is useful for asynchronous interrupts. Text chats are better than stopping by in person or calling on the phone since in-person drive-bys and phone calls interrupt whatever I'm thinking about or discussing. The purpose of the text chat is either a reminder or seeking a convenient time to talk. 

Email is useful for notifications or questions or setting up logistics. If I don't respond to a question or request for action within 2 days, please send a reminder. 

Video calls are my preferred method for group meetings. Group meetings via video call scheduled in advance on my calendar are best. Impromptu video calls are acceptable but as interruptive as a phone call. 

In-person discussions are my preferred method for one-on-one discussions. Stopping by my desk  without an appointment interrupts whatever I'm thinking about. If a calendar invitation is inconvenient, a text chat confirming my availability prior to the in-person discussion is helpful. 

Phone is useful when the caller is unable to be present in-person. Phone, being voice-only, is inferior to video calls. Unscheduled phone calls interrupt whatever I'm thinking about. I've intentionally disabled voicemail. 

Status updates in electronic issue trackers or wikis or forums are best (rather than email-only). Notifying in text chat or email that a wiki page or issue is updated is helpful but not mandatory. 

Given all those caveats, I prefer communication in any form at any time over surprises. 

If you're not sure whether you should communicate, default to communicating. If you're not sure whether communication would interrupt something, check my calendar and then communicate. If it's urgent or blocking progress, ignore my calendar. 
    \chapter{Communication within a Bureaucracy\label{sec:communication-within-bureaucracy}}
{\footnotesize Back to the \hyperref[sec:toc]{Main Table of Contents}}
\minitoc


\section{Communication is Critical for Bureaucracy}
\Gls{bureaucracy} 
\marginpar{[Tag] Glossary}
is a system of distributed knowledge and distributed decision making. Distributed decision making relies on effective communication. Bureaucrats communicate by writing and speaking, both of convey incomplete information and are imprecise. The inability to be comprehensive and precise in communication means iteration is key when establishing a shared mental model of the situation and plan.


The dependence on iterative interactions means relationships matter for communication. 
\href{https://en.wikipedia.org/wiki/Metcalfe\%27s_law}{Metcalfe's law} 
\index{Wikipedia!\href{https://en.wikipedia.org/wiki/Metcalfe\%27s_law}{Metcalfe's law}}
says
\marginpar{[Tag] Folk Wisdom}
\index{folk wisdom!\href{https://en.wikipedia.org/wiki/Metcalfe\%27s_law}{Metcalfe's law}}
the value of an organization is proportional to the square of the number of people interacting in the organization. A team of 5 people is not just the aggregated skills of five individuals -- there's the synergy arising from ten bilateral relationships. Broadening your network of collaborators broadens your potential effectiveness and that of your fellow bureaucrats. 

% https://graphthinking.blogspot.com/2018/08/confusion-leads-to-confusion.html
Suppose that you don't invest in relationships with your fellow bureaucrats. Then you'll be less likely to know what is happening in your organization, and you won't understand why certain decisions were made. Because of your confusion you might focus on your own self interests since that is what you have some control over and insight on. That doesn't leverage the potential synergy of collaboration, but at least you look busy. 

The analysis above applies to each person in the bureaucratic organization.
% A Nash equilibrium?
The consequence is that the organization's goals are not achieved when bureaucrats act independently. 
There is a way out of this: broadcast your intent to other people, be transparent with decisions, and share the results of your activities. That decreases the confusion other people may have about your actions and can improve effectiveness. You can make yourself and the organization more effective even when your investment in communication is not reciprocated.

\ \\

Information within a bureaucracy shapes the relationships among bureaucrats. You have a choice: you can decide to not share information, you can be transparent, and you can be vulnerable.  
Not sharing information can harm or protect relationships. 
Being transparent conveys ``here's what is happening." Being vulnerable expands that to ``here's why that's happening or what might happen."
Each of those options informs how your coworkers, peers, management, and team members interpret their perception of your intent. Your communication shapes your reputation.


Another way communication alters bureaucratic organizations is when actions informed by locally-available information produce \hyperref[sec:failure-to-comm]{suboptimal results for participants}, as illustrated by the
\href{https://en.wikipedia.org/wiki/Prisoner\%27s\_dilemma}{Prisoner's dilemma}.
\index{Wikipedia!\href{https://en.wikipedia.org/wiki/Prisoner\%27s\_dilemma}{Prisoner's dilemma}}
The concept of ``locally-available information'' refers to communication among bureaucrats that facilitates delegation of work, allocation of resources, relationship creation/maintenance, and carrying out processes. Each bureaucrat benefits from gathering information that shapes their actions, but communication has a cost: time spent building and verifying consensus delays action. There is an opportunity cost when time is spent talking with people.

% https://graphthinking.blogspot.com/2021/09/why-is-everything-so-hard-in-large.html


    % https://graphthinking.blogspot.com/2020/05/invisible-bureaucracy.html

\subsection*{Social and Bureaucratic interactions\label{sec:socializing}}

Change in a bureaucracy can apply to processes and people, but a more amorphous concept is changing the culture of a team or organization. What is meant by ``culture'' usually refers to norms -- the expectations of behavior that individuals hold to. That definition of culture is generic; what is meant within a bureaucratic context requires jargon for specific expectations.

To evaluate expectations, we start by introducing categories of interactions. 
Interactions among members of an organization are either a social interaction or a bureaucratic interaction. 

As examples of each of these,
\begin{itemize}
\item \textit{Social interaction example}: ``Did you see the game on TV last night? Our team did fantastic, right? I wanted to get tickets for the game, but they were sold out."
\item \textit{Bureaucratic interaction example}: ``You'll need to get approval from Sue before presenting your idea to the board for their review. Then talk with Russ and get his thoughts about how to proceed."
\end{itemize}
Both social and bureaucratic interactions are vital to cohesion in an organization. 


Bureaucratic interaction can be broken into two subcategories: 
\gls{visible bureaucracy} \iftoggle{glossaryinmargin}{\marginpar{[Glossary]}}{}%
(procedures and processes are written down and can be discovered by stakeholders) and 
\gls{invisible bureaucracy} \iftoggle{glossaryinmargin}{\marginpar{[Glossary]}}{}%
(procedures and processes are known to some stakeholders and are conveyed verbally to some of the other stakeholders).

Invisible bureaucracy is akin to related topics outside the professional environment: invisible domestic work\footnote{Cleaning your living space, raising children, caring for pets; see~\cite{1987_Daniels}.} and invisible relationship work.\footnote{Consistent need to delegate, being curious without reciprocation.} The work associated with emotional cohesion, logistics, planning, scheduling, and communicating is hard to quantify so it does not get counted.


The relevance of this jargon is to break down the components of an organization's ``culture" experienced by participants.
When someone in the organization advocates for changing the culture, which expectations are they specifically referring to? Invisible bureaucracy is the hardest to alter because it is undocumented and not counted.
%The ratio of social relationship to visible bureaucracy to invisible bureaucracy is a characterization of the culture. There are norms associated with each of these three categories.

Processes default to invisible bureaucracy because creating and maintaining documentation requires work. Making the documentation discoverable requires work.
%, and because some processes are embarrassingly inefficient. 
To make invisible bureaucracy visible, document the work and enable other people to find the documentation.
 % subsection
    \section{Communication Tips}

The role of verbal communication is critical for bureaucrats. 
There is a lot of advice on effective communication (enunciate, speak loud enough to be heard, be humble, be curious), so the advice below is highlighted because of prevalence in bureaucratic organizations. 
The following is generic to interactions outside of bureaucracy. 

For general writing tips, see Strunk's and White's Elements of Style and other resources \footnote{\href{https://www.youtube.com/watch?v=vtIzMaLkCaM}{LEADERSHIP LAB: The Craft of Writing Effectively} and \href{https://www.youtube.com/watch?v=aFwVf5a3pZM}{LEADERSHIP LAB: Writing Beyond the Academy}}.

\subsection*{Tip: Not all interaction challenges are communication problems.}
Sometimes an inability to discuss ideas is not a communication problem but a psychological deficit of personality. Distinguishing ``I'm an ineffective communicator'' from ``the person I'm talking with doesn't communicate effectively'' from ``that person has a diagnosed psychological reason they are unable to communicate'' is tough for those of us who are not psychologists or psychiatrists. 

%how to measure effectiveness: The waste or inefficiency in a bureaucracy is a measure of the lack of coordination or inconsistent decision making among the members

\subsection*{Tip: Avoid relying on stereotypes.}
Within an organization different teams may build up reputations for certain behaviors, or there may be significant events that the team is associated with. 
When interacting with members of a team that has a reputation, avoid relying on that stereotype or event as an opening for discussion. 
You're speaking to an individual, so address that person's behavior.



\subsection*{Tip: Avoid questions that have a binary response\label{sec:yes_no_questions}.}

Responding to a request with ``no'' is advantageous for the person replying to the question. There is less work required, less risk of failure, and better continuity. As an example of a poorly framed question, I could ask, ``Can I have a copy of the data you're using?'' The person I'm asking is less disrupted if they refuse to share. 

A more constructive phrasing is ``I need information on X to work on Y, and I think you have information about X. How can you help me get information on X?'' By clarifying my intent, I allow the person with the data to provide options I may not have considered.

Similarly, when I'm being asked for information, I try to learn the person's intent motivating the question. Sometimes the requester doesn't actually know what to ask for. Instead of ``no'' I try to figure out how to enable the person to be successful. 

\subsection*{Tip: Leverage the other person's experience while focusing on your own\label{sec:advice}}

Advice without context is less effective.\\
\textit{Bad}: ``Here is what I think you should do in that situation.''\\
\textit{Better}: ``Here is what I did in that situation.''

People usually find talking about themselves an easy topic if you are curious about their experiences. 
If you can learn the other person's background and history and motivations, you can weave that into the advice you provide. 
Tailoring your message increases the likelihood of implementation. 

\subsection*{Tip: Avoid Platitudes\label{sec:platitudes}.}
% https://graphthinking.blogspot.com/2017/10/why-platitudes-are-used.html
\href{https://en.wikipedia.org/wiki/Platitude}{Platitudes} are \gls{thought terminating}; the statement feels true and is resistant to debate. Platitudes capture a feeling with sufficient accuracy, but with imprecise language. As a result, there's no specific action.

Because platitudes result in a conclusion, the conversation participants may feel more bonded. However, that bond is shallow.

Example platitudes to avoid:
% https://graphthinking.blogspot.com/2017/02/phrases-which-serve-as-thinking-stoppers.html
% https://graphthinking.blogspot.com/2017/10/a-list-of-platitudes.html
\begin{itemize}
    \item pick your battles
    \item Some things you can't explain
    \item Your time will come.
    \item You can be anything that you want to be
    \item I just want to get through this day
    \item It is what it is
    \item I'm just one person
    \item That's that
    \item Life's not perfect
    \item Life's not fair
    \item There's only so much you can do about it
    \item What is meant to be will be
    \item It is God's will
    \item It's part of God's plan
\end{itemize}

If your goal is to understand a concept or issue deeply, you need to use precise language.

\subsection*{Tip: Strive to use Precise Language}

Imprecise language causes miscommunication. Intent is unclear, as is expected consequence.

If you have a specific definition for a word central to the topic of interest, ask your new collaborator for their definition. Do not expect others to share your definition even when there are established norms for the topic. 

Instead of asking a collaborator, ``Are you taking action on this topic?'' ask ``What actions are you taking on this topic?''

If someone claims, ``We plan to get to that action,'' ask for a timeline. A deadline can be for an artifact or a re-evaluation of the topic.

In the short-term imprecise language takes less work to create and can take less time to convey. 

The importance of precise language is proportional to the potential consequences of action/inaction/wrong action.
Precision also should be proportional to the complexity of the topic being discussed. 

\subsection*{Tip: Word is bond\label{sec:word-is-bond}}

Your communication (verbal, written) is your reputation. People rely on what you tell them even if there isn't legal recourse. Reliance on your word is why precision matters. 

Frustration and disappointment follow when you don't uphold your word, or others misinterpret your imprecise language, or you are misunderstood.

Communication implies responsibility for the content.  There is a corresponding accountability in the relationship between speaker and listener (or writer and reader).

\subsection*{Tip: Take care near the boundaries of knowledge}

Trying to find someone else's extent of knowledge is tricky -- they don't want to appear stupid. They may interpret the exploration at a trap. ``I don't know'' can be an embarrassing statement to make, even if you don't share their embarrassment. 

Knowing the limitations of your own knowledge and disclosing those boundaries to others is critical. Distinguish what you know from speculation about things you don't. 

\subsection*{Tip: Listen all the way to the last word of the speaker}

Formulating a response to the first part of an idea or a sentence is tempting. Waiting for the speaker to finish before thinking of how to response is courteous. Waiting creates a pause which makes you seem more thoughtful. 

This is complicated by the speakers who include long pauses for contemplation and then resume. 

\subsection*{Tip: Eliminate speaking over other people.\label{sec:crosstalk}}
% https://graphthinking.blogspot.com/2017/10/crosstalk.html

Crosstalk occurs two people who are communicating verbally experience interference from another audible conversation. That can occur because a third person is talking at one of the original two participants, or when four or more people are holding two separate conversations concurrently. 

Crosstalk in a bureaucracy is motivated by
limited time available to communicate. A meeting participant may feel inspired by something someone else said and want to interject. 
%Typically manifests as popcorn style stories based on experience. 
%Intended as wisdom for self-validation by others in our community. 
Crosstalk can indicate engagement and enthusiasm, or it can be due to the speaker wanting to dominate the topic through interruption. The likelihood of crosstalk is dependent on the level of aggressiveness of participants.
In either case (enthusiasm or power-seeking), the original speakers are disrespected. The original speaker may feel annoyed at being interrupted.



%Crosstalk has four roles and a minimum of two people participating: the discussion facilitator, the original speaker, the interrupter, and other meeting participants. 
%During crosstalk, the discussion facilitator loses control of the interaction to the interrupter.  
The audience is frustrated by the lack of clarity of where to focus. This distraction causes participants to lose of focus and productivity of the interaction decreases.

Bystander intervention for out-of-control meeting: raise your hand. \marginpar{[Tag] Actionable Advice} This non-verbally reverts focus back to the discussion facilitator. 

\subsection*{Tip: Account for Warnock's dilemma}
% https://graphthinking.blogspot.com/2018/09/dealing-with-warnocks-dilemma-in.html
\href{https://en.wikipedia.org/wiki/Warnock\%27s_dilemma}{Warnock's dilemma}
is the common experience of figuring out how to interpret not getting feedback. This is especially vital in meetings where the speaker or facilitator needs to gauge participant comprehension of delivered content. Simply asking ``Does anyone not understand what I just described?" is likely to get no response from attendees because individuals want to avoid looking stupid.

\ \\
\textit{Technique}: Pick an individual to provide a recap.\\
\textit{Technique}: Survey the audience using multiple choice to gauge understanding.\marginpar{[Tag] Actionable Advice}

\subsection*{Tip: Seek action with a deadline}

% https://graphthinking.blogspot.com/2017/11/collected-wisdom.html
When asking someone for help or input, specify a deadline for their response. \marginpar{[Tag] Actionable Advice}This helps the person prioritize their tasks.

\subsection*{Tip: Identify the cause of miscommunication}

% https://graphthinking.blogspot.com/2019/06/miscommunication-versus-inability-to.html
\begin{itemize}
    \item miscommunication the cause is often due to definitions of words used or differences in context. In this situation, additional time spent communicating and taking different approaches is sufficient to remedy the issue.
\item A speaker may be inarticulate. If the speaker is unable to coherently convey their internal experience to a listener, then the communication failure is of a distinct category. No amount of additional communication will lead to improved understanding on the part of the listener.
\item A speaker simply has nothing to say about a subject. Regardless of whether they are capable of articulating a concept, they may be unable to relate to the topic. Often a person in this situation still wants to participate, but they are unable to meaningfully contribute. 
\end{itemize}
% https://graphthinking.blogspot.com/2019/05/identifying-empty-talk.html
Empty talk is the use of words that are ill-defined, emotionally resonant, inactionable, and impersonal.

\subsection*{Tip: ask-tell-ask}

Collaborating with fellow bureaucrats who have expertise in areas you do not requires extra work. There may be differences in the words used to describe certain situations, more precision in wording that you're used to, or thinking about situations in ways you are not familiar with. In that context, to bridge the differences you can ask, tell, ask\footnote{\href{https://cepc.ucsf.edu/sites/cepc.ucsf.edu/files/Curriculum_sample_14-0602.pdf}{``The 10 Building Blocks of Primary Care: `Ask Tell Ask' Sample Curriculum''} and \href{https://www.the-hospitalist.org/hospitalist/article/125126/qi-initiatives/ask-tell-ask-simple-technique-can-help-hospitalists}{Ask-Tell-Ask: Simple Technique}}. 

The first step is asking what the other person's perspective is on the topic. This helps establish the appropriate level of nuance is and can tell you how that person frames the issue. The second step is to tell the person what you want to say. The ``tell'' step should leverage what you learned from the first ``ask'' step. Use phrasing that is consistent with what you just learned from the other person. The third step is to ask the person what they heard from you. If they are unable to tell you, you may need to refine your delivery. To improve the likelihood of success keep the content in the second step short. 

The ask-tell-ask technique can be used iteratively in the same conversation, especially in discussion complex topics with a new collaborator. \marginpar{[Tag] Actionable Advice}Using ask-tell-ask takes long than just telling but increases the effectiveness of the communication. You also get to learn more about the other person's perspective. 


\subsection*{Tip: Initial responsiveness and status updates}
In a bureaucracy requiring approval, or soliciting input, sometimes waiting can provide value to the person doing the waiting. The request may be overcome by events, or the person asking may remind which indicates priority.

\subsection*{Tip: Make deadlines explicit}

Typically requests have two deadlines. The first deadline is when a response is sought. The second deadline is when a response is no longer useful.  

As an example, suppose I am inviting people to a meeting. I send the invitation 5 days prior to the meeting and I want to know who is able to attend by 3 days before the meeting. The second deadline is the time of the meeting. Replies after that second deadline do not help me understand who is going to attend. 

\subsection*{Tip: Read each email/memo/report to determine the purpose }
% https://graphthinking.blogspot.com/2021/03/read-each-email-to-determine-purpose.html

\textit{Problematic behavior}: scan the text of a message, see if there is immediate action or response needed. If no action or response is needed, go to the next email. \\
 That does not work for emails that contain logistics associated with future events. 

Instead, consider possible intentions of the person writing the email. 

\textbf{Decision needed}. Typically includes context. \\
\textit{Action}: if the team maintains a decision log, update that.
Response is selection of a choice.

Tip: Instead of asking for a decision, ask for if the person is opposed.

Tip: instead of asking for a decision, ask for the go-ahead. This framing biases the respondent towards action (specifically approval) rather than thinking. 

\textbf{Situational awareness}.\\
\textit{Action}: Expected default is no action, but interject if there's an issue.


\textbf{Action or Tasking}.\\
\textit{Action}: Do something within some deadline

\textbf{Approval sought}.\\
\textit{Action}: Confirm or deny

\textbf{Feedback sought}.\\
\textit{Action}: Assessment of proposal


\textbf{Meeting logistics}. Can be an announcement (widely available), registration (limited attendance), or invitation (specific to you). Attendance is optional or require. \\
\textit{Action}: Create or update a calendar event
Response should restate the logistics (time/date/location/purpose) to confirm. 

\textbf{Brainstorming}\\
May provoke a response for building on an idea.
``For your situational awareness, no action needed." Notification of activity by someone else. Or change in plans. 
If needed, a correction to the described direction might trigger a response or even a meeting.

\textbf{Reference} e.g. describing a process or business workflow. Or a citation.\\
\textit{Action}: Copy process documentation to wiki. Copy citation to bibliography.
Acknowledgement response thanking the sender for the update/clarification.

\textbf{Setting a formal policy or issuing an informal edict}\\
\textit{Action}: move the policy/edict documentation to Confluence or Wiki
Acknowledgement response needed only if the edit is aimed at just me or the group I am leading

\textbf{Question}\\
If this is a recurring question, move to a ``Frequently Asked Questions" page on Confluence or Wiki.
Response needed that provides answer or seeks clarification.


Here I'm using ``action" to refer to activities outside the email channel. 

If I read email to figure out the purpose of the email, that will help me determine what action and response are relevant. 

Whether I am the only recipient or on of many receivers can change the intent of the email, and whether I'm in the ``to" or ``cc" field matters. Unfortunately, ``to" versus ``cc" are not reliable indicators since email senders do not reliably conform to the expected use. 



%This categorization of text within emails is a useful natural language processing challenge for machine learning. Currently a few email providers already do some of this with identifying meeting logistics, providing reminders to follow-up, and providing reply snippets. A browser plug-in that differentiates the various purposes of text could help readers determine relevant actions and responses. 

An email sent to multiple recipients may have different purposes for different readers. The reader's role or knowledge may factor into how they interpret the content. The inclusion or exclusion of recipients alters how the content is understood. 

\subsection*{Tip: Don't seek attribution for contributions; credit others\label{sec:credit-others}}

Give credit to others for good ideas and beneficial actions. Either they accept credit and you are seen as a contributor to their success, or they push back and you look generous. Credit is not a \href{https://en.wikipedia.org/wiki/Zero-sum_game}{zero sum game}.

\subsection*{Tip: Offer to take blame\label{sec:take-blame}}

Before an action commences, tell collaborators that you are willing to accept blame if something goes wrong. This alleviates their fear of risks.

\subsection*{Tip: Survey stakeholders}
% https://graphthinking.blogspot.com/2016/01/how-to-solve-and-not-solve-problems.html

Suppose you are a \href{http://www.peacecorps.gov/}{Peace Corps} worker in Africa. You show up and the village doesn't have easy access to clean water. Villagers walk a long ways in dangerous areas for dirty, unsafe water. This is a very obvious problem and all the villagers agree that they don't have good water and that this problem should be fixed.

Implementing the solution would take about a week - get the equipment to the village, drill a well, build a pump.

You could take additional time and involve the villagers in this project. They could participate in getting the equipment, which should lead to a sense of ownership.
But then when the equipment shows up, they don't take action to drill the well. If the well is drilled, it soon falls into disrepair and the villagers are back to doing things they way they used to. What happened?

The villagers don't see access to clean water as the most significant issue. You came in and imposed your view of what the problem is and how to fix it. When you impose your view of what the problem is, the solution won't be adopted by villagers because they don't prioritize it.
It is better to survey the community to see how they operate. What do they think the problems are?
Both leadership and the community members need to provide priorities.

This issue is exacerbated if you come to the village as a representative of a company providing wells. You are biased when you ask, ``Do you have any problems?"

Of course the villagers have water problems which could be fixed with better wells. However, when you get into the details of placing or improving a well, they lose interest. What the community really wants is free installation, zero maintenance, easy to use, and no operational costs. That would improve their life.

When you say there's cost (both initial investment of capital and then operations/maintenance) and a learning curve associated with the solution, then the user's interest wanes -- you are presenting another cost/benefit ratio for them to evaluate. Then they ask ``Can we get by without the well?" Yes, they don't need the well -- they've survived without it.

Novel solutions (in this example, drilling a well and installing a pump) have have barriers to adoption. Two barriers are the current priorities of the community and the incumbent solution/processes.

If there are problems with higher priority, the community will delay implementing your solution. That's fine if the higher-ranked priorities are bounded, but they are often not. An example of this is the following:
Suppose a person has three tasks, and you introduce a solution which is a fourth task.
If the first task is ``go from point A to point B," then that task will eventually be eliminated and there will be three remaining.
If the second task is ``secure your village," that is an unbounded task. The person won't get to or won't prioritize your low-ranked task.

How will your solution impact their higher-ranked priorities?

\ \\

\subsection*{Email Structure\label{sec:email-structure}}

% from https://graphthinking.blogspot.com/2021/10/structuring-email-content-for.html

If implementing all these tips sounds like a lot of work, that's because it is. Effective written communication requires intentional effort because of the lack of augmenting channels (compared to voice or video or in-person). 



Use consistent design and structure for your emails. Emails are part of your professional reputation.

Emails start with a greeting: Hi, Hello, Good morning, Good afternoon, Good evening. 
Email greetings include the name of the targeted recipient(s). 

Emails terminate with a professional closing, e.g., ``Kindly", ``Regards", etc

Emails contain a signature block with contact information -- phone number, normal hours of response, which timezone you're in if your team spans timezones, how long to wait for a response before asking again, which communication channel I prefer, etc.

\begin{figure}
\includegraphics[width=1\textwidth]{images/email_template.pdf}
\caption{Template for new email messages. Greeting has a space after the comma -- that is where the recipient's name will go. Signature block uses smaller after the name.}
\label{fig:email_template}
\end{figure}

Email signature blocks do not include unnecessary images, as that uses more storage for recipients. 
Email threads focused on a specific instance of a recurring event include the date (YYYY-MM-DD) in the subject line. 

Based on the purpose of the email, example key phrases for subject lines include: ``meeting notes" versus ``agenda" versus ``question about".

Revising an existing subject line can disrupt the ability of email software to thread conversations. However, sometimes the revision is worth breaking threading.

When replying to an ongoing thread, retain the original message as part of the thread to provide readers historical context.

When replying to threads with sensitive messages, sanitize the included content as appropriate by removing name or identifying details.

If an email contains multiple requests or questions, at the top of the email (after the greeting) explicitly say how many of each type. Then, in the body of the message, number them.

\begin{figure}
\includegraphics[width=1\textwidth]{images/email_two_questions.pdf}
\caption{Distinct items the recipient should address in a reply.}
\label{fig:email_two_questions}
\end{figure}

If an item corresponds to a requested action, separately highlight the action and indicate who is supposed to take the action and what the deadline for response is.

\begin{figure}
\includegraphics[width=1\textwidth]{images/email_meeting_notes.pdf}
\caption{Who has what action due when?}
\label{fig:email_meeting_notes}
\end{figure}

Computer commands should be distinct separate fixed-width font. This distinguishes the text from the rest of the narrative. 


\begin{figure}
\includegraphics[width=1\textwidth]{images/email_computer_font.pdf}
\caption{The computer commands use fixed width font. The author distinguished input from output through the use of bold vs non-bold respectively. The author highlighted the error message using red. Inline text like ``cat'' in the last line is also fixed width.}
\label{fig:email_computer_font}
\end{figure}

References to documents include a direct full path.

If referring to a previous separate thread, include the subject and the date+time that email was sent

For bullet points, explicitly specify that items are joined by one of the following: OR, XOR, AND

If you have an unordered list, explicitly state that order is irrelevant.

If you have a sequence of steps, number them appropriately and indicate which steps are required versus optional

Use visual sketches to illustrate concepts rather than always relying on text. Don't use pictures all the time, and don't have too many pictures in an email. 

Know how to both embed pictures inline and how to attach files and when to use which. 
Email replies should preferentially be at the top of the thread. 
If replying to multiple points in the previous email, embed replies inline, mark the distinction, and highlight the authorship. 

\begin{figure}
\includegraphics[width=1\textwidth]{images/email_reply.pdf}
\caption{Bob's reply to Sue's questions. The third question is not shown in this illustration.}
\label{fig:email_reply}
\end{figure}

If replying inline, explicitly state that at the top of the thread.

The email trilemma is to balance the amount of detail against providing sufficient context and being concise. 

If the email is longer than a paragraph, provide a \href{https://en.wikipedia.org/wiki/BLUF_(communication)}{B.L.U.F} or \href{https://en.wikipedia.org/wiki/Wikipedia:Too_long;_didn\%27t_read}{tl;dr} or summary. In general emails should be short. Longer discussions should be held on the phone or in person, with a summary report after the discussion. Reliance on a BLUF or tl;dr risks resulting in the reader skipping the content. 

Emails convey both emotional tone and facts. Your intent is practically irrelevant; the reader's perception is paramount. 

Every email should have a purpose. What are you asking the recipient to do? How do you want them to feel? How should they respond?

When replying, starting your email with an expression of gratitude for the work the recipient has done so far sets a positive tone by acknowledging their investment.

\ \\

\subsection*{Summary of what action should be carried out} 

As the outsider, you should help the community enumerate and document all of the problems they identify. Then you can help enumerate and document how the problems are related (dependencies). Only then can you help the community identify and document the root causes.

If the solution you, the outsider, identified really is the root cause, then the community will arrive at that independently. If that is the case, then you can enable them to implement a solution which addresses the root causes. The community will then have a sense of ownership.
 % subsection 
    \subsection*{Friction between teams within an organization}

Ideally there is a clear division of responsibilities among teams. Even in that context there is necessarily some interaction among teams -- one team may depend on the output from another team. Coordination among the teams regarding transfer of data or products or projects or knowledge is critical to the smooth operation of the organization. 

An organization has finite staffing, money, time. Therefore, teams within the organization face a \href{https://en.wikipedia.org/wiki/Zero-sum_game}{zero-sum}
\index{Wikipedia!\href{https://en.wikipedia.org/wiki/Zero-sum_game}{zero-sum game}}
distribution of resources.

Difference of decision making perspective based on local conditions, stupidity, or different incentives, different definitions of success

When attempting to resolve friction between teams, there is an authority common to the teams, but that person lacks the nuanced insight, doesn't have time to get involved in every challenge, and doesn't want to micromanage multiple teams.

    %There are many books on how to do meetings well. 
    % How is this section (in a book about bureaucracy) distinct?
    % how does understanding bureaucracy matter in the context of a meeting? what's the change in behavior for the effective bureaucrat?
    Stakeholders care about the outcome of a decision. 
    In some sense, every participant in an organization is invested the outcome of every decision made because there is consequence to how resources are allocated and the direction of the organization. Every org member has an opinion, even lacking experience or expertise. 
    
    \subsection{Well-run meeting\label{well-run_meeting}}

Identify essential attendees. If someone does not need to be present, notify them in advance that you will share the meeting notes afterwards. 


Bad: no meeting agenda\\
Good: agenda\\
Better: agenda share with other participants
For formal meetings, share agenda in writing prior to meeting. 

TODO: Why an agenda matters in a bureaucracy: 

TODO: forces conspiring against agendas


For formal in-person meetings, Verify meeting venue has sufficient space, seating, working IT equipment

For formal virtual meetings, ensure participants are familiar with virtual meeting controls

TODO: why logistics/infrastructure matter in a bureaucracy:

TODO: forces conspiring against logistics/infrastructure
    \subsection{Characterizing Meetings}
% relevance of this section:
Characterizing meetings is critical to distinguishing which norms are applicable, and what people expect from the different formats. 

types of meetings: internal meetings, customer meetings, conferences, scheduled one-on-ones, impromptu walk-around,  

% https://graphthinking.blogspot.com/2019/12/what-is-purpose-of-this-meeting.html
Many purpose to a meeting
\begin{itemize}
    \item To gather input from attendees
    \item To make a pronouncement to attendees
    \item To educate
    \item To brainstorm ideas
    \item To make progress towards an objective
\end{itemize}
When the purpose is not explicitly stated, confusion arises.

When multiple purposes occur in one meeting and the transition is not explicitly stated, confusion arises.
The reason for this confusion is that the assumptions and expectations and norms of each purpose are different. When the attendees don't know the purpose or the purpose shifts, the behaviors and roles are in question.

% https://graphthinking.blogspot.com/2014/12/how-to-understand-meetings-at-work.html
Level of formality, start time (early | on time | late), 
end time (early | on time | late), utility, 
duration, number of attendees, number of speakers, number of participants.


What is the purpose of a meeting?

meetings involve people, either known or strangers
meetings involve information, either relevant or irrelevant. Relevant information is either new or related to previous work
meetings either have a leader or no leader (brainstorming). If there's a leader, the leader may be disseminating info to participants, or gathering information from attendees
%    \subsection{Make Effective Presentations\label{sec:presentations}}

% https://graphthinking.blogspot.com/2011/10/presentation-notes.html

By paying attention to a bad presentation, you can find problems that you do not want to repeat.

Speaking is vital at decision points in your career progress - at interviews, competitions, gaining new collaborators at conf, impressing peers/boss/students. Thus, it's best not to make these mistakes in those situations.

Aspects to a presentation
\begin{itemize}
    \item speaker: appearance, verbal accent, enunciation, smell
    \item slides: layout, content
    \item audience: background
    \item venue: size, technology available, lighting
    \item lead time: how much time do you have to prepare?
\end{itemize}
The following list is meant to be reviewed as a checklist.

Planning your presentation

In introducing your research, there are a few ways to open the talk
\begin{itemize}
    \item compliment the audience
    \item humor: relevant joke
    \item how does this work make me feel
    \item explain context and relevance in terms of money, number of people involved, size of system
\end{itemize}
Estimate the experience/education background of your audience prior to creating the presentation. The question being answered in the presentation is independent of audience, but the level of delivery depends on audience background.


Speaking to an audience outside your field:
\begin{itemize}
    \item Use jargon your audience is familiar with. For example, a physicist uses "quantization of energy" whereas an audience of mathematicians may be more comfortable with "discrete energy levels."
    \item Look for commonality. For example, both physicists and mathematicians use assumptions and build models.
\end{itemize}
Visual presentation:
\begin{itemize}
    \item Switching between dark and light slides in a dark room stresses the eyes (need time to adjust to varying light levels)
    \item In figures, use both color contrast and distinct symbols, in order to deal with colorblind audiences. See also http://jfly.iam.u-tokyo.ac.jp/color/
    \item making slides appear "professional" means adding non-informational content. This added content should be consistent, not distracting
\end{itemize}
Software:
Common choices include Latex (beamer), Microsoft Powerpoint. Less well-known are Prezi (and its open source equivalent, InkScape+Sozi add-on).

If you use Microsoft Word, a document PDF, notepad, or any other non-presentation software to make a presentation, then you are sending a few messages to your audience:
The audience isn't worth your time needed to develop or learn a proper presentation
You are not technically savvy

Presenting multiple topics within one presentation:
Disparate topics require a segue to show why you are transitioning
Inter-relate the multiple topics 
Dealing with technical failure
If you plan to give a live demo, have screenshots of the process in the presentation. That way, if the demo fails you can show what was supposed to happen in the slides. If the demo works, skip the slides.
If possible, use the setup as close to reality as possible for practice sessions. Project onto the screen using the projector. This will show if the color contrast is sufficient.

Prior to giving a talk at a remote (or unfamiliar) environment

Questions to ask your host:

Will I have a projector and screen for the presentation?

If Power point available, what version is in use?

Will I be allowed to bring a USB drive and/or laptop into the location (security restrictions), or should I email the pptx file to you?

Day of talk

Appearance (suit and tie or jeans and t-shirt?):
\begin{itemize}
    \item under-dressed = I don't respect the audience
    \item overdressed = I'm better than you
    \item similar level of dress = I'm a peer
\end{itemize}
Some of these suggestions may seem glaringly obvious. The reason they are here is because I have personally seen them in "official" presentations given by a "professional"
\begin{itemize}
    \item Do not curse (profanity while speaking, or in the slide presentation, or even the name of the file)
    \item Practice the presentation (out loud in real time) at least once
    \item Run spell check before presenting
    \item Translate your slides from your native language to the language of your audience
    \item Tell a coherent story with a unified theme. Each slide should be logically connected to the following slide. Don't just put a bunch of slides with data together. You risk disorienting your audience.
\end{itemize}
Before the talk begins
\begin{itemize}
    \item Make sure your facility has power, a screen, a projector, pointer, any other necessary equipment. [Your host may not think of these things for you.]
    \item Make sure all equipment works and functions together
    \item Before the presentation begins, use a slide to make announcements and reminders:
\begin{itemize}
    \item list agenda if there are multiple speakers
    \item time talk begins, how long it will last
    \item Reminder: Turn OFF Cell Phones
\end{itemize}
    \item Announce whether to ask questions during talk (interrupt) or to hold questions until done
\end{itemize}
During the talk
\begin{itemize}
    \item Opening: Thank hosts/inviters/organizers. Establish a connection between you (the speaker) and the audience.
    \item Do not pace
    \item Do not stand frozen
    \item If you are the only person laughing at your joke, it isn't funny
\end{itemize}
 
After the talk
\begin{itemize}
    \item Let the person asking the question finish
    \item Restate the question to ensure the audience heard it and that you understood it
    \item Give the minimal answer. Save the long answer for followup
\end{itemize}

    % how to engage your audience
    % stages of dysfunction
%    % https://graphthinking.blogspot.com/2021/02/organizations-value-things-more-than.html
\section{meeting time compared to theft}

In large organizations, there can be significant bureaucracy associated with even small purchases. A multi-step review process may be incurred for a \$2000 acquisition.

Another measurement of value is that if an employee were to steal even \$200 worth of materials, the organization would likely punish that employee.


Those metrics apply to tangible goods, but not to people's time. Consider a meeting of 10 people and each person's cost is \$200 per hour. A wasted meeting is not unusual and certainly would not incur bureaucratic review processes. The cost to the organization is fiscally the same -- \$2000. Similarly, consider an employee who is late and causes a loss of productivity. Merely depriving the organization of \$200 worth of time is not punished in the same way theft is.

In fact, organizations default to meetings (even recurring meetings) rather than not meet. And being late to a meeting is accepted. 

We can debate the differences between theft of materials and theft of time. The financial argument is clear. 

Source: Andy Grove in "High Output Management"
    
    \subsection{One-on-one check-in meetings}

% https://graphthinking.blogspot.com/2021/05/the-agenda-for-one-on-one-meeting.html

One-on-one meeting questions for helping the manager understand the team member's status.
\begin{itemize}
    \item what are the objectives for the team?
    \item what have you been successful with since we last met?
    \item what is blocking our team's progress?
    \item what are your plans?
    \item how are you collaborating with the rest of the team?
\end{itemize}

Reflective prompts for one-on-one meetings:
\begin{itemize}
    \item If there was just one thing you could change about our organization, what would it be and why?
    \item How do you plan to train your coworkers on topics you understand and they don't?
    \item What have you learned in the past month?
    \item What are the biggest risks for the team?
    \item What's limiting your productivity?
\end{itemize}
Responding to these questions takes time (an hour) and willingness to be open. 

\ \\

The one-on-one check-in should be tailored to the phase of the employee's progression. 
\begin{itemize}
    \item new team member, either new to the team or new to the company. Here the focus of the one-on-one is to ensure a smooth on-boarding process. Does the employee have the necessary computer log-in accounts? Do they have an email account? Are they on the mailing list?\\
\textit{The duration of this phase could last between a day and two weeks.}
    \item team member is responsible for small tasks: the one-one-one is for discussions on training and sprint planning and sprint-reviews. Characterized by the team member being dependent on others for their success. In this phase the employee collaborates on tasks.
\textit{The duration of this phase could last a few months to years.}
    \item team member is responsible for large tasks (which get broken into subtasks): the one-on-one is to help the team member define their success. Activities include planning, resource allocation, assessment. Characterized by the need to coordinate with others on the team or other teams.
\textit{The duration of this phase could could be the rest of a career.}
    \item facilitating the productivity of others: rather than being task-oriented, this team member supports coworkers. 
    \item peer check-in: this one-on-one is a form of mentorship. The value of the exchange is to get a different perspective and to hold each other accountable.
\end{itemize}

TODO: How does the team member and the supervisor know when the next phase is appropriate?

TODO: What are the thresholds for change?

\ \\

https://news.ycombinator.com/item?id=30152268

\ \\

https://news.ycombinator.com/item?id=22341138
https://github.com/VGraupera/1on1-questions


%    \subsection{How to be a Successful Conference Attendee}
% https://graphthinking.blogspot.com/2012/02/how-to-succeed-as-attendee-at.html
%    \subsection{How to be a Successful Conference Organizer}

% https://graphthinking.blogspot.com/2011/11/how-to-organize-conference.html

\chapter{Last chapter}
\chapter{Conclusion\label{sec:last-chapter}}

You could operate within a bureaucracy and entirely focus on just doing the job you were hired for and ignore administrative distractions. Or you can be more effective in your job by understanding the role of engaging with other people -- peer bureaucrats, supervisors, and subjects. 

Learning bureaucracy as a skill doesn't mean you can ignore personalities of individuals. Bureaucracy as a skill separate from and in addition to being a good person, being an effective member of a team, being a good project manager, being a good product owner, skillful writing, excellent verbal communication skills, applying technical skills, etc. The distinction from those aspects is that as a bureaucrat you understand the complications and constraints of your environment and then can more effectively operate within those conditions.

If you are new to being a bureaucrat, then this book armed you with understanding your environment. On your first day of employment you are poised to frame incoming information constructively.
If you are an experienced bureaucrat, it is not too late to improve. Even on your last day of employment you can learn and be more effective.

In this book I described bureaucracy and provided options for action. If you are able to apply these generalized perspectives to your specific situation, you are applying the paradigm developed throughout this book. Concepts like learning the history of your situation, identifying and engaging stakeholders to learn their perspective. Brainstorming the incentives of the individuals involved, and listing what levers they have for action. What are the dilemmas? Are there feedback loops?


\section{Skills of an Effective Bureaucrat}

The narrow scope of your role (which hopefully leverages your education/training) does not capture all relevant aspects of your job. 
Based on the definition of bureaucracy (distributed knowledge and distributed decision making), the critical aspects of success are having knowledge, sharing your knowledge, leveraging the knowledge of others, effective communication, working well with other people, understanding the role of both processes and social influence, and how those interplay. 

Because bureaucracy is a system of distributed knowledge and distributed decision making, coordination is critical. 
\begin{itemize}
    \item You are able to facilitate meetings. This means agenda, providing rules on interaction (raising hands), taking notes, sharing notes, and following up after the meeting.
    \item You participate in meetings, whether that means actively contributing or intentionally supporting other attendees. You leverage relationships with other attendees. 
    \item You focus your educational investment in written communication (emails, text-based chats, reports). Your writing empathizes with readers, captures relevant context, is concise, and is clearly worded.
    \item You communicate verbally concisely and precisely. You listen and you teach. You seek shared definitions. You are professionally vulnerable when appropriate. 
    \item If in-person with colleagues, you walk around and talk with people one-on-one.  
    \item In your role as bureaucrat you leverage project management skills: you have a vision, you make and share plans, all while building consensus with stakeholders.
    \item You apply your negotiation skills~\cite{1982_Cohen} that improve your interactions and outcomes.
\end{itemize}

The attitude of the effective bureaucrat  is that of realistic optimism. A realistic optimist will occasionally be wrong but makes progress, whereas a pessimist will be right (because the prediction is self-fulfilling) but not make progress.


% If bureaucracy is a distributed knowledge distributed decision system, compare with paxos and Byzantine generals
% https://en.wikipedia.org/wiki/Paxos_(computer_science)
%https://blog.the-pans.com/understanding-paxos/
%https://martinfowler.com/articles/patterns-of-distributed-systems/paxos.html

% https://en.wikipedia.org/wiki/Complexity_theory_and_organizations


\appendix

\chapter{Assessments for Publication}
\section{Reader's Report\label{sec:reader_report}}
An view from inside bureaucracy as a \href{https://en.wikipedia.org/wiki/Complexity_theory_and_organizations}{complex adaptive system}. 

Central claims:
\begin{itemize}
    \item Everyone in modern society participates in bureaucracy. Therefore learning to be a skilled bureaucrat is useful.
    \item Individuals already have experience with bureaucracy from conventional roles in modern society. 
    \item Emergent behavior and being a \href{https://en.wikipedia.org/wiki/Wicked_problem}{wicked problem} make the complexity of bureaucracy irreducible to a simplistic model.
    \item The techniques of bureaucracy are meetings, processes, communication. These facilitate the coordination of distributed knowledge for distributed decision making.  
    \item Bureaucracy occurs when there is a lack of common quantitative feedback mechanism for individuals.
\end{itemize}
The consequences of thinking like a bureaucrat include
\begin{itemize}
    \item You are not limited to only the direct personal interactions that you have with other people. Bureaucratic processes that exceed your direct visibility and experience extend your influence.
    \item Help you recognize options beyond the naive defaults associated with fallacies in section~\ref{sec:fallacies}. What's feasible? What's negotiable? Once you realize rules and processes are subjectively created and enforced, negotiation (and identifying who to negotiate with) is more obvious.
    \item Decreased surprise when thing is not the way you naively expect. See the dilemmas in section~\ref{sec:dilemma_trilemma} and known hazards in section~\ref{sec:unavoidable_hazards}.
\end{itemize}


The effects of bureaucracy are not attributable to one person, but each person can improve bureaucracy.\newpage % section
\section{Audience Analysis}

Primary audience for this book is self-identified novice bureaucrats and people planning to work in a bureaucracy. Value of this book is advice beyond ``be a good person'' and specific to the role of a bureaucrat (and not project manager/team lead/software developer). 

Secondary audience is white collar workers who have experience working in a bureaucracy and are just realizing they are bureaucrats. This audience will find some of the information to be already known.  The value for this audience is a re-framing of the environment and problems. 

Tertiary audience is non-bureaucrats who want to better understand why bureaucratic systems are challenging. Americans consider themselves "individuals" and neglect the necessary integration of operating within a society. This book won't be a popular best seller, but it is intended for a lay audience and does not assume prior academic exposure to the topic.

Lastly, researchers of bureaucracy. This book is not a theoretical systemic analysis. \newpage  % section

\chapter{Book notes}
\section{``Dynamics of Bureaucracy'' by Blau}

\cite{1955_Blau}

Intended audience:

Ben Payne has read this book: in progress\\
Ben Payne has a copy: yes\\
Ben Payne's assessment:

\newpage
\section{``Values of Bureaucracy'' by Du~Gay\label{review:dugay_values}}

\cite{2005_DuGay}

Intended audience:

Ben Payne has read this book: no\\
Ben Payne has a copy: yes, electronic\\
Ben Payne's assessment:


 \newpage
\input{book_notes/farazmand_handbook_of_bureaucracy}\newpage
\input{book_notes/graeber_utopia_of_rules}\newpage
\input{book_notes/hupe_research_handbook_on_street-level_bureaucracy}\newpage
\section{``Street-Level Bureaucracy'' by Lipsky\label{review:lipsky_street}}

\cite{1983_Lipsky}

Intended audience:

author has read this book: yes\\
author has a copy: yes, physical\\
author's assessment:


pages 192 to the end.

Roles: customer, bureaucrat, boss of the bureaucrat

\subsection{Summary of claims}
From \cite{2015_Cooper}
\begin{quote}
Street-level bureaucracy (SLB) is a sociological theory that seeks to explain the working practices and beliefs of front-line workers in public services and the ways in which they enact public policy in their routine work. Developed by an American, Michael Lipsky,1,2 it examines the workplace in terms of systematic and practical dilemmas that must be overcome by employees, with a particular focus on public services such as welfare, policing, and education. The theory is based on the notion that public services represent ‘the coal mines of welfare where the “hard, dirty and dangerous work” of the state’ is done.’3 According to Lipsky,1,2 that is because:

\begin{itemize}
\item demand from clients will always outstrip supply due to finite resources (cost, time, or service access). Most clients are unable to obtain similar services elsewhere (such as private alternatives to state organisations). As a result, employees must resort to ‘mass processing’2 of excessive client caseloads.

\item extensive personal discretion is a critical component of the work of many front-line public sector employees, particularly those who undertake private, face-to-face interaction with clients to assess the credibility of cases. Employees must use their personal discretion to become ‘inventive strategists’ by developing ways of working to resolve excessive workload, complex cases, and ambiguous performance targets.4

    \item employees compromise the quality of their work by ‘creaming off’2 cases that are likely to be straightforward or to have a positive outcome. Alternatively, workers may act as an ‘advocate’2 for clients who are perceived as being at the tip of an iceberg of social vulnerability. Because workers are unable to offer all services to every individual they may be forced to ‘deny the basic humanity’2 of other clients. These pragmatic micro choices ultimately become the de facto policy of the organisation, which may contrast starkly with its official stated aims.
\end{itemize}
This theory has implications not just for the individual employee but also the overall system. In particular, Lipsky suggests that the extensive unmet demand from clients means that even substantial expansion of staff and budgets are unlikely to decrease workload pressures. Instead, he predicted that increased capacity would result in ongoing expansion of the same level of service quality at a higher volume.
\end{quote}


\subsection{Summary of remedies}
% https://graphthinking.blogspot.com/2018/08/how-to-decrease-bureacracy.html
In isolation, none of these ideas should be surprising.
\begin{itemize}
    \item automate recurring decision processes. This decreases bureaucracy by removing subjective influence of bureaucrats
\item where automation is infeasible, make customer advocates with end-to-end authority available
This decreases bureaucracy by improving customer's navigation of processes
\item make processes transparent to participants
This decreases bureaucracy by improving customer's understanding of processes
\item make information discoverable (e.g., via search engine) 
This decreases bureaucracy by 
\item make information directly available, rather than mediated by a person
\item after an interaction is completed, summarize the steps and outcome for the participant
\item when a process fails the needs of a participant, investigate the failure and improve the process
\item make the goals and priorities of the organization clear to all
\item define measurable standards of performance, both for individuals and teams
\item train bureaucrats how to engage participants effectively; these interactions determine the culture
\item train bureaucrats by addressing their immediate problems (e.g., through mentorship)
\item make employment desirable to people who have desirable characteristics (e.g., educated candidates)
\item enhance accountability to peers (e.g., peer review of actions and outcomes)
\item ensure that incentives for organizations and individuals encourage improvement rather than maintenance of status quo
\item decision making should be pushed down the hierarchy to the practitioner
\item when decision making requires cross-organization interaction, form a team of practitioners
\item bosses should share workload with their team in order to gain practical exposure to current challenges
\item seek feedback from process participants; then provide status updates on the implementation process
\end{itemize}
\newpage
\input{book_notes/ohearn_bureaucrats_handbook}\newpage
\section{``Bureaucracy: A Key Idea for Business and Society'' by Vine\label{review:vine_key}}

\cite{2020_Vine}

Intended audience:

Ben Payne has read this book: no\\
Ben Payne has a copy: no\\
Ben Payne's assessment:
\newpage
\section{``Bureaucracy'' by von~Mises\label{review:vonMises_bur}}

\cite{1996_Mises}

Intended audience:

Ben Payne has read this book: no\\
Ben Payne has a copy: yes, electronic\\
Ben Payne's assessment:
\newpage
\section{``Bureaucracy'' by Wilson}

\cite{1991_Wilson}

Intended audience: researchers

Ben Payne has read this book: no\\
Ben Payne has a copy: yes, physical\\
Ben Payne's assessment: Written from the perspective of an outside. Within that limitation it has useful analysis, though there are parts I disagree with. Well-written and easy to read.


Wilson defines ``operators'' as the street-level bureaucrats [Page 33].

Agencies typically have (ambiguous) goals, which are separated (subjectively) into tasks. [page 34]

Incentives (rewards and penalties) matter more than attitude.
[page 51]

Agencies operate under constraints set by Congress; businesses have more freedom to respond to clients.

What distinguishes business bureaucracy from government bureaucracy are feedback loops and self-determination of scope. -- From page 115.

Agencies are ``production" (ch8), or procedural/craft, or coping.
[page 245]

Procedural agencies have ambiguous outputs; Craft agencies have invisible operations
[page 250].

Coping or procedural agencies can discuss their activities but cannot verify their achievements
[page 252].

In chapter 5, agency environments were classified into four categories: majoritarian, entrepreneurial, clientist, and interest group.
[page 248].



\clearpage

\printglossaries

\nocite{*} % causes LaTeX to include every entry in your .bib file.

% http://www-math.ucdenver.edu/~billups/courses/guides/annotated_bibliography.html
\bibliographystyle{plain-annote}
\bibliography{biblio_bureaucracy}%,biblio_meetings}

\end{document}
