\documentclass{book}

\usepackage{graphicx}

\usepackage{hyphenat} % http://www.ctex.org/documents/packages/special/hyphenat.pdf

\usepackage{setspace}\onehalfspacing\frenchspacing\flushbottom\sloppy

% https://www.scivision.dev/include-svg-vector-latex/
%\usepackage{svg}
% see https://tex.stackexchange.com/questions/442077/is-it-possible-to-use-svg-images-with-overleaf

% https://tex.stackexchange.com/a/8459/235813
\usepackage[nottoc]{tocbibind}

\usepackage{hyperref}
\hypersetup{
    colorlinks=true,
    linkcolor=blue,
    filecolor=magenta,      
    urlcolor=cyan,
    pdftitle={Overleaf Example},
    pdfpagemode=FullScreen,
    }

% https://en.wikibooks.org/wiki/LaTeX/Glossary says
% "\usepackage{glossaries} and \makeglossaries in your preamble (after \usepackage{hyperref} if present)"

% https://www.overleaf.com/learn/latex/Glossaries
\usepackage[toc]{glossaries}

\makeglossaries % The command must be before the first glossary entry.

% https://en.wikibooks.org/wiki/LaTeX/Glossary says
% "define any number of \newglossaryentry and \newacronym glossary and acronym entries in your preamble"
% see https://en.wikibooks.org/wiki/LaTeX/Glossary


\newglossaryentry{bureaucracy}{
    name=bureaucracy,
    description={definition here}
}

\newglossaryentry{visible bureaucracy}{
    name=visible bureaucracy,
    description={procedures and processes are written down and can be discovered by stakeholders}
}
\newglossaryentry{invisible bureaucracy}{
    name=invisible bureaucracy,
    description={procedures and processes are known to some stakeholders and are conveyed verbally to some of the other stakeholders.}
}

\newglossaryentry{process}{
name=process,
description={a task broken into a specified set of subtask dependencies.}
}

% https://graphthinking.blogspot.com/2021/07/bureaucracy-book-outline.html
\newglossaryentry{bureaucrat}{
    name=bureaucrat,
    description={a person responsible for subjective implementation of someone else's intent, with unquantifiable results. Examples of a bureaucrat role: teacher, police, government employee. Not bureaucrats: factory line worker, student}
}

\title{How to be an Effective Bureaucrat\\
A Guidebook for Everyone}
\author{Ben Payne}
\date{\today}

\begin{document}

\begin{titlepage}
\maketitle
\thispagestyle{empty}
\end{titlepage}
\newpage

%\thispagestyle{empty}
\frontmatter % the front of the book has roman numerals

%\pagenumbering{gobble}
\thispagestyle{empty}

Copyright \copyright 2022 Ben Payne

Creative Commons \href{https://creativecommons.org/licenses/by-nc-nd/4.0/}{Attribution-NonCommercial-NoDerivs}

CC BY-NC-ND
\clearpage
\thispagestyle{empty}

Thank you to my coworkers. Our interactions helped me learn how to be a better bureaucrat.%\clearpage
%\pagenumbering{roman}

\chapter*{Foreword}% * excludes from Contents)
When a person has a positive experience engaging with bureaucracy, positive attribution is made to the people involved. Or ease of a solution makes the bureaucracy less visible and the solution seems obvious. 

When a person has a negative experience with bureaucracy, complaints are about the incompetence of the people involved, or the incomprehensibleness of the system. Don't these bureaucrats know how to do their job? Why isn't the solution obvious? Why does this system not work for me?

% Who this book is for

% from https://graphthinking.blogspot.com/2021/07/bureaucracy-book-outline.html
This book is for you if you are curious about bureaucracies, or you are thinking about working as a bureaucrat, or you are employed as a bureaucrat, or your job is shifting to be more bureaucratic. If you don't think of yourself as a bureaucrat, or if the term bureaucrat has negative connotations, I hope to change your mind on this vital topic. 


% What you should expect reading this book: 
The purpose of this book is to decrease surprise and arm you (both emotionally and intellectually) for the toil of being a bureaucrat. 

This book does not have a narrow focus on one topic like leadership, managing a team, being a team member, planning, time management, project management, advancing your career, or self-improvement. Some lessons may apply in those domains.

% What is the benefit of reading this book?
As a result of reading this book, you will be better able to recognize and navigate complex professional environments, both within your career and outside of work. The perspectives offered in this book can benefit you directly, whether by promotion of title or increase in pay; successful completion of a project; or through decreased stress of understanding how the world works.

There's harm in not recognizing yourself as a bureaucrat, as the role and responsibilities are distinct

Automation and computers will not eliminate or decrease bureaucracy. They merely obfuscate the processes and make negotiation more challenging. 

% my experience
% I wrote this book for a younger version of me.
 I was sufficiently self-aware when I first started my job in a large organization to recognize I didn't know much about working in that environment. Over the years I learned from my mistakes by reflecting on my (in)actions and the consequences. This approach has been an expensive education: mistakes delay progress and damage relationships.


% Caveats
Simplifying to "this interaction is characterized merely as human relations" is an easier perspective. However, that misses emergent phenomena. 

There's a risk of overanalysis. Sometimes a pipe is just a pipe. Avoiding conjecture about conspiracy and malice is a difficult boundary when insufficient information is available. 

I recognized the importance of navigating bureaucracy early in my career, but the insights were most clear when I entered middle management. 

My experiences cannot be generalized to every situation. Some of the observations here may be analogous to your context if you squint hard. 

Nothing in this book is domain specific, nothing is tied to engineering of products, and nothing is applicable solely in science research or policy development. While this material is intended to be timeless and generic, it is culturally specific to the USA. As a privileged white male, I did not encounter systemic hurdles in my career so there are blindspots not addressed in this book. 

% ? 
There are no alternatives to bureaucracy, so gaining skills in navigating bureaucracy are helpful. 

% Source of this content: 
This material is based on personal experience, reading published materials, and anecdotes from other people. No surveys were taken to support the claims made. No double blind experiments were conducted. 

% How the book should be read: 
Reading this book front-to-back is feasible. Each section is intended to be stand-alone. The book is intended to spark contemplation. 


% as per https://tex.stackexchange.com/q/393238/235813
\begin{flushright}
Ben Payne\\
\today\\
USA
\end{flushright}


%\clearpage


\tableofcontents

\mainmatter % the main part of the book will have standard pages



\chapter{Introduction to Bureaucracy}
% essentials

\section{Fundamentals of Bureaucracy\label{fundamentals_of_b}}
This section provides terminology and definition for a bureaucratic lens. Specifically bureaucracy is defined and labels for roles in bureaucracy are named. Structure in organizations is often characterized by hierarchy, and that hierarchy is described by an ``org chart.'' Lastly, meetings and written communication are described as the way in which consensus among bureaucrats is established.
\section{What is bureaucracy?\label{sec:define_bureaucracy}}

Bureaucracy is the specialization of roles necessitated by scaling and complexity for the distribution of common resources or widespread policy

In the course of carrying out someone else's subjectively defined policy, you have to make your own subjective decisions in the execution and enforcement in the course of carrying out someone else's subjectively defined policy, you have to make your own subjective decisions in the execution and enforcement


While you may know it when you see it or experience it, for this book definitions are useful. There are three distinct roles in bureaucracy: policy creator, policy enforcer, and the person upon whom policy is inflicted.

The existence of bureaucracy is independent of an organization's purpose.


The policy creator is either a politician or a bureaucrat. 

A \gls{bureaucrat} is a person subjectively interpreting policies on behalf of an organization and has discretionary enforcement to facilitate coordination of stakeholders. 

Let's break that down piece-by-piece. First, ``subjective interpretation'' means there is a person making a decision about how to do something. The subjectivity arises from different reasons one might choose an option over a competing option.  ``Policies" is a set of actions in a given circumstance. ``An \gls{organization}" is the collection of people for who the policy is made. ``Discretionary enforcement'' means the person is choosing how to apply the policy in the specific circumstances. ``Facilitating coordination'' means bureaucracy is about getting multiple people (or sometimes a person at different instances in time) to work together. The ``stakeholders'' is a group of people who care about the application of the action in each circumstance.  That's still pretty dense, so the rest of the book is spent expanding on the nuances and implications of this definition.

Bureaucracy is neither good nor bad. Bureaucracy is not tied to politics, or any specific institution (corporations, governments, academics). Bureaucracy is not defined to be efficient nor, does it have to be inefficient. Bureaucracy is not restricted to paperwork, or record keeping, or quantification, or gathering metrics. 

Bureaucracy is about delegation of control, communication, decision making, coordination, and processes. Involves negotiation, primarily informal. 

An organization comprised of bureaucrats is a \gls{bureaucracy}. The definition of bureaucracy used in this book is independent of government. Nothing in this definition involves paperwork or an office building. Definitions that limit the concept of bureaucracy to specific contexts result in a decreased ability to describe complex large-scale human organizations. 

The protagonist within a \gls{bureaucracy} is the \gls{bureaucrat} -- the person who is a member of an organization and is responsible for subjective implementation of policy for the organization. The person that a bureaucrat's decisions are inflicted on a \gls{subject}.  Depending on context, a subject may be a student (when the bureaucrat is a teacher) or a subject may be a citizen if the bureaucrat is a police officer or government official. Sometimes a bureaucrat's decisions are inflicted on other bureaucrats-as-subjects, such as when a Chief of Police creates guidelines for police in their district, or when a senior diplomat sets policy for embassy employees. 

A critical aspect of bureaucracy is that everything is made up, specifically by other humans. The consequence is that everything is negotiable. You (in the role of either a subject or a bureaucrat) need to know who to negotiate with and how to negotiate the desired changes. The only actual rules are mathematical physics that describe nature. Everything else is either naturally occurring macroscopic emergent phenomena (e.g., chemistry, biology) or humans making up labels and norms. 

Bureaucracy arises when there is no common objectively quantifiable feedback mechanism for individual participants in the organization. This aspect is why governments, schools, and prisons are characterized as bureaucratic. The military doesn't rank soldiers by ``number of enemies killed'' and is bureaucratic. Even profit-driven commercial organizations are bureaucratic when the impacts of individual employees are not coupled to sales metrics. 

Profit-based feedback makes some roles in a business context slightly more predictable and understandable, though there are still trade-offs like long-term profit versus short-term profit and externalization of harm. 

The concept of bureaucracy is most visible for complex, long lasting, and recurring situations involving many people. The apparent friction can be lower when there are only a few people involved (``I'm just talking to my collaborator" or ``I'm just buying groceries from a clerk at the store'' or ``I'm using a website for a government service''), but there is a continuous gradient. 

There is the external resource (mail delivery for USPS, public safety for FBI, environment for EPA) and there are resources internal to the bureaucracy. The focus of this book is on internal resources. In that context, bureaucracy is for the disseminated responsibility for use of resources: attention, skill, expertise. Time, money, staffing are proxy measures.



A useful way to think about bureaucracy is as a system for distributed knowledge and distributed decision making. That is in contrast to easier-to-understand concepts like centralized knowledge and centralized decision making. A government run by dictatorship is easy to conceptualize compared to democracies because there is a central character around which a narrative can be formed. Similarly, telling stories about the \href{https://en.wikipedia.org/wiki/Chief_executive_officer}{CEO} of a company is much easier than capturing the thousands of interactions conducted by the many employees of that company. Linear story-telling with a small number of protagonists does not map well to the complexities of bureaucracy. 
% are there alternatives to Bureaucracy that accomplish the same non-centralized non-consensus approach to complexity?


% https://graphthinking.blogspot.com/2017/09/market-friction-and-bureaucratic.html
Distributed knowledge and distributed decision making are hindered by
\begin{itemize}
    \item limited bandwidth between people, specifically the bureaucrats involved
    \item non-zero latency of information between people, specifically the bureaucrats involved
    \item the cost of measurement (getting data)
    \item the cost of analysis of the data
    \item making decisions that are suboptimal
\end{itemize}


\subsection{Bureaucracy as an economics model}
Firms exist in a market because negotiating contracts and prices for every interaction is burdensome. 
% https://www.kellogg.northwestern.edu/faculty/hubbard/htm/research/ec174/lectures/3coase.htm

Doesn't address small vs large companies, and doesn't distinguish between profit-oriented and non-profit and government. 

\subsection{Bureaucracy as emergent phenomenon}
Bureaucracy as a set of many bilateral interactions may not need to invoke emergence. However, there's a universality that hints at emergence. 

Above the threshold for emergence, there is scale-free behavior. The same patterns are observable at large organizations and extremely large organizations.

All those choices faced by the individual are not independent choices with respect to other bureaucrats in their environment. There is a flocking behavior of my choices are informed by the choices of those around me. Not necessarily in space.

Everyone is playing by different rules and has different objectives and everything is dynamic (both individuals and the conditions). 

Bureaucracy as a macroscopic phenomenon is emergent at sufficient scale. The scale is important because there is no longer dependence on individual relationships (beyond \href{https://en.wikipedia.org/wiki/Dunbar\%27s_number}{Dunbar's number}. There are people in the organization that you don't know and for which there is no common accountability. An organization subdivided into team recursively until there is local person-to-person accountability.  

The local rules bureaucrats employ to enable distributed decisions using distributed knowledge is meetings, processes, and communications. 

The relevance of making a claim that something is emergent is that there is behavior occurring at the macroscopic scale, and Knowing that individual motives and actions of every player at the microscopic level is not relevant.

% https://www.preposterousuniverse.com/podcast/2021/10/11/168-anil-seth-on-emergence-information-and-consciousness/
What does ``emergent'' mean? Nominal emergence example: a circle is emergent from a collection of points. Weak emergence is measurable using \href{https://en.wikipedia.org/wiki/Granger_causality}{Granger causality} or, equivalently\footnote{https://arxiv.org/abs/0910.4514}, \href{https://en.wikipedia.org/wiki/Transfer_entropy}{transfer entropy} (information theory). 


\subsection{Bureaucracy in Game Theory}
Bureaucracy does not fit cleanly into game theory categories of cooperative or competitive.

Maybe all the interactions within a bureaucracy are a bunch of small games?

Bureaucracy is self-modifying. 

Bureaucracy is in constant flux due to external conditions, externally imposed constraints, staff turn-over, internal dilemmas, disagreements of individuals. 


\subsection{Bureaucracy resists characterization}
Actually, bureaucracy is worse than emergent - the system rules can be altered or ignored by the stakeholders. \href{https://en.wikipedia.org/wiki/Wicked_problem}{Wicked problem}. This is why coming up with a holistic theory of bureaucracy is difficult. 

As soon as a claim is made, then a group can respond to that claim by behaving in an opposing manner. 

\subsection{Money as a fitness function}
Commercial businesses have a different accountability -- money. Common across all participants within the organization, and common with external stakeholders. The goal of a company is to generate profit. Commercial businesses have people who make subjective decisions and enforce policies, but there is a common metric for feedback. The feedback mechanism is not perfect. Being a good commercial bureaucrat does not necessarily result in monetary success.

Prisons, schools, medical, government, military all consume and spend money, but money isn't the goal. When faced with a decision, choice is not guided by which will generate more profit. 


\subsection{Bureaucracy as evolutionary outcome}

Biological, Genetic -- individual level
Biological, Genetic -- Group selection
Memetic

\subsection{Bureaucracy as Psychological Phenomenon}

Are you doing what's best for you, the group you're in, or everyone?
Altruistic or reciprocal? Retaliation
The answer changes time the time and situation of situation and person to person

Just a mixture of pathologies?
% define the core concepts 
\subsection*{Hierarchy of Roles\label{sec:hierarchy-of-roles}}


In an ideal situation, sufficient depth and breath for decision making would be embodied in one person. That might not be possible in every situation. One way to resolve this is to identify distinct scopes of responsibility and then assign different members of an organization separate scopes for decision making. Within a decision making scope there may be more work than one person can handle, so a team is formed. That team may have some members focused on tactical work and other members focused on strategy and coordination. Hierarchy within an organization is the formalization of separate decision-making scopes and associated specialization. 

Partitioning knowledge and decision making enables complex work beyond what one person can accomplish and causes friction among members. An expert reporting to a manager knows things the manager does not, and the manager may have context that the expert lacks. Both bureaucrats (the expert and the manager) need to convey their respective understanding and seek the holistic view.

Hierarchical decision making is one option for coordination among alternatives (like consensus), so why is hierarchy so common? Members of an organization gravitate towards hierarchy because it helps define task scope, assigns responsibility, and obviates a need for building consensus. Reaching consensus for every decision would take time and be more burdensome than appointing a person as the decision maker.

\ \\

A hierarchical organization with partitioned knowledge introduces a challenge: the order in which you share information with others matters. Your choices for who to first describe an idea to are your peers, your management, and your subordinates. 
\marginpar{[Tag] Trilemma} 
\index{trilemma!communication priority: peers, management, subordinates}
The people subordinate to you know more about the topic and are exposed to the consequences. Giving them a chance to vet the idea results in a more robust idea and validates their value in the organization. Alternatively, first sharing your idea with management  allows your superiors to provide context you might not be aware off. And choosing to first start the conversation with your peers indicates you value the relationship and decreases the risk of duplicating work.

\ \\

I've included hierarchy in the section on Fundamentals of Bureaucracy, but that does not imply that hierarchy is a required feature of bureaucracy. Hierarchies of bureaucrats are a common \href{https://en.wikipedia.org/wiki/Organizational_structure}{organizational structure} and are worth studying even if not essential to bureaucracy. The relevance of understanding hierarchy is to identify recurring behavior and patterns to leverage.
Organizations of bureaucrats can intentionally work against the use of hierarchy for decisions, but the amount of effort needed to enable alternatives results in hierarchy being a common approach.

The benefits of formal hierarchy include improved capacity for the number of policy decisions made, enabling consistency of decisions, and leveraging specialization of knowledge. 
Hierarchical decision making has costs: higher latency (compared to a signle decider), inconsistency among bureaucrats (dissemination isn't perfect), waste due to inefficiency, and 
\hyperref[sec:unavoidable-hazards]{many others}.
\ifsectionref
described in section~\ref{sec:unavoidable-hazards}. 
\fi
As another example of the potential for harm, hierarchy enables strategic ignorance. Bureaucrats in positions of power can deny having knowledge of improper activity\footnote{L.~McGoey, ``The Unknowers: How Strategic Ignorance Rules the World" (2019)
% review: https://www.tandfonline.com/doi/abs/10.1080/19460171.2020.1768422?journalCode=rcps20
and 
L.~McGoey, ``The logic of strategic ignorance" (2012). DOI 
10.1111/j.1468-4446.2012.01424.x
}. 



The structure of an organization is dynamic, but at each point in time an organization typically has a defined set of roles. Each role is distinguished by different scopes of decision authority. 
Roles are often confused with titles. What matters is the role (scope of decisions) and who reports to whom. The names of teams can be similarly not descriptive.




Roles in an organization are defined by the boundaries of responsibility. The purpose of a role is to minimize conflict, decrease the need for coordination, reduce redundancy, and allow for control of resources. Clear responsibility boundaries enables effective bureaucracy. 


A conventional characterization of an organization's hierarchy involves two criteria: the depth and breadth of the org chart.
The more people a supervisor oversees, the flatter the organization -- that's the breadth of the organization. The depth of the hierarchy is how many layers there are. See the Valve handbook~\cite{2012_Valve} and Joreen's essay~\cite{1972_Joreen} for contrasting views on the merits of an organization's hierarchy. 

A more practical view of an organization's hierarchy also involves two criteria. The two choices of how a hierarchy is shaped are 
1) how many people a supervisor oversees and 
2) how many supervisors a person has. 
Though you might naively expect that an employee has one boss, but that is \href{https://en.wikipedia.org/wiki/Matrix_management}{not a requirement}. A supervisor for a given topic may have many people reporting to them, and a bureaucrat with multiple roles may report to more than one supervisor.

\ \\

Acting as part of a group means ceding part of your autonomy. Hierarchy cedes more of your responsibility and adds expectations about relationships.
The consequence of hierarchy in an organization is that, as a member of the bureaucracy, you do not have full autonomy -- otherwise you would not be a member of the hierarchy. At the same time, you are not under strict control of the organization -- you still have some subjective decision making authority as a bureaucrat.

The person at the top of the hierarchy does not know everything. The person at the top of the hierarchy does not have input on every decision made in the organization. Some autonomy is retained by all members of the bureaucracy.

Independent of the defined roles and titles in an organization's hierarchy, there are a set of implicit roles and a separate social hierarchy of informal influencers and decision makers. Informal influencers in a bureaucracy usually have long relationships with the decision maker or relevant credentials or both. The credentials can be formal (e.g., a \href{https://en.wikipedia.org/wiki/Doctor_of_Philosophy}{PhD}) or informal (demonstrated success on a project). In either case, the decision maker is relying on another person's expertise. 

Another set of informal relations within an organization is mentors and mentees. These relations allow mentors to share institutional knowledge to mentees, and allows people in senior positions to access the novice perspective. 


\ \\

One consequence of hierarchy is a sense of fear felt by people who report to other people. This fear stems from the loss of control (less autonomy) that leaves the person feeling disempowered. 

For example, consider the following relationship. Sue, is perceived to have power over another person, Amy, because Amy gave up some control to Sue. Amy not having full control over decisions triggers the feeling of fear in Amy, regardless of how Sue behaves. Having complete responsibility for decisions also induces anxiety.

If Sue is aware of the potential for this emotional experience, Sue can compensate for Amy's fear by being friendly and receptive towards Amy. Alternatively Sue may exploit or rely on the fear felt by subordinates. Sue not noticing or accounting for Amy's fear does not invalidate Amy's emotional experience.


% Active bystander when the person doing wrong is in a position of authority
% PACT (Probe, Alert, Challenge, Take Action)
% https://mobile.twitter.com/GeorgetownABLE/status/1408498438203969541


% Mintzberg's Coordination Mechanisms
% https://www.youtube.com/watch?v=IZET8VjSifQ

 % approval process
\subsubsection*{Organizational chart as a Guide and a Lie\label{sec:org-chart-as-guide-and-lie}}

An \href{https://en.wikipedia.org/wiki/Organizational_chart}{organizational chart} 
\index{Wikipedia!\href{https://en.wikipedia.org/wiki/Organizational_chart}{organizational chart}}\iftoggle{WPinmargin}{\marginpar{[Wikipedia] Organizational\\chart}}{}
(hereafter an ``\gls{org chart}'') identifies formal roles and the formal relations among roles. An org chart is at best a snapshot in time, and more often aspirational than descriptive. Despite possible deficiencies, an org chart helps outsiders and newcomers understand the scope of responsibilities and interactions.\footnote{The organization chart hasn't always existed. The \href{https://en.wikipedia.org/wiki/George_Holt_Henshaw\#First_organization_chart}{first known org chart} 
\index{Wikipedia!\href{https://en.wikipedia.org/wiki/George_Holt_Henshaw}{George Holt Henshaw}}
was created in the 1850s.}

Org charts are a lie because undocumented relationships can matter more than official roles. Org charts fail to capture the informal roles and network of relations that facilitate progress in any organization. Org charts document titles instead of describing roles.

Org charts foster a second separate lie by creating a sense of power dynamics based on visual orientation. For more on this issue see the discussion of~\hyperref[sec:org-chart-orientation]{org chart orientation}\iftoggle{haspagenumbers}{ on page~\pageref{sec:org-chart-orientation}}{}.

\subsection{Meetings for coordination\label{sec:meetings-for-coordination}}
In an organization comprised of more than one person, meetings are a necessary artifact for facilitating coordination. The coordination accomplished by a meeting can be explicit (verbal or written), or it can be indirect through signaling (who attended the meeting, when the meeting was held, where the meeting was held, how much notice was provided). 

Meetings are a vital aspect of coordination in a bureaucracy. Tips are in \S~\ref{well-run_meeting}.


% https://graphthinking.blogspot.com/2021/07/thought-terminating-concepts-in.html
\textit{No plan; I just do what you tell me.}\\
Employee: I just do what you tell me to do. With that approach, either I will be successful because I worked hard on what you directed, or I will fail because I was directed to do the wrong thing by you.\\
Manager: Is that how you want your career to go? I think you're smart, and I think you're capable of shaping your career.

\ \\

\textit{No point in making a plan}\\
Employee: There is no point in making a plan, because everything changes so frequently.\\
Manager: But the software has an end goal, right?\\

Employee: Yes, so then the plan is to get from where we are now to that end goal.\\
Manager: And there are no intermediary steps? Milestones?

Manager: Is it better to have no plans and just put up fires reactively, or to have a plan that is subject to change?

\ \\

\textit{What is a plan anyways?}\\
Manager: I think there is value in creating a goal, enumerating tasks that would support the goal, identifying the dependencies among the sub-tasks, and time-binning the dependencies with defined milestones and deliverables. That is my definition of a plan. And having that is more useful than merely reacting.

\ \\

\textit{Who's plan?}\\
Employee: I don't need to come up with that plan, you already have a plan. Just tell me what the plan is.\\

Manager: That's not as effective as coming up with independent plans and then resolving the differences. There's value in resolving the differences, Even though that will cost time and frustration and displace time to implement.
\subsection{Written Communication}



Reports, memos, emails are artifacts of bureaucracy in an organization. A written record creates evidence about policies and decisions. Existence of a record can be used for good or for harm.


% https://graphthinking.blogspot.com/2020/09/identifying-and-eliminating.html
\subsection{Decision making in a bureaucracy\label{sec:decision-making}}


Decision are not the only source of change in an organization. Occasionally events unfold without decisions being made. This might be because the decision makers are not informed, or there is an \href{https://en.wikipedia.org/wiki/Willful_blindness}{intentional neglect}. This section focuses on situations where bureaucrats recognize the need for a decision and want to make the best decision.

There are multiple types of decisions. 
A \gls{simple decision} has one correct or beneficial choice and one or more wrong or harmful choices. The work of decision making is then to gather information that identifies which is the correct or beneficial choice and select that option.

The best case scenario for any decision making is one person making a well-informed simple decision that has immediate consequence and the consequence is to the decision maker. Examples from elementary formal education include arithmetic math problems, multiple choice quizzes, spelling tests, and memorization tests. A bureaucrat's \href{https://en.wikipedia.org/wiki/Moral_injury}{moral injury} can come from decision making that involves multiple people, weak feedback loops, and complex decisions.

\subsubsection{Example Decision Method: Pareto Frontier\label{sec:pareto}}

A complex decision may have many choices, and there are might not be a best option. Then a \href{https://en.wikipedia.org/wiki/Pareto_front}{Pareto frontier} might exist where trade-offs can be made. 

As an example of a complex decision made by one person with immediate consequence and direct relevance to the decision maker, suppose you want to purchase a car. You car about only two aspects: fuel efficiency and cost. 

\begin{figure}[ht]
    \centering
    \includegraphics[width=1\textwidth]{images/pareto_frontier_car_options.pdf}
    \caption{Four cars, L, M, N, and P. The goal of the buyer is to spend less money and get better fuel efficiency. Choices not on the frontier should be avoided, but that doesn't yield a single result.}
    \label{fig:pareto_frontier_cars}
\end{figure}

Visualzing a Pareto frontier for two quantitative variables is easy, but typically decisions involve more factors. For example, evaluating the trade-off of three quantitative variables like speed, accuracy, and cost creates a surface. With more than three variables visualization are less useful, though the analysis technique still applies. 

Another constraint on using Pareto frontier analysis is that it works well when there are many options relative to the number of variables being optimized for. 
The assessment does not work as well when there are few choices relative to the number of variables. For example, suppose there are 10 choices of car and you want high fuel efficiency, sufficient cargo capacity, maximum number of passengers, stylish, low cost, low maintenance, good durability, and high resale value. 

For a set of quantitative variables, a Pareto frontier does not account for relative importance of different variables. Assign weights to each of these factors merely stretches one axis relative to the other axes. 

There are many possible decision making frameworks besides Pareto frontiers, but in practice a typical bureaucratic decision is ill-informed, has diffuse consequences, delayed impact, and does not affect the decision maker. In bureaucratic processes there is rarely a formal assessment of options. 
Decisions are rarely recorded. 
Even afterwards a decision can be difficult to evaluate for correctness because there are multiple stakeholders.


\subsubsection{Risks of Using Decision Frameworks}

Decision making frameworks can be attractive to bureaucrats intending to formalize processes (see \S\ref{sec:process}) and encourage predictability. There are potential risks worth being aware of. 

% https://graphthinking.blogspot.com/2019/01/political-decisions-versus-science.html
A decision is political when the basis is historical relationships, maintenance or creation of a relationship, or to enable future relationships. A decision is subjective when someone else faced with the same scenario would have come to a different conclusion.
A decision is quantitative when it is based on measurements. To avoid the appearance of subjective decision making or political decision making, a decision may be framed as ``data driven." 
% https://graphthinking.blogspot.com/2018/06/data-driven-decisions-versus-data.html
A good approach for data driven quantitative analysis involves coming up with a testable hypothesis, then performing experiments and collecting data to evaluate the hypothesis. More commonly, a decision is made, then data is gathered which supports the desired outcome. Forming an opinion and then looking for evidence to back the outcome yields suboptimal results for the organization.

Even if a bureaucrat is not intentionally biased towards an outcome, there are many ways to gather evidence and some approaches have biased sampling and produce biased results.

With valid and representative data measurement, decision makers can be led astray by poor modeling. A model may use inapplicable techniques, or may have implementation bugs.

For all the dangers of decision making methods, there are worse approaches that do not rely on measurement. People rely on history (if they are aware of it) and perpetuate bad ideas, or take action based on what is best for their career, or decide based on how to accumulate more power, or simply choose based on what someone else says to do.  


\subsubsection{Decision Making Delay}

What appears from the outside as ``organizational inertia'' is internal delay of decision making and the delay of dissemination. 
Delay comes from
\begin{itemize}
    \item It takes time for each decision maker to gather information, arrive at a decision, modify processes, disseminate their selection, justify their selection. 
    \item forcing a continuous variable into a discrete set of choices. Typically the number of choices is small. Discrete choices for a continuous variable is a loss of effectiveness.
    % https://dynomight.net/teaching/
    \item processes designed to account for cheaters and people with malicious intent, whether that means a malicious bureaucrat or malicious subject. 
\item Analysis paralysis, due to {insufficient information, too much information, which framing is unclear}
\item When other people who are needed to carry out the action push back, either in disagreement or seeking clarification. The fact that the organization is not profit driven is important because the justification for the action isn't quantitatively obvious. Therefore there's a higher burden for communication.
\end{itemize}


% The following characterization has no consequence
% Decision making by bureaucrats can be informal or formal, consensus-based or solo. 


\subsubsection{A Bureaucratic Decision involves many Decisions}

A decision is actually a collection of dependent choices. After recognizing the need for a decision, follow-on decisions include identifying the stakeholders (who to include in an impact analysis) and identifying options. Who is a stakeholder and what are the options are interrelated. Involving more people expands the number of options and the complexity. Additional choices associated with reducing uncertainty are how much time to spend on the decision, how much information to gather for the decision, whether the make the decision or push the decision to someone with more expertise, whether to push the decision to someone with more exposure to the consequences.

Most decisions you make as a bureaucrat do not have hard deadlines. Instead, there are trade-offs in allocation of your time. Sooner is preferable since the consequence of the decision benefits the organization and allows you to focus on other tasks, but delaying allows for more information gathering for a better informed decision. (See Dilemma~\ref{table:gather_data_lots-vs-little} and other related Dilemmas in that section.)


If a bureaucrat is going to rely on expert consultation (see \S\ref{sec:expertise} on expertise), the decision maker needs to be confident the expert is not of straying outside their area of expertise. For example, I don't rely on a botanist with many published papers to tell me how to change the oil in my car. 
Besides knowing their own limitations, the expert should be clear about whether their input is a factual summary, a predictive assessment, or a value judgement. This nuance complicates what you as a bureaucrat are interested in (what's the best choice?).



\section{History of Bureaucracy\label{sec:history}}


Bureaucracy has repeatedly arisen independently in various cultures
\footnote{See the Wikipedia entry on the \href{https://en.wikipedia.org/wiki/Bureaucracy\%23History}{history of bureaucracy}.
\index{Wikipedia!\href{https://en.wikipedia.org/wiki/Bureaucracy\%23History}{history of bureaucracy}}
}
lasting for timescales that exceed the lifespan of one person.\footnote{See the YouTube video on the \href{https://www.youtube.com/watch?v=B_nsZlcC12g}{History of bureaucracy}.} That indicates the current situation is not a fluke or coincidence. There is some utility (or pathology) that is consistently recurring. 


Bureaucracy predates writing and language and even humans! Subjective policy enforcement in support of an organization arises in pre-human tribes, visible in groups of modern apes who have to manage access to shared resources~\cite{2016_Suchak}. 



Though bureaucracy is not new, the pervasiveness is. Before the industrial revolution the scale of employment and government were small, with organizations limited by the speed of communication. For the past 100+ years the size of organizations (commercial, governmental, and academia) have grown beyond \href{https://en.wikipedia.org/wiki/Dunbar\%27s_number}{Dunbar's number} -- the number of human relationships you can maintain, about 150. \iftoggle{WPinmargin}{\marginpar{[Wikipedia] Dunbar's\\number}}{}
\index{Wikipedia!\href{https://en.wikipedia.org/wiki/Dunbar\%27s_number}{Dunbar's number}}
More people participate in more organizations that are more bureaucratic. Driving this increase is the support for more complex products and processes. 

% claim: bureaucracy grow faster than the growth of human population?



\section{Scope of bureaucracy}
\subsection{Who is a bureaucrat?}

A cashier in a gas station is a bureaucrat. The ``policy'' might simply be ``take money from customer in exchange for items and gas,'' but the subjective application of that policy leaves a lot of room for the cashier to shape the customer's experience. Does the cashier greet the customer when the customer enters the store? Does the cashier look at the customer to acknowledge the customer? Smile? How quickly does the cashier engage the customer? Minor nuances that are left to the cashier in the execution of the store policy means there is room for subjective application of the policy. 

This same discretionary application of policy applies to commercial bureaucrats like sandwich makers, car salespeople, grocery clerks, retail clerks. Teachers, police, tax collectors, and other state works are government bureaucrats. 

% bureaucracy is not limited to white collard office workers
Factory line workers subjectively apply policies. Enforcement of quality standards is the most obvious area. Pacing of work is a negotiation.

Sometimes bureaucrats do not work directly with customers or citizens or products. Then the bureaucratic process is inflicted on fellow bureaucrats. In this scenario, a bureaucrat is subjectively applying a policy to other bureaucrats. 

Identifying yourself as a bureaucrat matters, both to the employee and to the business. The risk of not self-identifying as a bureaucrat is that you won't grasp how much control you have in implementing and enforcing policy. If you think of yourself as having to blindly follow rules, you will harm the people you are applying the rules to and you will harm the business/institution you are applying the rules for. Adapting policies to circumstances is the value of having judgement capacity. 

In a similar sense from the consumer/citizen perspective, if you don't think you are interacting with a bureaucracy, you won't perceive the opportunity to negotiate.  If you view rules as fixed and inflexible, you will harm your ability to make progress. If a rule was made by a human, then that rule is flexible. Who made the rule? Who enforces the rule? If you can talk to them, could they be convinced to make a modification or an exception?
 % subsection
\subsection{Number of people in a bureaucracy}
Although bureaucracy can be present for one person, and bureaucracy is often apparent on teams (e.g., 3 to 20 people), this book primarily focuses on the situation of multiple teams comprising an organization. This might be a few hundred people (above \href{https://en.wikipedia.org/wiki/Dunbar's_number}{Dunbar's number}) up to millions of people. 

There are tens of companies that employ more than a million people\footnote{see \href{https://en.wikipedia.org/wiki/List_of_largest_employers}{Wikipedia's list of largest employers}}, including Walmart, Amazon, and McDonald's. Bureaucracy emerges at a far smaller scale, is scale invariant, and is generic across sectors. 

Small companies with a few people incur bureaucracy. Non-profit organizations encounter bureaucracy. The complexity of the tasks may be different, but the same scale-independent patterns can emerge because of a common factor: human behavior.

Size of bureaucracy scales with the complexity of the problem. 

Scaling a bureaucracy up is easier than scaling down. % subsection
\subsection{What does Bureaucracy imply?}
Bureaucracy is neither good nor bad. Bureaucracy is not tied to politics, or any specific institution (corporations, governments, academics). Bureaucracy is not defined to be efficient nor, does it have to be inefficient. Bureaucracy is not restricted to paperwork, or record keeping, or quantification, or gathering metrics. 

Bureaucracy is about delegation of control, communication, decision making, coordination, and processes. Involves negotiation, primarily informal. In that context, wouldn't it be useful to be skilled at bureaucracy?  % section
\subsection{Why does bureaucracy exist? Can't we just do the work?}

The short answer is that bureaucracy is a response to the complexity of a problem being solved. To see why that is, let's start simple and then increase the complexity. 

The minimal scenario to start from is to imagine a single person working on a single task that does not last long (a few minutes), is relatively easy (cognitively and physically and emotionally), and does not recur. In that situation, building consensus is irrelevant and no process is required. 

Most of what you do occurs outside those limits and thus incurs some concept of \gls{process} (breaking a task into subtasks). Staying with the one-person constraint, a complex task can benefit from being broken into subtasks. Sometimes the order of the subtasks matters, so we need to track the dependencies. A recurring multi-step process with documentation is starting to have features of bureaucracy, but lacks the need for consensus. 

If one person lacks the skills relevant to a multi-step process, they may engage another person to help. The interaction may be informal (anarchy) or formalized in a contract (\href{https://en.wikipedia.org/wiki/Libertarianism}{libertarian}). If the parties working on the task fail to reach consensus, what is the recourse? Options include physical violence, threats, or involving a third party (e.g., a court with lawyers and judges). 


The bureaucrat's identity is subsumed into service for the organization they are part of. At the same time, bureaucracy enables the bureaucrat to amplify their presence by being part of a larger organization. A bureaucrat can accomplish more as part of an organization than by working alone. Sometimes the cost of being part of the organization exceeds the force multiplier of working together. 

% https://graphthinking.blogspot.com/2021/09/why-is-everything-so-hard-in-large.html

What if we completely avoided bureaucracy? That question is better worded by replacing ``bureaucracy" with ``coordination of stakeholders". If you avoid coordination of stakeholders, you either are constrained to only work on tasks that involve one person, or you get is random (uncoordinated) interactions. 

What if we minimized bureaucracy? Again, try replacing ``bureaucracy" in that question with ``coordination of stakeholders". The goal of ``minimizing coordination" probably isn't the real objective. To be more precise, a specific objective might be ``minimize time spent executing the task" (which takes a lot of coordination prior to the task execution) or ``minimize the level of distraction to stakeholders" (chunk the coordination time). Another strategy for minimizing bureaucracy is to reduce the number of stakeholders involved. For a given task complexity, this means having smarter people who have more skills. 

\begin{figure}
\includegraphics[width=0.8\textwidth]{images/people-per-task-for-skill-level.pdf}
\caption{Three levels of task complexity are shown. As task complexity increases, the size of the team needs to grow. The growth may be less if the team members are brilliant. Those brilliant people cost more and there are fewer of them.}
\end{figure}
 % section
\section{Subject's View of Bureaucracy\label{sec:subjects-view}}

This section takes on the perspective of the subject of bureaucracy but is meant to be read by bureaucrats who want to improve their \gls{process empathy}. This book doesn't provide advice for subjects of bureaucracy.\footnote{For advice navigating bureaucracy, listen to National Public Radio's \href{https://www.npr.org/2022/03/16/1086915600/get-what-you-want-customer-service}{How to talk with customer service}~\cite{2022_LifeKit}.} 

\ \\

When you have a positive experience engaging with bureaucracy, your positive attribution is to the people involved. Or the ease of a solution makes bureaucracy less visible and the solution seems obvious. 
When you have a negative experience with bureaucracy, complaints are about the incompetence of the people involved or the incomprehensibleness of the system. Don't these bureaucrats know how to do their job? Why isn't the solution obvious? Why does this system not work for me? David Graeber summarized this view:\footnote{\href{https://harpers.org/archive/2015/03/in-regulation-nation/}{In Regulation Nation. Harpers Magazine, 2015.}}
% May have actually come from "The Utopia of Rules"
\begin{quote}
 Amongst working-class Americans, government is generally seen as being made up of two sorts of people: `politicians,' who are blustering crooks and liars but can at least occasionally be voted out of office, and `bureaucrats,' who are condescending elitists almost impossible to uproot.   
\end{quote}


%\subsection{Sources of complexity}

The scale of bureaucracy (the number of people in an organization) and the processes of an organization can seem disproportionate to the complexity of the task. Typically when a person (being a subject of bureaucracy) interacts with the bureaucratic organization the artifacts are simple, like a form to fill out. The simplicity of the artifact does not correlate to the number of decisions made, the tracking of information, or the precautions taken by the organization. All these aspects are invisible to the subject because they are internal to the organization.

As an example, consider when you go to the doctor and they mend your broken arm with a cast. That seems straightforward because all you see is the doctor putting a cast on. You don't get insight into the decisions they had to make. Why did they need 20 years of focused schooling to carry out a procedure that took 15 minutes?

Judging bureaucracy by the artifact visible to the individual subject undersells the complexity of the decision-making necessary to take action. The contingencies that you were not exposed to because everything went well make the amount of investment from the bureaucrat appear wasteful.

As the subject of bureaucracy, you also lack the ability to distinguish how much work is attributable to the bureaucrat generating justifications for their actions (colloquially, \href{https://en.wikipedia.org/wiki/Cover_your_ass}{covering the bureaucrat's ass}). 
\index{Wikipedia!\href{https://en.wikipedia.org/wiki/Cover_your_ass}{cover your ass}}
\iftoggle{WPinmargin}{\marginpar{[Wikipedia] Cover\\your ass}}{}
These justifications are needed both within the organization and potentially for external stakeholders. Each bureaucrat's rationalization may not be reviewed, but it needs to be available for review later.

As the subject of bureaucracy, you typically can't distinguish when work is caused by a bureaucrat's ill-informed decisions. Is the person stupid, mistaken, or is there something you are not taking into account?
Sometimes the work is carried out by insufficiently trained bureaucrats, but you don't get to know whether you're working with an experienced and knowledgeable bureaucrat or a new untrained bureaucrat. 

As the subject of bureaucracy, you don't have visibility on the many nuances of an organization. The internal power struggles and organizational politics that depend on personalities, resources, and competing prioritization are not clear to outsiders.
The inconsistency of an organization's policies may not be felt by bureaucrats in that organization. Each bureaucrat may have a different opinion, resulting in a lack of consistent guidance.
There may be legacy policies in effect.

% summary paragraph
Effective action by a bureaucratic organization is complicated by the need for relevant information. Gathering, analyzing, and sharing that information persistently requires bureaucratic processes. To further complicate the ideal process, sometimes the organization lacks the staffing with relevant skills. 


% Transition paragraph
The next section documents why bureaucracy is hard from the perspective of the bureaucrat. Even without getting into the specifics of a bureaucrat's role or the purpose of an organization, there are generic reasons that bureaucracy is a burdensome responsibility.

\chapter{Bureaucracy throughout life\label{b_throughout_life}}
This chapter provides the view of a person interacting with bureaucracy as a user. 

Personal routines are self-imposed bureaucracy
\section{Avoiding Bureaucracy is Nearly Impossible}
\iftoggle{shortsectiontitle}{\sectionmark{Avoiding Bureaucracy}}{}

The only situation where bureaucracy might not exist is if you live entirely independently and have no interaction with other people. That means completely disengaging from society. Even then, personal routines are a self-imposed form of bureaucracy, with the roles of policy maker, bureaucrat, and subject collapsed to a single person -- you.

Self-sufficiency and autonomy are attractive alternatives to bureaucracy. The way participants in modern society strive for self-sufficiency is by denying their dependence on modern society. That's a relabeling of selfishness which feels better. 

For the rest of us who operate as members of  society, bureaucracy is necessary for our rights. You validate your name using paperwork, forms, and records. These artifacts are used, in cooperation with other people, to determine your claim of citizenship and associated rights. 
That's a subjective policy that \iftoggle{glossarysubstitutionworks}{\glspl{stakeholder}}{stakeholders} in society agree to. 


Bureaucracy is necessary because it is a response to the \iftoggle{WPinmargin}{\marginpar{$>$Wikipedia: Collective action problem}}{}%
\href{https://en.wikipedia.org/wiki/Collective_action_problem}{collective action problem} -- everyone would benefit from cooperation, but each person has to sacrifice their self-interests.  
\index{Wikipedia!collective action problem@\href{https://en.wikipedia.org/wiki/Collective_action_problem}{collective action problem}}
As long as humans form communities, we will address the challenge of \iftoggle{glossarysubstitutionworks}{\glspl{shared resource}}{shared resources}
 -- whether tangible (e.g., water, land, air) or intangible (e.g., expertise, information). 
That is why bureaucracy is culturally invariant and persistent across time.
Learning to be an effective bureaucrat improves your chances of success in society. 

The specific way society is constructed (democratic, authoritarian, dictatorship, anarchy) is irrelevant -- bureaucracy is still present. Even the libertarian approach of relying on contract enforcement implies some bureaucracy (e.g., forums for resolving contract disputes like a court system, enforcing decisions through violence). 


Not all bureaucracy is due to the state, nor is bureaucracy confined to companies. Parenting involves creating situation-specific requirements for children, with the organization being the family as mentioned in the section 
on \hyperref[sec:bureaucracy-early-childhood]{bureaucracy in childhood}.
\marginpar{See page~\pageref{sec:bureaucracy-early-childhood}.}%
Dress codes for sports teams are arbitrary standards. 
Store clerks are bureaucrats, as are website forum moderators.  Content moderation is the process of (inconsistently) enforcing arbitrary standards. This mindset even permeates individuals as internalized expectations of policy and enforcement when no one else is present. 

Recognizing instances of bureaucracy enables more skillful interaction, whether as a bureaucrat or subject. The rest of this section illustrates both the view of a person interacting with bureaucracy as a \gls{subject} and the perspective of bureaucrats working within organizations. 





\section{Beginnings and Endings}
Birth

Education

Hiring
Getting hired

Firing
Being fired
Quitting

Death

\section{Relationships throughout life}
In the conventional Western progression, relationships include
\begin{itemize}
    \item pre-education childhood: primarily family, possible community members, caretakers
    \item primary education: family, friends, teachers
    \item college, graduate school: friends, teachers, advisors
    \item employment: managers above you, peer employees, people you manage
    \item healthcare
\end{itemize}

\subsection{school vs business}
School (high school, undergraduate, graduate school) is  different from working in a large organization. 


\subsection{undergrad vs graduate}
 education process roles and expectations vary over time

\subsection{military}
Less than 0.5\% of the United States population serves in the military. \footnote{source: \href{https://www.cfr.org/backgrounder/demographics-us-military}{Council on Foreign Relations}}

The military, with rigid hierarchy and defined protocols, is also a distinct experience. 



\chapter{Bureaucracies are made of Humans\label{b_made_of_humans}}
This chapter provides an insider's perspective of working in a bureaucracy. What are all the aspects to consider?
% unordered essays to be clustered later

\subsection{Hiring into a Bureaucracy}
Regardless of the specifics of the job, there are specific attributes that make a candidate more likely to be successful in a bureaucracy. 

In addition to role-specific skills, hire for \href{https://en.wikipedia.org/wiki/Metacognition}{metacognition}, social skills, and intrinsic motivation.

% https://graphthinking.blogspot.com/2021/04/screening-for-metacognition-in-job.html

% https://graphthinking.blogspot.com/2021/07/screening-for-intellectual-empathy-in.html

% https://graphthinking.blogspot.com/2021/04/questions-to-ask-interviewer-when.html


\subsection{Professional Training}

There is often a disconnect between the results formal education and the specific needs of the organization. Professional training is intended to fill that gap.  
\section{Folk Wisdom}
While most of the entries on
 https://github.com/dwmkerr/hacker-laws
don't apply to bureaucracy, the list is useful to review. 

Folk wisdom is an attempt to explain bureaucratic features but ends up being \gls{thought terminating}.

\href{https://en.wikipedia.org/wiki/Murphy\%27s_law}{Murphy's law}

\href{https://en.wikipedia.org/wiki/Wiio\%27s_laws}{Wiio's laws}

\href{https://en.wikipedia.org/wiki/Hanlon\%27s_razor}{Hanlon's razor}

\href{https://en.wikipedia.org/wiki/Parkinson\%27s_law}{Parkinson's law}

\href{https://en.wikipedia.org/wiki/Putt\%27s_Law_and_the_Successful_Technocrat}{Putt's Law}

\href{https://en.wikipedia.org/wiki/Hick\%27s_law}{Hick's law}

\href{https://en.wikipedia.org/wiki/Allen_curve}{Allen curve}

\href{https://en.wikipedia.org/wiki/Peter_principle}{Peter principle}

\href{https://en.wikipedia.org/wiki/Dilbert_principle}{Dilbert principle}

\href{https://en.wikipedia.org/wiki/Law_of_triviality}{Law of Triviality}
\subsubsection*{Organization chart orientation
\label{sec:org-chart-orientation}}

A common method of describing relations within the bureaucracy is the organization chart (commonly the ``\gls{org chart}"). \iftoggle{glossaryinmargin}{\marginpar{[Glossary]}}{}%
Normally the Chief Executive Officer (CEO) is at the top of the chart, middle management is in the middle, and managed employees are at the bottom. See Figure~\ref{fig:org_chart_orientation_ceo-at-top}\iftoggle{haspagenumbers}{ on page~\pageref{fig:org_chart_orientation_ceo-at-top}.}{.}

Artifacts like org charts subtly convey an organization's culture. 
% What's the point of this section? Is there a consequence, or is this just an observation?
There are emotional connotations to alternative layouts. You can alter expected relations (culture and norms) by playing with the orientation of the org chart.
Org chart orientation can be overanalyzed, so the exploration in this section is limited.

The point of thinking about org chart orientation is to frame how you perceive your supervisors, peers, and the bureaucrats you manage. Notice that the framing is embedded in the words -- prefixes super (over) and sub (under). 
These concepts inform what you expect from relations.
Do I seek support or direction and guidance from my boss? What do I expect from my boss? My peers? The people I oversee? What do I expect to provide them?

%\begin{itemize}
%\item 
%\end{itemize}

The relative orientation of the \href{https://en.wikipedia.org/wiki/Chief_executive_officer}{CEO} 
\index{Wikipedia!Chief Executive Officer@\href{https://en.wikipedia.org/wiki/Chief_executive_officer}{Chief Executive Officer}}\iftoggle{WPinmargin}{\marginpar{$>$Wikipedia: CEO}}{}
to the workers sets expectations for relations. 
Options for orientation are the conventional CEO at the top
(Figure~\ref{fig:org_chart_orientation_ceo-at-top}), 
CEO at the bottom (Figure~\ref{fig:org_chart_orientation_ceo-at-bottom}),
CEO on the right (Figure~\ref{fig:org_chart_orientation_ceo-leads}),
CEO on the left (Figure~\ref{fig:org_chart_orientation_ceo-follows}),
CEO as the center of a star 
(for example, the diagram for the \href{https://en.wikipedia.org/wiki/File:League_of_Nations_Organization.png}{League of Nations} in 1930.)
\index{Wikipedia!League of Nations diagram@\href{https://en.wikipedia.org/wiki/File:League_of_Nations_Organization.png}{League of Nations diagram}}

\begin{figure}
\begin{center}
\includegraphics[width=1\textwidth]{images/org-chart-orientation-ceo-at-top.pdf}
\end{center}
\caption{Standard orientation. The role with the most responsibility and authority is at the top. Left-right ordering is intended to be irrelevant in this view, though left-to-right reading order emphasizes importance.}
\label{fig:org_chart_orientation_ceo-at-top}
\end{figure}

\begin{figure}
\begin{center}
\includegraphics[width=1\textwidth]{images/org-chart-orientation-ceo-at-bottom.pdf}
\end{center}
\caption{Flipping the orientation presents a more realistic view of the CEO's responsibility. The crushing burden of servant leadership is clear. Left-right ordering is intended to be irrelevant in this view.}
\label{fig:org_chart_orientation_ceo-at-bottom}
\end{figure}

\begin{figure}
\begin{center}
\includegraphics[width=0.7\textwidth]{images/org-chart-orientation-ceo-leads.pdf}
\end{center}
\caption{Conventionally time flows from left (old) to right (new), so in this graph the CEO leads the charge into the unknown. Is the CEO dragging workers forward, or are the workers pushing the CEO? The top-to-bottom order can be read as importance. }
\label{fig:org_chart_orientation_ceo-leads}
\end{figure}

\begin{figure}
\begin{center}
\includegraphics[width=0.7\textwidth]{images/org-chart-orientation-workers-lead.pdf}
\end{center}
\caption{The ``chariot view'' with the CEO in the chariot and the workers out front. Workers are in the future; the CEO is in the past operating on old information. As with Figure~\ref{fig:org_chart_orientation_ceo-leads}, top-to-bottom ordering can be read as importance. }
\label{fig:org_chart_orientation_ceo-follows}
\end{figure}

\begin{figure}
\begin{center}
\includegraphics[width=0.8\textwidth]{images/org_chart_wedding_cake_dependencies_-_manufacturing.pdf}
\end{center}
\caption{An internal-customer-oriented view rather than a reporting-oriented view. The center of the bullseye is the team that generates the value that is the focus of the business or the organization.
The outer rings support teams that exist in the inner rings. The diagram is specific to an organization's domain. This visualization identifies which teams are the customers of which other teams in an organization.}
\label{fig:org_chart_wedding_cake_manufacturing}
\end{figure}



%extension of 
% \href{https://en.wikipedia.org/wiki/Conway\%27s_law}{Conway's law}: seating chart reflects org chart
\section{Bureaucracy as Accidental, Legacy, or Essential}
\gls{essential bureaucracy} is the minimum processes and staffing and skills necessary to address the complexity of the managing a community's access to a \gls{shared resource}. Achieving this minimum is tricky since the optimization can be with respect to resilience to change, resilience to edge cases, staff turn-over, speed experienced by consumer, financial cost, time spent by the organization, and number of staff. Miss any one of those and the bureaucracy is deemed inefficient.

Undesirable bureaucracy is categorized as either accidental or legacy. Accidental bureaucracy arises when someone misunderstands what is needed, or when skills of the bureaucrats involved are insufficient for the complexity a problem. Legacy bureaucracy occurs when the situation changes but the processes do not. Resolving each of these suboptimal conditions may seem easy: have better knowledge of the problem, assign the right people to the problem, and change processes as problems evolve. 

Having enough knowledge is often infeasible, especially for complex problems at large scale. Having the people with the right skills assumes that a pipeline of people with relevant talents exists and that people in the pipeline won't be poached to work on other challenges. Keeping up with evolving problems depends on having resources to change (beyond the maintenance baseline), and having a defined approach for changing the process. 
% https://graphthinking.blogspot.com/2021/02/how-to-have-efficient-bureaucracy.html

\section{Efficient Bureaucracy}

In an ideal scenario with no bureaucracy, everyone comes to the same conclusion when presented with the same information. Then the management process of building consensus becomes unnecessary. There is no need to fight over resources (money, staffing) and no need to fight over direction.

While that ideal scenario is not going to happen, it points to how to improve bureaucratic efficiency:
\begin{itemize}
\item each person has the same information. 
\item each person applies the same decision making process consistently
\item every person has the same incentives
\end{itemize}
The reason bureaucracy is inefficient is
\begin{itemize}
\item not everyone has the same information
\item processes are inconsistent
\item incentives vary
\end{itemize}
Also, add the issue that each person's reference experiences are unique. As a consequence, decision making is subjective. 

Can any action be taken to improve bureaucratic efficiency? Yes!
\begin{itemize}
\item you can share information with other stakeholders
\item you can seek information from other stakeholders
\item you can strive for and demonstrate transparency
\item apply consistent processes 
\item hold others (and yourself) accountable 
\item account for varying incentives and reference experiences
\end{itemize}

\subsection{Motivation of Bureaucrats\label{sec:motivations}}

Appreciating the diverse motives of the bureaucrats you work with is instructive. If you expect everyone to have the same motives as you then you will be surprised by the friction created by diverse motives. 

Motivations of participants are rarely ``how can I make the organization more successful" or even ``how can I sell/produce more product"? Usually motivation is based on personal success in various manifestations, which leads to emergent phenomena which appears confounding to observers outside the bureaucracy. 


% see https://en.wikipedia.org/wiki/Social_influence

Each bureaucrat has a motive, even the bureaucrats who do nothing. 
% https://graphthinking.blogspot.com/2020/02/there-is-no-idle-status-for-paid.html
In an organization where you are a paid bureaucrat, you are either actively working for improvement of the organization, or your existence is parasitic to the organization. There is no ``idle" status for paid employees in an organization with limited resources.

As an example motive for a bureaucrat, I want to avoid being too efficient such that I eliminate the need for my job, and not so inefficient that the organization fails and I lose my job. Increasing the efficiency of bureaucracy is good for the organization and the outcomes, but can be harmful to the bureaucrat's career.

Career stability within an organization is a benefit, and it can be leveraged to take more risk. However, it typically manifests as inaction by an employee. There's no harm to the employee in not taking action. If an employee doesn't do anything, nothing bad will happen to that employee. Career stability decreases extrinsic motivation.


Example motivations for bureaucrats: 
stability (aka job security, the comfort of a routine),
money (current pay or future earnings), 
travel, 
problem solving, 
status, 
exerting power or control, 
credibility of being associated with the organization (if the organization has a positive reputation), 
logistical convenience (``the office is near where I lived''), 
service to people the organization serves.



The consequence of diverse motives is that expecting bureaucratic organizations to be logical, fair, consistent, and efficient is unreasonable even when every participant wants those features. Each bureaucrat thinks, ``I am logical, fair, consistent, and efficient.'' Therefore each bureaucrat expects other bureaucrats to meet those same (unrealistic) standards. Next, anthropomorphize the team or organization and expect the group to meet those standards. 

Even if each bureaucrat were logical, fair, consistent, and efficient (they are not, and neither are you), each person has a different motivation. Each person wants to accomplish something different using their unique skills and referencing their own experiences. Compounding the confusion, each bureaucrat has to coordinate using communication that has latency and limited bandwidth and isn't precise.

An expectation that bureaucracy feels illogical, unfair, inconsistent, and inefficient is a useful baseline. Working against bureaucratic entropy yields improvements even though perfection is inaccessible.
% https://graphthinking.blogspot.com/2021/04/laffer-curve-and-minimum-viable.html

The Laffer curve is a claim in economics that there is a relation between government tax rates and the revenue from taxes collected. The relation, based on Rolle's theorem, says that between a tax rate of 0% and 100%, there must be some amount of tax that corresponds to the maximum of revenue. 

While the mathematical statement may be provable, the use in economics seems hand-wavy. In this post, I'll extend that hand-waviness to a different domain: bureaucratic processes in organizations. The relation to the Laffer curve is that bureaucratic processes a tax on productivity. 
\section{Bureaucratic Fallacies\label{sec:fallacies}}
\index{bureaucratic fallacy}

Discussions about bureaucracy by non-experts often rely on common conceptions that are \gls{thought-terminating}. Identifying these enables you to understand both why the fallacy is attractive during a discussion with fellow bureaucrats and how each idea is incomplete.

These fallacies may at first feel right but are in fact misleading. In contrast, there are  
\hyperref[sec:unavoidable-hazards]{unavoidable hazards} \iftoggle{haspagenumbers}{(see page~\pageref{sec:unavoidable-hazards})}{} that may feel bad but reveal underlying truths.

\ \\
\begin{samepage}
\textit{Bureaucratic fallacy}: \textbf{Bureaucracy is bad}. \\
\index{bureaucratic fallacy!bureaucracy is bad}
\textit{Why this feels true}: When a person subjected to bureaucracy has a negative experience, the easiest attribution is to the least-understood aspect -- the bureaucracy.\\
\textit{What this is missing}: 
\iftoggle{glossarysubstitutionworks}{\Gls{bureaucracy}}{Bureaucracy}
 as defined in this guide is neither good nor bad. Bureaucracy is merely a way of managing  resources shared amongst a community. 
 \end{samepage}

\ \\
\begin{samepage}
\textit{Bureaucratic fallacy}: 
\textbf{There is no point in planning ahead since everything (staffing, funding, purpose, scope) is always changing.}\\
\index{bureaucratic fallacy!no point in planning}
\textit{Why this feels true}: Change can feel disorienting, especially when it is unexpected. A change of the assumptions for a plan may make the plan less relevant. \\
\textit{What this is missing}: Preparing for change and thinking ahead about contingencies enables effective use of resources. Have a vision and work towards it while accounting for and adapting to change. This approach requires extra work, some of which will be left unused. See Dilemma \ref{table:dilemma-personal-emergencies-vs-ignore}\iftoggle{haspagenumbers}{ on page~\pageref{table:dilemma-personal-emergencies-vs-ignore}.}{.}
\end{samepage}

\ \\
\begin{samepage}
\textit{Bureaucratic fallacy}: \textbf{Bureaucracy is an aberration, a mistake, due to poor planning or incompetent participants}. \\
\index{bureaucratic fallacy!bureaucracy is a mistake}
\textit{Why this feels true}: Mistakes are made, poor or insufficient planning does happen, and some participants are incompetent.\\
\textit{What this is missing}: Bureaucracy occurs even if no mistakes are made, effort is spent on effective planning, and participants are competent. That's because, in the use of distributed knowledge and distributed decision-making, bureaucrats face \hyperref[sec:dilemma-trilemma]{dilemmas}.
\marginpar{See page~\pageref{sec:dilemma-trilemma}.}
%\ifsectionref
%(see section~\ref{sec:dilemma-trilemma}).
%\fi
\end{samepage}

\ \\
\begin{samepage}
\textit{Bureaucratic fallacy}: \textbf{Bureaucracy is inefficient}. \\
\index{bureaucratic fallacy!bureaucracy is inefficient}
\textit{Why this feels true}: Expressed by both subjects and bureaucrats who observe seemingly wasteful processes.\\
\textit{What this is missing}: If bureaucracy were truly inefficient (not allocating resources in the most efficient way), then in a competitive environment it would be replaced by a more efficient approach. The key is to ask, ``efficient with respect to what metric?'' The metric of money, or time, or number of people, or stability, or robustness to perturbation?  Second, what would motivate improved efficiency? Without incentives, change is less likely. 
\end{samepage}

\ \\
\begin{samepage}
\textit{Bureaucratic fallacy}: \textbf{Bureaucracy is due to malfeasance.}\\
\index{bureaucratic fallacy!bureaucracy is due to malfeasance}
The specific number of malicious bureaucrats in variations on this fallacy ranges from ``all of the participants'' to ``just enough to be problematic.'' \\
\textit{Why this feels true}: There are bad actors present in any system comprised of humans. \\
\textit{What this is missing}: Most participants are earnestly trying to help make a positive contribution, even though that can be hard to see from the view of subjects or even other bureaucrats. Processes like isolation or promotion exist within bureaucracy to deal with malicious bureaucrats.
\end{samepage}

\ \\
\begin{samepage}
\textit{Bureaucratic fallacy}: \textbf{Bureaucracy is a sign of decay from within the organization.} \\
\index{bureaucratic fallacy!bureaucracy indicates decay}
\textit{Why this feels true}: Relationships within an organization have a half-life and require ongoing investment to renew. At the same time, new bureaucratic processes are constantly being developed by other bureaucrats. The number of processes increases as the organization ages. Bureaucracy seems to arise without effort and countering it takes effort.  \\
\textit{What this is missing}: Bureaucracy unavoidably emerges in every organization because coordination is required. The negative connotation of decay should be replaced with a sense of neutral evolution.
\end{samepage}

\ \\
\begin{samepage}
\textit{Bureaucratic Fallacy}: \textbf{If the response to a request I make can't be expedited, my request must not be important}.  \\
\index{bureaucratic fallacy!importance is measured by response latency}
\textit{Why this feels true}: Other people would demonstrate they care about what I am working on by prioritizing things I am dependent on.\\
\textit{What this is missing}: When everything gets prioritized, that's the same as nothing getting priority.
\end{samepage}

\ \\
\begin{samepage}
\textit{Bureaucratic Fallacy}: \textbf{The expected duration of a task is how long it would take one person to accomplish}.  \\
\index{bureaucratic fallacy!task duration for one person}
\textit{Why this feels true}: When I imagine carrying out a task, the default is a story with one character. \\
\textit{What this is missing}: This narrative fails to account for the overhead of interaction among participants and delays due to asynchronous dependencies.~\cite{1975_brooks}
\end{samepage}

\ \\
% https://graphthinking.blogspot.com/2019/08/two-misleading-simplifications-when.html
\begin{samepage}
\textit{Bureaucratic Fallacy}: \textbf{When developing or altering policy, focus on the average or majority (to the exclusion of outliers)}. \\
\index{bureaucratic fallacy!ignore outliers}
\textit{Why this simplification is misleading}: Sometimes outliers are not just more of the same; they alter the outcome. During the transition from horses-for-transportation to cars, cars could initially have been considered outliers. 
\end{samepage}

\ \\
\begin{samepage}
\textit{Bureaucratic Fallacy}: \textbf{People learn from their mistakes}. \\
\index{bureaucratic fallacy!people learn from their mistakes}
\textit{Why this feels true}: There's an optimistic desire for this to be true. \\
\textit{What this is missing}: 
People repeat mistakes without noticing. People do not naturally reflect on their failings in a constructive way and then apply insights to future situations.
People \textit{can} learn from their mistakes. Doing so requires a low latency feedback loop and incentive to change. In bureaucracies feedback loops are weak so learning may not happen.
\end{samepage}

\ \\
\begin{samepage}
\textit{Bureaucratic Fallacy}: \textbf{Processes are serial}.\\
\index{bureaucratic fallacy!processes are serial}
A conventional approach to process design is a sequence of tasks. As an example, consider approval chains. \\
\textit{Why this feels true}: Serial processes are easier to understand. \\
\textit{What this is missing}: Some tasks that are independent can be carried out concurrently; see the section on \hyperref[sec:reducing-overhead]{reducing overhead}\iftoggle{haspagenumbers}{on page~\pageref{sec:reducing-overhead}.}{.}
\end{samepage}

\ \\
\begin{samepage}
\textit{Bureaucratic Fallacy}: \textbf{Hard work creates results}.\\
\index{bureaucratic fallacy!hard work creates results}
\textit{Why this feels true}: Some results do require hard work. The alternative (results arise serendipitously or with little effort) is not inspiring. \\
\textit{What this is missing}: Hard work can be invested on wasteful effort. Don't confuse being busy with being productive. Sometimes insight is more useful than hard work. 
\end{samepage}

%\ \\

% NOT USEFUL
%\textit{Bureaucratic Fallacy}: \textbf{Motivations for bureaucrats are categorized as individualistic, tribal, organizational, societal, or humanity}.\\



\ \\
\begin{samepage}
\textit{Bureaucratic Fallacy}: 
\textbf{You cannot pay a little and get a lot}; see the \href{https://en.wikipedia.org/wiki/Common_law_of_business_balance}{Common law of business balance}. 
\index{Wikipedia!\href{https://en.wikipedia.org/wiki/Common_law_of_business_balance}{Common law of business balance}}
\marginpar{$>>$ Folk Wisdom} 
\index{folk wisdom!\href{https://en.wikipedia.org/wiki/Common_law_of_business_balance}{Common law of business balance}} \\
\index{bureaucratic fallacy!cannot pay a little and get a lot}
\textit{Why this feels true}: If small investments made a big difference we'd already be making the investment.\\
\textit{What this is missing}: This doesn't account for creative solutions and ignores \href{https://en.wikipedia.org/wiki/Nudge_theory}{nudge theory}
\index{Wikipedia!\href{https://en.wikipedia.org/wiki/Nudge_theory}{nudge theory}}
from behavioral economics. 
\end{samepage}

\ \\

When you are reasoning about bureaucratic systems, there may be a conclusion that is concise and feels explanatory. Then you should try to come up with counter-examples, either logically or from experience.  

Some fallacies are based on an expectation that other people should be more like what you imagine your best self to be. That mindset fails to account for your  shortcomings and the diversity of other bureaucrats. 


\subsection{Building, Managing, and Spending Reputation\label{sec:reputation}}

Your personal reputation within the organization dramatically impacts your effectiveness.

Organizations have reputations externally. 
Internal-to-the-org there is cultural norms. 
% https://graphthinking.blogspot.com/2021/01/why-active-shaping-of-culture-is.html


% https://graphthinking.blogspot.com/2018/05/my-evolving-view-on-role-of-my.html


Building reputation through multiple small wins or larger risk on bigger bets


Relation between Reputation and Brand and Political Capital? Same thing?

% the following article is useless
% https://www.indeed.com/career-advice/career-development/build-a-reputation
% since it reduces to "be a good person"

How does an individual create and accumulate political capital? What does political capital mean with respect to teams?

% https://graphthinking.blogspot.com/2021/09/notes-from-class-on-being-politically.html

Reputation matters for influence. How other people perceive you impacts what you can accomplish and when people seek out your help or input.

Your reputation is actively changing based on your activities and communication -- both your communication and the stories others tell about you.

Neglecting to manage your reputation means you lose input to the stories others tell about you. Active management of your reputation requires engaging people and generating evidence. 

Reputation is set whenever and where ever you are observed, or artifacts are associated with you. What you wear, when you show up, how your emails appear, body posture in meetings. 


Reputation is perception of the person the bureaucrat is engaging with.  What does that person think of you?

Ideally that would be a function of their technical skill, ability to collaborate with other people, the strength of their network, creativity. None of this matters if the person you're engaging with doesn't know those things. 

Based on your reputation, what trust does that person have? 

To spend reputation is to bend the rules. 

Spending reputation means taking risk that involve other people

Build reputation by doing useful things that are visible to other people

\section{product deployment}

\subsection{external to the org}
\subsection{internal to the org}
\subsection{Actual time spent working}

In a bureaucracy the actual time spent working is less than the number of hours you get paid for. Breaks during work, vacation from work, holidays, sick leave. 

% https://rescuetime.wpengine.com/work-life-balance-study-2019/

\begin{figure}
    \centering
    \includegraphics[width=0.8\textwidth]{images/hours_per_activity_per_employed_year}
    \caption{Hours of ``work'' per year when accounting for the rest of life. Assumes 5 weeks of vacation, 2 days of sick leave, and 11 holidays.}
    % footnotes in caption is not recommended; see https://texfaq.org/FAQ-ftncapt
    % however it can be done; see https://stackoverflow.com/questions/67621322/footnote-in-caption-of-figure-on-latex
    \label{fig:my_label}
\end{figure}

Fig~\ref{fig:my_label}
\footnote{\href{https://docs.google.com/spreadsheets/d/1ZaOZZXWkEzX4fFltUdlR4A6ENrAXnkzTW4YrjA4tDO8/edit?usp=sharing}{source for calculations}}
% https://graphthinking.blogspot.com/2021/07/patterns-anti-patterns-in-bureaucracy.html

% intra- and inter- team dynamics

\subsection{Teams are Subdivisions of an Organization}

\cite{2015_Katzenbach}

In the context of altering teams, there are a few major levers available: create new team, merge teams, dissolve a team. 


For a given set of teams, the lateral interactions are competitive or cooperative. Coordination is required (or conflict will occur) for money, staffing, and resources. Examples of resources include access to or control of data, computer equipment, hardware, floor space, prestige, products (output).

\subsubsection{Roles of Management versus Leadership}

Teams include managers and leaders. Those roles are not necessarily the same person. 

A manager's role involves time management, task tracking, employee evaluations, promotion, pay, requesting resources for team members, and hiring. 

A leader's role focuses on coordinating vision and principles. The vision and principles do not have to originate from the leader -- other team members can contribute ideas. The leader's responsibility is primarily social consensus. 
\section{Communication within a Bureaucracy}
% https://graphthinking.blogspot.com/2020/05/invisible-bureaucracy.html

\subsection*{Social and Bureaucratic interactions\label{sec:socializing}}

Change in a bureaucracy can apply to processes and people, but a more amorphous concept is changing the culture of a team or organization. What is meant by ``culture'' usually refers to norms -- the expectations of behavior that individuals hold to. That definition of culture is generic; what is meant within a bureaucratic context requires jargon for specific expectations.

To evaluate expectations, we start by introducing categories of interactions. 
Interactions among members of an organization are either a social interaction or a bureaucratic interaction. 

As examples of each of these,
\begin{itemize}
\item \textit{Social interaction example}: ``Did you see the game on TV last night? Our team did fantastic, right? I wanted to get tickets for the game, but they were sold out."
\item \textit{Bureaucratic interaction example}: ``You'll need to get approval from Sue before presenting your idea to the board for their review. Then talk with Russ and get his thoughts about how to proceed."
\end{itemize}
Both social and bureaucratic interactions are vital to cohesion in an organization. 


Bureaucratic interaction can be broken into two subcategories: 
\gls{visible bureaucracy} \iftoggle{glossaryinmargin}{\marginpar{[Glossary]}}{}%
(procedures and processes are written down and can be discovered by stakeholders) and 
\gls{invisible bureaucracy} \iftoggle{glossaryinmargin}{\marginpar{[Glossary]}}{}%
(procedures and processes are known to some stakeholders and are conveyed verbally to some of the other stakeholders).

Invisible bureaucracy is akin to related topics outside the professional environment: invisible domestic work\footnote{Cleaning your living space, raising children, caring for pets; see~\cite{1987_Daniels}.} and invisible relationship work.\footnote{Consistent need to delegate, being curious without reciprocation.} The work associated with emotional cohesion, logistics, planning, scheduling, and communicating is hard to quantify so it does not get counted.


The relevance of this jargon is to break down the components of an organization's ``culture" experienced by participants.
When someone in the organization advocates for changing the culture, which expectations are they specifically referring to? Invisible bureaucracy is the hardest to alter because it is undocumented and not counted.
%The ratio of social relationship to visible bureaucracy to invisible bureaucracy is a characterization of the culture. There are norms associated with each of these three categories.

Processes default to invisible bureaucracy because creating and maintaining documentation requires work. Making the documentation discoverable requires work.
%, and because some processes are embarrassingly inefficient. 
To make invisible bureaucracy visible, document the work and enable other people to find the documentation.
 % subsection
\section{Communication Tips}

The role of verbal communication is critical for bureaucrats. 
There is a lot of advice on effective communication (enunciate, speak loud enough to be heard, be humble, be curious), so the advice below is highlighted because of prevalence in bureaucratic organizations. 
The following is generic to interactions outside of bureaucracy. 

For general writing tips, see Strunk's and White's Elements of Style and other resources \footnote{\href{https://www.youtube.com/watch?v=vtIzMaLkCaM}{LEADERSHIP LAB: The Craft of Writing Effectively} and \href{https://www.youtube.com/watch?v=aFwVf5a3pZM}{LEADERSHIP LAB: Writing Beyond the Academy}}.

\subsection*{Tip: Not all interaction challenges are communication problems.}
Sometimes an inability to discuss ideas is not a communication problem but a psychological deficit of personality. Distinguishing ``I'm an ineffective communicator'' from ``the person I'm talking with doesn't communicate effectively'' from ``that person has a diagnosed psychological reason they are unable to communicate'' is tough for those of us who are not psychologists or psychiatrists. 

%how to measure effectiveness: The waste or inefficiency in a bureaucracy is a measure of the lack of coordination or inconsistent decision making among the members

\subsection*{Tip: Avoid relying on stereotypes.}
Within an organization different teams may build up reputations for certain behaviors, or there may be significant events that the team is associated with. 
When interacting with members of a team that has a reputation, avoid relying on that stereotype or event as an opening for discussion. 
You're speaking to an individual, so address that person's behavior.



\subsection*{Tip: Avoid questions that have a binary response\label{sec:yes_no_questions}.}

Responding to a request with ``no'' is advantageous for the person replying to the question. There is less work required, less risk of failure, and better continuity. As an example of a poorly framed question, I could ask, ``Can I have a copy of the data you're using?'' The person I'm asking is less disrupted if they refuse to share. 

A more constructive phrasing is ``I need information on X to work on Y, and I think you have information about X. How can you help me get information on X?'' By clarifying my intent, I allow the person with the data to provide options I may not have considered.

Similarly, when I'm being asked for information, I try to learn the person's intent motivating the question. Sometimes the requester doesn't actually know what to ask for. Instead of ``no'' I try to figure out how to enable the person to be successful. 

\subsection*{Tip: Leverage the other person's experience while focusing on your own\label{sec:advice}}

Advice without context is less effective.\\
\textit{Bad}: ``Here is what I think you should do in that situation.''\\
\textit{Better}: ``Here is what I did in that situation.''

People usually find talking about themselves an easy topic if you are curious about their experiences. 
If you can learn the other person's background and history and motivations, you can weave that into the advice you provide. 
Tailoring your message increases the likelihood of implementation. 

\subsection*{Tip: Avoid Platitudes\label{sec:platitudes}.}
% https://graphthinking.blogspot.com/2017/10/why-platitudes-are-used.html
\href{https://en.wikipedia.org/wiki/Platitude}{Platitudes} are \gls{thought terminating}; the statement feels true and is resistant to debate. Platitudes capture a feeling with sufficient accuracy, but with imprecise language. As a result, there's no specific action.

Because platitudes result in a conclusion, the conversation participants may feel more bonded. However, that bond is shallow.

Example platitudes to avoid:
% https://graphthinking.blogspot.com/2017/02/phrases-which-serve-as-thinking-stoppers.html
% https://graphthinking.blogspot.com/2017/10/a-list-of-platitudes.html
\begin{itemize}
    \item pick your battles
    \item Some things you can't explain
    \item Your time will come.
    \item You can be anything that you want to be
    \item I just want to get through this day
    \item It is what it is
    \item I'm just one person
    \item That's that
    \item Life's not perfect
    \item Life's not fair
    \item There's only so much you can do about it
    \item What is meant to be will be
    \item It is God's will
    \item It's part of God's plan
\end{itemize}

If your goal is to understand a concept or issue deeply, you need to use precise language.

\subsection*{Tip: Strive to use Precise Language}

Imprecise language causes miscommunication. Intent is unclear, as is expected consequence.

If you have a specific definition for a word central to the topic of interest, ask your new collaborator for their definition. Do not expect others to share your definition even when there are established norms for the topic. 

Instead of asking a collaborator, ``Are you taking action on this topic?'' ask ``What actions are you taking on this topic?''

If someone claims, ``We plan to get to that action,'' ask for a timeline. A deadline can be for an artifact or a re-evaluation of the topic.

In the short-term imprecise language takes less work to create and can take less time to convey. 

The importance of precise language is proportional to the potential consequences of action/inaction/wrong action.
Precision also should be proportional to the complexity of the topic being discussed. 

\subsection*{Tip: Word is bond\label{sec:word-is-bond}}

Your communication (verbal, written) is your reputation. People rely on what you tell them even if there isn't legal recourse. Reliance on your word is why precision matters. 

Frustration and disappointment follow when you don't uphold your word, or others misinterpret your imprecise language, or you are misunderstood.

Communication implies responsibility for the content.  There is a corresponding accountability in the relationship between speaker and listener (or writer and reader).

\subsection*{Tip: Take care near the boundaries of knowledge}

Trying to find someone else's extent of knowledge is tricky -- they don't want to appear stupid. They may interpret the exploration at a trap. ``I don't know'' can be an embarrassing statement to make, even if you don't share their embarrassment. 

Knowing the limitations of your own knowledge and disclosing those boundaries to others is critical. Distinguish what you know from speculation about things you don't. 

\subsection*{Tip: Listen all the way to the last word of the speaker}

Formulating a response to the first part of an idea or a sentence is tempting. Waiting for the speaker to finish before thinking of how to response is courteous. Waiting creates a pause which makes you seem more thoughtful. 

This is complicated by the speakers who include long pauses for contemplation and then resume. 

\subsection*{Tip: Eliminate speaking over other people.\label{sec:crosstalk}}
% https://graphthinking.blogspot.com/2017/10/crosstalk.html

Crosstalk occurs two people who are communicating verbally experience interference from another audible conversation. That can occur because a third person is talking at one of the original two participants, or when four or more people are holding two separate conversations concurrently. 

Crosstalk in a bureaucracy is motivated by
limited time available to communicate. A meeting participant may feel inspired by something someone else said and want to interject. 
%Typically manifests as popcorn style stories based on experience. 
%Intended as wisdom for self-validation by others in our community. 
Crosstalk can indicate engagement and enthusiasm, or it can be due to the speaker wanting to dominate the topic through interruption. The likelihood of crosstalk is dependent on the level of aggressiveness of participants.
In either case (enthusiasm or power-seeking), the original speakers are disrespected. The original speaker may feel annoyed at being interrupted.



%Crosstalk has four roles and a minimum of two people participating: the discussion facilitator, the original speaker, the interrupter, and other meeting participants. 
%During crosstalk, the discussion facilitator loses control of the interaction to the interrupter.  
The audience is frustrated by the lack of clarity of where to focus. This distraction causes participants to lose of focus and productivity of the interaction decreases.

Bystander intervention for out-of-control meeting: raise your hand. \marginpar{[Tag] Actionable Advice} This non-verbally reverts focus back to the discussion facilitator. 

\subsection*{Tip: Account for Warnock's dilemma}
% https://graphthinking.blogspot.com/2018/09/dealing-with-warnocks-dilemma-in.html
\href{https://en.wikipedia.org/wiki/Warnock\%27s_dilemma}{Warnock's dilemma}
is the common experience of figuring out how to interpret not getting feedback. This is especially vital in meetings where the speaker or facilitator needs to gauge participant comprehension of delivered content. Simply asking ``Does anyone not understand what I just described?" is likely to get no response from attendees because individuals want to avoid looking stupid.

\ \\
\textit{Technique}: Pick an individual to provide a recap.\\
\textit{Technique}: Survey the audience using multiple choice to gauge understanding.\marginpar{[Tag] Actionable Advice}

\subsection*{Tip: Seek action with a deadline}

% https://graphthinking.blogspot.com/2017/11/collected-wisdom.html
When asking someone for help or input, specify a deadline for their response. \marginpar{[Tag] Actionable Advice}This helps the person prioritize their tasks.

\subsection*{Tip: Identify the cause of miscommunication}

% https://graphthinking.blogspot.com/2019/06/miscommunication-versus-inability-to.html
\begin{itemize}
    \item miscommunication the cause is often due to definitions of words used or differences in context. In this situation, additional time spent communicating and taking different approaches is sufficient to remedy the issue.
\item A speaker may be inarticulate. If the speaker is unable to coherently convey their internal experience to a listener, then the communication failure is of a distinct category. No amount of additional communication will lead to improved understanding on the part of the listener.
\item A speaker simply has nothing to say about a subject. Regardless of whether they are capable of articulating a concept, they may be unable to relate to the topic. Often a person in this situation still wants to participate, but they are unable to meaningfully contribute. 
\end{itemize}
% https://graphthinking.blogspot.com/2019/05/identifying-empty-talk.html
Empty talk is the use of words that are ill-defined, emotionally resonant, inactionable, and impersonal.

\subsection*{Tip: ask-tell-ask}

Collaborating with fellow bureaucrats who have expertise in areas you do not requires extra work. There may be differences in the words used to describe certain situations, more precision in wording that you're used to, or thinking about situations in ways you are not familiar with. In that context, to bridge the differences you can ask, tell, ask\footnote{\href{https://cepc.ucsf.edu/sites/cepc.ucsf.edu/files/Curriculum_sample_14-0602.pdf}{``The 10 Building Blocks of Primary Care: `Ask Tell Ask' Sample Curriculum''} and \href{https://www.the-hospitalist.org/hospitalist/article/125126/qi-initiatives/ask-tell-ask-simple-technique-can-help-hospitalists}{Ask-Tell-Ask: Simple Technique}}. 

The first step is asking what the other person's perspective is on the topic. This helps establish the appropriate level of nuance is and can tell you how that person frames the issue. The second step is to tell the person what you want to say. The ``tell'' step should leverage what you learned from the first ``ask'' step. Use phrasing that is consistent with what you just learned from the other person. The third step is to ask the person what they heard from you. If they are unable to tell you, you may need to refine your delivery. To improve the likelihood of success keep the content in the second step short. 

The ask-tell-ask technique can be used iteratively in the same conversation, especially in discussion complex topics with a new collaborator. \marginpar{[Tag] Actionable Advice}Using ask-tell-ask takes long than just telling but increases the effectiveness of the communication. You also get to learn more about the other person's perspective. 


\subsection*{Tip: Initial responsiveness and status updates}
In a bureaucracy requiring approval, or soliciting input, sometimes waiting can provide value to the person doing the waiting. The request may be overcome by events, or the person asking may remind which indicates priority.

\subsection*{Tip: Make deadlines explicit}

Typically requests have two deadlines. The first deadline is when a response is sought. The second deadline is when a response is no longer useful.  

As an example, suppose I am inviting people to a meeting. I send the invitation 5 days prior to the meeting and I want to know who is able to attend by 3 days before the meeting. The second deadline is the time of the meeting. Replies after that second deadline do not help me understand who is going to attend. 

\subsection*{Tip: Read each email/memo/report to determine the purpose }
% https://graphthinking.blogspot.com/2021/03/read-each-email-to-determine-purpose.html

\textit{Problematic behavior}: scan the text of a message, see if there is immediate action or response needed. If no action or response is needed, go to the next email. \\
 That does not work for emails that contain logistics associated with future events. 

Instead, consider possible intentions of the person writing the email. 

\textbf{Decision needed}. Typically includes context. \\
\textit{Action}: if the team maintains a decision log, update that.
Response is selection of a choice.

Tip: Instead of asking for a decision, ask for if the person is opposed.

Tip: instead of asking for a decision, ask for the go-ahead. This framing biases the respondent towards action (specifically approval) rather than thinking. 

\textbf{Situational awareness}.\\
\textit{Action}: Expected default is no action, but interject if there's an issue.


\textbf{Action or Tasking}.\\
\textit{Action}: Do something within some deadline

\textbf{Approval sought}.\\
\textit{Action}: Confirm or deny

\textbf{Feedback sought}.\\
\textit{Action}: Assessment of proposal


\textbf{Meeting logistics}. Can be an announcement (widely available), registration (limited attendance), or invitation (specific to you). Attendance is optional or require. \\
\textit{Action}: Create or update a calendar event
Response should restate the logistics (time/date/location/purpose) to confirm. 

\textbf{Brainstorming}\\
May provoke a response for building on an idea.
``For your situational awareness, no action needed." Notification of activity by someone else. Or change in plans. 
If needed, a correction to the described direction might trigger a response or even a meeting.

\textbf{Reference} e.g. describing a process or business workflow. Or a citation.\\
\textit{Action}: Copy process documentation to wiki. Copy citation to bibliography.
Acknowledgement response thanking the sender for the update/clarification.

\textbf{Setting a formal policy or issuing an informal edict}\\
\textit{Action}: move the policy/edict documentation to Confluence or Wiki
Acknowledgement response needed only if the edit is aimed at just me or the group I am leading

\textbf{Question}\\
If this is a recurring question, move to a ``Frequently Asked Questions" page on Confluence or Wiki.
Response needed that provides answer or seeks clarification.


Here I'm using ``action" to refer to activities outside the email channel. 

If I read email to figure out the purpose of the email, that will help me determine what action and response are relevant. 

Whether I am the only recipient or on of many receivers can change the intent of the email, and whether I'm in the ``to" or ``cc" field matters. Unfortunately, ``to" versus ``cc" are not reliable indicators since email senders do not reliably conform to the expected use. 



%This categorization of text within emails is a useful natural language processing challenge for machine learning. Currently a few email providers already do some of this with identifying meeting logistics, providing reminders to follow-up, and providing reply snippets. A browser plug-in that differentiates the various purposes of text could help readers determine relevant actions and responses. 

An email sent to multiple recipients may have different purposes for different readers. The reader's role or knowledge may factor into how they interpret the content. The inclusion or exclusion of recipients alters how the content is understood. 

\subsection*{Tip: Don't seek attribution for contributions; credit others\label{sec:credit-others}}

Give credit to others for good ideas and beneficial actions. Either they accept credit and you are seen as a contributor to their success, or they push back and you look generous. Credit is not a \href{https://en.wikipedia.org/wiki/Zero-sum_game}{zero sum game}.

\subsection*{Tip: Offer to take blame\label{sec:take-blame}}

Before an action commences, tell collaborators that you are willing to accept blame if something goes wrong. This alleviates their fear of risks.

\subsection*{Tip: Survey stakeholders}
% https://graphthinking.blogspot.com/2016/01/how-to-solve-and-not-solve-problems.html

Suppose you are a \href{http://www.peacecorps.gov/}{Peace Corps} worker in Africa. You show up and the village doesn't have easy access to clean water. Villagers walk a long ways in dangerous areas for dirty, unsafe water. This is a very obvious problem and all the villagers agree that they don't have good water and that this problem should be fixed.

Implementing the solution would take about a week - get the equipment to the village, drill a well, build a pump.

You could take additional time and involve the villagers in this project. They could participate in getting the equipment, which should lead to a sense of ownership.
But then when the equipment shows up, they don't take action to drill the well. If the well is drilled, it soon falls into disrepair and the villagers are back to doing things they way they used to. What happened?

The villagers don't see access to clean water as the most significant issue. You came in and imposed your view of what the problem is and how to fix it. When you impose your view of what the problem is, the solution won't be adopted by villagers because they don't prioritize it.
It is better to survey the community to see how they operate. What do they think the problems are?
Both leadership and the community members need to provide priorities.

This issue is exacerbated if you come to the village as a representative of a company providing wells. You are biased when you ask, ``Do you have any problems?"

Of course the villagers have water problems which could be fixed with better wells. However, when you get into the details of placing or improving a well, they lose interest. What the community really wants is free installation, zero maintenance, easy to use, and no operational costs. That would improve their life.

When you say there's cost (both initial investment of capital and then operations/maintenance) and a learning curve associated with the solution, then the user's interest wanes -- you are presenting another cost/benefit ratio for them to evaluate. Then they ask ``Can we get by without the well?" Yes, they don't need the well -- they've survived without it.

Novel solutions (in this example, drilling a well and installing a pump) have have barriers to adoption. Two barriers are the current priorities of the community and the incumbent solution/processes.

If there are problems with higher priority, the community will delay implementing your solution. That's fine if the higher-ranked priorities are bounded, but they are often not. An example of this is the following:
Suppose a person has three tasks, and you introduce a solution which is a fourth task.
If the first task is ``go from point A to point B," then that task will eventually be eliminated and there will be three remaining.
If the second task is ``secure your village," that is an unbounded task. The person won't get to or won't prioritize your low-ranked task.

How will your solution impact their higher-ranked priorities?

\ \\

\subsection*{Email Structure\label{sec:email-structure}}

% from https://graphthinking.blogspot.com/2021/10/structuring-email-content-for.html

If implementing all these tips sounds like a lot of work, that's because it is. Effective written communication requires intentional effort because of the lack of augmenting channels (compared to voice or video or in-person). 



Use consistent design and structure for your emails. Emails are part of your professional reputation.

Emails start with a greeting: Hi, Hello, Good morning, Good afternoon, Good evening. 
Email greetings include the name of the targeted recipient(s). 

Emails terminate with a professional closing, e.g., ``Kindly", ``Regards", etc

Emails contain a signature block with contact information -- phone number, normal hours of response, which timezone you're in if your team spans timezones, how long to wait for a response before asking again, which communication channel I prefer, etc.

\begin{figure}
\includegraphics[width=1\textwidth]{images/email_template.pdf}
\caption{Template for new email messages. Greeting has a space after the comma -- that is where the recipient's name will go. Signature block uses smaller after the name.}
\label{fig:email_template}
\end{figure}

Email signature blocks do not include unnecessary images, as that uses more storage for recipients. 
Email threads focused on a specific instance of a recurring event include the date (YYYY-MM-DD) in the subject line. 

Based on the purpose of the email, example key phrases for subject lines include: ``meeting notes" versus ``agenda" versus ``question about".

Revising an existing subject line can disrupt the ability of email software to thread conversations. However, sometimes the revision is worth breaking threading.

When replying to an ongoing thread, retain the original message as part of the thread to provide readers historical context.

When replying to threads with sensitive messages, sanitize the included content as appropriate by removing name or identifying details.

If an email contains multiple requests or questions, at the top of the email (after the greeting) explicitly say how many of each type. Then, in the body of the message, number them.

\begin{figure}
\includegraphics[width=1\textwidth]{images/email_two_questions.pdf}
\caption{Distinct items the recipient should address in a reply.}
\label{fig:email_two_questions}
\end{figure}

If an item corresponds to a requested action, separately highlight the action and indicate who is supposed to take the action and what the deadline for response is.

\begin{figure}
\includegraphics[width=1\textwidth]{images/email_meeting_notes.pdf}
\caption{Who has what action due when?}
\label{fig:email_meeting_notes}
\end{figure}

Computer commands should be distinct separate fixed-width font. This distinguishes the text from the rest of the narrative. 


\begin{figure}
\includegraphics[width=1\textwidth]{images/email_computer_font.pdf}
\caption{The computer commands use fixed width font. The author distinguished input from output through the use of bold vs non-bold respectively. The author highlighted the error message using red. Inline text like ``cat'' in the last line is also fixed width.}
\label{fig:email_computer_font}
\end{figure}

References to documents include a direct full path.

If referring to a previous separate thread, include the subject and the date+time that email was sent

For bullet points, explicitly specify that items are joined by one of the following: OR, XOR, AND

If you have an unordered list, explicitly state that order is irrelevant.

If you have a sequence of steps, number them appropriately and indicate which steps are required versus optional

Use visual sketches to illustrate concepts rather than always relying on text. Don't use pictures all the time, and don't have too many pictures in an email. 

Know how to both embed pictures inline and how to attach files and when to use which. 
Email replies should preferentially be at the top of the thread. 
If replying to multiple points in the previous email, embed replies inline, mark the distinction, and highlight the authorship. 

\begin{figure}
\includegraphics[width=1\textwidth]{images/email_reply.pdf}
\caption{Bob's reply to Sue's questions. The third question is not shown in this illustration.}
\label{fig:email_reply}
\end{figure}

If replying inline, explicitly state that at the top of the thread.

The email trilemma is to balance the amount of detail against providing sufficient context and being concise. 

If the email is longer than a paragraph, provide a \href{https://en.wikipedia.org/wiki/BLUF_(communication)}{B.L.U.F} or \href{https://en.wikipedia.org/wiki/Wikipedia:Too_long;_didn\%27t_read}{tl;dr} or summary. In general emails should be short. Longer discussions should be held on the phone or in person, with a summary report after the discussion. Reliance on a BLUF or tl;dr risks resulting in the reader skipping the content. 

Emails convey both emotional tone and facts. Your intent is practically irrelevant; the reader's perception is paramount. 

Every email should have a purpose. What are you asking the recipient to do? How do you want them to feel? How should they respond?

When replying, starting your email with an expression of gratitude for the work the recipient has done so far sets a positive tone by acknowledging their investment.

\ \\

\subsection*{Summary of what action should be carried out} 

As the outsider, you should help the community enumerate and document all of the problems they identify. Then you can help enumerate and document how the problems are related (dependencies). Only then can you help the community identify and document the root causes.

If the solution you, the outsider, identified really is the root cause, then the community will arrive at that independently. If that is the case, then you can enable them to implement a solution which addresses the root causes. The community will then have a sense of ownership.

\subsection*{Friction between teams within an organization}

Ideally there is a clear division of responsibilities among teams. Even in that context there is necessarily some interaction among teams -- one team may depend on the output from another team. Coordination among the teams regarding transfer of data or products or projects or knowledge is critical to the smooth operation of the organization. 

An organization has finite staffing, money, time. Therefore, teams within the organization face a \href{https://en.wikipedia.org/wiki/Zero-sum_game}{zero-sum}
\index{Wikipedia!\href{https://en.wikipedia.org/wiki/Zero-sum_game}{zero-sum game}}
distribution of resources.

Difference of decision making perspective based on local conditions, stupidity, or different incentives, different definitions of success

When attempting to resolve friction between teams, there is an authority common to the teams, but that person lacks the nuanced insight, doesn't have time to get involved in every challenge, and doesn't want to micromanage multiple teams.
\section{Meetings within a Bureaucracy}
types of meetings: internal meetings, customer meetings, conferences 
% presenting
% audience
\subsection{Well-run meeting\label{well-run_meeting}}

Identify essential attendees. If someone does not need to be present, notify them in advance that you will share the meeting notes afterwards. 


Bad: no meeting agenda\\
Good: agenda\\
Better: agenda share with other participants
For formal meetings, share agenda in writing prior to meeting. 

TODO: Why an agenda matters in a bureaucracy: 

TODO: forces conspiring against agendas


For formal in-person meetings, Verify meeting venue has sufficient space, seating, working IT equipment

For formal virtual meetings, ensure participants are familiar with virtual meeting controls

TODO: why logistics/infrastructure matter in a bureaucracy:

TODO: forces conspiring against logistics/infrastructure
% https://graphthinking.blogspot.com/2021/02/organizations-value-things-more-than.html
\section{meeting time compared to theft}

In large organizations, there can be significant bureaucracy associated with even small purchases. A multi-step review process may be incurred for a \$2000 acquisition.

Another measurement of value is that if an employee were to steal even \$200 worth of materials, the organization would likely punish that employee.


Those metrics apply to tangible goods, but not to people's time. Consider a meeting of 10 people and each person's cost is \$200 per hour. A wasted meeting is not unusual and certainly would not incur bureaucratic review processes. The cost to the organization is fiscally the same -- \$2000. Similarly, consider an employee who is late and causes a loss of productivity. Merely depriving the organization of \$200 worth of time is not punished in the same way theft is.

In fact, organizations default to meetings (even recurring meetings) rather than not meet. And being late to a meeting is accepted. 

We can debate the differences between theft of materials and theft of time. The financial argument is clear. 

Source: Andy Grove in "High Output Management"
\subsection{One-on-one check-in meetings}

% https://graphthinking.blogspot.com/2021/05/the-agenda-for-one-on-one-meeting.html

One-on-one meeting questions for helping the manager understand the team member's status.
\begin{itemize}
    \item what are the objectives for the team?
    \item what have you been successful with since we last met?
    \item what is blocking our team's progress?
    \item what are your plans?
    \item how are you collaborating with the rest of the team?
\end{itemize}

Reflective prompts for one-on-one meetings:
\begin{itemize}
    \item If there was just one thing you could change about our organization, what would it be and why?
    \item How do you plan to train your coworkers on topics you understand and they don't?
    \item What have you learned in the past month?
    \item What are the biggest risks for the team?
    \item What's limiting your productivity?
\end{itemize}
Responding to these questions takes time (an hour) and willingness to be open. 

\ \\

The one-on-one check-in should be tailored to the phase of the employee's progression. 
\begin{itemize}
    \item new team member, either new to the team or new to the company. Here the focus of the one-on-one is to ensure a smooth on-boarding process. Does the employee have the necessary computer log-in accounts? Do they have an email account? Are they on the mailing list?\\
\textit{The duration of this phase could last between a day and two weeks.}
    \item team member is responsible for small tasks: the one-one-one is for discussions on training and sprint planning and sprint-reviews. Characterized by the team member being dependent on others for their success. In this phase the employee collaborates on tasks.
\textit{The duration of this phase could last a few months to years.}
    \item team member is responsible for large tasks (which get broken into subtasks): the one-on-one is to help the team member define their success. Activities include planning, resource allocation, assessment. Characterized by the need to coordinate with others on the team or other teams.
\textit{The duration of this phase could could be the rest of a career.}
    \item facilitating the productivity of others: rather than being task-oriented, this team member supports coworkers. 
    \item peer check-in: this one-on-one is a form of mentorship. The value of the exchange is to get a different perspective and to hold each other accountable.
\end{itemize}

TODO: How does the team member and the supervisor know when the next phase is appropriate?

TODO: What are the thresholds for change?

\ \\

https://news.ycombinator.com/item?id=30152268

\ \\

https://news.ycombinator.com/item?id=22341138
https://github.com/VGraupera/1on1-questions


\subsection{Walk-around impromptu meetings}
\subsection{How to be a Successful Conference Attendee}
% https://graphthinking.blogspot.com/2012/02/how-to-succeed-as-attendee-at.html
\subsection{How to be a Successful Conference Organizer}

% https://graphthinking.blogspot.com/2011/11/how-to-organize-conference.html



% If bureaucracy is a distributed knowledge distributed decision system, compare with paxos and Byzantine generals
% https://en.wikipedia.org/wiki/Paxos_(computer_science)

% https://en.wikipedia.org/wiki/Complexity_theory_and_organizations

% thought experiment: 
% * What if everybody in a bureaucracy were the same?
% * What if everybody in a bureaucracy had a different opinion?



\clearpage

\printglossaries

\nocite{*} % causes LaTeX to include every entry in your .bib file.
\bibliographystyle{plain-annote}
\bibliography{biblio}

\end{document}
