\section[Communicating Professional Vulnerability]{Communicating a Professional Sense of Vulnerability\label{sec:professional-vulnerability}\iftoggle{shortsectiontitle}{\sectionmark{Communicating Professional Vulnerability}}{}}
\iftoggle{shortsectiontitle}{\sectionmark{Communicating Professional Vulnerability}}{}

% LONG1 shows up in the TOC
% LONG2 is the section title


Emotionally vulnerable communication is personal and is a form of intimacy. You are exposing personal issues. This personal emotional vulnerability deepens relationships. The risk is that information could be used to manipulate or harm. Whether you choose to form social friendships with your coworkers is up to you, but isn't a requirement for an effective bureaucrat. There is a more relevant practice of vulnerability.

Professional vulnerability is about going beyond transparency regarding bureaucratic issues. Transparency is about what is happening, while professional vulnerability is about why something is happening. For example, professional topics of conversation include processes and incentives. Bureaucratic processes may be intended to be impersonal, but the consequences are felt by participants.


Discussions of internal intrigues of an organization are a form of gossip among professionals. \href{https://en.wikipedia.org/wiki/Gossip}{Gossip} 
\index{Wikipedia!gossip@\href{https://en.wikipedia.org/wiki/Gossip}{gossip}}\iftoggle{WPinmargin}{\marginpar{$>$Wikipedia: Gossip}}{}
can be constructive (finding aspects to remedy),  lead to insights, and shapes cultural expectations within the bureaucracy. As with personal gossip, professional gossip can risk harm if used against other organization members. 

As with personal vulnerability, professional vulnerability involves learning who to share what information with and when. In interacting with a person who is new to you, you can experiment by being professionally vulnerable and see whether they reciprocate or at least explore the topic with you. Being open, direct, and curious can help the person you're talking with feel more comfortable. Shared introspection is the objective. 

As an example, you can say something like, ``Perhaps the reason behind (observation) is (reason 1) or (reason 2).'' That gives the other person a chance to brainstorm with me without committing to a position. If the other person is unwilling to explore in-depth, reverting to safer topics is easy.

Being vulnerable does not mean the other person will reciprocate. That is a risk on your part. 