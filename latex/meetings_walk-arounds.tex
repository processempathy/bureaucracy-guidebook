\subsection*{Walk-around impromptu meetings\label{sec:walk-arounds}}

Meetings aren't constrained to occur in conference rooms around a table with everyone seated. Walking around the shared office space and having conversations with coworkers can be an intentional form of informal meeting. 

Ambushing coworkers when they aren't expecting interruption takes tact. As the (potential) interrupter, you need to be sensitive that other people may not want to be interrupted, while others seek diversion from their current task. If you're the person being interrupted by the coworker walking around, you have the right of deferral. 

Walking-around and finding people to talk with can be for 
\hyperref[sec:socializing]{social engagement} (see more on 
page~\pageref{sec:socializing}) 
or to discuss tasks you are working on. 


%The \href{https://en.wikipedia.org/wiki/Allen_curve}{Allen curve}; 

% TODO: Presence creates priority. (already stated elsewhere in ths book)


% https://graphthinking.blogspot.com/2016/04/role-models-for-great-leaders.html
In the American Revolution, \href{https://en.wikipedia.org/wiki/Friedrich_Wilhelm_von_Steuben}{Baron von Steuben}
\index{Wikipedia!\href{https://en.wikipedia.org/wiki/Friedrich_Wilhelm_von_Steuben}{Baron von Steuben}}
and 
\href{https://en.wikipedia.org/wiki/George_Washington}{George Washington}
\index{Wikipedia!\href{https://en.wikipedia.org/wiki/George_Washington}{George Washington}}
both did walk-around meetings to understand needs of the members of their army\footnote{\href{https://www.battlefields.org/learn/articles/winter-valley-forge}{https://www.battlefields.org/learn/articles/winter-valley-forge}: ``Like Steuben, Lafayette engaged directly with his soldiers and became well known for enduring the same hardships as his men''}.
