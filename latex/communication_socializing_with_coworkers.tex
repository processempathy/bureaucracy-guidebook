\section{Socializing with Coworkers\label{sec:socializing}}

interact outside of work

team building:
People not like you (different experiences and motives) outnumber the people like you.
Understanding how people not like you think and behave improves your chance of success regardless of the specific issue.

% https://graphthinking.blogspot.com/2020/07/how-to-have-useful-conversation.html
Seek to understand the other person's
\begin{itemize}
\item Interests
\item Reference experiences
\item Paradigm or perspective
\item Default behavior
\item Internal inconsistencies
\item Strengths and weaknesses
\end{itemize}

Building a \href{https://en.wikipedia.org/wiki/Theory_of_mind}{theory of mind}
\index{Wikipedia!\href{https://en.wikipedia.org/wiki/Theory_of_mind}{theory of mind}}
is not pure curiosity. The value of this work is that you can leverage this insight. For example, you can present explanations or a set of choices in a way that is more likely to be received as you intended and is more efficiently conveyed.  

If you are aware of the above context for one or more of the participants, you can help translate perspectives and you can mediate discussions more effectively. 

% https://graphthinking.blogspot.com/2016/01/the-value-of-small-talk.html
Occasionally talking about trivial things (``small talk") is important for relationships.
The other option is to not talk until there's a need to, typically about something big. Then there's a larger emotional barrier and no reference experience to transition from.

Talking frequently about trivial things demonstrates that you are thinking about the other person, and that they are important to spend time on.
Talking about small things enables the conversation to have an ability to transition to important topics which require emotional engagement.

Role of Small talk
\begin{enumerate}
\item not just interested in the important highlights. Also show interest in what might seem to you like the minutiae. 
\item logistics is not the focus, but can lead to emotional and motivation insights. enables building a model of how they think/act
\item comparison of experience - leverage the difference
\item the value is mutual insight into a different lifestyle and set of choices
\end{enumerate}

What can I learn from this interaction?
What is the person's passion?
What is the emotional response to various stimulus?


