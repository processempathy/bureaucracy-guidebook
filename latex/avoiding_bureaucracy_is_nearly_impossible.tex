\section{Avoiding bureaucracy is nearly impossible}

The only situation where bureaucracy might not exist is if you live completely on your own, with no interaction with other people. That means completely disengaging with society. Even then, personal routines are a self-imposed bureaucracy.


For the rest of us, as members of a society bureaucracy is necessary for our rights. We prove our name by cooperating with other people, and our name helps us claim our citizenship. That's a subjective policy that \glspl{stakeholder} in society agree to. 

The specific way a society is constructed (democratic, authoritarian, dictatorship, anarchy) is irrelevant -- bureaucracy is still present. Even the libertarian view of relying on contract enforcement implies some about of bureaucracy. 


Not all bureaucracy is due to the state, nor is bureaucracy confined to companies. Parenting involves coming up with situation-specific requirements for children, with the organization being the family. Dress codes for team are arbitrary standards inconsistently enforced. 
Store clerks are bureaucrats, as are website forum moderators.  Content moderation is the process of inconsistently enforcing arbitrary standards. This mindset even permeates individuals as internalized expectations of policy and enforcement when no one else is present. 

Recognizing instances of bureaucracy enables more skillful interaction, whether as a bureaucrat or as a subject. The remainder of this section  illustrates both the view of a person interacting with bureaucracy as a \gls{subject}, and the perspective of bureaucrats working within organizations. 




