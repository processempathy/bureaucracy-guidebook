\section{Avoiding Bureaucracy is Nearly Impossible}
\iftoggle{shortsectiontitle}{\sectionmark{Avoiding Bureaucracy}}{}

The only situation where bureaucracy might not exist is if you live entirely independently and have no interaction with other people. That means completely disengaging from society. Even then, personal routines are a self-imposed form of bureaucracy, with the roles of policy maker, bureaucrat, and subject collapsed to a single person -- you.

Self-sufficiency and autonomy are attractive alternatives to bureaucracy. The way participants in modern society strive for self-sufficiency is by denying their dependence on modern society. That's a relabeling of selfishness which feels better. 

For the rest of us who operate as members of  society, bureaucracy is necessary for our rights. You validate your name using paperwork, forms, and records. These artifacts are used, in cooperation with other people, to determine your claim of citizenship and associated rights. 
That's a subjective policy that \iftoggle{glossarysubstitutionworks}{\glspl{stakeholder}}{stakeholders} in society agree to. 


Bureaucracy is necessary because it is a response to the \iftoggle{WPinmargin}{\marginpar{$>$Wikipedia: Collective action problem}}{}%
\href{https://en.wikipedia.org/wiki/Collective_action_problem}{collective action problem} -- everyone would benefit from cooperation, but each person has to sacrifice their self-interests.  
\index{Wikipedia!collective action problem@\href{https://en.wikipedia.org/wiki/Collective_action_problem}{collective action problem}}
As long as humans form communities, we will address the challenge of \iftoggle{glossarysubstitutionworks}{\glspl{shared resource}}{shared resources}
 -- whether tangible (e.g., water, land, air) or intangible (e.g., expertise, information). 
That is why bureaucracy is culturally invariant and persistent across time.
Learning to be an effective bureaucrat improves your chances of success in society. 

The specific way society is constructed (democratic, authoritarian, dictatorship, anarchy) is irrelevant -- bureaucracy is still present. Even the libertarian approach of relying on contract enforcement implies some bureaucracy (e.g., forums for resolving contract disputes like a court system, enforcing decisions through violence). 


Not all bureaucracy is due to the state, nor is bureaucracy confined to companies. Parenting involves creating situation-specific requirements for children, with the organization being the family as mentioned in the section 
on \hyperref[sec:bureaucracy-early-childhood]{bureaucracy in childhood}.
\marginpar{See page~\pageref{sec:bureaucracy-early-childhood}.}%
Dress codes for sports teams are arbitrary standards. 
Store clerks are bureaucrats, as are website forum moderators.  Content moderation is the process of (inconsistently) enforcing arbitrary standards. This mindset even permeates individuals as internalized expectations of policy and enforcement when no one else is present. 

Recognizing instances of bureaucracy enables more skillful interaction, whether as a bureaucrat or subject. The rest of this section illustrates both the view of a person interacting with bureaucracy as a \gls{subject} and the perspective of bureaucrats working within organizations. 




