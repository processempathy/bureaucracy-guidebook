\section{Avoiding Bureaucracy is Nearly Impossible}

The only situation where bureaucracy might not exist is if you live completely on your own and have no interaction with other people. That means completely disengaging from society. Even then, personal routines are a self-imposed form of bureaucracy, with the roles of policy maker, bureaucrat, and subject collapsed to a single person -- you.

Self-sufficiency and autonomy are attractive alternatives bureaucracy. The way participants in modern society strive for self-sufficiency is by denying their dependence on modern society. That's a relabeling of selfishness which feels better. 

For the rest of us who operate as members of a society, bureaucracy is necessary for our rights. You validate your name using paperwork and forms and records. These artifacts are used, in cooperation with other people, to determine your claim citizenship and associated rights. That's a subjective policy that \glspl{stakeholder} in society agree to. 

The necessity of bureaucracy, and the reason it is culturally invariant and persistent across time, is because it is a response to the 
\href{https://en.wikipedia.org/wiki/Collective_action_problem}{collective action problem}. 
\index{Wikipedia!\href{https://en.wikipedia.org/wiki/Collective_action_problem}{collective action problem}}
As long as humans form communities, we will need to address the challenge of \glspl{shared resource} -- whether tangible (water, land, air) or intangible (expertise). Therefore, learning how to be an effective bureaucrat improves your chances of success. 

The specific way a society is constructed (democratic, authoritarian, dictatorship, anarchy) is irrelevant -- bureaucracy is still present. Even the libertarian approach of relying on contract enforcement implies some amount of bureaucracy (e.g., forums for resolving contract disputes like a court system). 


Not all bureaucracy is due to the state, nor is bureaucracy confined to companies. Parenting involves coming up with situation-specific requirements for children, with the organization being the family as mentioned in section 
on \hyperref[sec:bureaucracy-early-childhood]{bureaucracy in childhood}.
%~\ref{sec:bureaucracy-early-childhood}. 
Dress codes for sports teams are arbitrary standards. 
Store clerks are bureaucrats, as are website forum moderators.  Content moderation is the process of (inconsistently) enforcing arbitrary standards. This mindset even permeates individuals as internalized expectations of policy and enforcement when no one else is present. 

Recognizing instances of bureaucracy enables more skillful interaction, whether as a bureaucrat or as a subject. The rest of this section  illustrates both the view of a person interacting with bureaucracy as a \gls{subject}, and the perspective of bureaucrats working within organizations. 




