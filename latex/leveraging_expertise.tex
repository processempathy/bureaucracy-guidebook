\subsection{Leveraging Expertise\label{sec:expertise}}

Expertise is a resource administered by the bureaucracy.

*************

Don't measure things otherwise you'll be held accountable for them.

This is related to \href{https://en.wikipedia.org/wiki/Goodhart\%27s_law}{Goodhart's law} that ``when a measure becomes a target, it ceases to be a good measure.''

*************

In the following two sections two viewpoints are described: that of an expert providing help and that of a bureaucrat needing expert help. 

\subsubsection{For experts looking to help non-expert decision makers}

Non-experts think and speak less precisely

*************

Evaluation of expertise:
* do people defer to the candidate because of the candidate's title, the candidate's experience, or the candidate's knowledge of technical details (aka expertise)?
* when asked about the topic, does the candidate respond defensively (emotionally) or with an anecdote or with a direct answer (logical)?
* does the candidate use the relevant jargon for the topic?
* does the candidate use the jargon for the topic in the right way?
* can the candidate explain the topic with minimal use of jargon?
* does the candidate know the history of the topic? The important people in the field? The history matters for understanding the evolution of mistakes and innovations of the topic.
* does the candidate know who the stakeholders are for the current situation? 

*************

% https://graphthinking.blogspot.com/2012/05/notes-on-consulting-experience.html
You, the expert, may not win any debates based on number of years of experience. Some audiences may have many years in the field. However, expertise (rather than experience) is what they can benefit from. 

Identify which party (you, the expert, or the audience) has broader diversity of situations.

respect and account for the expertise of the audience (especially when they work in a different field). Acknowledge your own limits.

Don't rely completely on expertise as a single dimensional relation. Find ways to relate personally to audience members. They will need to build trust in you personally.

Facilitate the audience setting the goals. They should be able to identify how success is defined.

Identify strengths your audience has and build on those.

You don't have to teach a group all-at-once. One-on-one teaching is more customizable, and the audience may feel less worried about asking dumb questions.

Once you've taught more than one person, they can help each other translate the skill. 


\subsubsection{For non-experts looking for expertise}

Experts may lack experience communicating their knowledge. 

Experts are cautious to include caveats and limitations

Translation process is necessary from "here's what's known in general" + "situation specific facts" to "plan of (in)action" and what's realistic with what confidence. 

Experts straying outside their expertise are dangerous because the boundary may not be clear to the audience.

*************

How to recruit an advisor if you don't have expertise yourself: triangulate. Seek guidance from your existing social network. Advisor should be opinionated (not meek) but adaptive to new information. 

*************

Being the sole subject matter expert in an organization can feel isolating. You are constantly educating others but the stakeholders are not expected to become your peers. Typically stakeholders want to know just enough to make a decision. 

You are unable to ask someone smarter than you. 

Membership in professional organizations

Cultivate your replacement. 