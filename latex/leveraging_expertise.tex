\section{Leveraging Expertise\label{sec:expertise}}


%**************************************
% Generic observations on Expertise
%**************************************

The definition of \gls{bureaucracy} is the management of shared resources on behalf of a community. Besides tangible items like land, water, and air, expertise is another shared resource. Finding people with expertise, vetting expertise, and deploying expertise for the benefit of a community are all critical skills for bureaucrats. 

Experts who accumulate knowledge and experience look for ways to apply their expertise. Finding the relevant decision-makers and providing context for decisions can feel satisfying when positive  outcomes benefit the community. 


In the following two sections address both viewpoints: that of a bureaucrat seeking expert help, and that of an expert providing help. Bridging the cultural divide and knowledge gap in both directions is necessary for the success of an organization.

\subsection*{Non-experts seeking Experts}
%*******************************************************
% Advice for Bureaucrats collaborating with Experts
%*******************************************************

Before you seek expert help, your first instinct may be to try and proceed on your own. \textit{How hard could this be? Looks straightforward enough...} That is a plateau you can remain on for an entire career. Or perhaps you encounter difficulties and realize your own limitations. You may seek help from a peer or a trusted consultant. Seeking help from those around you is a second plateau that may suffice for the issue you're responding to. 

Finding an expert can be difficult to distinguish from talking with non-experts who know something you don't. If you decide a challenge requires skills you don't have, anyone with those skills has more expertise than you. Expertise is relative. 

\subsubsection*{Finding Experts}

One reason you as a bureaucrat seek an expert is you don't have expertise yourself. (If your motivation is merely to seek an external scapegoat that can be blamed, then getting any expert that appears convincing will suffice.)

To find a relevantly qualified expert, you can triangulate. 
\marginpar{[Tag] Actionable Advice}
Seek guidance from your existing social network. Multiple referrals from independent parties are a positive indicator. 

Once you've identified a potential expert, do people defer to the candidate because of the candidate's title, the candidate's position in the hierarchy, the candidate's experience, or the candidate's knowledge of technical details? Those can each qualify someone as an expert, but the title and position are least-well coordinated with expertise. Even experience can be misleading: ten years of doing a job can sometimes mean just doing the same first-year activities for ten years. %Formal education can serve as a signal of expertise, but practical experience is 

Within an area of expertise there may be sub-disciplines that you are not familiar with. Don't be embarrassed to ask the expert you find what subfield they have experience in, and their evaluation of how well that aligns with the effort you're recruiting them for.

If you're tackling a complex challenge, you may need experts from different disciplines. This introduces added challenges of cross-discipline communication. 



%\begin{itemize}
%    \item When asked about the topic, does the candidate respond defensively (emotionally) or with an anecdote or with a direct answer (logical)?
    %\item Does the candidate know the history of the topic? The important people in the field? The history matters for understanding the evolution of mistakes and innovations of the topic.
 %   \item Does the candidate know who the stakeholders are for the current situation? 
%\end{itemize}


\subsubsection*{Communicating with Experts}
Once you find a candidate expert, you'll need to evaluate their expertise and the ability to collaborate. This is inherently difficult since you lack the expertise you are seeking. 

When signaled solely by credentials, expertise is independent of experience. The expert may lack ability to communicate their knowledge with people outside their domain. A common barrier is use of jargon by experts.

Jargon can serve two purposes: to signal group membership (which is harmful to people outside the group); and to be more precise about a concept (desirable). An pseudo-expert may use jargon relevant to a topic but not be able to explain the concept to you.

As a way of distinguising people who know more than you do, consider the following levels of expertise.
\begin{enumerate}
    \item Self-designated expert is unfamiliar with domain-specific jargon.
    % Does the candidate use the relevant jargon for the topic?
    \item Pseudo-expert can use jargon in discussion but cannot expand the definition or provide analogs.
    % Does the candidate use the jargon for the topic in the right way?
    \item Pseudo-expert has memorized definitions and has analogies but doesn't grasp underlying relations and principles. Boundaries of what to apply where are uncertain.
    \item Expert can map situation-of-interest to underlying relations and principles, then use that mapping to remedy the issue (or explain why a fix is infeasible). The result works and the solution wasn't obvious to non-experts.
    % Can the candidate explain the topic with minimal use of jargon?
\end{enumerate}

%Experts are cautious to include caveats and limitations.

%\subsubsection*{What to use an Expert For}

%Translation process is necessary from ``here's what's known in general" + ``situation specific facts" to ``plan of (in)action" and what's realistic with what confidence. 

%Experts straying outside their expertise are dangerous because the boundary may not be clear to the audience.


%***************************************************
% Advice for Experts collaborating with Bureaucrats
%***************************************************

\subsection*{For Experts looking to help non-expert decision-makers}

Providing help to decision-makers can be an emotionally rewarding application of your skills. The interaction with bureaucrats can also feel devastating when you aren't listened to, you aren't understood, or your input is perverted in ways you didn't want. %The relationship you have with the decision-maker 
Identify strengths your audience has and build on those.

%\subsubsection*{Finding the relevant person to provide advice to}

%\subsubsection*{How to talk with non-experts}

%Coherence of data can lead to increased confidence about the results.
%Having less data may be more coherent. 

%More data is likely to be less consistent and therefore lowers confidence.

%Non-experts think and speak less precisely.

%You will need to simplify in the sense of applying to the relevant decisions - funding, staffing, planning.


% https://graphthinking.blogspot.com/2012/05/notes-on-consulting-experience.html
You, the expert, may not win any debates based on number of years of experience. Some audiences may have many years in the field. However, expertise (rather than experience) is what they can benefit from. 

Identify which party (you, the expert, or the non-expert audience) has broader diversity of situations.

Respect and account for the expertise of the audience (especially when they work in a different field). Acknowledge your  limits.

Don't rely completely on expertise as a single dimensional relation. Find ways to relate personally to audience members. They will need to build trust in you personally.

Facilitate the audience setting the goals. They should be able to identify how success is defined.


You don't have to teach a group all-at-once. One-on-one teaching is more customizable, and the audience may feel less worried about asking dumb questions.

Once you've taught more than one person, they can help each other translate the skill. 



\subsubsection*{Advice for Experts}

Being the sole subject matter expert in an organization can feel isolating. You are constantly educating others but the stakeholders are not expected to become your peers. Typically stakeholders want to know just enough to make a decision. 

You are unable to ask someone smarter than you. Membership in professional organizations.

% TODO
Cultivate your replacement. 
\marginpar{[Tag] Actionable Advice}