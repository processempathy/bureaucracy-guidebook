\subsection*{Fundamental: Decision-making in a Bureaucracy\label{sec:decision-making}}

Decisions are central to the operation of a bureaucratic organization. In this book we explore the topic of decisions, but decisions are not the only source of change in an organization. Occasionally events unfold without decisions being made either because the decision-makers are not informed or there is  \href{https://en.wikipedia.org/wiki/Willful_blindness}{intentional neglect}. 
\index{Wikipedia!\href{https://en.wikipedia.org/wiki/Willful_blindness}{willful blindness}}\iftoggle{WPinmargin}{\marginpar{[Wikipedia] Willful\\blindness}}{}
This book focuses on situations where bureaucrats recognize the need for a decision and want to make the best decision.

There are multiple types of decisions. 
A \gls{simple decision} \iftoggle{glossaryinmargin}{\marginpar{[Glossary]}}{}
has one correct or beneficial choice and one or more wrong or harmful choices. The work of decision-making is then to gather information that identifies the correct or beneficial choice and select that option.

The best case scenario for any decision-making is one person making a well-informed, simple decision that has immediate consequence and the consequence is to the decision-maker. Examples from elementary school include arithmetic math problems, multiple-choice quizzes, spelling tests, and memorization tests. A bureaucrat's \href{https://en.wikipedia.org/wiki/Moral_injury}{moral injury}
\index{Wikipedia!\href{https://en.wikipedia.org/wiki/Moral_injury}{moral injury}}\iftoggle{WPinmargin}{\marginpar{[Wikipedia] moral\\injury}}{}
comes from decision-making that involves multiple people, weak feedback loops with high latency, and complex decisions with multiple objectives.

\subsubsection{Example Decision Method: Pareto Frontier\label{sec:pareto}}

A complex decision may have many choices, and there may not be a best option. There are formal techniques for navigating the process of arriving at a result. One approach for describing the situation is a \href{https://en.wikipedia.org/wiki/Pareto_front}{Pareto frontier}. 
\index{Wikipedia!\href{https://en.wikipedia.org/wiki/Pareto_front}{Pareto frontier}}\iftoggle{WPinmargin}{\marginpar{[Wikipedia] Pareto\\frontier}}{}
If a Pareto frontier exists, then you can evaluate trade-offs. 

As an example of a complex decision made by one person with immediate consequence and direct relevance to the decision-maker, suppose you want to buy a car. You care about only two aspects: fuel efficiency and cost. See Figure~\ref{fig:pareto_frontier_cars} for an example of the Pareto frontier.

\begin{figure}[ht]
    \centering
    \includegraphics[width=1\textwidth]{images/pareto_frontier_car_options.pdf}
    \caption{Four cars: L, M, N, and P. The buyer's goal is to spend less money (lower on the vertical axis) and get better fuel efficiency (right on the horizontal axis). Choices not on the frontier should be avoided, but that doesn't yield a single result.}
    \label{fig:pareto_frontier_cars}
\end{figure}

Visualizing a Pareto frontier for two quantitative variables is easy, but typically decisions involve more factors. For example, evaluating the trade-off of three quantitative variables like 
passenger capacity, cost, and fuel efficiency creates a surface. With more than three variables visualization is less useful, though the analysis technique still applies. 

Another constraint on using Pareto frontier analysis is that it works well when there are many options relative to the number of variables you are optimizing for. 
The assessment does not work well when there are few choices relative to the number of variables. For example, suppose there are five choices of car and you want high fuel efficiency, sufficient cargo capacity, maximum number of passengers, stylish, low cost, low maintenance, good durability, and high resale value. Then defining a Pareto frontier is less helpful.

For a set of quantitative variables, a Pareto frontier does not account for the relative importance of different variables. Assigning weights to each of these factors merely stretches one axis relative to the other axes. 

There are many possible %decision-making 
% 2023-01-11: Brian Black says to remove redundancy
frameworks besides Pareto frontiers, but in practice a typical bureaucratic decision is ill-informed, has diffuse consequences, delayed impact, and does not affect the decision-maker. In bureaucratic processes there is rarely a formal assessment of options. 
Decisions are rarely recorded. 
Even afterward a decision can be difficult to evaluate for correctness because there are multiple stakeholders.

\textit{Tip}: If the people you want to convince are not swayed by data, then you can augment the \href{https://en.wikipedia.org/wiki/Cost\%E2\%80\%93benefit_analysis}{cost-benefit analysis} with stories. 
\index{Wikipedia!\href{https://en.wikipedia.org/wiki/Cost\%E2\%80\%93benefit_analysis}{cost-benefit analysis}}
\iftoggle{WPinmargin}{\marginpar{[Wikipedia] cost-\\benefit analysis}}{}

\subsubsection{Risks of Using Decision Frameworks}

%Decision-making 
% 2023-01-11: Brian Black says to remove redundancy
Frameworks can be attractive to bureaucrats intending to formalize 
\hyperref[sec:process]{processes} 
\marginpar{See page~\pageref{sec:process}.}
%(see section~\ref{sec:process}) 
and encourage predictability. There are potential risks worth being aware of.  Cost-benefit analysis can decrease the apparent responsibility of the decision-maker. Cost-benefit analysis can be used to deflect criticism for an action. 

% https://graphthinking.blogspot.com/2019/01/political-decisions-versus-science.html
A decision is political when the basis is historical relationships, maintenance or creation of a relationship, or to enable future relationships. A decision is subjective when someone else faced with the same scenario would have come to a different conclusion.
A decision is quantitative when based on measurements. To avoid the appearance of subjective decision-making or political decision-making, a bureaucrat may frame a decision as ``data-driven." 
% https://graphthinking.blogspot.com/2018/06/data-driven-decisions-versus-data.html
A good approach for data-driven quantitative analysis involves coming up with a testable hypothesis, pre-registering what actions are to be taken after assessing the results, then performing experiments and collecting data to evaluate the hypothesis. 

More commonly, a decision is made, then data is gathered which supports the desired outcome. Forming an opinion and then looking for evidence to back the outcome yields suboptimal results for the organization. The value to the individual using that suboptimal approach is extracted from other members of the organization or subjects of the bureaucracy. 

Even when a bureaucrat is not intentionally biased towards an outcome, there are many ways to gather evidence. Some approaches have biased sampling and produce biased results.

Even with valid and representative data measurement, decision-makers can be led astray by poor modeling. A model may use inapplicable techniques or may have implementation bugs.

For all the dangers described above,
%of decision-making methods, 
% 2023-01-11: Brian Black says to remove redundancy
there are worse approaches that do not rely on measurement. People rely on history (if they are aware of it) and perpetuate bad ideas, or take action based on what is best for their career, or decide based on how to accumulate more power, or choose based on what someone else says to do.  

%\subsubsection{Alternative Decision-Making Approaches}
% 2023-01-11: Brian Black says to remove redundancy
\subsubsection{Alternative Approaches}
Identifying the spectrum of ways bureaucrats make decisions can decrease the surprise when you encounter the approaches in practice. 
\index{decrease surprise!decision-making techniques}
Enumerating the choices of how to make a decision allows for conversation about the pros and cons of each. The spectrum below is ordered from least effective to most effective and is not exhaustive.
\begin{enumerate}
    \item Random selection from the options (e.g., roll of the dice, shake the Magic 8 ball shown in figure~\ref{fig:magic8ball}).
    \item Rely on a hunch.
    \item Rely on experience, whether yours or someone else's.
    \item Use a \href{https://en.wikipedia.org/wiki/Cost\%E2\%80\%93benefit_analysis}{cost-benefit model}, 
    \index{Wikipedia!\href{https://en.wikipedia.org/wiki/Cost\%E2\%80\%93benefit_analysis}{cost-benefit analysis}}
    \iftoggle{WPinmargin}{\marginpar{[Wikipedia] cost-\\benefit analysis}}{} gather data, and analyze.
    \item Make a hypothesis, design an experiment, carry out a test, collect data, and analyze.
\end{enumerate}

\begin{figure}
    \centering
    \includegraphics[width=0.4\textwidth]{images/magic8ball.pdf}
    \caption{Magic 8 ball for decision-making. The least effective method of selecting an option.}
    \label{fig:magic8ball}
\end{figure}

\subsubsection{Why cost-benefit analysis models are not used}

Applying quantitative models to decision-making may seem a valuable path to optimal outcomes. Improved decision-making that helps stakeholders seems attractive. There are various reasons this approach is not taken. I outline the reasoning here not out of cynicism, but so that you can appropriately respond when confronted with arguments against \href{https://en.wikipedia.org/wiki/Cost\%E2\%80\%93benefit_analysis}{cost-benefit modeling}.\iftoggle{WPinmargin}{\marginpar{[Wikipedia] cost-\\benefit modeling}}{}
\index{Wikipedia!\href{https://en.wikipedia.org/wiki/Cost\%E2\%80\%93benefit_analysis}{cost-benefit analysis}}
\index{decrease surprise!reasons not to use cost-benefit modeling}
Decreasing your surprise through the inoculation of exposure enables you to develop counterarguments. 

One reason to avoid cost-benefit modeling is that there is a lot of work to do, so coming up with a model is an \href{https://en.wikipedia.org/wiki/Opportunity_cost}{opportunity cost}. 
\index{Wikipedia!\href{https://en.wikipedia.org/wiki/Opportunity_cost}{opportunity cost}}\iftoggle{WPinmargin}{\marginpar{[Wikipedia] opportunity\\cost}}{}
Modeling every decision in detail is unrealistic, so triage is needed -- which decisions warrant what degree of investment in analysis. A person inexperienced in modeling decisions will take longer and is more likely to create a bad model. Practice designing models is expensive. Identifying relevant parameters for inclusion in a model is a learnable skill. 

If a person has a bad experience with cost-benefit modeling, they may be less likely to try again with other decisions. Bad experiences can come from using incorrect data, missing a critical variable, picking the wrong scope for a question, implementing the right model with the wrong math, misinterpreting the result, using a model that is too in-depth or not sufficiently deep, or failing to sell the result of an analysis to stakeholders.

Quantitative cost-benefit modeling doesn't account for personalities or organizational politics. The best options may decrease the decider's power or prestige, or harm relationships. 

Measurements are needed to inform a cost-benefit analysis. In addition to measurement being costly, measurements get perverted -- see \href{https://en.wikipedia.org/wiki/Campbell\%27s_law}{Campbell's law}. 
\index{Wikipedia!\href{https://en.wikipedia.org/wiki/Campbell\%27s_law}{Campbell's law}}
\marginpar{$>>$ Folk Wisdom}
\index{folk wisdom!\href{https://en.wikipedia.org/wiki/Campbell\%27s_law}{Campbell's law}}

\subsubsection{Decision-Making Delay\label{sec:decision-delay}}

From an outsider's view, what appears as ``organizational inertia'' is the delay of internal decision-making and the delay of dissemination. 
Delay comes from:
\begin{itemize}
    \item Time used by each decision-maker to gather information, arrive at a decision, change processes, share their choice, and justify their choice. 
    %\item Forcing a continuous variable into a discrete set of choices. Typically the number of choices is small. Discrete choices for a continuous variable is a loss of effectiveness.
    % https://dynomight.net/teaching/
    \item Processes designed to detect and counter cheaters and people with malicious intent, whether that means a malicious bureaucrat or malicious subject. 
\item \href{https://en.wikipedia.org/wiki/Analysis_paralysis}{Analysis paralysis} 
\index{Wikipedia!\href{https://en.wikipedia.org/wiki/Analysis_paralysis}{Analysis paralysis}}\iftoggle{WPinmargin}{\marginpar{[Wikipedia] Analysis\\paralysis}}{}
due to insufficient information, too much information, and lack of clarity about which framing is applicable.
\item When other people who are needed to carry out the action push back, either in disagreement or seeking clarification. If the impact on profit doesn't exist or is indirect, then justification for action isn't quantitatively obvious. Therefore there's a higher burden for communication.
\end{itemize}


% The following characterization has no consequence
% Decision-making by bureaucrats can be informal or formal, consensus-based or solo. 


\subsubsection{A Bureaucratic Decision involves many Decisions}

A decision is a collection of interdependent choices. After recognizing the need for a decision, follow-on decisions include identifying the stakeholders (who to include in an analysis) and identifying options. Who is a stakeholder and what options are available are interrelated. Involving more people expands the number of options and the complexity of coordination. Additional choices associated with reducing uncertainty are how much time to spend on the decision, how much information to gather for the decision (see Dilemma~\ref{table:dilemma-personal-gather-data-lots-vs-little}),
\marginpar{See page~\pageref{table:dilemma-personal-gather-data-lots-vs-little}.}
whether to make the decision or push the decision to someone with more expertise (see Dilemma~\ref{table:dilemma-personal-scope-of-speaking}), or whether to push the decision to someone with more exposure to the consequences.

Most decisions you make as a bureaucrat do not have hard deadlines. Instead, there are trade-offs in the allocation of your attention. Sooner is preferable since the consequence of the decision helps the organization and allows you to focus on other tasks, but delaying allows you to gather more information for a better-informed decision. See 
Dilemma~\ref{table:dilemma-personal-gather-data-lots-vs-little}
\marginpar{See page~\pageref{table:dilemma-personal-gather-data-lots-vs-little}.}
and \hyperref[sec:dilemma-trilemma]{other related Dilemmas}.


If a bureaucrat relies on consulting an
\hyperref[sec:expertise]{expert},
\marginpar{See page~\pageref{sec:expertise}.}
%(see section~\ref{sec:expertise} on expertise),
the decision-maker needs to be confident the expert is not  straying outside their area of expertise. For example, I don't rely on a botanist  to tell me how to change the oil in my car. 
Besides knowing their limitations, the expert should be clear about whether their input is a factual summary, a predictive assessment, or a value judgment. This complicates what you as a bureaucrat are interested in when deciding what's the best choice.


\subsubsection{Transparency of Decision-Making\label{sec:transparency-of-decisions}}

Your %decision-making 
% 2023-01-11: Brian Black says to decrease redundancy
process might change if made visible to stakeholders. You may provide more precise justifications or spend more time gathering evidence to support a claim. Strengthening the arguments is beneficial but takes time and resources. 

Frank conversations and exploratory brainstorming need protections to allow participants to be vulnerable. 
This is the motive for the \href{https://en.wikipedia.org/wiki/Chatham_House_Rule}{Chatham House Rule}. 
\index{Wikipedia!\href{https://en.wikipedia.org/wiki/Chatham_House_Rule}{Chatham House Rule}}\iftoggle{WPinmargin}{\marginpar{[Wikipedia] Chatham\\House Rule}}{}
The concept of transparency can apply before a decision is issued, while a policy is in effect, or after the consequence (as part of a review). 
Making decisions as part of a transparent process can make participants more risk-averse because of the potential for failure.

People affected by the decisions benefit from understanding how the decisions were made -- see examples like 
\href{https://en.wikipedia.org/wiki/Government_in_the_Sunshine_Act}{Sunshine laws} and
\index{Wikipedia!\href{https://en.wikipedia.org/wiki/Government_in_the_Sunshine_Act}{Sunshine Act}}
\href{https://en.wikipedia.org/wiki/Freedom_of_Information_Act_(United_States)}{FOIA}. 
\index{Wikipedia!\href{https://en.wikipedia.org/wiki/Freedom_of_Information_Act_(United_States)}{Freedom of Information Act}}\iftoggle{WPinmargin}{\marginpar{[Wikipedia] Freedom\\of Information Act}}{}
Even the symbolism of surveillance changes behavior.\footnote{K.~J.~Haley and D.~Fessler (2005). ``Nobody’s watching? Subtle cues affect generosity in an anonymous dictator game.'' Evolution and Human Behavior, 26, 245–256.\\
T.~Burnham and B.~Hare (2007). ``Engineering human cooperation.'' Human Nature, 18, 88–108.}


% TODO: TRANSITION to hierarchy

