\section{Processes within a Bureaucracy}

Processes are designed to limit bad behavior or to allocate scarce resources

\href{https://en.wikipedia.org/wiki/Tragedy_of_the_commons}{Tragedy of the Commons} says that when there is a shared resource, someone will try to get away with behavior that is harmful to the organization.
Create processes for oversight/review/approval
Each process may be justifiable, but the aggregate is unreasonably burdensome


Processes with fewer people and fewer steps can be quicker and use fewer resources, but they are more fragile and less representative. Having more people involved helps with edge cases, but slows down the process. 

Any Nash equilibrium is constantly being upset by the change in conditions and change in people (who have varying motives).


    Processes can be undocumented. Then oral folklore is the mechanism. 
    
    Processes guardrails to prevent harm and keep stakeholders informed. 
    
    Without oversight processes, \href{https://en.wikipedia.org/wiki/Tragedy_of_the_commons}{tragedy of commons} occurs and malicious actors dominate.
    


\begin{figure}
    \centering
    \includegraphics{images/process_loop_harm-oversight-improvement}
    \caption{loop: harm-oversight-improvement}
    \label{fig:harm-oversight-improvement}
\end{figure}


