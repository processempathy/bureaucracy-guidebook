\subsection{Motivation of participants}

% see https://en.wikipedia.org/wiki/Social_influence

% https://graphthinking.blogspot.com/2020/02/there-is-no-idle-status-for-paid.html
In an organization where you are a paid bureaucrat, you are either actively working for improvement of the organization, or your existence is parasitic to the organization. There is no "idle" status for paid employees.

Motivations of participants are rarely ``how can I make the company more successful" or even ``how can I sell/produce more product"? Usually motivation is based on personal success in various manifestations, which leads to emergent phenomena which appears confounding to observers outside the bureaucracy. 

Not too efficient such that I lose my job, and not so inefficient that the organization fails and I lose my job. Increasing the efficiency of bureaucracy is good for the organization and the outcomes, but can be harmful to the bureaucrat's career.

Career stability is a benefit, and it can be leveraged to take more risk. However, it typically manifests as inaction by an employee. There's no harm to the employee in not taking action. If an employee doesn't do anything, nothing bad will happen to that employee. Career stability decreases extrinsic motivation.