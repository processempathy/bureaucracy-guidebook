\section{Motives of Bureaucrats\label{sec:motivations}}

Learning the diverse motives of bureaucrats you depend upon is instructive for finding causes of delay or opposition. If you expect everyone to have the same motives as you then you will be surprised by the friction created by diverse motives. 

Motivations of participants are rarely, ``How can I make the organization more successful?" or even, ``How can I sell or produce more product?" Usually motivation is based on personal fulfillment in various manifestations. Internal motives and external incentives can combine into emergent phenomena where teams and organizations seem to act in ways no individual member would. The dissonance of how personified organizations behave is confounding to observers outside the bureaucracy. If you were to talk to each member you could find a reason for that person's behavior. 


% see https://en.wikipedia.org/wiki/Social_influence

Each bureaucrat has a motive, even the bureaucrats who do nothing. Bureaucratic nihilism (showing up for your job but choosing to not provide value) is an option but it is not without cost.
% https://graphthinking.blogspot.com/2020/02/there-is-no-idle-status-for-paid.html
In an organization where you are a paid bureaucrat, you are either actively working for improvement of the organization, or your existence is parasitic to the organization. There is no ``idle" status for paid employees in an organization with limited resources.

The reason an employee may choose to stop contributing value is that they perceive that a lack of action will not harm their options. There is a perceived disconnect of their actions and the consequences for the organization. This is not a pure nihilism, as the person still gets benefit from showing up (usually a paycheck).

Rationalizing money as the cause for action is just one of many potential sources.
There are a variety of motivations and incentives for bureaucrats: 
stability (aka job security, the comfort of a routine),
money (current pay or future earnings), 
travel, 
problem solving, 
status, 
exerting power or control, 
credibility of being associated with the organization (if the organization has a positive reputation), 
logistical convenience (``the office is near where I live''), 
service to people the organization serves.


As an example incentive for a bureaucrat, I want to avoid being too efficient such that I eliminate the need for my job, 
\marginpar{[Tag]\href{https://en.wikipedia.org/wiki/Goldilocks_principle}{Goldilocks balance}}
and not so inefficient that the organization fails and I lose my job. Increasing the efficiency of bureaucracy is good for the organization and the outcomes, but can be harmful to the bureaucrat's career.


The consequence of diverse motives is that expecting bureaucratic organizations to be logical, fair, consistent, and efficient is unreasonable even when every participant wants those features. Each bureaucrat thinks, ``I am logical, fair, consistent, and efficient.'' Therefore each bureaucrat expects other bureaucrats to meet those same (unrealistic) standards. Next, anthropomorphize the team or organization and expect the group to meet those standards. 

Even if each bureaucrat were logical, fair, consistent, and efficient (they are not, and neither are you), each person has a different motivation. Internal motivation depends on what you value, in the sense of a
\index{Wikipedia!\href{https://en.wikipedia.org/wiki/Utility}{utility}}
\href{https://en.wikipedia.org/wiki/Utility}{utility}
\marginpar{[Wikipedia] Utility}
function.  Each person wants to accomplish something different using their unique skills and referencing their own experiences. Compounding the confusion, each bureaucrat has to coordinate using communication that has latency and limited bandwidth and isn't precise.

When viewed from the outside, an expectation that bureaucracy feels illogical, unfair, inconsistent, and inefficient is a useful baseline. From inside the bureaucracy, a baseline expectation is that every bureaucrat is driven by different motives and uses distinct ways to tackle challenges. Working against bureaucratic entropy (a ceaseless investment) yields improvements even though perfection is inaccessible.


\ \\

% TRANSITION to tropes

The above discussion might leave you with the impression that individual bureaucrats are distinct. While that's true, with a sufficient number of different people there are recurring themes which are detectable. The next section identifies common tropes you might encounter.