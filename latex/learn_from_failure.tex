\section{Learning from Failure\label{sec:learn-from-failure}}

This section on individual failure is separate from
%\ifsectionref
%section~\ref{sec:org-failure-and-success} on 
%\fi
\hyperref[sec:org-failure-and-success]{how an organization characterizes failure}\iftoggle{haspagenumbers}{ (see page~\pageref{sec:org-failure-and-success}).}{.}

\ \\

Making progress in a bureaucracy is not a linear sequence of steps. Ideally there is an ongoing cycle of trying something, not succeeding, and then applying what you learn towards the next try. This idealized ``\href{https://en.wikipedia.org/wiki/Fail-fast\%23Business}{fail fast}'' 
\index{Wikipedia!\href{https://en.wikipedia.org/wiki/Fail-fast\%23Business}{fail fast}}\iftoggle{WPinmargin}{\marginpar{$>$Wikipedia: Fail-fast}}{}
cycle does not occur naturally -- the participants have to either have intrinsic motivation or external incentives to make progress. Failing as a path for learning is justified after having exhausted less expensive alternative education opportunities like reading books, going to school, and talking with experts. 

The recognition of failure depends on a clear measurement. Do the participants know the measure, and can they make the measurement? Having a defined measure of failure and regularly making the measurement relies on having an understanding of the expectations for the situation. Expectations are assumptions about the future.

Assumptions about the future can be categorized as
\begin{itemize}
    \item Opinions. For example, the interpretation of policy. Policy interpretation can be bent or exceptions can be made. 
    \item Guesses; more formally interpolation -- ``Given multiple options, X seems most likely."
    \item Experience; more formally extrapolation -- ``Last time X happened, so next time ...''
\end{itemize}

Failure with respect to expectations implies thinking about the future so that you can measure your progress (or failure). That is a subset of planning. You can fail if you don't plan, and you can fail if you do plan. This does not indicate that planning is irrelevant. Plans can be overly detailed and prescriptive, or plans can be inadequately specified; both are unhelpful.

If your measurements indicate failure, this could be due to an invalid assumption (with a valid objective) or you may have selected a bad objective. Declaring failure means re-evaluating assumptions, resetting objectives, or giving up on the concept and doing something else.

\ \\

There are ways to avoid failing: by not setting an objective, or setting an objective that isn't measurable. You can avoid the appearance of failure by not telling anyone the objective. Extrinsic accountability for failure requires other people to have awareness. If these tactics are being employed, it may be because the person (or
%\ifsectionref
%, in section~\ref{sec:org-failure-and-success}, 
%\fi
\marginpar{See page~\pageref{sec:org-failure-and-success}.}
\hyperref[sec:org-failure-and-success]{the organization}) doesn't want to share how they've failed until they have a success. 

Understanding rationales like the failure avoidance mentality is critical if you want to avoid feeling baffled the first time you encounter it. By learning these bureaucratic anti-patterns, you can inoculate yourself emotionally and cognitively. You now have a chance to think ahead before encountering the explanation.

A counterargument when you hear that someone is avoiding failure is to explain that noticing failure is essential to improving. Detecting indicators of failure, addressing the cause, and then changing is the process of improvement. When someone wants to avoid failure, I don't understand how they expect to improve. 

