\section{Learning from Failure\label{sec:learn-from-failure}}

Framing of emotions is crucial for learning. Wisdom comes from experience comes from failure, but if you are afraid to fail you'll have fewer opportunities to learn. While you should learn from the experience of other people, you will also need to cope with the anxiety that can come with decision-making.
This section on individual failure is separate from
%\ifsectionref
%section~\ref{sec:org-failure-and-success} on 
%\fi
\hyperref[sec:org-failure-and-success]{how an organization characterizes failure}\iftoggle{haspagenumbers}{ (see page~\pageref{sec:org-failure-and-success}).}{.}


Making progress in a bureaucracy is not a linear sequence of steps. Ideally there is an ongoing cycle of trying something, not succeeding, and then applying what you learn towards the next try. This idealized ``\href{https://en.wikipedia.org/wiki/Fail-fast\%23Business}{fail-fast}'' 
\index{Wikipedia!fail-fast@\href{https://en.wikipedia.org/wiki/Fail-fast\%23Business}{fail-fast}}\iftoggle{WPinmargin}{\marginpar{$>$Wikipedia: Fail-fast}}{}
cycle does not occur naturally -- participants need either intrinsic motivation or external incentives to make progress. Failing as a path for learning is justified after having exhausted less expensive alternative education opportunities like reading books, going to school, and talking with experts. 

The recognition of failure depends on a clear measurement. Do the participants know the relevant measure? Can they make the measurement? Having a defined measure of failure and regularly making the measurement relies on having an understanding of the expectations for the situation. 
%Expectations are assumptions about the future.

Assumptions can be categorized as opinions, guesses, or based on experience. 
%\begin{itemize}
For example, opinions on the interpretation of policy. In that context policy interpretation can be bent or exceptions can be made. 
Guesses can be stated as, ``Given multiple options, X seems most likely." If someone else is making a guess, you can express curiosity about the basis for their reasoning. 
Lastly, experience can be extrapolated: ``Last time X happened, so next time...'' In this situation it can be worth asking others if they have similar or different experiences. 
%\end{itemize}

Failure with respect to expectations implies thinking about the future so that you can measure your progress (or failure). That is a subset of planning. You can fail if you don't plan, and you can fail if you do plan. This does not indicate that planning is irrelevant. Plans can be overly detailed and prescriptive, or plans can be inadequately specified; both are unhelpful.
\marginpar{\href{https://en.wikipedia.org/wiki/Goldilocks_principle}{Goldilocks principle}}%
\index{Goldilocks principle!level of detail for plan}%

If your measurements indicate failure, this could be due to an invalid assumption (with a valid objective) or you may have selected a bad objective. Declaring failure means re-evaluating assumptions, resetting objectives, or giving up on the concept and doing something else.

\ \\

There are ways to avoid failing: by not setting an objective, or setting an objective that isn't measurable. You can avoid the appearance of failure by not telling anyone the objective. Extrinsic accountability for failure requires other people to have awareness. If the criteria for failure are not known to other people, it may be because the person (or, \hyperref[sec:org-failure-and-success]{in the next section},
%\ifsectionref
%, in section~\ref{sec:org-failure-and-success}, 
%\fi
%\marginpar{Page~\pageref{sec:org-failure-and-success}.}%
the organization) doesn't want to share how they've failed until they have a success. 

Understanding rationales like the failure avoidance mentality is critical if you want to avoid feeling baffled the first time you encounter it. By learning these bureaucratic anti-patterns, you can inoculate yourself emotionally and cognitively -- there will be less surprise when you encounter it. You now have a chance to think ahead before encountering the situation.

A counterargument when you hear that someone is avoiding failure is to explain that noticing failure is essential to improving. Detecting indicators of failure, addressing the cause, and then changing is the process of improvement. Being more efficient involves decreasing incidents of failure, but that can also become brittle (not robust). When someone wants to avoid failure, how do they expect to find novel effective ways to do things? 

