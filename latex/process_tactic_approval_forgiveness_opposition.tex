\subsection*{Tactic: Approval, Forgiveness, Opposition\label{sec:approval-forgiveness-opposition}}
% https://graphthinking.blogspot.com/2017/10/flipping-approval-mentatlity.html

Bureaucracy involves distributed decision-making. 
A common bureaucratic task is seeking consensus regarding action or spending resources. There are distinct options for how to get that consensus:
\begin{itemize}
    \item Seek approval before taking action. This approach incurs both providing justification and waiting.
    \item Ask forgiveness after taking action. Often viewed as being in contrast to seeking approval. Less delay, and usually works if things go well or if no one notices. 
    \item Notification of Intent with a deadline for response. The window for a response should be sufficient to  allow feedback. If no response is provided default is for action to be taken.
    \item Solicit opposition before taking action~\cite{2009_Perr}. This is a different framing from approval or forgiveness. It decreases the risk the approver has to take on.
\end{itemize}
The best way to proceed depends on the personalities of the people involved in building consensus and their relationships. 

Most organizations default to approval-based  processes. Each new idea needs to be signed off as approved by a sequential list of 
\iftoggle{glossarysubstitutionworks}{\glspl{bureaucrat}}{bureaucrats}. The sequential (not concurrent) process may be known in advance, or it may be ad hoc if the request is novel.

Relying on approval is harmful to innovation because sign-off by each bureaucrat is interpreted as ``I am 100\% in agreement with this.'' Each stakeholder has to bless innovation and tie their reputation to the outcome.

Soliciting opposition and a response of ``I won't stop this'' is a more useful paradigm. With the consensus process language changed to ``I won't stop this," then the bureaucrat reviewing the idea can avoid taking responsibility for the idea and therefore is not tying their reputation to the result.

% https://news.ycombinator.com/item?id=15407757

%\ \\

% TRANSITION to next section: process_deployment
%The section above described tactics for changing processes when the changes are driven by changes to resources or staffing while assuming the underlying policy remained the same. The next section describes the case where changes to a policy require that processes be revised. 