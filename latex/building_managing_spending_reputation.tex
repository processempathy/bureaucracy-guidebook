\section{Building, Managing, and Spending Reputation\label{sec:reputation}}

% distinct from
%Social Capital https://en.wikipedia.org/wiki/Social_capital
%Political Capital https://en.wikipedia.org/wiki/Political_capital


%How does an individual create and accumulate political capital? What does political capital mean for teams?

This section is about actions you can take to create a reputation. Even if you don't intentionally build a reputation, you still get one because you interact with other people. Also in this section: how to spend political capital -- both yours and that of your team. 

\subsection*{Relevance of Reputation in a Bureaucracy}

As a bureaucrat, you may want to be all things to all people. Since that is not feasible, you have to limit your scope and therefore disappoint some people. This same dynamic of a limited scope applies to the team you are on, and again to the organization the team is part of. The purpose of a bureaucratic team or bureaucratic organization is to manage a \gls{shared resource} for a constituency. The constituency will invariably be disappointed because access to the shared resource is limited. 

The good news is that you can influence how others perceive your limitations. Their perception of your value is shaped by your investment in relationships and by your creative negotiations with other bureaucrats. %, other teams, and other organizations shape . 

\subsection*{Definition of Reputation}

Your \gls{reputation} as a bureaucrat is what other people expect from you. \iftoggle{glossaryinmargin}{\marginpar{[Glossary]}}{}
\index{reputation}
Reputation is perception. What does that person think of you? Your team? Your organization? 
Reputation is set whenever and where ever you are observed, or artifacts are associated with you. 
What you wear matters to how you are perceived. When you show up impacts your reputation. How your written communication is read alters perceptions. People are informed by your body posture in meetings. 

\subsection*{Reputation, Brand, Image}

There are multiple phrases that all refer to the same concept of being perceived and generating associated expectations. Individuals create (or get) a reputation; organizations have brands. A team of bureaucrats can have a reputation. 

\subsection*{Your Reputation matters}

Your reputation within the organization dramatically impacts your effectiveness. People will let you do things (or prevent you from doing things) based on their expectations of you. 

Your reputation alters your influence. Whether other people turn to you for input depends on what they expect from you. How others perceive you impacts what you can accomplish and when people seek your help or input.

\subsection*{You can Manage your Reputation}

Managing reputation means acknowledging that your interaction with others is partially performative. This may feel disappointing if you want to be judged solely on your productivity or knowledge. 

Your success is limited if you focus exclusively on doing the work, and your success is limited if you focus exclusively on performative aspects. 
Neglecting to manage your reputation means you lose input to the stories others tell about you. Active management of your reputation requires engaging with people and generating evidence. 

Your reputation is dynamically changing based on your activities and communication. Your communication matters, and the stories other people tell about you matter.

\subsection*{Techniques for Building Reputational Capital}

Ideally, your reputation would be based on your technical skills, your ability to collaborate with other people, the strength of your network, and your creativity. None of those matters if the person you're engaging with doesn't know those things. 
You build your reputation by doing things that are useful contributions, are visible to other people, and are associated with you (or your team or organization). This positive association is what you are creating.

Because the definition of reputation is about expectations other people have about you, what you choose to work on matters. How you work on your tasks (creativity, enthusiasm, dedication), the artifacts produced, and your ability to communicate all shape your reputation. 

When building your reputation, you can work on multiple small wins or take larger risks on bigger bets. The bigger bets provide faster leverage of reputational capital as you have shown your worthiness and skill. 

The same concepts apply to teams of bureaucrats and bureaucratic organizations. Just as individuals compete for resources, teams compete for budget, staffing, and glory. How the team's budget and staffing are spent impact reputation. 

A specific technique for building a good reputation is 
\hyperref[sec:credit-others]{giving others credit}. 
%(section~\ref{sec:credit-others}).
\marginpar{$>>$ Actionable Advice}
\index{actionable advice}
This may seem counter-intuitive if you are focused on assignment of credit. What  matters is that other people observe you (whether directly or indirectly) giving credit. 

Similarly, another technique for building good reputation is offering to \hyperref[sec:take-blame]{take blame}.
%(section~\ref{sec:take-blame}).
\marginpar{$>>$ Actionable Advice}
\index{actionable advice}
Again, this can seem counter-intuitive since being blamed doesn't sound good. The value is in proclaiming your willingness to be blamed such that other people observe your offer. 

\subsection*{Using Good News (or Early News) to build Influence}

Rather than tell good news directly to a top-level decision-maker, first inform the person who influences the decision-maker.
\marginpar{$>>$ Actionable Advice}
\index{actionable advice}
You can tell the influencer the information does not need to be credited to yourself.

The benefits of this approach include
\begin{itemize}
    \item Improves the influencer's reputation with the decision-maker.
    \item The influencer can contextualize the information for the decision-maker.
    \item The influencer can leverage the information in ways you would not have.
    \item My reputation with the influencer is enhanced.
    \item You reaffirm the influencer's role and status.
    \item Influencers can be easier to access.
\end{itemize}

\subsection*{How to Spend Reputation}

Based on your reputation (what the other person expects), what trust does that person have?  You can then use that trust to do things that might not otherwise be feasible. You can spend your reputation to bend  rules. 

Reputation is not the only thing you as a bureaucrat can spend. Other examples include your time (up to your entire career), your ability to be a member of the team or organization, your self-respect, and your ability to be promoted or receive bonuses. At the level of the team and the organization, additional aspects to spend are  budget and  staff. Each of these (reputation, time, membership, budget, staffing) are potential investments. 
%Confusingly, the investments are not independent. 

When considering what to spend to build reputation, consider what boundaries you have. 
Things you might not want to spend include your integrity and your health (physical, mental, emotional). Similar boundaries apply at the team and organization levels -- trust in the team, the well-being of team members. 


Whenever you engage within your team, you are either actively spending or building your reputation within the team.
Whenever you are engaging with people from outside your team, you are either actively spending or building your team's reputation.
Because of that, spending reputation means taking risks that involve other people.

\ \\

Examples of spending reputation and not getting a return on the investment:
\begin{itemize}
    \item Ask for a favor and provide no value.
    \item Explore options that other people don't see as worthwhile and then there's no payoff.
    \item Produce nothing of value to the organization or other people.
    \item Talk with people outside your team and misrepresent the efforts of your team.
    \item Speak honestly about the faults of your team to people outside your team.
\end{itemize}




%Internal-to-the-org there is cultural norms. 
% https://graphthinking.blogspot.com/2021/01/why-active-shaping-of-culture-is.html


% https://graphthinking.blogspot.com/2018/05/my-evolving-view-on-role-of-my.html


% the following article is useless
% https://www.indeed.com/career-advice/career-development/build-a-reputation
% since it reduces to "be a good person"


% https://graphthinking.blogspot.com/2021/09/notes-from-class-on-being-politically.html


