\section{Make Effective Presentations\label{sec:presentations}}

% https://graphthinking.blogspot.com/2011/10/presentation-notes.html

By paying attention to a bad presentation, you can find problems that you do not want to repeat.

Speaking is vital at decision points in your career progress - at interviews, competitions, gaining new collaborators at conferences, impressing peers/boss/subordinates. Thus, it's best not to make these mistakes in those situations.

Aspects to a presentation
\begin{itemize}
    \item speaker: appearance, verbal accent, enunciation, smell
    \item slides: layout, content
    \item audience: background
    \item venue: size, technology available, lighting
    \item lead time: how much time do you have to prepare?
\end{itemize}
The following list is meant to be reviewed as a checklist.

Planning your presentation

In introducing your research, there are a few ways to open the talk
\begin{itemize}
    \item compliment the audience
    \item humor: relevant joke
    \item how does this work make me feel
    \item explain context and relevance in terms of money, number of people involved, size of system
\end{itemize}
Estimate the experience/education background of your audience prior to creating the presentation. The question being answered in the presentation is independent of audience, but the level of delivery depends on audience background.


Speaking to an audience outside your field:
\begin{itemize}
    \item Use jargon your audience is familiar with. For example, a physicist uses ``quantization of energy" whereas an audience of mathematicians may be more comfortable with ``discrete energy levels."
    \item Look for commonality. For example, both physicists and mathematicians use assumptions and build models.
\end{itemize}
Visual presentation:
\begin{itemize}
    \item Switching between dark and light slides in a dark room stresses the eyes (need time to adjust to varying light levels).
    \item In figures, use both color contrast and distinct symbols to deal with colorblind audiences. See also \href{http://jfly.iam.u-tokyo.ac.jp/color/}{http://jfly.iam.u-tokyo.ac.jp/color/}.
    \item Making slides appear ``professional" means adding non-informational content. This added content should be consistent, not distracting.
\end{itemize}
Software:
Common choices include Latex (beamer), Microsoft Powerpoint. Less well-known are Prezi (and its open source equivalent, InkScape+Sozi add-on).

If you use Microsoft Word, a document PDF, notepad, or any other non-presentation software to make a presentation, then you are sending a few messages to your audience:
The audience isn't worth your time needed to develop or learn a proper presentation. 
You are not technically savvy.

Presenting multiple topics within one presentation:
Disparate topics require a segue to show why you are transitioning
Inter-relate the multiple topics 
Dealing with technical failure
If you plan to give a live demo, have screenshots of the process in the presentation. That way, if the demo fails you can show what was supposed to happen in the slides. If the demo works, skip the slides.
If possible, use the setup as close to reality as possible for practice sessions. Project onto the screen using the projector. This will show if the color contrast is sufficient.

Prior to giving a talk at a remote (or unfamiliar) environment

Questions to ask your host:

Will I have a projector and screen for the presentation?

If Power point available, what version is in use?

Will I be allowed to bring a USB drive or laptop into the location (i.e., are there security restrictions), or should I email the presentation file to you?

Day of talk

Appearance (suit and tie or jeans and t-shirt?):
\begin{itemize}
    \item Under-dressed = I don't respect the audience.
    \item Overdressed = I'm better than you.
    \item Similar level of dress = I'm a peer.
\end{itemize}
Some of these suggestions may seem glaringly obvious. The reason they are here is because I have personally seen them in ``official" presentations given by a ``professional."
\begin{itemize}
    \item Do not curse (profanity while speaking, or in the slide presentation, or even the name of the file).
    \item Practice the presentation (out loud in real time) at least once.
    \item Run spell check before presenting.
    \item Translate your slides from your native language to the language of your audience.
    \item Tell a coherent story with a unified theme. Each slide should be logically connected to the following slide. Don't just put a bunch of slides with data together. You risk disorienting your audience.
\end{itemize}
Before the talk begins,
\begin{itemize}
    \item Make sure your facility has power, a screen, a projector, pointer, any other necessary equipment. [Your host may not think of these things for you.]
    \item Make sure all equipment works and functions together.
    \item Before the presentation begins, use a slide to make announcements and reminders:
\begin{itemize}
    \item List the agenda if there are multiple speakers.
    \item Time talk begins, how long it will last.
    \item Reminder: Turn OFF Cell Phones.
\end{itemize}
    \item Announce whether to ask questions during talk (interrupt) or to hold questions until done.
\end{itemize}
During the talk
\begin{itemize}
    \item Opening: Thank hosts/inviters/organizers. Establish a connection between you (the speaker) and the audience.
    \item Do not pace.
    \item Do not stand frozen.
    \item If you are the only person laughing at your joke, it isn't funny.
\end{itemize}
 
After the talk
\begin{itemize}
    \item Let the person asking the question finish.
    \item Restate the question to make sure the audience heard it and that you understood it.
    \item Give the minimal answer. Save the long answer for followup.
\end{itemize}
