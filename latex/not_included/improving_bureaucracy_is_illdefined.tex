\subsection{Improvement is ill-defined}

Bureaucrats create and carry out policy governing access to shared resources. Measuring how effectively that is being carried out depends on what the goal is -- fair access to shared resource, latency of access, prioritized access on a basis of need, low-overhead, minimize fraudulent access. 

Separate from the question of managing the shared resource, measuring change of the administrative process is technically feasible:
\begin{enumerate}
    \item Identify recurring workflows.
    \item Given the steps in a workflow, count number of active tasks for each step.
    \item For each task in each step, measure the duration.
\end{enumerate}

For example, suppose your organization is going to revise an existing policy. That's a recurring activity that gets broken into steps. 
\begin{itemize}
    \item Legal review: would the language be useful in a court case.
    \item Compliance: is the policy consistent with laws and regulations.
    \item Finance: how will the new policy be paid for?
    \item Deployment of revision to policy.
    \item Updates to software that support the policy.
    \item Training of staff on the changes to the policy.
\end{itemize}


% * Implementing a new policy, or revising an existing policy
% ** Staffing 
% *** Training
% *** Hiring: recruiting, screening, interview
% *** Payroll, benefits
% *** Promotion 
% *** Have a building to sit in, or if remote-only, collaborative software and hardware 
% **** Building maintenance
% **** Building security 
% ** Legality evaluation
% ** Financial support for all listed 
% ** Compliance with data protection 
% ** Vetting (does app already exist), prototype, deployment, sustainment, decommission
% ** Hardware and software infrastructure for analytic
% *** Support for when hardware or software breaks
% *** Security updates for software
% *** Scanning for security issues in software 


Problems with trying to measure durations of tasks for a recurring workflow: 
\begin{itemize}
    \item Flux of processes due to turnover and due to evolving best practices.
    \item Localization: every team's process is different. Meaningful comparison relies on consistency between teams.
    \item Centralized aggregation of count and duration is burdensome to every single pipeline segment. Collecting data takes time away from the work.
    \item Enumeration: a complete workflow touches every role that every employee has.
    \item Incomplete data (not reported).
    \item Collection lag (not live, not real-time).
    \item Getting benefits from data requires someone with the skills to make sense.
\end{itemize}

Suppose we enumerated every workflow and collected every metric.
What is the consequence of the observations?
Optimize allocation of staffing and budget to maximize pipeline throughput and minimize latency.

That wouldn't tell 
\begin{itemize}
    \item Whether the good ideas succeeded.
    \item If smart and kind people are retained.
    \item If the improvements of latency and throughput benefit users.
    \item If improvements are occurring at the cost of higher burnout.
    \item If employees find the work emotionally rewarding 
\end{itemize}


Practical responses: 
\begin{itemize}
    \item Implement small scope rather than holistic (pet projects, iterative).
    \item Wait for complaints, fix as detected.
    \item Optimize for political relevance (high profile).
\end{itemize}
