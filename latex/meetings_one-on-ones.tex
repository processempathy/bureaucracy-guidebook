\subsection*{One-on-one check-in meetings\label{sec:meetings-one-on-one}}

% See https://graphthinking.blogspot.com/2022/10/one-on-one-questions-to-ask-in-meeting.html


% CONTENT CHOICE: I'm avoiding tips on "how to manage" and "how to coach"?
% I have a lot of potential content on 
% https://graphthinking.blogspot.com/2021/09/notes-from-half-day-course-on-coaching.html

% TODO:~\cite{1995_Grove}

In contrast to the ambush of a walk-around, formal one-on-one meetings are typically a planned interaction between a supervisor and a team member. These discussions can be mutually beneficial. The meeting is an opportunity for the supervisor to coach the team member and for the supervisor to learn what the problems are on the team. From the team member's perspective, a one-on-one is an opportunity to ensure alignment with the team's direction and provide insight on how to improve the team. There can be education in both directions. 

% https://graphthinking.blogspot.com/2021/09/notes-from-half-day-course-on-coaching.html
When a supervisor oversees multiple team members, holding one-on-ones can take a lot of time. With ten team members and an hour every other week, that's 12.5\% of the supervisor's time without counting prep and documenting the discussion outcomes.  There's a risk that the mindset becomes ``no news is good news" or ``everything is going well, no need to engage." The consequence of that minimal mode is that only 
\href{https://en.wikipedia.org/wiki/Seagull_management}{negative feedback causes interaction}. 
\index{Wikipedia!Seagull management@\href{https://en.wikipedia.org/wiki/Seagull_management}{Seagull management}}
\iftoggle{WPinmargin}{\marginpar{$>$Wikipedia: Seagull management}}{}%

\ \\

% https://graphthinking.blogspot.com/2021/05/the-agenda-for-one-on-one-meeting.html

A constructive check-in requires forethought for both the supervisor and the team members. For one-on-one meetings there are generic questions that can help the supervisor understand the team member's status. 

A supervisor should ask the team member questions
\footnote{Not sure what to discuss? An extensive list of questions on topics like career development, conversation starters, and job satisfaction are available on 
\href{https://github.com/VGraupera/1on1-questions}{https://github.com/VGraupera/1on1-questions}.}
%See also \href{https://news.ycombinator.com/item?id=22341138}{comments}.}
like:
\begin{itemize}
    \item What have you been successful with since we last met?
    \item What is blocking our team's progress?
    \item What are your plans?
    \item How are you collaborating with the rest of the team?
    \item If you could change one thing about our organization, what would it be and why?
    \item How do you plan to train your coworkers on topics you understand and they don't?
    \item What have you learned in the past month?
    \item What are the biggest risks for the team?
    \item What's limiting your productivity?
\end{itemize}
Responding to these questions takes time (an hour) and a willingness to be open. 

If the supervisor doesn't ask these questions, the team member can include them in the agenda and bring them up. 

Preparation for the one-on-one is a shared responsibility. The supervisor can review notes from the previous discussion and any artifacts that have been in progress. Before the meeting, the team member should document responses to the following questions:
\begin{itemize}
    \item What was discussed previously?
    \item What progress has been made since the previous meeting?
    \item What is blocking progress?
\end{itemize}

The generic questions above may not fit all situations, like when a new bureaucrat joins an existing team. 
%In addition to phase-specific questions, there are questions that fit all phases.
The one-on-one check-in should be tailored to the phase of the employee's progression. Frequency of check-ins depends on the newness of the team member, the complexity of the work, or how quickly the conditions are changing.
\begin{enumerate}
    \item \textit{Name of phase}: \underline{Welcome to the team!}\\
    \textit{Scenario: New team member, either new to the team or new to the organization. }\\
    Here the focus of the one-on-one is to ensure a smooth onboarding process. Get them up-to-speed on the technical challenges, professional norms, and integrated with other team members. Resolve administrative blockers like the following: Does the employee have the necessary computer log-in accounts? Do they have an email account? Are they on the mailing list? \\
    \textit{Questions you can ask}:
    \begin{itemize}
        \item What are the goals for the team?
        \item What items on the onboarding checklist are not yet completed?
        \item Who have you met on the team? What is your understanding of their role on the team?
    \end{itemize}
\textit{The duration of this phase could last between a day and two weeks.}
    \item \textit{Name of phase}: \underline{Initial contributions}\\
    \textit{Scenario: Team member handles small tasks. }\\
    The purpose of this one-on-one is for discussions on training, planning, and task reviews. This phase is characterized by the team member being dependent on others for their success. In this phase the employee collaborates on tasks.\\
    \textit{Questions you can ask}:
    \begin{itemize}
        \item What are the goals for the team?
        \item What are your task goals?
        \item What are you expecting to deliver to the team? When? 
        \item What dependencies does that deliverable have (external to the team or internal to the team)?
    \end{itemize}
\textit{The duration of this phase could last a few months to years.}
    \item \textit{Name of phase}: \underline{Experienced contributor}\\
    \textit{Scenario: Team member handles large tasks (which they break into subtasks). }\\
    The purpose of this one-on-one is to help the team member define their success. Activities include planning, resource allocation, and assessment. This phase is characterized by the need to coordinate with others on the team or other teams. Team member understands task scope and intent and relevant processes. Team member decomposes task into subtasks.\\
    \textit{Questions you can ask}:
    \begin{itemize}
        \item How do the artifacts you're working on support your plan for the team's progress?
        \item What dependencies does that deliverable have (external to the team or internal to the team)?
        \item What insights do you have about the team or organization?
        \item What insights do you have about the relevance of the task relative to the purpose of the organization?
        \item What should management be doing to enable the team's success?
    \end{itemize}
\textit{The duration of this phase could be the rest of a career.}
    \item \textit{Name of phase}: \underline{Facilitator}\\
    \textit{Scenario: Facilitating the productivity of others.}\\
    Rather than being task-oriented, this team member supports coworkers. \\
    \textit{Questions you can ask}:
    \begin{itemize}
        \item What observations from mentoring team members do you have?
        \item What collaborations should we be fostering?
    \end{itemize}
    \item \textit{Name of phase}: \underline{Peer}\\
    \textit{Scenario: Peer check-in.}\\ 
    This one-on-one is a form of mentorship. The value of the exchange is to get a different perspective and to hold each other accountable.
\end{enumerate}

To evaluate when a team member and their supervisor should move to the next phase in the evolution, have an explicit conversation about the threshold for progression. Your practice of Process Empathy involves accounting for different rates of maturation for team members.


%\ \\

% lots of comments, not much useful info
% https://news.ycombinator.com/item?id=30152268

%\subsubsection*{One-on-one meeting questions to spur discussion}

