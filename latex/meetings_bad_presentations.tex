\subsection*{Listening to Bad Presentations\label{sec:bad-presentations}}

Occasionally you attend a meeting with a bad presentation. The slides may appear slick but the content is poorly thought out. Or the presenter does not understand the topic well. Or the presenter has good content but does not convey it well. Or the presenter is wrong about the topic. Regardless of the cause, you should assume the presenter is making their best effort. 

You could remain silent, complain, criticize, ask leading questions, or offer constructive feedback. Your silence may result in other attendees and the presenter leaving with incomplete or wrong information. If you speak up you'll prolong the meeting or limit the presenter's time to convey their material. 

Your assessment of the presentation may be wrong. You may lack relevant information. A reliable technique for interjection is to assume a state of confusion instead of confidently asserting that the presenter is wrong. 
If you believe the presenter is wrong, asking about the source of their information is a good entry point.

You should ask for clarification when the information is correct but presented poorly (or above the level you understand). 

\ \\

% TRANSITION

By paying attention to a bad presentation, you can  identify issues that you do not want to repeat. The next section summarizes a few insights so you don't need to learn from bad presentations.