\section[Communication Reflects Your Thinking]{Communication Reflects and Shapes Your Thinking\iftoggle{shortsectiontitle}{\sectionmark{Communication and Your Thinking}}{}}
\iftoggle{shortsectiontitle}{\sectionmark{Communication and Your Thinking}}{}

% LONG1 shows up in the TOC
% LONG2 is the section title


Before you talk to another person, you have an internal dialog about how to interpret events and what you think will happen in the future. Your inner monologue shapes your expectations and your actions. When you talk with fellow bureaucrats, you're sharing some of your thoughts and allowing their thoughts to shape your contemplation. 

As a bureaucrat, you might have a few complaints you've thought about regarding your team or organization. Complaining about bureaucracy is a common way for bureaucrats to bond. Complaints occasionally lead to insights, but that's typically not the goal. 

By changing how you think about the challenges you face, you can alter your perception and your reputation. As an example of this change, can you turn complaints into impact statements? 
\marginpar{$>>$ Actionable Advice}%
\index{actionable advice}%
The following examples show how to spin a negative experience. You don't need to provide a solution. 

\ \\
\textit{Negative observation}: ``Logging into my computer takes a long time.''\\
\textit{Positive statement with explanation of impact}: ``If I were able to log into my computer more quickly, then I could do more tasks.''

\ \\
\textit{Negative observation}: ``I need support from a team that doesn't offer a ticket tracking system.''\\
\textit{Positive statement with explanation of impact}: ``If the service team I need support from offered a ticket tracking system, then I would be able to know the status of my request.''

\ \\
\textit{Negative observation}: ``When I submit a ticket for a support request to the team, I don't have visibility on the status of the request.''\\
\textit{Positive statement with explanation of impact}: ``If the service team I need support from offered visibility into their ticket tracking status, then I could proceed with other tasks knowing my request wasn't lost.''

\ \\
Sharing negativity is a way of commiserating, but it doesn't indicate to other people that you're adding value.  
When you notice a complaint, can you frame it as describing why the issue matters? 

\ \\

Another framing that is harmful to your thinking and how you communicate is the use of generalizations. Personifying other teams and organizations is a simpler-to-understand and simpler-to-describe model, though the simplification is often incorrect. 

Here's a progression that shows how to transform generalizations into action: 
\marginpar{$>>$ Actionable Advice}%
\index{actionable advice}%
\begin{enumerate}
    \item This organization does not like blueberry pie.
    \item No one in this organization likes blueberry pie.
    \item I don't know of anyone in the organization who likes blueberry pie.
    \item I like blueberry pie. How would I find someone else in the organization who wants blueberry pie?
\end{enumerate}
The first statement personifies the organization. Organizations cannot like or dislike tangible items, so that statement is meaningless. The second statement is slightly more precise (each person can have likes and dislikes) though still a generalization and therefore likely inaccurate. Did you confirm with every member of the organization that they do not like blueberry pie? The source of the generalization might be your lack of relationships with every member of the organization. That leads to the fourth statement which is correct, precise, and curious. 

\ \\

Now that you've reflected on how your internal dialog alters your perceptions and actions, the next section describes how communication goes wrong.