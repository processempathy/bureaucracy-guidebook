\section{``Street-Level Bureaucracy'' by Lipsky\label{review:lipsky_street}}

\cite{1983_Lipsky}

Intended audience:

Ben Payne has read this book: yes\\
Ben Payne has a copy: yes, physical\\
Ben Payne's assessment: Well-written and easy to read. Outsider perspective based on theoretical assessment of conventional roles. 


Lipsky separates the roles in a bureaucracy as: customer, bureaucrat, boss of the bureaucrat

\subsection*{Summary of claims}
From \cite{2015_Cooper}
\begin{quote}
Street-level bureaucracy (SLB) is a sociological theory that seeks to explain the working practices and beliefs of front-line workers in public services and the ways in which they enact public policy in their routine work. Developed by an American, Michael Lipsky, it examines the workplace in terms of systematic and practical dilemmas that must be overcome by employees, with a particular focus on public services such as welfare, policing, and education. The theory is based on the notion that public services represent ‘the coal mines of welfare where the “hard, dirty and dangerous work” of the state’ is done.’ According to Lipsky, that is because:

\begin{itemize}
\item demand from clients will always outstrip supply due to finite resources (cost, time, or service access). Most clients are unable to obtain similar services elsewhere (such as private alternatives to state organisations). As a result, employees must resort to ‘mass processing’ of excessive client caseloads.

\item extensive personal discretion is a critical component of the work of many front-line public sector employees, particularly those who undertake private, face-to-face interaction with clients to assess the credibility of cases. Employees must use their personal discretion to become ‘inventive strategists’ by developing ways of working to resolve excessive workload, complex cases, and ambiguous performance targets.4

    \item employees compromise the quality of their work by ‘creaming off’ cases that are likely to be straightforward or to have a positive outcome. Alternatively, workers may act as an ‘advocate’ for clients who are perceived as being at the tip of an iceberg of social vulnerability. Because workers are unable to offer all services to every individual they may be forced to ‘deny the basic humanity’ of other clients. These pragmatic micro choices ultimately become the de facto policy of the organisation, which may contrast starkly with its official stated aims.
\end{itemize}
This theory has implications not just for the individual employee but also the overall system. In particular, Lipsky suggests that the extensive unmet demand from clients means that even substantial expansion of staff and budgets are unlikely to decrease workload pressures. Instead, he predicted that increased capacity would result in ongoing expansion of the same level of service quality at a higher volume.
\end{quote}


