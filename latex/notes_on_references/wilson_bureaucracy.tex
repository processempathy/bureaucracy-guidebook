\section{``Bureaucracy'' by Wilson}

\cite{1991_Wilson}

Intended audience: researchers

Ben Payne has read this book: no\\
Ben Payne has a copy: yes, physical\\
Ben Payne's assessment: Written from the perspective of an outside. Within that limitation it has useful analysis, though there are parts I disagree with. Well-written and easy to read.


Wilson defines ``operators'' as the street-level bureaucrats [Page 33].

Agencies typically have (ambiguous) goals, which are separated (subjectively) into tasks. [page 34]

Incentives (rewards and penalties) matter more than attitude.
[page 51]

Agencies operate under constraints set by Congress; businesses have more freedom to respond to clients.

What distinguishes business bureaucracy from government bureaucracy the specified on page 115 as feedback loops and self-determination of scope.

Agencies are ``production" (ch8), or procedural/craft, or coping.
[page 245]

Procedural agencies have ambiguous outputs; Craft agencies have invisible operations
[page 250].

Coping or procedural agencies can discuss their activities but cannot verify their achievements
[page 252].

In chapter 5, agency environments were classified into four categories: majoritarian, entrepreneurial, clientist, and interest group.
[page 248].
