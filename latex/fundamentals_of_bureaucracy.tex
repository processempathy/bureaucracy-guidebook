\section{Fundamentals of Bureaucracy\label{sec:fundamentals-of-b}}
  
Bureaucracy is a system of distributed knowledge and distributed decision-making for managing shared resources. The decision-making is not purely logical; it relies on particular facets of human interaction. This section outlines the critical roles of decision-making, hierarchy, communication in bureaucracy, and feedback loops. 
% the following sentence acknowledges there's a difference but doesn't explain why there's a difference.
My list is smaller than Max Weber's~\cite{2015_Weber}\footnote{\href{https://en.wikipedia.org/wiki/Bureaucracy\#Max_Weber}{Bureaucracy: Max Weber}.
\index{Wikipedia!\href{https://en.wikipedia.org/wiki/Bureaucracy\#Max_Weber}{bureaucracy}}
} and I see bureaucracy as more widespread.

\hyperref[sec:decision-making]{Decision-making}
\marginpar{See page~\pageref{sec:decision-making}.}
%\ifsectionref
%(section~\ref{sec:decision-making}) 
%\fi
is central to bureaucracy. Every other aspect of bureaucracy derives from decision-making. The decision-making in bureaucracy is for the subjective management of resources. In my definition of bureaucracy, ``\iftoggle{glossarysubstitutionworks}{\glspl{shared resource}}{shared resources}'' 
\iftoggle{glossaryinmargin}{\marginpar{[Glossary]}}{} refers to both expertise and tangible goods like air, water, and land. 

When there are multiple people present, or even when one person is trying to be self-consistent, coordination of decision-making is crucial to the management of shared resources. Who gets to make which decision is managed using
\hyperref[sec:hierarchy-of-roles]{hierarchy}.
\marginpar{See page~\pageref{sec:hierarchy-of-roles}.}
%\ifsectionref
%; see section~\ref{sec:hierarchy-of-roles}.
%\fi
While coordination can occur without hierarchy when there are a small number of people, typically an organization of people leads to both formal and informal hierarchies. 

\hyperref[sec:meetings-for-coordination]{Meetings} and 
\marginpar{See page~\pageref{sec:meetings-for-coordination}.}
\hyperref[sec:written-communication]{written communication} help with consensus among bureaucrats, though agreement isn't necessarily the outcome.
Once a decision is made, the choice selected by a bureaucrat propagates throughout the organization to achieve some level of consistency. 

A less prominent feature of bureaucracy is the weak 
\gls{feedback loop}. 
\iftoggle{glossaryinmargin}{\marginpar{[Glossary]}}{} This distinguishes bureaucracy from a market-based system where the consequences of decisions manifest as profits and losses. If you sell shoes, you can evaluate decisions you've made by measuring how much money you make. In contrast, the decisions of a bureaucrat are barely felt by the bureaucrat. That weak feedback loop (few consequences for the decision-maker) means learning from mistakes is harder and positive reinforcement for good outcomes is negligible. A weak feedback loop does not mean there are no consequences. The actions of a bureaucrat create ripples that other people feel. This is described in more detail in the section on \hyperref[sec:feedback-loop-and-ripples]{feedback loops and ripples}.
\marginpar{See page~\pageref{sec:feedback-loop-and-ripples}.}

\ \\

If you're not an academic researcher of bureaucracy, you might think that you don't want to think about hierarchy or collaboration or coordination. Meetings and email are not your goal; you just want to do the tasks trained for. 

As you progressed through school you were graded as an individual student. Now that you are employed your job has pay, promotion, hiring/firing, and a title. School, and now employment, may seem focused on your individual abilities. The following sections explain why the individualist mentality is ineffective when you are part of a system of distributed knowledge and distributed decision-making. 


% there are examples where the ability of the team is measured: 
% An example of hiring a team is when a company buys another company for talent acquisition.