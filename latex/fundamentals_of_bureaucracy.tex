\section{Fundamentals of Bureaucracy\label{fundamentals_of_b}}
  
Bureaucracy as a system of distributed knowledge and distributed decision making for management of shared resources relies on particular facets of human interaction. I outline the roles of decision making, hierarchy, and communication as central to bureaucracy. My list is smaller than that of Weber \cite{2015_Weber}\footnote{\href{https://en.wikipedia.org/wiki/Bureaucracy\#Max_Weber}{Bureaucracy: Max Weber}}.

Decision making (\S\ref{sec:decision-making}) is central to bureaucracy. The management of resources requires policies to be created. 

Coordination of decision making within an organization of people motivates formal and informal hierarchies. Who gets to make which decision is managed using hierarchy; see \S\ref{sec:hierarchy_of_roles}.

Meetings and written communication help establish consensus among bureaucrats.
Once a decision is made, the choice selected by a person have to propagate throughout the organization to achieve consistency; see \S\ref{sec:meetings-for-coordination} on meetings and \S\ref{sec:written-communication} on written communication.
