\section{Failure to Communicate\label{sec:failure-to-comm}}

Communication is critical in bureaucracy because bureaucracy is a system of distributed knowledge and distributed decision-making. When less effective communication occurs, individual bureaucrats are less able to rely on the knowledge of other experts, and they have to make decisions with less consensus. 

This section on failure outlines how ineffective communication happens. Identifying generic sources of difficulty is intended to help you distinguish what is common in bureaucratic organizations versus what is specific to the people you collaborate with. 


\ \\

Sometimes failure to communicate is caused by too much ineffective communication. 
You can get overwhelmed when there is a lot of communication happening (whether from too many sources or too much content).
Feeling saturated with meetings, emails, phone calls, and other coordination may stem from ineffective communication. 
Communication that is incorrect, imprecise, redundant, and insufficient can be improved. 
% https://graphthinking.blogspot.com/2016/04/impact-of-communication-on-negotiation.html
Improvement in communication effectiveness is more than just a time saver --
poor communication yields poor negotiation. When negotiations are not going well,  the interaction defaults to an emotional struggle of willpower. 


\subsection*{Why written communication does not happen\label{sec:written-comm-does-not-happen}}
If you can identify specific causes of why communication isn't happening, then you can intervene and address the concerns others may have. The following list is a set of common reasons that you can remedy by developing a workaround or by discussing explicitly. 
\begin{itemize}
    \item Too many reports, emails, and messages to read and process and respond to. Bureaucrats feel emotionally and cognitively overwhelmed.
\item The receiving bureaucrat may read slowly.
\item The bureaucrat-as-author may type slowly, or their handwriting is poor.
\item Potential participants fear imperfect communication. What if the email is incomplete or inaccurate or ambiguous?
\item The bureaucrat views written communication as ``official" or ``plan of record'' and feels uncomfortable brainstorming or creating contingency plans.
\item The bureaucrat wants to avoid accountability for their statements.
\item The bureaucrat may not be confident in their writing ability -- spelling, grammar, sentence composition, structuring content. \footnote{For tips on writing, see page~\pageref{sec:resources-for-writing}.}
\end{itemize}
Once you can identify why written communication is not happening, you can work with the other people involved to develop creative solutions. 

Another source of challenge for written communication is the latency of asynchronous interactions. By labeling distinct types of delay, you are better equipped to respond when the issue arises. 

\subsection*{Slowing Communication\label{sec:slowing-communication}}

In an ideal scenario there would be no delay associated with communication -- you would get the information you need when you need it. In practice, there are various causes for why your progress is blocked when you depend on other people. 

Tactics that delay communication within a bureaucratic organization are stonewalling, slow-rolling, bikeshedding, and red herrings. By learning these concepts you will be better able to identify and then respond to their use.

\index{responsiveness!stonewalling}
\iftoggle{glossaryinmargin}{\marginpar{[Glossary]}}{}
\iftoggle{glossarysubstitutionworks}{\Gls{stonewalling}}{Stonewalling} 
is when the recipient of a request or question  doesn't reply. There are \hyperref[sec:email-responsiveness]{legitimate reasons for the lack of response}\iftoggle{haspagenumbers}{; see page~\pageref{sec:email-responsiveness}.}{.}

The person may be busy and didn't see your message, or they did see your message but didn't have a chance to reply yet because a response to you is lower priority than other tasks they have. You cannot differentiate those reasonable causes from when the recipient doesn't want to enable you to proceed. They may disagree with your objective and see silence as \href{https://en.wikipedia.org/wiki/Passive-aggressive_behavior}{less confrontational}
\index{Wikipedia!passive-aggressive behavior@\href{https://en.wikipedia.org/wiki/Passive-aggressive_behavior}{passive-aggressive behavior}}\iftoggle{WPinmargin}{\marginpar{$>$Wikipedia: Passive-aggressive behavior}}{}
than explicit rejection. 

One way of circumventing stonewalling is to ask if the respondent is opposed to your idea. 
\marginpar{$>>$ Actionable Advice} 
\index{actionable advice}
Then a lack of response indicates no opposition. This tactic applies if you are confident the recipient will read or hear the message.

An unintentional source of stonewalling is when you ask on the wrong channel. Sending an email may result in what appears to be stonewalling if the person relies on chat messages or the phone. The solution for this 
\marginpar{$>>$ Actionable Advice} 
\index{actionable advice}
is feasible, but action is required by the person who doesn't respond. The person who only uses certain channels should explicitly indicate that. An automatic out-of-office email that says, ``Contact me by phone'' tells the sender the \hyperref[sec:communication-preferences]{preferred channel}. 
\marginpar{See page~\pageref{sec:communication-preferences}.}

From the view of the person doing the stonewalling, if you need time to think or gather information before responding, 
\marginpar{$>>$ Actionable Advice} 
\index{actionable advice}
tell the person who sent a request that you acknowledge their message and will follow up in more detail later (with a specific timeline). While better than no response, this leads to the next challenge.

\index{responsiveness!slow-rolling}
\iftoggle{glossaryinmargin}{\marginpar{[Glossary]}}{}
\iftoggle{glossarysubstitutionworks}{\Gls{slow-rolling}}{Slow-rolling} 
is when you get a response to your request or question, but the response isn't helpful. Progress is delayed because you have to iterate to get an answer. There are valid reasons for a slow-roll and there are uncool reasons for a slow-roll response. Perhaps the person wants to acknowledge your request but doesn't currently have time to provide a full explanation. The person may need to gather more information for the complete response. Or the person is \href{https://en.wikipedia.org/wiki/Passive-aggressive_behavior}{passive-aggressive}
\index{Wikipedia!passive-aggressive behavior@\href{https://en.wikipedia.org/wiki/Passive-aggressive_behavior}{passive-aggressive behavior}}\iftoggle{WPinmargin}{\marginpar{$>$Wikipedia: Passive-aggressive behavior}}{}
and may understand your question but does not want to enable your progress. 

The reason for a slow-roll should be made explicit by the respondent, 
\marginpar{$>>$ Actionable Advice}
\index{actionable advice}
and a timeline for a complete response is helpful. 


\label{concept:bikeshedding}
\index{responsiveness!bikeshedding}
\iftoggle{glossaryinmargin}{\marginpar{[Glossary]}}{}
\iftoggle{glossarysubstitutionworks}{\Gls{bikeshedding}}{Bikeshedding} 
is when the recipient of a question or request 
\href{https://en.wikipedia.org/wiki/Law_of_triviality}{focuses on unimportant details relative to the primary topic}. 
\index{Wikipedia!Law of Triviality@\href{https://en.wikipedia.org/wiki/Law_of_triviality}{Law of Triviality}}\iftoggle{WPinmargin}{\marginpar{$>$Wikipedia: Law of Triviality]}}{}
Whether this behavior is intentional or not, the best response is to refocus the conversation on the core issue. The amount of time allocated for various topics should be proportional to their consequence. 

\index{responsiveness!red herring}
A \gls{red herring}\iftoggle{glossaryinmargin}{\marginpar{[Glossary]}}{}
response is misleading, whether intentional or not. The respondent provides what looks like a reasonable answer but results in unproductive work. Occasionally there is a coincidental benefit of discovering something unexpected, but that wasn't the respondent's intent. 


Even though bikeshedding and a red herring can be a dark pattern, that may not be the intent of the speaker or author. Perception matters more than intent.
\marginpar{$>>$ Mantra}
\index{mantra!perception matters more than intent}
Process Empathy applies both when you are sharing information (how could the information be perceived?) and when you are receiving information (what was the author's intent?).

\subsection*{Decreased Effectiveness in Communication and Some Remedies}

This section describes a few common challenges and what you can do if you find yourself in this situation.

\textit{Challenge}: The \href{https://en.wikipedia.org/wiki/Allen_curve}{Allen curve} 
\index{Wikipedia!\href{https://en.wikipedia.org/wiki/Allen_curve}{Allen curve}}
is 
\marginpar{[Tag] Folk Wisdom}
\index{folk wisdom!\href{https://en.wikipedia.org/wiki/Allen_curve}{Allen curve}}
%%%CANTDO\marginpar{[Wikipedia] Allen\\curve}
an ``exponential drop in frequency of communication between engineers as the distance between them increases."

\index{mantra!presence creates priority}
Presence creates priority - go to their desk. 
\marginpar{[Tag] Story Time}
\index{story time!Presence creates priority}
\begin{mdframed}
I needed some data from a coworker. After trying email and phone calls multiple times, I ended up flying across the country. Once I arrived the person was able to provide the data in a few hours.
\end{mdframed}

Merely sitting next to a coworker, even with no official purpose of interaction, results in spontaneous informal discussions. See the discussion of 
\hyperref[sec:prisoner-exchange]{Prisoner exchange}.
\marginpar{See page~\pageref{sec:prisoner-exchange}.}

Take advantage of the \href{https://en.wikipedia.org/wiki/Allen_curve}{Allen curve} 
\index{Wikipedia!\href{https://en.wikipedia.org/wiki/Allen_curve}{Allen curve}}
by implementing the Inverse Conway Maneuver: if you know what interfaces a product or process need, then design the placement of your team members to reflect that.

\ \\
\textit{Challenge}: \href{https://en.wikipedia.org/wiki/Wiio\%27s_laws}{Wiio's law}: 
\index{Wikipedia!\href{https://en.wikipedia.org/wiki/Wiio\%27s_laws}{Wiio's law}}
\marginpar{[Tag] Folk Wisdom}
\index{folk wisdom!\href{https://en.wikipedia.org/wiki/Wiio\%27s_laws}{Wiio's laws}}
``Communication usually fails, except by accident.''\\
This pessimistic take is similar to \href{https://en.wikipedia.org/wiki/Murphy\%27s_law}{Murphy's law}
\index{Wikipedia!\href{https://en.wikipedia.org/wiki/Murphy\%27s_law}{Murphy's law}}
and is indicates the level of investment needed for effective communication. 

\ \\
\textit{Challenge}: Periodic status reports sent up the chain of command get sanitized so that only good news is shared. The removal of ``bad'' information impedes risk analysis. \\
If your reports are getting sanitized, ask for a copy of the sanitized version. If you have the responsibility for consolidating and aggregating reports, aim for conciseness rather than good news. 

\ \\
\textit{Challenge}: Decisions by bureaucrats high in the \href{https://en.wikipedia.org/wiki/Command_hierarchy}{chain of command}
\index{Wikipedia!\href{https://en.wikipedia.org/wiki/Command_hierarchy}{command hierarchy}}
\marginpar{[Wikipedia] Command hierarchy}
are not pushed down the chain. \\
You can request management provide a summary of their activities.

\ \\
% Role of assumptions 
\textit{Challenge}: To assume makes an ass out of you and me, 
\marginpar{[Tag] Folk Wisdom}
\index{folk wisdom!To assume makes an ass out of you and me}
yet assumptions are necessary to making progress in communication.\\ 
You can address this dissonance by looking for sources of difference and then talking about them. For example, when you first talk with someone you can ask what their educational background is. You can tune your language to their academic training if they have a different degree than yours. You can make your story more relatable. 

Another technique for detecting differences is to ask about the person's previous experience. What did they work on previously in this organization? What were their jobs before joining this organization? This backstory can provide context for decisions that need to be made in the current context. 


\subsection*{Disrupting the Path to Failure}

% https://graphthinking.blogspot.com/2021/09/why-is-everything-so-hard-in-large.html 

To break the 
\href{https://en.wikipedia.org/wiki/Prisoner\%27s\_dilemma}{Prisoner's dilemma}, 
\index{Wikipedia!\href{https://en.wikipedia.org/wiki/Prisoner\%27s\_dilemma}{Prisoner's dilemma}}
\marginpar{[Wikipedia] Prisoner's\\dilemma}
options are 
\begin{itemize}
    \item Expose all participants to the consequence of outcomes. In practice this feels unfair to each participant because the outcome is partially attributable to other people involved in the process. Dividing responsibility limits exposure to consequences.
    \item Have all participants communicate. In practice communication takes time and skill. Not everyone is willing to invest since communication is not seen as ``doing the work.'' Accounting for the \href{https://en.wikipedia.org/wiki/Allen\_curve}{Allen curve}
    \index{Wikipedia!\href{https://en.wikipedia.org/wiki/Allen\_curve}{Allen curve}}
    \marginpar{[Wikipedia] Allen\\curve}
    takes effort. The time needed to arrive through consensus at an optimal approach for a given situation may exceed time available for solving the problem.
    \item Limit everything to what can be accomplished by one person. This hero-based approach is limited to the person's attention-bandwidth and skills. As the complexity increases the necessary skills increase and the number of candidate heroes decreases. Large organizations accomplish complicated tasks by leveraging diverse skill-sets of teams of bureaucrats.
\end{itemize}
To break the Prisoner's Dilemma that is rampant within bureaucratic organizations depends on recognizing the issue and identifying possible routes. You will have to invest effort beyond what other bureaucrats are doing.

%  How does communication among individuals fail?

%**************************

\ \\

% TRANSITION to communication_preferences

The multitude of potential causes of communication failure is made more complicated by the myriad ways individuals prefer to communicate. The next section provides an example why of a bureaucrat might prefer one channel over another. 
 
 