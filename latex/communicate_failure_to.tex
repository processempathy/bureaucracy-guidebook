\section{Failure to Communicate\label{sec:failure-to-comm}}

Communication is critical in bureaucracy because bureaucracy is a system of distributed knowledge and distributed decision making. When less effective communication occurs, individual bureaucrats are less able to rely on the knowledge of other experts, and they have to make decisions with less consensus. 

\subsection*{Decreased Effectiveness in Communication and Some Remedies}

\textit{Challenge}: The \href{https://en.wikipedia.org/wiki/Allen_curve}{Allen curve} 
\index{Wikipedia!\href{https://en.wikipedia.org/wiki/Allen_curve}{Allen curve}}
is 
\marginpar{[Tag] Folk Wisdom}
\index{folk wisdom!\href{https://en.wikipedia.org/wiki/Allen_curve}{Allen curve}}
an ``exponential drop in frequency of communication between engineers as the distance between them increases.'

Presence creates priority - go to their desk. I once needed some data from a coworker. After trying email and phone calls, I ended up flying across the country. Once I arrived the person was able to provide the data in a few hours.

Merely sitting next to a coworker, even with no official purpose of interaction, results in spontaneous informal discussions. See the discussion of 
\hyperref[sec:prisoner-exchange]{Prisoner exchange} on 
page~\pageref{sec:prisoner-exchange}.

Take advantage of the \href{https://en.wikipedia.org/wiki/Allen_curve}{Allen curve} 
\index{Wikipedia!\href{https://en.wikipedia.org/wiki/Allen_curve}{Allen curve}}
by implementing the Inverse Conway Maneuver. If you know what structure is needed for a product, then design the placement of your team to reflect that.

\ \\
\textit{Challenge}: \href{https://en.wikipedia.org/wiki/Wiio\%27s_laws}{Wiio's law}: 
\index{Wikipedia!\href{https://en.wikipedia.org/wiki/Wiio\%27s_laws}{Wiio's law}}
\marginpar{[Tag] Folk Wisdom}
\index{folk wisdom!\href{https://en.wikipedia.org/wiki/Wiio\%27s_laws}{Wiio's laws}}
``Communication usually fails, except by accident.''\\
This pessimistic take is similar to \href{https://en.wikipedia.org/wiki/Murphy\%27s_law}{Murphy's law}
\index{Wikipedia!\href{https://en.wikipedia.org/wiki/Murphy\%27s_law}{Murphy's law}}
and is indicative of the level of ongoing investment needed for effective communication. 

\ \\
\textit{Challenge}: Periodic status reports sent up the chain of command get sanitized so that only good news is shared. This impedes risk analysis. \\
If your reports are getting sanitized, as for a copy of the sanitized version. If you are the person consolidating and aggregating reports, aim for conciseness rather than good news. 

\ \\
\textit{Challenge}: Decisions by bureaucrats high in the \href{https://en.wikipedia.org/wiki/Command_hierarchy}{chain of command}
\index{Wikipedia!\href{https://en.wikipedia.org/wiki/Command_hierarchy}{command hierarchy}}
are not pushed down the chain. \\
You can request management provide a summary of their activities.

\ \\
% Role of assumptions 
\textit{Challenge}: To assume makes an ass out of you and me, 
\marginpar{[Tag] Folk Wisdom}
\index{folk wisdom!To assume makes an ass out of you and me}
yet assumptions are necessary to making progress in communication.\\ 
This dissonance can be addressed by looking for sources of difference and then talking about them. For example, when I talk with someone for the first time I ask what their educational background it. If they have a different degree than mine, I can tune my language to their academic training. I can make my story more relatable. 

Another technique for detecting differences is to ask about the person's previous experience. What did they work on previously in this organization? What were their jobs before joining this organization? This backstory can provide context for decisions that need to be made in the current context. 

\subsection*{Leveling up Your Communication}

There are levels of enlightenment for bureaucrats. If you know the progression, you can step up to the next level more easily.
\begin{enumerate}
    \item I feel bad.
    \item (complaint) I can't do what I want.
    \item (complaint) I can't do what I want in the way I want.
    \item (complaint) There is a problem.
    \item I have a solution.
    \item I have an implementation.
    \item I tried but my solution didn't work.
    \item (burnout) Life sucks but I get a pay check.
    \item I quit (in hopes of being more successful somewhere else).
\end{enumerate}


\subsection*{Theory of What Could be Done and Why the Theories are Impractical}

% https://graphthinking.blogspot.com/2021/09/why-is-everything-so-hard-in-large.html 

To break the 
\href{https://en.wikipedia.org/wiki/Prisoner\%27s\_dilemma}{Prisoner's dilemma}, 
\index{Wikipedia!\href{https://en.wikipedia.org/wiki/Prisoner\%27s\_dilemma}{Prisoner's dilemma}}
options are 
\begin{itemize}
    \item Expose all participants to the consequence of outcomes. In practice this feels unfair to each participant because the outcome is partially attributable to other people involved in the process. Dividing responsibility limits exposure to consequences.
    \item Have all participants communicate. In practice communication takes time and skill. Not everyone is willing to invest since communication is not seen as ``doing the work.'' Accounting for the \href{https://en.wikipedia.org/wiki/Allen\_curve}{Allen curve}
    \index{Wikipedia!\href{https://en.wikipedia.org/wiki/Allen\_curve}{Allen curve}}
    takes effort. The time needed to arrive through consensus at an optimal approach for a given situation may exceed time available for solving the problem.
    \item Limit everything to what can be accomplished by one person. This hero-based approach is limited to the attention-bandwidth of the person and their skills. As the complexity increases the necessary skills increase and the number of candidate heros decreases. Large organizations accomplish complicated tasks by leveraging diverse skillsets of teams of bureaucrats.

\end{itemize}

%  How does communication among individuals fail?

 
 