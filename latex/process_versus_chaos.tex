\section{Riding the Chaos}

Bureaucratic organizations are not perfect. Somethings work, others do not. A bureaucratic organization is a complex and chaotic\footnote{Chaotic in the \href{https://en.wikipedia.org/wiki/Chaos_theory}{mathematical sense} of non-linear dependence on initial conditions, not just colloquial sense of disorder} environment. A natural response is to impose processes and hierarchy and roles. Imposing structure can feel intellectually fulfilling and emotionally satisfying. Imposing structure looks like progress and may even help with your promotion in the hierarchy. The challenge (or continual source of employment) is that chaos is dynamic. A process created in response to chaos create new problems and is disrupted by changes in number of staff, changes to who is part of the process, changes to amount of work. The half-life of structure depends on the rate of change of tasking and the rate of \hyperref[sec:turnover]{personnel turn-over}; see page~\pageref{sec:turnover}.

A bureaucrat in the organization could strive for perfection, run away from the chaos (to something less chaotic), exploit the chaos, or adopt an attitude of ``I do what I can'' or ``I'll wait out the problem.'' 
Another option is to build skills for navigating the chaos of a bureaucratic organization.

Build and maintain your network of connections with fellow bureaucrats. 
\marginpar{[Tag] Actionable Advice} 
This augments the chain of command of a hierarchy. There is a turn-over in both hierarchical roles and in your professional network, which means you need to continually invest effort in creating new bonds and maintain existing relations. 

See change as an opportunity for improvement rather than a destruction of your previous investments. The \href{https://en.wikipedia.org/wiki/Sunk_cost}{sunk cost fallacy} applies to your emotional state and the resources you invested. Fear of change is understandable -- it disrupts the comfort of stability and known processes. Anticipating change and preparing for it ease the stress of change.

Leverage personalities and unique traits rather than expecting everyone to be interchangeable and treating differences as flaws. 
\marginpar{[Tag] Actionable Advice}
By knowing the strengths, interests, and weaknesses of coworkers you can facilitate the process of change. 