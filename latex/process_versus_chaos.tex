\section{Riding the Chaos\label{sec:process-chaos}}

Bureaucratic organizations are imperfect because they are composed of humans. %Somethings work, others do not. 
A bureaucratic organization is a complex and chaotic environment.\footnote{Chaotic in the \href{https://en.wikipedia.org/wiki/Chaos_theory}{mathematical sense} 
\index{Wikipedia!chaos theory@\string\href{https://en.wikipedia.org/wiki/Chaos_theory}{chaos theory}}
of non-linear dependence on initial conditions, not just colloquial sense of disorder.} A sensible response is to impose processes, hierarchy, and roles. Imposing structure can feel intellectually fulfilling and emotionally satisfying. Imposing structure looks like progress and may even help with your promotion in the hierarchy. The challenge (or continual source of employment) is that chaos is dynamic. A process created in response to chaos creates new challenges and is disrupted by changes in the number of staff, changes to who is part of the process, and changes to the amount of work. The half-life of structure depends on the rate of change of tasking and the rate of 
\marginpar{See page~\pageref{sec:turnover}.}%
\hyperref[sec:turnover]{personnel turnover}.
%\iftoggle{haspagenumbers}{; see page~\pageref{sec:turnover}.}{.}

A bureaucrat in the organization could strive for perfection, run away from the chaos (to something less chaotic), exploit the chaos, or adopt an attitude of ``I do what I can'' or ``I'll wait out the problem.'' 
Another option, and the focus of this section, is to build skills for navigating the chaos of a bureaucratic organization.

You can build and maintain your network of professional connections with fellow bureaucrats. 
\marginpar{$>>$ Actionable Advice}%
\index{actionable advice}%
This augments the chain of command of a hierarchy. There is a turnover in both hierarchical roles and in your professional network, so you need to continually invest effort in creating new bonds and maintaining existing relations. 

You can see change as an opportunity for improvement rather than a destruction of your previous investments. The \href{https://en.wikipedia.org/wiki/Sunk_cost}{sunk cost fallacy}
\index{Wikipedia!sunk cost fallacy@\href{https://en.wikipedia.org/wiki/Sunk_cost}{sunk cost fallacy}}
applies both to your emotional state and the resources you invested. Fear of change is purdent -- change disrupts the comfort of stability and known processes. Anticipating change and preparing for it eases the stress of change.

You can leverage personalities and unique traits rather than expecting everyone to be interchangeable and treating differences as flaws. 
\marginpar{$>>$ Actionable Advice}
\index{actionable advice}
By knowing the strengths, interests, and weaknesses of coworkers, you can facilitate the process of change. 