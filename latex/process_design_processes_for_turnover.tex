\subsection*{Design Processes for Turnover of Staff\label{sec:turnover}}

% https://graphthinking.blogspot.com/2020/02/design-for-turnover-rather-than-rely-on.html

When designing a \gls{process}, there are a few goals to optimize for: time-to-first-result, average latency, initial financial cost, total financial cost, flexibility to input conditions, throughput, scalability. In a theoretical sense all these factors should inform decision-making. An often neglected aspect which is harder to predict and harder to measure is the importance of employee turnover. On a long time scale, turnover of \glspl{bureaucrat} is a significant source of risk for any team or project. 

Processes are implemented differently than intended because the people implementing them are not the same people who came up with and designed them.

Designing processes relies on roles: participants are treated as interchangeable with other people who have similar skills. As a bureaucrat coming up with a novel process, accounting for differences in enthusiasm or communication among participants is difficult. 
Designing processes that are robust to turnover does not mean ignoring the unique talents of participants. 
Emphasize documentation (how and why) and training of new participants. 
\marginpar{$>>$ Actionable Advice}
\index{actionable advice}



\ \\

Making processes resilient to change requires  bureaucrats be educated beyond the requirements of the immediate task. That education occurs during onboarding of the bureaucrat, while carrying out the process, and as bureaucrats exit participation in the process. 

Onboarding new bureaucrats involves technical training, explanation of norms, learning the processes, and creating a professional network of coworkers. During this onboarding the new bureaucrat should be documenting their observations. Postponing the creation of documentation until the new person has experience results in a skewed and incomplete capture of the challenges.

Trained and experienced bureaucrats executing a process are responsible for coordinating with other bureaucrats. Bureaucrats should be cross-trained in other roles to gain familiarity with other parts of a process. 

Bureaucrats exiting the team or organization or changing roles have a responsibility to document their knowledge for team members. An exit interview can inform how the team improves. 

In each phase documentation is the mechanism for spreading knowledge. Documenting the why (in addition to the how) is critical for the reader. Access to documentation should be measured to determine whether there is value to the investment. 


\noindent\hrulefill

\ \\

% TRANSITION to process_change_existing
The opportunity for you to create novel processes is likely to be rare if you are a bureaucrat in an old organization or team. A more frequent activity is updating and revising existing processes. The next section covers topics specific to changeing a process.