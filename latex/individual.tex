\chapter{You are an Individual in an Organization\label{sec:individual-in-org}}
\iftoggle{shortsectiontitle}{\chaptermark{You are an Individual}}{}
  \iftoggle{showbacktotoc}{{\footnotesize Back to the \hyperref[sec:toc]{Main Table of Contents}}}{}

\iftoggle{showbacktotoc}{\ \\}{}
  
  \iftoggle{showminitoc}{\minitoc}{}
  
    This chapter focuses on the perspective of an individual bureaucrat operating within an organization. What actions can you take?  This chapter is foundational to the next chapter which describes   
    % 2024-08-10: TL thinks "working with others" sounds demeaning, and "collaborating with others" sounds more professional
    \hyperref[sec:working-with-other-bureaucrats]{collaborating with others}.%\iftoggle{haspagenumbers}{ (see page~\pageref{sec:working-with-other-bureaucrats}).}{.}
    
    As a bureaucrat, understanding your options for action matters. You're less effective if you don't know what your options are. If you are okay with being ineffective or are not feeling the harm of your ineffectiveness, keep in mind that
    there is no neutral member of a bureaucracy. Either you are contributing to the organization, or you are sapping resources from the organization. This dichotomy exists because the organization has a zero-sum allocation of resources.

    \ \\

    This chapter starts with the hiring and onboarding process, then focuses on your emotional state as a bureaucrat, and ends with how you can be effective. 
    % https://graphthinking.blogspot.com/2019/06/standard-issues-that-organization-must.html
    %Voluntary membership in a long-lasting organization implies a turnover rate. 
    Even though you may join an organization once, organizations that are not shrinking have to add new members due to turnover and sometimes growing size.
    Even with this ingest of new people an organization is distinct from the rest of society due to the differences in norms and goals. If an organization does not invest in onboarding new members, the norms of broader society will prevail. Littering, crime, pollution, scams, and fraud are usually aspects organizations try to constrain.

    %In section~\ref{sec:hiring} 
The next section on \hyperref[sec:hiring]{hiring} %\iftoggle{haspagenumbers}{ (see page~\pageref{sec:hiring})}{}
 is discussed from both the perspective of the person hiring and the person being hired. 
%In section~\ref{sec:promotion} 
Likewise, incentives for \hyperref[sec:promotion]{promotion}\iftoggle{haspagenumbers}{ (see page~\pageref{sec:promotion})}{}
are described from the view of the person promoting a bureaucrat and the person seeking promotion. 
%In section~\ref{sec:professional-training} 
%I explain the relevance of \hyperref[sec:professional-training]{professional training}
%\marginpar{See page~\pageref{sec:professional-training}}
%from the perspective of the employer and the employee. 
The reason to cover both roles is to help people in each position build empathy with the other person's experience. 

