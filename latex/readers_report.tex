\section{Reader's Report\label{sec:reader_report}}
An view from inside bureaucracy as a \href{https://en.wikipedia.org/wiki/Complexity_theory_and_organizations}{complex adaptive system}. 

Central claims:
\begin{itemize}
    \item Everyone in modern society participates in bureaucracy. Therefore learning to be a skilled bureaucrat is useful.
    \item Individuals already have experience with bureaucracy from conventional roles in modern society. 
    \item Emergent behavior and being a \href{https://en.wikipedia.org/wiki/Wicked_problem}{wicked problem} make the complexity of bureaucracy irreducible to a simplistic model.
    \item The techniques of bureaucracy are meetings, processes, communication. These facilitate the coordination of distributed knowledge for distributed decision making.  
    \item Bureaucracy occurs when there is a lack of common quantitative feedback mechanism for individuals.
\end{itemize}
The consequences of thinking like a bureaucrat include
\begin{itemize}
    \item You are not limited to only the direct personal interactions that you have with other people. Bureaucratic processes that exceed your direct visibility and experience extend your influence.
    \item Help you recognize options beyond the naive defaults associated with fallacies in section~\ref{sec:fallacies}. What's feasible? What's negotiable? Once you realize rules and processes are subjectively created and enforced, negotiation (and identifying who to negotiate with) is more obvious.
    \item Decreased surprise when thing is not the way you naively expect. See the dilemmas in section~\ref{sec:dilemma_trilemma} and known hazards in section~\ref{sec:unavoidable_hazards}.
\end{itemize}


The effects of bureaucracy are not attributable to one person, but each person can improve bureaucracy.