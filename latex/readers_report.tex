\section{Reader's Report\label{sec:reader_report}}
An view from inside bureaucracy as a \href{https://en.wikipedia.org/wiki/Complexity_theory_and_organizations}{complex adaptive system}. 

Central claims:
\begin{itemize}
    \item Everyone in modern society participates in bureaucracy. Therefore becoming skilled is useful.
    \item Individuals already have experience with bureaucracy from conventional roles in modern society. 
    \item Emergent behavior and being a wicked problem make the complexity of bureaucracy irreducible to a simplistic model.
    \item Bureaucracy requires 
    \begin{itemize}
        \item sufficient scale because it is an emergent phenomenon for distributed knowledge and distributed decision making. 
        \item lack of common quantitative feedback mechanism for individuals.
    \end{itemize}
    \item The techniques of bureaucracy are what lead to emergence at scale: meetings, processes, communication. 
\end{itemize}
The consequence of thinking like a bureaucrat is that
\begin{itemize}
    \item effects of bureaucracy are not the fault of one person, but each person can improve bureaucracy
    \item you are not limited to only the direct personal interactions that you have with other people. Framing allows you to think about processes that exceed your direct visibility and experience.
    \item Help you recognize options beyond the naive defaults associated with fallacies (\ref{sec:fallacies}). What's feasible? What's negotiable? Once you realize rules and processes are subjectively created and enforced, negotiation (and identifying who to negotiate with) is more obvious.
    \item Decrease surprise when thing is not the way you naively expect. Name dilemmas/trilemmas (\ref{sec:dilemma_trilemma}), known hazards (\ref{sec:unavoidable_hazards}).
\end{itemize}
