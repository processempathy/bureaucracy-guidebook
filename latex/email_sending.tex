%https://graphthinking.blogspot.com/2024/06/emotional-response-to-logistics-of-email.html

\subsection*{Logistics of Sending an Email\label{sec:sending_email}}


This section focuses on sending one email. Whether you are the only recipient or one of many receivers can change how you interpret the intent of the email. Whether you are in the ``to" or ``cc" field matters. Unfortunately, ``to" versus ``cc" are not reliable indicators since email senders do not reliably conform to the expected use. 


An email sent to multiple recipients may have different purposes for different readers. The reader's role or knowledge may factor into how they interpret the content. The inclusion or exclusion of recipients alters how the content is understood.  Knowing when to switch channels (e.g., to phone or in person) is vital.

\subsubsection*{Send One Email to One Person}

Consider the scenario where you send an email to a person. 
The recipient reads the content and reacts emotionally.
Alternatively, the recipient reads the content and attempts to project what the author may have intended.
%(For the scope of this post I'm excluding the topic of email forwarding and just focusing on how the recipient reacts.)

In this simplest case, as the author you have to account for how the reader might respond. Merely sending a set of facts, questions, requests, or personal observations misses half the situation.


\subsubsection*{An email to Two People}
Consider a different scenario where you send an email to two people. Now the possible responses include:
\begin{itemize}
\item Recipient reads the content and responds emotionally. This is typical.
\item Recipient reads the content and attempts to project what the author may have intended. Especially relevant when wording is ambiguous.
\item Recipient reads the email and attempts to project how the other recipient might react emotionally.
\end{itemize}
That last outcome is distinct from when you sent an email to one person. The challenge of a reader projecting how the other recipient might react makes writing content that doesn't get misconstrued complicated. 

There are a couple fields in email (to, cc, bcc) to convey the sender's intent. 
\begin{itemize}
\item ``to" can serve multiple purposes: ``for your situational awareness," or ``please take action," or ``please respond."
\item CC is equivalent to ``for your situational awareness, here's what you told the people in the `to' field." There's no expectation for action by the cc'd recipients, but there is the opportunity to interject nuance, clarification, or opposition.
\item BCC is similar to CC, but not telling other recipients (to, cc) that the BCC'd person has awareness of the content.
\end{itemize}
Permutations available when emailing 2 people include:
\begin{itemize}
\item You could send the email to the two people both as ``to" recipients. Both recipients know the other person received the email. You're treating them as equals. You may be expecting action or response from each person, or this might be a notice to both people.
\item You could send the email to the two people with one person being ``to" and the other person being ``cc". Typically You're not expecting a response from the cc'd person, though they can respond as needed. The ``to" person can see that you sent the content to both people.
\item You could use BCC for one person and put the other person in the ``to" field.
\item You could BCC both recipients. Neither recipient would know who else received the email.
\item You could send two separate emails, one to each recipient.
\item You could send two separate emails, one to one recipient, and then forward that email to the second recipient.
\item You can do the same as above except swapping the order of who gets the email first and who sees that the email was sent to the person.
\end{itemize}
Each of the above 7 options has a different emotional connotation for the recipients of the email. Beyond the issue that different people have different reactions to the same content, recipients will respond differently based on who else received the information and the order in which the information was shared.

A complicating factor in the analysis of sending to two people is whether they are peers, competitors, members of the same team with the same role, members of the same team with different roles, if there is a power disparity between the two recipients,  or one of the manager and the other is not.

A bad thing for BCC people to do is to reply to the thread and indicate to the other recipients that they received the content.

A common use case for BCC is with introduction emails. You know Bob and Sue but Bob and Sue do not know each other.  You send one email with the two of them as recipients introducing the two of them to each other. Bob can then send a reply with Sue as a ``to'' recipient and you BCC'd. In the email body, Bob will explicitly say that he has moved you to the BCC. The consequence is that you get to see that Bob replied to Sue, but if Sue replies to Bob by email, you don't need to be part of that conversation.



\subsubsection*{An Email to More than Two People}
The new aspect that arises with 3 or more recipients is whether there are factions among the 3 or more people. In the smallest case, options are:
\begin{itemize}
\item All three have the same reaction.
\item All three have three different reactions.
\item Two people have the same reaction, and that is different from the third recipient's reaction.
\end{itemize}
When there are groups of people (here two people form a group) then there are insiders and outsiders. In the previous scenario when you sent email to 2 recipients, there is no insider/outsider dichotomy (just power imbalances). When you send email to 3 or more recipients, there can be groups and that means having insiders (a group with a common reaction to the email content) and outsiders (who don't have the same reaction).

A recipient projecting how someone else might respond to the content still applies, but now in addition to ``How did the other person respond?" when there were just 2 recipients the projection can also be ``How did the group respond?" or ``How did the outsider respond?"



\subsubsection*{Tactics for Emails with Many Recipients}
The easiest escape from this complexity is to restrict the emails you send to just one person at a time. As the sender, you no longer have to account for how different audiences will react, nor do you have to account for groups with different reactions.

When communicating with a collection of groups, splitting the message by group can help. Partition the email into like-minded cohorts. That way each person in their group is less concerned about how someone in another group would respond. For example, suppose you have a message for members of the workforce and management.
\begin{itemize}
\item Talk with members of the workforce to understand the systemic issue.
\item Tell management the issue, that it is with input of the team, and here are solution options.
\item Forward the message you sent management to the team so they can see that the topic was presented to management.
\end{itemize}
In general, team members and management getting the same message should be two separate emails.

Personalized emails (using mail merge) are a way to increase the likelihood of content consumption. This tactic is the additional benefit the recipient doesn't know who else the message went to. Personalized emails also remove the risk of a projected emotional reaction of a hypothetical third party.