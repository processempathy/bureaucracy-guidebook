
Phases of life is academic

**********************

Formalized titles for roles in the bureaucracy refers to two different aspects.
First you are given the title, the title is not self-anointed.
Second, other people make assumptions about you based on that title and they accept it. They are giving up their autonomy and power in negotiation in difference to your title

**********************

== why processes exist ==
I previously thought that processes were for new people who didn't have relationships and that older members who would have relationships didn't want process. Now I have seen that older members who are not aggressively social want processes so that they can be told what to do. People who rely on their relationships don't want processes. Not everyone takes initiative to form and maintain relationships.

Processes are a way of relying on a top down directive to avoid responsibility.People want processes for consistency and fairness. When there is not processes, an interpersonal conflict arises which most people fear and avoid.

**********************
There is no transparency on the input or output sides for subjects. Forms ask for what but do not provide the whySimilarly on the output side, decisions are made but without justification or reasoning provided
The subject has no input on who gets to make the decisions, nor does the subject feel the decision maker represents them, nor is there any obligation of competence in the decision making
The experience of a subject outside the organization is not the same as the experience of a bureaucrat as a subject within the organization. Both can complain to management. However, the hierarchical distance is significantly different

**********************

The roles and titles and hierarchy are formalized in order to provide consistency in the distribution of shared resources.

The bureaucrat's experience is more focused on the policies associated with the task of allocating and managing the shared resource. Bureaucrats are executing task assigned by someone else's policy

**********************

Bureaucracy requires three people, the policymaker, the bureaucrat to execute the task associated with policy, and the subject upon which the bureaucracy is inflicted.That requires a minimum of three different roles.

Marriage licenses are a limited resource in the sense that only the state gets the issue them

**********************

Informal implicit bartering for attention in a hierarchical structure

Attention here applies both to listen to my argument and take the time to tell me your information and awareness

The shared resources like money and staffing equipment our secondary to the attention bartering

The value of organizations, teams, and companies is that they are a simplifying assumption that depersonalize the function of inputs and outputs.

The purpose of a team or organization or company is to obfuscate the intrinsic bureaucracy of the function


==== hierarchy - communication up and down ====

What's the information conveyed in a hierarchy?
* Bottom up is situational awareness
* Top down is relative prioritization of objectives
This is applicable in the military. Military members are providing situational awareness of the battle space.
This is applicable in business. Workers provide information about customer interactions to decision makers
And some businesses workers provide technical insight to decision makers

This is incorrect -- situational awareness is conveyed in both directions. Priority is conveyed in both directions. Tactics and strategy each go up and down the chain

************************

== work distribution model==
to measure: time spent coordinating versus working per person
how does that change based on organization size, task complexity, social circle size

Scenario: A2, b2 or...a0,b0; task_duration = coordination_duration
Result: coordination means lower throughput (with social size 0).
Result: social circle size matters to throughput


***************************

== todo list ==

use Grammarly
search for Todo
search for \ \\
search for Figure H
In the "dilemmas" section, use \iftoggle{boundbook}{ (\ref{})}{}

**************


Journal section: self-reflection; drawing section
**************

Incentives and feedback loops on managing the shared resources have drawbacks
* Taking a cut of taxes collected
* Profit sharing leads to corruption
The alternative is no feedback loop in terms of risk or profit
Incentive is stability and predictability
source:
Ezra Klein show
https://podcasts.google.com/feed/aHR0cHM6Ly9mZWVkcy5zaW1wbGVjYXN0LmNvbS84MkZJMzVQeA/episode/NjE5YTc4MWYtYTVlMC00OTVkLTkyNWMtNzZlNzBmMjNmZTJk?ep=14

Nicholas Bagley, law professor
Procedural fetishism

Feedback loops are in tension with management of shared resources
For example you get corruption or regulatory capture as soon as there is incentive to make certain policies or favor of certain constituents

**************
My book contains neither principles nor strategy. Just tactics derived from the definition

******************
In the innovation section, labels that get used:
Iconoclast, Mavericks, rebels, innovators, cowboys
******************

Hypothetically there'd be "lessons learned", but without a feedback loop bureaucrats just do whatever they did before. 

Example of a bureaucrat's thinking:
I am unable to describe in detail what I actually want. I don't know how to do it, I just need it to be done I don't know how to oversee the people who might be able to do the work, and I don't know how to evaluate whether they can do the work. I have not learned project management skills.

Example of a bureaucrat's thinking:
I'd rather not look into details, especially if I'm not empowered to make the changes or if I don't have the resources to do so.

**************
meeting attendees need to have context prior to meeting, otherwise time is spent educating novices. 

The impact of context variance increases as the number of participants increases. solution: read-ahead documentation, attending previous meetings, participating in processes (experience)

*************************
How much friction is the result of misaligned incentives?

*******************
This book is not a simple claim like "the 7 types of bureaucrat". 

*********************
Senior positions in the hierarchy are more "political" because there is less process and more reliance on relationships.
*******************************
Outcomes of Crowding in Emergency Departments; a Systematic Review
https://www.ncbi.nlm.nih.gov/pmc/articles/PMC6785211/ 
****************************
[speculation] Cost to the organization of an employee is 2X the salary as per university grants
*******************

*****************
systems based practice 
https://www.google.com/search?q=%E2%80%9Csystems+based+practice%E2%80%9D

and “health systems science”https://www.google.com/search?q=%E2%80%9Chealth+systems+science%E2%80%9D*************************
Google search for dilemmas and bureaucracyprocesses and bureaucracy
trilemma bureaucracy

https://freakonomics.com/?s=bureaucracy

**************************
FDA example of scale
https://thebrowser.com/notes/rohit-krishnan/
https://www.strangeloopcanon.com/p/why-do-we-dislike-rules-so-much?s=r
**************************
process patterns
https://www.blogger.com/blog/post/preview/6676997041585153703/1414396592095465493

reasoning using logic or emotion?
https://www.blogger.com/blog/post/preview/6676997041585153703/9159354612473133556
**************************

==== numerical model of maven density as a function of community size ====
Assume a gaussian distribution of intelligenceAssume a gaussian distribution of relationship connectivity between 20 and 150Assume there's no correlation between intelligence and connectivityAssume a graph of connectivity with maximum separation of six degrees
What is the distribution of distances between two members of the population that are in the top 5% of intelligence?
