
BLUF: The lobotomy of management is caused by the challenge of coordination

Management may oversee multiple different disciplines (e.g., engineering, marketing, manufacturing) and different types of people (e.g., outgoing, introverted) and as a result have insufficient time to address everyone's needs. 

Management is supposed to facilitate coordination in the hierarchy between teams. Management often lacks technical expertise and lacks time needed to understand each issue deeply. Management adds latency to the process of coordination. The alternative, not including management in coordination and instead of relying on direct interaction between peers, is chaotic.

Management is supposed to make decisions that affect multiple different teams, but lacks expertise and the time that understand the depth for each issue.

Management has the holistic perspective to identify gaps and inefficiencies. Management has the authority to redirect resources to address these issues.

Manage is supposed to coordinate up, down, and laterally. Each of those is supposed to be bidirectional.

The choice of the manager is to either dive deeply into one topic (and neglect the other needs) or try and superficially respond to all needs and seem like a dilettante.Because management cannot address issues satisfactorily, they appear incompetent.

************

The action associated with the observation that management isn't leadership is that there is a void to fill and the void is addressed by influence without authority

One option is to separate the narrow subject matter experts from the roles of management listed above. Another option is referred to in academia as citizenship duties. Serving on committees, interviewing applicants, promotion boards, mentoring students. Those are in addition to the duties of teaching and research.
 
